\documentclass{article}
\usepackage{amsmath,amsthm,amsfonts,amssymb,fullpage,bbm}

\newtheorem{problem}{Problem}

\begin{document}

\begin{flushright}
Kris Harper\\

MATH 26300\\

April 27, 2010
\end{flushright}

\begin{center}
Homework 4
\end{center}

\begin{problem}
Construct a simply-connected covering space of the space $X \subseteq \mathbb{R}^3$ that is the union of a sphere and a diameter. Do the same when $X$ is the union of a sphere and a circle intersecting it in two points.
\end{problem}
\begin{proof}
Consider an infinite string of $2$-spheres connected by line segments as in the picture. We can break the chain into a disjoint union of sets that are the union of a closed interval, a sphere and a half-open interval.
\vspace{100pt}
Now define the covering map $p$ to take the sphere to the sphere and the two intervals to the diameter. Extend the map to the whole chain in the same way. This is a covering map because we can cover the sphere with two open sets which are taken to a disjoint union of open sets in the covering space under $p^{-1}$, then cover the diameter with two open intervals (one end of each will contain a small open set of the sphere). These two intervals will get mapped to the corresponding connecting intervals in the covering space. The covering space is simply-connected because it's a connected union of line segments and spheres which are simply-connected.

In the next case we have a network of $2$-spheres connected by line segments as in the picture in which there are two line segments connected to the north and south poles of each sphere. Each north pole has a line segment connecting to the south pole of one sphere and another connecting to the north pole of another sphere. We can break the chain into a disjoint union of sets which are a union of a sphere and four line segments.
\vspace{150pt}
In each of these sets the sphere maps onto the sphere. The two line segments directed to the right map to the diameter and the two line segments directed to the left map to a semi-circle connected to the diameter forming a circle intersecting the sphere in two points. Since every sphere in the network has the same structure (i.e., two line segments emanating from each pole connecting to opposite poles of other spheres) we see that this is indeed a covering space since it's locally homeomorphic to $X$. Furthermore it's simply connected because it's the connected union of simply connected spaces.
\end{proof}

\begin{problem}
Let $a$ and $b$ be the generators of $\pi_1(S^1 \vee S^1)$ corresponding to the two $S^1$ summands. Draw a picture of the covering space of $S^1 \vee S^1$ corresponding to the normal subgroup generated by $a^2$, $b^2$, and $(ab)^4$, and prove that this covering space is indeed the correct one.
\end{problem}
\begin{proof}
The covering space looks as follows.
\vspace{150pt}
We know that this is the correct covering space because $a^2$, $b^2$ and $(ab)^4$ are clearly the only nontrivial loops in this space. Moreover the fundamental group for a $1$-cell structure is generated by these nontrivial loops.
\end{proof}

\begin{problem}
Find all the connected covering spaces of $\mathbb{R}P^2 \vee \mathbb{R}P^2$.
\end{problem}
\begin{proof}
Note that $\pi_1(\mathbb{R}P^2) \approx \mathbb{Z}/2\mathbb{Z}$ so by Van Kampen's Theorem we know $\pi_1(\mathbb{R}P^2 \vee \mathbb{R}P^2) \approx \mathbb{Z}_2 * \mathbb{Z}_2$. To categorize the covering spaces of $\mathbb{R}P^2 \vee \mathbb{R}P^2$ we first categorize the subgroups of $\mathbb{Z} * \mathbb{Z} = \langle a,b \mid a^2, b^2 \rangle$. Note that since $a^2 = b^2 = 1$ we see that every word in $\mathbb{Z} * \mathbb{Z}$ is a word alternating in $a$ and $b$. The words of odd length start and end with the same letter and thus have order $2$. The words of even length are of the form $(ab)^n$ for $n \geq 0$. Note that $(ba)^n$ is the inverse for $(ab)^n$ so if $(ba)^n$ is in a generating set we also know $(ab)^n$ is in the set. Clearly $\langle (ab)^n \rangle$ is cyclic so we can replace any generating set of the form $\langle (ab)^{n_1}, (ab)^{n_2}, \dots \rangle$ with $\langle (ab)^n \rangle$ for some $n$. We are left with groups of the form $\langle (ab)^n, w \rangle$ for $w$ of odd length. Note that if $w$ and $w'$ are distinct words with odd length then $ww' = (ab)^n$ for some $n$ so any generating set of the form $\langle (ab)^n, w, w' \rangle = \langle (ab)^m \rangle$ for some $m$. Thus the only possible subgroups of $\mathbb{Z} * \mathbb{Z}$ are $\langle (ab)^n \rangle$ and $\langle (ab)^n, w \rangle$ where $n \geq 0$ and $w$ has odd length.

We now make covering spaces which correspond to these generators. In the case of $\langle (ab)^n \rangle$ we take a chain of $2n$ $2$-spheres connected at the ends. We can arbitrarily pick a basepoint between two spheres and then number the spheres in a clockwise fashion. The even $2$-spheres will map to the first copy of $\mathbb{R}P^2$ by the antipodal map and the odd spheres will map to the second copy of $\mathbb{R}P^2$ using the same map. Note that any loop in this space which doesn't circle all the way back the basepoint will be nullhomotopic since $\pi_1(S^2 \vee S^2) = \pi_1(S^2) * \pi_1(S^2) = 0$. Thus the only nontrivial loop in this space will be $(ab)^n$ making it the only generator of the fundamental group.
\vspace{200pt}

In the second case we have $\langle w \rangle$ where $w$ is a word of odd length. Take an infinite sequence of $2$-spheres with a copy of $\mathbb{R}P^2$ placed in the middle at some point. If $w$ has length $2n+1$ then place the basepoint $n$ spheres away from the copy of $\mathbb{R}P^2$. Map alternating spheres to each copy of $\mathbb{R}P^2$ using the antipodal map as above and map $\mathbb{R}P^2$ the the corresponding copy of $\mathbb{R}P^2$ as it falls in line. Switching which copies of $\mathbb{R}P^2$ each sphere maps onto will interchange the generators in $w$.

Finally we're left with subgroups of the form $\langle (ab)^n, w \rangle$ where $m > 0$ and $w$ has odd length of $2m+1$. First we need to consider the case where $n < m+1$. In this case the the length of the word $w$ exceeds the length of $(ab)^n$. Then consider the group $\langle (ab)^n, w(ab)^n \rangle$ where $w$ starts with $a$ (or possibly $\langle (ab)^n, (ab)^n w \rangle$ if $w$ starts with $b$). This group is clearly contained in the earlier group, but the earlier group is also contained in this one since $w(ab)^n(ab)^n = w$. Therefore we can continually take a group presentation in which $w$ has a shorter length until $n \geq m+1$.

We can now draw the covering space which will consist of $\mathbb{R}P^2 \vee \mathbb{R}P^2$ with a circle of $n-1$ $2$-spheres attached to each end. Pick a basepoint which is $m$ spheres away from one of the copies of $\mathbb{R}P^2$. Then we have a generator of length $2m + 1$ corresponding to $w$.
\vspace{200pt}
We also have a generator of length $2(n-1-m) + 1 = 2m' + 1$ formed by going the other direction around the other copy of $\mathbb{R}P^2$. Putting these two generators together we have a generator of length $2m + 1 + 2(n-1-m) + 1 = 2n$ which corresponds to $(ab)^n$. Each sphere must map to alternating copies of $\mathbb{R}P^2$ with the particular copy determined by $w$. Each copy of $\mathbb{R}P^2$ maps to the opposite copy of $\mathbb{R}P^2$ that the sphere next to it maps to.

All of these spaces are connected and they're covering spaces of $\mathbb{R}P^2 \vee \mathbb{R}P^2$ because they are locally homeomorphic to that space.
\end{proof}

\begin{problem}
Given maps $X \to Y \to Z$ such that $Y \to Z$ and the composition $X \to Z$ are covering spaces, show that $X \to Y$ is a covering space if $Z$ is locally path-connected, and show that this covering space is normal if $X \to Z$ is a normal covering space.
\end{problem}
\begin{proof}
Let $f : X \to Y$, $g : Y \to Z$ and $h : X \to Z$ such that $h = g \circ f$. Since $h$ and $g$ are both covering spaces of $Z$, there are two open covers of $Z$, $\{ U_{\alpha} \}$ and $\{ V_{\beta} \}$ such that $h^{-1}$ and $g^{-1}$ respectively take these to a disjoint union of open sets in $X$ and $Y$ which are local homeomorphisms. For each $z \in Z$ take a $U_{\alpha}$ and a $V_{\beta}$ containing it so that $z \in U_{\alpha} \cap V_{\beta}$. Since $Z$ is locally path connected we can find an open neighborhood of $z$ contained in this set which is path-connected. We can do this for each point in $Z$, so call the resulting open cover $\{ W_{\gamma} \}$. Since each $W_{\gamma} \subseteq V_{\beta}$, we know $g^{-1}(W_{\gamma})$ is still a disjoint union of open sets in $Y$ and the same can be said about $h^{-1}$ and $X$. Note that the collection $g^{-1}(W_{\gamma})$ for each $\gamma$ is an open cover of $Y$. Suppose we consider the preimage of these sets under $f$. But this amounts to looking at the sets under $h^{-1}g$. Since $g^{-1}(W_{\gamma})$ is locally homeomorphic to $W_{\gamma}$ and $h^{-1}(W_{\gamma})$ is a disjoint union of open sets in $X$, for each $g^{-1}(W_{\gamma})$ we see that the preimage under $f$ is a disjoint union of open sets in $X$. Note further that each of these sets is mapped homeomorphically to $W_{\gamma}$ by $h$ and each $W_{\gamma}$ is locally mapped homeomorphically to $g^{-1}(W_{\gamma})$ by $g^{-1}$. Since we've picked all the $W_{\gamma}$ to be path connected, it follows that $f$ must be a covering space.
\end{proof}

\begin{problem}
Given a covering space action of a group $G$ on a path-connected, locally path-connected space $X$, then each subgroup $H \subseteq G$ determines a composition of covering spaces $X \to X/H \to X/G$. Show:\\
(a) Every path-connected covering space between $X$ and $X/G$ is isomorphic to $X/H$ for some subgroup $H \subseteq G$.\\
(b) Two such covering spaces $X/H_1$ and $X/H_2$ of $X/G$ are isomorphic iff $H_1$ and $H_2$ are conjugate subgroups of $G$.\\
(c) The covering space $X/H \to X/G$ is normal iff $H$ is a normal subgroup of $G$, in which case the group of deck transformations of this cover is $G/H$.
\end{problem}
\begin{proof}
(a) If $p : X \to X/G$ is a covering space then by the classification theorem giving a bijection between covering spaces and subgroups of $\pi_1(X/G)$ we know that $X$ is isomorphic to $X/H$ for some subgroup $H \leq \pi_1(X/G)$.

(b) If $X/H_1$ and $X/H_2$ are isomorphic then there exists some isomorphism taking $\widetilde{x_0}$ to $\widetilde{x_0}'$. But since $X$ is path-connected and locally path-connected this amounts to changing the basepoint of $\pi_1(X,x_0)$ so the corresponding subgroups $H_1$ and $H_2$ must be conjugate by a path in $X$. Conversely, if $H_1$ and $H_2$ are conjugate then there exists some path in $X$ conjugating them. This path defines a change of basepoint isomorphism from $X/H_1$ to $X/H_2$.

(c) If $H$ is a normal subgroup of $G$ then $\pi_1(X/G)/\pi_1(X)$ is normal in $G$. But then $p_*\pi_1(X/H)$ is normal in $\pi(X/G)$ which shows that $X/H \to X/G$ is normal. Conversely, if $X/H \to X/G$ is normal then $\pi_1(X/H)$ is normal in $\pi_1(X/G)$. But then $H$ must be normal in $G$. In this case the group of deck transformations is given by $G/H$ since $X/H \to X/G$ is the covering space and we have a normal cover.
\end{proof}

\begin{problem}
A \emph{topological group} is a group $G$ equipped with a topology such that the maps
\[
\begin{tabular}{cc}
$i : G \to G$, & $\mu : G \times G \to G$\\
$g \mapsto g^{-1}$ & $(g,h) \mapsto gh$\\
\end{tabular}
\]
are continuous. Let $G$ be a topological group, with identity $e$.\\
(a) For any two paths $\gamma$, $\delta$ in $G$, define a path $\gamma * \delta (t) = \gamma (t) \delta (t)$. If $\gamma$ and $\delta$ are loops at $e$, show that $\gamma . \delta \simeq \gamma * \delta \simeq \delta . \gamma$. Conclude that $\pi_1(G,e)$ is abelian.\\
(b) Assume $G$ is path-connected and locally contractible, and $p: (\widetilde{G}, \widetilde{e}) \to (G,e)$ is a path-connected covering space. Show that $\widetilde{G}$ naturally has the structure of a topological group such that $p$ is a group homomorphism. Show, further, that $\pi_1(G,e)/\pi_1(\widetilde{G},\widetilde{e})$ is a discrete subgroup of $\widetilde{G}$.
\end{problem}
\begin{proof}
(a) For any two paths at $e$, $\alpha$ and $\beta$ suppose that $\alpha \simeq \alpha'$ and $\beta \simeq \beta'$. Then note that since $\mu$ is continuous, the product of the homotopies taking $\alpha$ to $\alpha'$ and $\beta$ to $\beta'$ is a again a homotopy taking $\alpha * \beta$ to $\alpha' * \beta'$. Let $\epsilon$ be the constant path at $e$. Note that $\gamma.\delta = (\gamma .\epsilon)*(\epsilon .\delta)$ and from a homotopy point of view this means that $[\gamma].[\delta] = [\gamma . \epsilon]*[\epsilon . \delta] = [\gamma] * [\delta]$. Now if we want a homotopy between $\gamma * \delta$ and $\delta * \gamma$ we can simply take $H(s,t) = \gamma^{-1}(st)\gamma(t)\delta(t)\gamma(st)$. When $s = 0$ this is $\gamma(t)\delta(t)$ and when $s = 1$ this is $\delta(t)\gamma(t)$.

(b) Let $\alpha$ and $\beta$ be paths in $\widetilde{G}$ starting at $\widetilde{e}$ and terminating at $a$ and $b$. We can define the path $\gamma (t) = p(\alpha (t))p(\beta (t))$. By the unique lifting property of covering spaces this lifts to a unique path $\widetilde{\gamma}$ with starting point $\widetilde{e}$. We define the product of $a$ and $b$ to be $\widetilde{\gamma}(1)$. To show that this is well defined, take another path $\alpha'$ starting at $\widetilde{e}$ and ending at $a$. We wish to show that $\gamma' = p(\alpha')p(\beta)$ lifts to a path which is homotopic to $\widetilde{\gamma}$. Note that $p(\alpha)$ and $p(\alpha')$ are both paths in $G$ starting at $e$ and ending at $p(a)$. By the homotopy lifting property we see that the lift $\widetilde{\gamma'}$ of $p(\alpha')p(\beta)$ is homotopic to $\widetilde{\gamma}$ so multiplication is well-defined. The inverse map on $\widetilde{G}$ is defined similarly by using the inverse map $i$ in $G$ and lifting this to a path in $\widetilde{G}$ then taking the endpoint.

The fact that multiplication and inverses are is continuous follows immediately from the fact that $\mu$ and $I$ are continuous since the composition of continuous maps is continuous. Furthermore, the element $\widetilde{e}$ clearly acts as an identity element for $\widetilde{G}$. Finally, $\widetilde{G}$ is associative because path composition is associative in the fundamental group and elements of $\widetilde{G}$ are defined by paths in $G$.
\end{proof}

\end{document}