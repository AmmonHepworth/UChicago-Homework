\documentclass{article}
\usepackage{amsmath,amsthm,amssymb,amsfonts,fullpage,fancyhdr}

\pagestyle{fancy}
\renewcommand{\headheight}{50pt}
\renewcommand{\footskip}{10pt}
\renewcommand{\textheight}{609pt}
\renewcommand{\headrulewidth}{0pt}

\newcommand{\tor}{\textup{Tor}}
\newcommand{\Hom}{\textup{Hom}}
\newcommand{\End}{\textup{End}}

\newtheorem{problem}{Problem}

\begin{document}

\rhead{Kris Harper\\MATH 25800\\February 15, 2010\\}
\chead{Homework 4\\}

\begin{problem}[9.4.2]
Prove that the following polynomials are irreducible in $\mathbb{Z}[x]$:\\
(a) $x^4-4x^3+6$\\
(c) $x^4+4x^3+6x^2+2x+1$.
\end{problem}
\begin{proof}
(a) This follows from Eisenstein's Criterion since $2 \mid -4$ and $2 \mid 6$ but $4 \nmid 6$.

(c) Substituting $x-1$ for $x$ gives the polynomial $x^4-2x+2$, which is irreducible in $\mathbb{Z}[x]$ by Eisenstein's Criterion since $2 \mid 2$ but $4 \nmid 2$. But this means the original polynomial must also be irreducible in $\mathbb{Z}[x]$ since any factorization of it would give a factorization of $x^4-2x+2$ by substituting $x-1$.
\end{proof}

\begin{problem}[9.4.5]
\label{finite}
Find all the monic irreducible polynomials of degree $\leq 3$ in $\mathbb{F}_2[x]$, and the same in $\mathbb{F}_3[x]$.
\end{problem}
\begin{proof}
Note that since we're only interested in polynomials of degree less than or equal to $3$, they will only be reducible in $\mathbb{F}_2[x]$ if they have a root in $\mathbb{F}_2$.

First consider degree three polynomials of the form $p(x) = x^3 + ax^2 + bx + c$. Since $p(0) = c$, if $c = 0$, then $p(x)$ has a root. Therefore $c = 1$. Also $p(1) = 1 + a + b + 1 = a + b$ so we need $a + b \neq 0$. Thus, either $a = 1$ and $b = 0$ or $b = 1$ and $a = 0$. Therefore the only possibilities for $p(x)$ are $p(x) = x^3 + x^2 + 1$ and $p(x) = x^3 + x + 1$.

Now suppose $p(x) = x^2 + ax + b$. We immediately see that $b = 1$ as before. Since $p(1) = 1 + a + 1 = a$, we also have $a = 1$ so the only choice is $p(x) = x^2 + x + 1$. Finally, $p(x) = x + 1$ is an irreducible polynomial in $\mathbb{F}_2[x]$.

Now suppose we're working in $\mathbb{F}_3[x]$. We have $p(x) = x^3 + ax^2 + bx + c$ and we have right away that $c = 1$ or $c = 2$. If $c = 1$ then $p(1) = 1 + a + b + 1 = a + b + 2$ so $a + b \neq 1$ and $p(2) = 2 + a + 2b + 1 = a + 2b$ so $a + 2b \neq 0$. This gives $a = 0$ and $b = 2$, $a = 1$ and $b = 2$, $a = 2$ and $b = 1$ or $a = 2$ and $b = 0$. Otherwise if $c = 2$ then $p(1) = 1 + a + b + 2 = a + b$ so $a + b \neq 0$ and $p(2) = 2 + a + 2b + 2 = a + 2b + 1$ so $a + 2b + 1 \neq 0$. This gives $a = 0$ and $b = 2$, $a = 1$ and $b = 0$, $a = 1$ and $b = 1$ or $a = 2$ and $b = 2$. So the possible polynomials are $p(x) = x^3 + 2x + 1$, $p(x) = x^3 + x^2 + 2x + 1$, $p(x) = x^3 + 2x^2 + x + 1$, $p(x) = x^3 + 2x^2 + 1$, $p(x) = x^3 + 2x + 2$, $p(x) = x^3 + x^2 + 2$, $p(x) = x^3 + x^2 + x + 2$ and $p(x) = x^3 + 2x^2 + 2x + 2$.

Consider $p(x) = x^2 + ax + b$ in $\mathbb{F}_3[x]$. We see that $b = 1$ or $b = 2$. If $b = 1$ then $p(1) = 1 + a + 1 = a + 2$ so $a \neq 1$. Also $p(2) = 1 + 2a + 1 = 2a + 2$ so $a \neq 2$. Therefore the only choice is $a = 0$. Similarly if $b = 2$ then $p(1) = 1 + a + 2 = a$ so $a \neq 0$ and $p(2) = 1 + 2a + 2 = 2a$ so $a \neq 0$. Thus the possible choices are $p(x) = x^2 + 1$, $p(x) = x^2 + x + 2$ and $p(x) = x^2 + 2x + 2$.

Finally note that the polynomials $x + 1$, $x + 2$, $2x + 1$ and $2x + 2$ are irreducible in $\mathbb{F}_3[x]$.
\end{proof}

\begin{problem}[9.4.6]
Construct fields of each of the following orders: (a) $9$, (b) $49$, (c) $8$, (d) $81$ (you may exhibit these as $F[x]/(f(x))$ for some $F$ and $f$).
\end{problem}
\begin{proof}
(a) We know that if $F$ is a field with order $q$ and $f(x)$ is an irreducible polynomial of degree $n$, then $F[x]/(f(x))$ is a field with $q^n$ elements. So we need an irreducible polynomial of degree $2$ over the field $\mathbb{F}_3$. Using Problem~\ref{finite} we see that $f(x) = x^2 + 1$ will work. Then $\mathbb{F}_3[x]/(x^2 + 1)$ is a finite field with $3^2 = 9$ elements.

(b) As in part (a) we need a degree $2$ polynomial over $\mathbb{F}_7$ which is irreducible. Let $p(x) = x^2 + 1$ and note that $p(0) = 1$, $p(1) = 2$, $p(2) = 5$, $p(3) = 3$, $p(4) = 3$, $p(5) = 5$ and $p(6) = 2$. Thus $p(x)$ has no root in $\mathbb{F}_7$ and since it's degree $2$, this means it's irreducible. Therefore $\mathbb{F}_7/(x^2 + 1)$ is a field with $7^2 = 49$ elements.

(c) We wish to find an irreducible polynomial of degree $3$ over $\mathbb{F}_2$. Using Problem~\ref{finite} again we see that $x^3 + x + 1$ will work. Now $\mathbb{F}_2/(x^3 + x + 1)$ is a field with $2^3 = 8$ elements.

(d) We wish to find an irreducible polynomial of degree $2$ in $(\mathbb{F}_3/(x^2 + 1))[y]$. Choose $p(y) = y^2 + y + 2$. Note that if $p(y) = 0$ then $y^2 + y + 2 = \overline{x^2 + 1} = \overline{x}^2 + 1$ so $y^2 + y + 1 = \overline{x}^2$. It's now clear that $p(\overline{0}) \neq 0$, $p(\overline{1}) \neq 0$, $p(\overline{2}) \neq 0$, $p(\overline{x}) \neq 0$ and $p(2\overline{x}) \neq 0$. Note also that $p(\overline{x+1}) = \overline{x^2} \neq 0$, $p(\overline{x+2}) = \overline{x^2 + 2x} \neq 0$, $p(\overline{2x+1}) = \overline{x^2} \neq 0$ and $p(\overline{2x+2}) = \overline{x^2 + x} \neq 0$. Therefore $p(y)$ has no roots and is a degree $2$ polynomial. It is thus irreducible and $(\mathbb{F}_3/(x^2 + 1))[y]/(y^2+y+2)$ is a field with $9^2 = 81$ elements.
\end{proof}

\begin{problem}[9.4.7]
Prove that $\mathbb{R}[x]/(x^2 + 1)$ is a field which is isomorphic to the complex numbers.
\end{problem}
\begin{proof}
It follows immediately that $\mathbb{R}[x]/(x^2 + 1)$ is field since $(x^2 + 1)$ is irreducible in $\mathbb{R}[x]$ (it has no root and is of degree $2$). Let $\varphi : \mathbb{R}[x] \to \mathbb{C}$ be given by $\varphi(p(x)) = p(i)$, that is, evaluation at $i$. This is a ring homomorphism since evaluation maps are ring homomorphisms. It's also surjective since given $a + bi \in \mathbb{C}$, the polynomial $a + bx \in \mathbb{R}[x]$ is mapped to it. Also note that $\varphi(x^2+1) = 0$ and so $(x^2 + 1) \subseteq \ker \varphi$. For $p(x) \in \ker \varphi$, we have $p(i) = 0$. Since $\mathbb{R}[x]$ is a U.F.D, we must have some factor $(x^2 + 1) \mid p(x)$ so $p(x) \in (x^2 + 1)$. Thus $\ker \varphi \subseteq (x^2 + 1)$ and $\ker \varphi = (x^2 + 1)$. Now from the First Isomorphism Theorem we have $\mathbb{R}[x]/(x^2 + 1) \cong \mathbb{C}$.
\end{proof}

\begin{problem}[9.4.14]
Factor each of the two polynomials: $x^8 - 1$ and $x^6 - 1$ into irreducibles over each of the following rings: (a) $\mathbb{Z}$, (b) $\mathbb{Z}/2\mathbb{Z}$, (c) $\mathbb{Z}/3\mathbb{Z}$.
\end{problem}
\begin{proof}
(a) $x^8 - 1 = (x^4 + 1)(x^2 + 1)(x + 1)(x - 1)$ and $x^6 - 1 = (x^2 + x + 1)(x^2 + x -1)(x + 1)(x - 1)$.

(b) $x^8 - 1 = (x+1)^8$ and $x^6 - 1 = (x + 1)^2(x^2 + x + 1)^2$.

(c) $x^8 - 1 = (x^2 + 2x + 2)(x^2 + x + 2)(x^2 + 1)(x + 2)(x + 1)$ and $x^6 - 1 = (x + 2)^3(x + 1)^3$.
\end{proof}

\begin{problem}[9.4.19]
Let $F$ be a field and let $f(x) = a_nx^n + a_{n-1}x^{n-1} + \dots + a_0 \in F[x]$. The \emph{derivative}, $D_x(f(x))$, of $f(x)$ is defined by
\[
D_x(f(x)) = na_nx^{n-1} + (n-1)a_{n-1}x^{n-2} + \dots + a_1
\]
where, as usual, $na = a + a + \dots + a$ ($n$ times). Note that $D_x(f(x))$ is again a polynomial with coefficients in $F$.

The polynomial $f(x)$ is said to have a \emph{multiple root} if there is some field $E$ containing $F$ and some $\alpha \in E$ such that $(x-\alpha)^2$ divides $f(x)$ in $E[x]$. For example, the polynomial $f(x) = (x-1)^2(x_2) \in \mathbb{Q}[x]$ has $\alpha = 1$ as a multiple root and the polynomial $f(x) = x^4 + 2x^2 + 1 = (x^2 + 1)^2 \in \mathbb{R}[x]$ has $\alpha = \pm i \in \mathbb{C}$ as multiple roots. We shall prove in Section 13.5 that a nonconstant polynomial $f(x)$ has a multiple root if and only if $f(x)$ is not relatively prime to its derivative (which can be detected by the Euclidean Algorithm in $F[x]$). Use the criterion to determine whether the following polynomials have multiple roots:\\
(a) $x^3 - 3x - 2 \in \mathbb{Q}[x]$\\
(b) $x^3 + 3x + 2 \in \mathbb{Q}[x]$\\
(c) $x^6 - 4x^4 + 6x^3 + 4x^2 - 12x + 9 \in \mathbb{Q}[x]$\\
(d) Show that for any prime $p$ and any $a \in \mathbb{F}_p$ that the polynomial $x^p - a$ has a multiple root.
\end{problem}
\begin{proof}
(a) The derivative of this polynomial is $3x^2 - 3$ and these two polynomials share a common factor of $x+1$ so they're not relatively prime. The original polynomial therefore has no multiple roots.

(b) The derivative of this polynomial is $3x^2 + 3$ and these two polynomials are relatively prime so the original must have a multiple root.

(c) The derivative of this polynomial is $6x^5 - 16x^3 + 18x^2 + 8x - 12$ and these two polynomials share a common factor of $x^3 - 2x + 3$ so they're not relatively prime. The original polynomial therefore has no multiple roots.

(d) The derivative of $x^p - a$ is $px^{p-1}$. But in $\mathbb{F}_p$, $p = 0$ so the derivative is $0$. Therefore these two polynomials are not relatively prime and $x^p - a$ has no multiple roots.
\end{proof}

\begin{problem}[9.4.20]
(a) Show that the reduction of $f(x)$ modulo both of the nontrivial ideals $(2)$ and $(3)$ of $\mathbb{Z}/6\mathbb{Z}[x]$ is an irreducible polynomial, showing that the condition that $R$ be an integral domain in Proposition 12 is necessary.
\end{problem}
\begin{proof}
(a) Note that $(\mathbb{Z}/6\mathbb{Z})/(2) \cong \mathbb{Z}/2\mathbb{Z}$ and $(\mathbb{Z}/6\mathbb{Z})/(3) \cong \mathbb{Z}/3\mathbb{Z}$. It's clear that $x$ is not reducible over either of these rings.
\end{proof}

\begin{problem}[9.5.2]
For each of the fields constructed in Exercise 6 of Section 4 exhibit a generator for the (cyclic) multiplicative group of nonzero elements.
\end{problem}
\begin{proof}
(a) We wish to find an element of $\mathbb{F}_3[x]/(x^2+1)$ which has order $8$. Consider the element $\overline{x+1}$. Note that $|(\mathbb{F}_3[x]/(x^2+1))^{\times}| = 8$ so we only need to check that $\overline{x+1}^2 \neq 1$ and $\overline{x+1}^4 \neq 1$. But $\overline{x+1}^2 = \overline{x^2 + 2x + 1} = \overline{2x}$ and $\overline{x+1}^4 = \overline{x^4 + 4x^3 + 6x^2 + 4x + 1} = \overline{2}$ using the division algorithm. Thus $|\overline{x+1}| = 8$ and this is a generator for the group.

(b) This group has order $48$ so it suffices to check that an element is not $1$ when raised to the $24^{\textup{th}}$ power. Using the division algorithm, we see that $\overline{x+2}^24 = \overline{6}$ so $\overline{x+2}$ is a generator for this set.

(c) This group has order $7$ which is prime. It thus suffices to find a nonzero, nonidentity element. Choose $\overline{x+1}$.

(d) We need to find an element which when raised to the $40^{\textup{th}}$ power is not the identity. Note that from part (a), $\overline{x+1}$ is an element of the underlying field $\mathbb{F}_3[x]/(x^2+1)$ which has order $8$. Choose this as a coefficient for $\overline{y+1}$ to get $\overline{\overline{x+1}(y+1)}$. This polynomial is not the identity when raised to the $40^{\textup{th}}$ power and so must be a generator for the multiplicative group of our finite field.
\end{proof}

\begin{problem}[9.5.5]
Let $\varphi$ denote Euler's $\varphi$-function. Prove the identity $\sum_{d \mid n} \varphi(d) = n$, where the sum is extended over all the divisors $d$ of $n$.
\end{problem}
\begin{proof}
Let $A$ be a cyclic group of order $n$. Let $d \mid n$ and note that $A$ contains a unique subgroup of order $d$. Note that the number of generators of $Z_d$ is simply $\varphi(d)$ since this is the number of elements which are relatively prime to $d$. Therefore there are $\varphi(d)$ generators for each subgroup of order $d$ and each of these generators must have order $d$ (since $Z_d$ is cyclic). Now, since order is unique to each element, we can use Lagrange's Theorem and sum over all possible divisors $d \mid n$ to $\sum_{d \mid n} \varphi(d) = n$.
\end{proof}

\begin{problem}[10.1.3]
Assume that $rm = 0$ for some $r \in R$ and some $m \in M$ with $m \neq 0$. Prove that $r$ does not have a left inverse (i.e., there is no $s \in R$ such that $sr = 1$).
\end{problem}
\begin{proof}
We know that $1m = 1$ and $s0 = 0$ for all $m \in M$ and $s \in R$. Then multiplying on the left by such an $s$ would give $m = 1m = (sr)m = s(rm) = s0 = 0$ and we've assumed $m \neq 0$. Thus such an $s$ cannot exist.
\end{proof}

\begin{problem}[10.1.4]
Let $M$ be the module $R^n$ described in Example 3 and let $I_1, I_2, \dots , I_n$ be left ideals of $R$. Prove that the following are submodules of $M$:\\
(a) $\{(x_1, x_2, \dots , x_n) \mid x_i \in I\}$\\
(b) $\{(x_1, x_2, \dots , x_n) \mid x_i \in R \text{ and } x_1 + x_2 + \dots + x_n = 0\}$.
\end{problem}
\begin{proof}
(a) Let $N$ be the set in question. We see that $N \neq \emptyset$ since each $I_i \neq \emptyset$. Let $(x_1, \dots , x_n), (y_1, \dots , y_n) \in N$ and let $r \in R$. Then
\[
(x_1, \dots , x_n) + r(y_1, \dots , y_n) = (x_1 + ry_1 + \dots + x_n + ry_n).
\]
Since $I_i$ is a left ideal, $ry_i \in I_i$. Furthermore, since ideals are subgroups of $R$, $x_i + ry_i \in I_i$. Therefore, this element is an element of $N$ so $N$ fits the submodule criterion and is a submodule of $M$.

(b) Let $N$ be the set in question as before. Once again, $N$ isn't empty since $0 \in I_i$ for each $i$ so $(0, \dots , 0) \in N$. Let $(x_1, \dots , x_n), (y_1, \dots , y_n) \in N$ and let $r \in R$. Then
\[
(x_1, \dots , x_n) + r(y_1, \dots , y_n) = (x_1 + ry_1 + \dots + x_n + ry_n).
\]
Note that each coordinate is an element of $I_i$ as in part (a). Furthermore,
\[
\sum_{i=1}^{n} (x_i + ry_i) = \sum_{i=1}^{n} x_i + \sum_{i=1}^{n}ry_i = \sum_{i=1}^{n} x_i + r \sum_{i=1}^{n} y_i = 0 + r0 = 0.
\]
Therefore, this element is in $N$ and $N$ fits the submodule criterion so $N$ is a submodule of $M$.
\end{proof}

\begin{problem}[10.1.7]
Let $N_1 \subseteq N_2 \subseteq \dots$ be an ascending chain of submodules of $M$. Prove that $\bigcup_{i=1}^{\infty} N_i$ is a submodule of $N$.
\end{problem}
\begin{proof}
Let $A$ be the set in question. Since $N_1 \subseteq A$, we see that $A \neq \emptyset$. Suppose $x,y \in A$ and let $r \in R$. Then $y \in N_i$ for some $i$ and thus $ry \in N_i$ as well since $N_i$ is a submodule. But then $ry \in A$ as well. Likewise, $x \in N_j$ for some $j$. Without loss of generality, let $i \leq j$ so that $ry \in N_j$ as well. But the $x+ry \in N_j$ and $x + ry \in A$. Therefore $A$ is a submodule.
\end{proof}

\begin{problem}[10.1.8]
An element $m$ of the $R$-module $M$ is called a \emph{torsion element} if $rm = 0$ for some nonzero element $r \in R$. The set of torsion elements is defined
\[
\tor(M) = \{m \in M \mid rm = 0 \text{ for some nonzero } r \in R\}.
\]
(a) Prove that if $R$ is an integral domain then $\tor(M)$ is a submodule of $M$ (called the \emph{torsion} submodule of $M$).\\
(b) Give an example of a ring $R$ and an $R$-module $M$ such that $\tor(M)$ is not a submodule.
(c) If $R$ has zero divisors show that every nonzero $R$-module has nonzero torsion elements.
\end{problem}
\begin{proof}
(a) Suppose $R$ is an integral domain. Note that $0 \in \tor(M)$ so $\tor(M) \neq \emptyset$. Let $x,y \in \tor(M)$ and $a,b,r \in R$ such that $ax = by = 0$ with $a$ and $b$ nonzero. Then $ab(x+y) = 0$ and $a(rx) = (ar)x = 0$ so $\tor(M)$ is a submodule of $M$.

(b) Let $R = \mathbb{Z}/6\mathbb{R}$ and $M = R$. Then $2,3 \in \tor(M)$ since $2 \cdot 3 = 0$ but $2 \neq 0$ and $3 \neq 0$. But then $2 + 3 = 5$ and $5 \notin \tor(M)$. Thus $\tor(M)$ is not a submodule.

(c) Let $m \in M$ and let $x,y \in R$ such that $x$ and $y$ are nonzero but $xy = 0$. Then $ym \in \tor(M)$ since $x(ym) = (xy)m = 0m = 0$.
\end{proof}

\begin{problem}[10.1.14]
Let $z$ be an element of the center of $R$, i.e., $zr = rz$ for all $r \in R$. Prove that $zM$ is a submodule of $M$, where $zM = \{zm \mid m \in M\}$. Show that if $R$ is the ring of $2 \times 2$ matrices over a field and $e$ is the matrix with a $1$ in position $1,1$ and zeros elsewhere then $eR$ is \emph{not} a left $R$-submodule (where $M = R$ is considered as a left $R$-module as in Example 1) --- in this case the matrix $e$ is not in the center of $R$.
\end{problem}
\begin{proof}
Note that $zM \neq \emptyset$ since $0 \in zM$. Let $zx,zy \in zM$ and let $r \in R$. Then $zx + rzy = zx + zry = z(x + ry)$ so this is an element of $zM$ as well and $zM$ is a submodule of $M$. Now let $R$ be the ring of matrices defined above with $e$ defined as above as well. Note that a matrix in $eR$ will have two possibly nonzero elements in the first row, and two zero elements in the second row. But these matrices are clearly not closed under multiplication by elements of $R$. For example, multiplying by the matrix with all $1$s as entries puts the upper left coordinate in the lower left coordinate. This is not an element of $eR$ so $eR$ is not an $R$-submodule.
\end{proof}

\begin{problem}[10.1.22]
Suppose that $A$ is a ring with identity $1_A$ that is a (unital) left $R$-module satisfying $r \cdot (ab) = (r \cdot a)b = a(r \cdot b)$ for all $r \in R$ and $a,b \in A$. Prove that the map $f : R \to A$ defined by $f(r) = r \cdot 1_A$ is a ring homomorphism mapping $1_R$ to $1_A$ and that $f(R)$ is contained in the center of $A$. Conclude that $A$ is an $R$-algebra and that the $R$-module structure on $A$ induced by its algebraic structure is precisely the original $R$-module structure.
\end{problem}
\begin{proof}
Let $r,s \in R$. Note that
\[
f(r+s) = (r+s) \cdot 1_A = 1_A((r+s) \cdot 1_A) = 1_A(r \cdot 1_A + s \cdot 1_A) = 1_A(r \cdot 1_A) + 1_A(s \cdot 1_A) = r \cdot 1_A + s \cdot 1_A = f(r) + f(s).
\]
Likewise
\[
f(rs) = (rs) \cdot 1_A = 1_A((rs) \cdot 1_A) = 1_A(r \cdot (s \cdot 1_A)) = (r \cdot 1_A)(s \cdot 1_A) = f(r)f(s).
\]
We also have $f(1_R) = 1_R \cdot 1_A = 1_A$. Now let $f(r) \in f(R)$ so that $f(r) = r \cdot 1_A$ and let $a \in A$. Then
\[
f(r)a = (r \cdot 1_A)a = r \cdot (1_A a) = r \cdot (a 1_A) = a(r \cdot 1_A) = af(r)
\]
so $f(r)$ is in the center of $A$ and $f(R)$ is a subset of the center of $A$. This shows that $A$ is an $R$-algebra. When viewed as an $R$-module, this $R$-algebra is precisely the same as if we view $A$ itself as an $R$-module.
\end{proof}

\begin{problem}[10.2.4]
Let $A$ be an $\mathbb{Z}$-module, let $a$ be any element of $A$ and let $n$ be a positive integer. prove that the map $\varphi_a : \mathbb{Z}/n\mathbb{Z} \to A$ given by $\varphi_a(\overline{k}) = ka$ is a well defined $\mathbb{Z}$-module homomorphism if and only if $na = 0$. Prove that $\Hom_{\mathbb{Z}}(\mathbb{Z}/n\mathbb{Z}, A) \cong A_n$, where $A_n = \{a \in A \mid na = 0\}$ (so $A_n$ is the annihilator in $A$ of the ideal $(n)$ of $\mathbb{Z}$---cf. Exercise 10, Section 1).
\end{problem}
\begin{proof}
First we show that $\varphi_a$ is a homomorphism. Let $\overline{x}, \overline{y} \in \mathbb{Z}/n\mathbb{Z}$ and let $m \in \mathbb{Z}$. Then
\[
\varphi_a(\overline{x} + \overline{y}) = \varphi_a(\overline{x+y}) = (x+y)a = xa + ya = \varphi_a(\overline{x}) + \varphi_a(\overline{y})
\]
and
\[
\varphi_a(m\overline{x}) = (mx)a = m(xa) = m\varphi_a(\overline{x}).
\]
Now suppose $\varphi_a$ is well defined. Then consider
\[
0 = 0 \cdot a = \varphi_a(\overline{0}) = \varphi_a(\overline{n-1} + \overline{1}) = \varphi_a(\overline{n-1}) + \varphi_a(\overline{1}) = (n-1)a + a = na
\]
so $na = 0$. Conversely, suppose that $na = 0$. Let $\overline{x} = \overline{y}$. Write $x = x' + cn$ and $y = y' + dn$ where $0 \leq x' \leq n-1$ and $0 \leq y' \leq n-1$. Then since $\overline{x} = \overline{y}$, we must have $x' = y'$ so that
\[
\varphi_a(\overline{x}) = xa = (x' + cn)a = x'a + cna = x'a = y'a = y'a + dna = (y'+dn)a = ya = \varphi_a(\overline{y}).
\]
Thus $\varphi_a$ is well-defined. Let $\psi : \Hom_{\mathbb{Z}}(\mathbb{Z}/n\mathbb{Z},A) \to A_n$ be defined by $\psi(\varphi) = \varphi(\overline{1})$. Note that
\[
\psi(\varphi_1 + \varphi_2) = (\varphi_1 + \varphi_2)(\overline{1}) = \varphi_1(\overline{1}) + \varphi_2(\overline{1}) = \psi(\varphi_1 + \varphi_2)
\]
and for $m \in \mathbb{Z}$
\[
\psi(m\varphi) = (m\varphi)(\overline{1}) = m(\varphi(\overline{1})) = m\psi(\varphi)
\]
so $\psi$ is a module homomorphism. We've shown that $\psi$ is surjective since the homomorphisms $\varphi_a$ are each mapped to $a \in A_n$. Furthermore, if $\varphi_1(\overline{1}) = \varphi_2(\overline{1})$ then it follows that $\varphi_1(\overline{k}) = \varphi_2(\overline{k})$ because this is simply $\varphi_i(\overline{1})$ added to itself $k$ times. Thus $\varphi_1 = \varphi_2$ and $\psi$ is injective. Thus $\Hom_{\mathbb{Z}}(\mathbb{Z}/n\mathbb{Z}, A) \cong A_n$.
\end{proof}

\begin{problem}[10.2.7]
Let $z$ be a fixed element of the center of $R$. Prove that the map $m \mapsto zm$ is an $R$-module homomorphism from $M$ to itself. Show that for a commutative ring $R$ the map from $R$ to $\End_R(M)$ given by $r \mapsto rI$ is a ring homomorphism (where $I$ is the identity endomorphism).
\end{problem}
\begin{proof}
Let $m,n \in M$ and $r \in R$. Then we have $z(m+n) = zm + zn$ and $z(rm) = (zr)m = (rz)m = r(zm)$ so the map is both additive and scalar multiplicative and is thus a module homomorphism. Now suppose $\varphi : R \to \End_R(M)$ and let $r,s \in R$. Then $\varphi(r+s) = (r+s)I = rI + sI = \varphi(r) + \varphi(s)$ and $\varphi(rs) = (rs)I = (rI)(sI) = \varphi(r)\varphi(s)$. Note that we need $R$ to be commutative to ensure that each $rI$ is actually an $R$-module homomorphism from $M$ to itself (since now the center of $R$ is all of $R$).
\end{proof}

\begin{problem}[10.2.8]
Let $\varphi : M \to N$ be an $R$-module homomorphism. Prove that $\varphi(\tor(M)) \subseteq \tor(N)$ (cf. Exercise 8 in Section 1).
\end{problem}
\begin{proof}
Let $m \in \tor(M)$ such that $rm = 0$ for $r \neq 0$. Then since $\varphi$ is an $R$-module homomorphism we have $r\varphi(m) = \varphi(rm) = \varphi(0) = 0$ in $N$. Thus $\varphi(r) \in \tor(N)$ and $\varphi(\tor(M)) \subseteq \tor(N)$.
\end{proof}

\end{document}