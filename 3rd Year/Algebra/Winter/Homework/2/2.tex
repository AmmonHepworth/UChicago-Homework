\documentclass{article}
\usepackage{amsmath,amsthm,amssymb,amsfonts,fullpage,fancyhdr}

\pagestyle{fancy}
\renewcommand{\headheight}{50pt}
\renewcommand{\footskip}{10pt}
\renewcommand{\textheight}{609pt}
\renewcommand{\headrulewidth}{0pt}

\newtheorem{problem}{Problem}

\begin{document}

\rhead{Kris Harper\\MATH 25800\\January 22, 2010\\}
\chead{Homework 2\\}

\begin{problem}[7.4.4]
Assume $R$ is commutative. Prove that $R$ is a field if and only if $0$ is a maximal ideal.
\end{problem}
\begin{proof}
Assume that $R$ is a field. Then then only ideals are $0$ and $R$. Since $0$ is contained in no proper ideal other than itself, it must be maximal. Conversely, suppose $0$ is a maximal ideal. We know that an ideal $M$ is maximal if and only if $R/M$ is a field. Thus $R/0 \cong R$ is a field.
\end{proof}

\begin{problem}[7.4.5]
Prove that if $M$ is an ideal such that $R/M$ is a field then $M$ is a maximal ideal (do not assume $R$ is commutative).
\end{problem}
\begin{proof}
Let $R/M$ be a field and suppose to the contrary that $M$ is not maximal so that there exists some ideal $M'$ with $M \subsetneq M' \subsetneq R$. Let $\varphi : R/M \to R/M'$ be a function defined by $\varphi(r + M) = r + M'$. Then
\[
\varphi((r + M)+(s + M)) = \varphi(r+s+M) = r+s+M' = (r + M') + (s + M') = \varphi(r + M) + \varphi(s + M)
\]
and
\[
\varphi((r + M)(s + M)) = \varphi(rs + M) = rs + M' = (r + M')(s + M') = \varphi(r + M)\varphi(s + M)
\]
so $\varphi$ is a homomorphism. Note that since $M$ is strictly smaller than $M'$, $\varphi$ can't be injective. But this is a contradiction because $R/M$ is a field and any homomorphism from a field to another ring must be an injection. Therefore $M$ must be a maximal ideal.
\end{proof}

\begin{problem}[7.4.7]
Let $R$ be a commutative ring with $1$. Prove that the principle ideal generated by $x$ in the polynomial ring $R[x]$ is a prime ideal if and only if $R$ is an integral domain. Prove that $(x)$ is a maximal ideal if and only if $R$ is a field.
\end{problem}
\begin{proof}
We know that $(x)$ is a prime ideal if an only if $R[x]/(x)$ is an integral domain. The problem is then reduced to showing $R[x]/(x) \cong R$. Let $\varphi : R[x]/(x) \to R$ be the function which takes $p(x) + (x) \in R[x]$ to the constant term of $p(x)$. It's clear that $\varphi$ is a ring homomorphism since adding and multiplying two polynomials will add or multiply their constant terms respectively. Let $p(x) = r_nx^n + \dots + r_0$ and $q(x) = s_nx^n + \dots + s_0$. Then
\[
p(x) + (x) = r_nx^n + \dots + r_0 + (x) = (r_nx^n + (x)) + \dots + (r_0 + (x)) = 0 + \dots + r_0 + (x).
\]
In the same way, $q(x) + (x) = s_0 + (x)$ and we see that two elements of $R[x]/(x)$ are equal precisely when their constant terms are the same. Thus, if we assume $p(x) \neq q(x)$, then $r_0 \neq s_0$ and $\varphi(p(x) + (x)) = r_0 \neq s_0 = \varphi(q(x) + (x))$. Therefore, $\varphi$ is injective. It's clear that $\varphi$ is surjective since $\varphi$ applied to a constant is just the identity function. Thus, $\varphi$ is a bijection from $R[x]/(x)$ to $R$ so $R[x]/(x) \cong R$. Therefore $(x)$ is a prime ideal if and only if $R$ is an integral domain.

As before, we know that $(x)$ is a maximal ideal if and only if $R[x]/(x)$ is a field. But we've just shown that $R[x]/(x) \cong R$ so $(x)$ is a maximal ideal if and only if $R$ is a field.
\end{proof}

\begin{problem}[7.4.10]
Assume $R$ is commutative. Prove that if $P$ is a prime ideal of $R$ and $P$ contains no zero divisors then $R$ is an integral domain.
\end{problem}
\begin{proof}
Let $a, b \in R$ be two elements such that $ab = 0$. Since $P$ is an ideal, $0 \in P$ and so either $a \in P$ or $b \in P$. But since $P$ contains no zero divisors, we must have $a = 0$ or $b = 0$. Thus $R$ is an integral domain.
\end{proof}

\begin{problem}[7.4.13]
Let $\varphi : R \to S$ be a homomorphism of commutative rings.\\
(a) Prove that if $P$ is a prime ideal of $S$ then either $\varphi^{-1}(P) = R$ or $\varphi^{-1}(P)$ is a prime ideal of $R$. Apply this to the special case when $R$ is a subring of $S$ and $\varphi$ is the inclusion homomorphism to deduce that if $P$ is a prime idea of $S$ then $P \cap R$ is either $R$ or a prime ideal of $R$.\\
(b) Prove that if $M$ is a maximal ideal of $S$ and $\varphi$ is surjective then $\varphi^{-1} (M)$ is a maximal ideal of $R$. Give an example to show that this need not be the case if $\varphi$ is not surjective.
\end{problem}
\begin{proof}
(a) Let $P$ be a prime ideal of $S$. We've already shown that $\varphi^{-1}(I)$ is an ideal of $R$ for any ideal $I$ of $S$. It's possible that $\varphi^{-1}(P) = R$, in which case we're done, so assume otherwise. Now let $ab \in \varphi^{-1}(P)$. Then $\varphi(ab) \in P$ so $\varphi(a)\varphi(b) \in P$. Since $P$ is prime, either $\varphi(a) \in P$ or $\varphi(b) \in P$, which means either $a \in \varphi^{-1}(P)$ or $b \in \varphi^{-1}(P)$. Thus $\varphi^{-1}(P)$ is prime.

In the special case that $\varphi$ is an inclusion homomorphism, $\varphi$ is the identity on $R$, so $\varphi^{-1}(P)$ consists of elements of $R$ which are also elements of $P$. That is, $\varphi^{-1}(P) = P \cap R$ and by the above proof, we know this is now either $R$ itself, or a prime ideal of $R$.

(b) Let $M$ be a maximal ideal of $S$ and suppose that $\varphi$ is surjective. We know that $\varphi^{-1} \neq R$ since $M \neq S$ and $\varphi$ is surjective. Suppose there exists some ideal $M'$ such that $\varphi^{-1}(M) \subseteq M' \subseteq R$. Since $\varphi$ is surjective, $\varphi(M')$ is an ideal of $S$ and $M \subseteq \varphi(M')$. Since $M$ is maximal, we either have $M = \varphi(M')$ or $\varphi(M') = S$. Suppose the former and let $x \in M'$. Then $\varphi(x) \in \varphi(M')$ so $\varphi(x) \in M$. Then $x \in \varphi^{-1}(M)$ and we have $M' \subseteq \varphi^{-1}(M)$. This shows $M' = \varphi^{-1}(M)$. Secondly, suppose $\varphi(M') = S$ and let $x \in R$. Then $\varphi(x) \in S$ and $\varphi(x) \in \varphi(M')$. Thus there exists $y \in M'$ such that $\varphi(x) = \varphi(y)$. Then we have $\varphi(x) - \varphi(y) = \varphi(x-y) = 0$ so $x-y \in \ker \varphi$. Note that $\ker \varphi = \varphi^{-1}(0) \subseteq M'$. Therefore $x = y + (x-y)$ is in $M'$ which shows $R \subseteq M'$ and $R = M'$. In all cases we either have $M' = \varphi^{-1}(M)$ or $M' = R$ so $\varphi^{-1}(M)$ is maximal in $R$.
\end{proof}

\begin{problem}[7.4.16]
\label{polyrem}
Let $x^4 - 16$ be an element of the polynomial ring $E = \mathbb{Z}[x]$ and use the bar notation to denote passage to the quotient ring $\mathbb{Z}[x]/(x^4 - 16)$.\\
(a) Find a polynomial of degree $\leq 3$ that is congruent to $7x^{13} - 11x^9 + 5x^5 - 2x^3 + 3$ modulo $(x^4 - 16)$.\\
(b) Prove that $\overline{x - 2}$ and $\overline{x + 2}$ are zero divisors in $\overline{E}$.
\end{problem}
\begin{proof}
(a) We need to find a polynomial with degree less than or equal to $3$ which has the same remainder as $7x^{13} - 11x^9 + 5x^5 - 2x^3 + 3$ when divided by $x^4 - 16$. Note that $(7x^{13} - 11x^9 + 5x^5 - 2x^3 + 3)/(x^4 - 16)$ has remainder $-2x^3 + 25936x + 3$. This remainder is then a polynomial which cannot be reduced by dividing by $x^4 - 16$ and so it serves as its own remainder. Thus $7x^{13} - 11x^9 + 5x^5 - 2x^3 + 3 \equiv -2x^3 + 25936x + 3 \pmod{x^4 - 16}$.

(b) Note that $x^4 - 16 = (x-2)(x+2)(x^2 + 4)$. Thus
\[
(\overline{x-2})(\overline{x^3 + 2x^2 + 4x + 8}) = \overline{0}
\]
and
\[
(\overline{x+2})(\overline{x^3 - 2x^2 + 4x - 8}) = \overline{0}.
\]
Since $x+2$, $x-2$, $x^3 - 2x^2 + 4x - 8$ and $x^3 + 2x^2 + 4x + 8$ all have degree less than or equal to three, they can't be equal to $0$ in $\overline{E}$. Thus, they are all zero divisors.
\end{proof}

\begin{problem}[7.4.17]
Let $x^3 - 2x + 1$ be an element of the polynomial ring $E = \mathbb{Z}[x]$ and use the bar notation to denote passage to the quotient ring $\mathbb{Z}[x]/(x^3-2x+1)$. Let $p(x) = 2x^7 - 7x^5 + 4x^3 - 9x + 1$ and let $q(x) = (x-1)^4$.\\
(a) Express each of the following elements of $\overline{E}$ in the form $\overline{f(x)}$ for some polynomial $f(x)$ of degree $\leq 2$: $\overline{p(x)}$, $\overline{q(x)}$, $\overline{p(x) + q(x)}$ and $\overline{p(x)q(x)}$.\\
(b) Prove that $\overline{E}$ is not an integral domain.\\
(c) Prove that $\overline{x}$ is a unit in $\overline{E}$.
\end{problem}
\begin{proof}
(a) As in part (a) of Problem~\ref{polyrem}, we note that $p(x)$ is congruent to it's remainder when divided by $x^3 - 2x + 1$ modulo $x^3 - 2x + 1$. If these remainders have degree less than or equal to $2$, then we're done. Dividing and looking at the remainders gives the following equalities. We have $\overline{p(x)} = \overline{-x^2 - 11x + 3}$, $\overline{q(x)} = \overline{8x - 5}$, $\overline{p(x) + q(x)} = \overline{7x^2 - 24x + 8}$ and $\overline{p(x)q(x)} = \overline{146x - 90}$.

(b) We see that $x^3 - 2x + 1 = (x-1)(x^2 + x - 1)$ and so $\overline{x-1}$ is a zero divisor.

(c) Note that $\overline{x^3 - 2x} + \overline{1} = \overline{0}$ and so $\overline{1} = \overline{-x^3 + 2x}$. Thus $\overline{x}\overline{-x^2 + 2} = \overline{-x^3 + 2x} = \overline{1}$ and $\overline{x}$ is a unit.
\end{proof}

\begin{problem}[7.6.3]
Let $R$ and $S$ be rings with identities. Prove that every ideal of $R \times S$ is of the form $I \times J$ where $I$ is an ideal of $R$ and $J$ is an ideal of $S$.
\end{problem}
\begin{proof}
Let $K$ be an ideal of $R \times S$ and write $K \subseteq I \times J$ where $I$ is the subset of $R$ which makes up the left components of $K$ and $J$ is the subset of $S$ which makes up the right components. Let $a,b \in I$ and $c,d \in J$ such that $(a,c),(b,d) \in K$. Note that $(a,c) - (b,d) = (a-b,c-d)$ so $a-b \in I$ and $(a,c)(b,d) = (ab,cd)$ so $ab \in I$. Furthermore, for $r \in R$ we have $(a,c)(r,c) = (ar,c^2)$ and $(r,c)(a,c) = (ra,c^2)$ so $I$ is closed under left and right multiplication by elements of $R$. This shows that $I$ is an ideal of $R$ and similarly, that $J$ is an ideal of $S$.

Finally, let $(a,c)$ be an arbitrary element of $I \times J$. This means there exists some $(a,c') \in K$ and since $K$ is closed under multiplication by elements from $R \times S$, we have $(a,c')(1,0) = (a,0)$ is an element of $K$ as well. Similarly, $(0,c) \in K$. But now $(a,c) = (a,0) + (0,c)$ and so $(a,c) \in K$ since $K$ is closed under addition. Therefore $I \times J \subseteq K$ and $K = I \times J$ where $I$ is an ideal of $R$ and $J$ is an ideal of $S$.
\end{proof}

\end{document}