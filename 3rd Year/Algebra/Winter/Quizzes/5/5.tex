\documentclass{article}
\usepackage{amsmath,amsthm,amssymb,amsfonts,fullpage,fancyhdr}

\pagestyle{fancy}
\renewcommand{\headheight}{50pt}
\renewcommand{\footskip}{10pt}
\renewcommand{\textheight}{600pt}
\renewcommand{\headrulewidth}{0pt}

\newcommand{\ann}{\textup{Ann}}

\newtheorem{problem}{Problem}

\begin{document}

\rhead{Kris Harper\\MATH 25800\\March 10, 2010\\}
\chead{Quiz 5\\}

\begin{problem}
Let $R = \mathbb{Z}[x]$ and let $M = (2,x)$ be the ideal generated by $2$ and $x$, considered as a submodule of $R$. Show that $\{2,x\}$ is not a basis of $M$. Show that the rank of $M$ is $1$ but that $M$ is not free of rank $1$.
\end{problem}
\begin{proof}
Note that $x(2) + (-2)(x) = 0$ but $x \neq 0$ and $-2 \neq 0$ so $\{2, x\}$ cannot be a basis for $M$ since this set is linearly dependent. This shows that the rank of $M$ must be less than $2$. Since $M$ contains a nonzero element, it must have rank $1$. But $M$ is not free of rank $1$ since we know the ideal $(2,x)$ is not principal.
\end{proof}

\begin{problem}
Let $R$ be a P.I.D., let $B$ be a torsion $R$-module and let $p$ be a prime in $R$. Prove that if $pb = 0$ for some nonzero $b \in B$, then $\ann(B) \subseteq (p)$.
\end{problem}
\begin{proof}
Since $b \neq 0$, we can form the nontrivial $R$-module $Rb$ and note that $Rb \subseteq B$. This means that $\ann(B) \subseteq \ann(Rb)$. But note that $(p) \subseteq \ann(Rb)$ since $pb = 0$ and $R$ is commutative. Since $p$ is prime and $R$ is a P.I.D., we know $(p)$ is maximal. Thus either $\ann(Rb) = (p)$ or $\ann(Rb) = R$. Suppose the latter. Then for each $r \in R$ we have $rm = 0$ for all $m \in Rb$. In particular, we have $rb = 0$ for all $r \in R$. Then $Rb = 0$ and we trivially have $\ann(B) \subseteq (p) \subseteq \ann(Rb) = 0$. On the other hand, if $\ann(Rb) = (p)$ then we have $\ann(B) \subseteq \ann(Rb) = (p)$ as desired.
\end{proof}

\end{document}