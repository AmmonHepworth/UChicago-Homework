\documentclass{article}
\usepackage{amsmath,amsthm,amssymb,amsfonts,fullpage,fancyhdr}

\pagestyle{fancy}
\renewcommand{\headheight}{50pt}
\renewcommand{\footskip}{40pt}
\renewcommand{\textheight}{590pt}
\renewcommand{\headrulewidth}{0pt}

\newtheorem{problem}{Problem}

\newcommand{\normal}{\unlhd}

\begin{document}

\rhead{Kris Harper\\MATH 25700\\October 30, 2009\\}
\chead{Homework 4\\}

\begin{problem}[3.1.4]
\label{normalpowers}
Prove that in the quotient group $G/N$, $(gN)^{\alpha} = g^{\alpha}N$ for all $\alpha \in \mathbb{Z}$.
\end{problem}
\begin{proof}
First take $\alpha > 0$. Since $G/N$ is a group we have $(gN)^{\alpha} = gN \cdot gN \cdot \dots \cdot gN$ where there are $\alpha$ $gN$s. From the generalized associative property and the fact that $gN \cdot gN = (g \cdot g) N$, this reduces to $(g \cdot g \cdot \dots \cdot g) N = g^{\alpha} N$. For $\alpha = 0$ we get $(gN)^0 = N = 1 N = g^0N$. Finally, if $\alpha < 0$ then again since $G/N$ is a group we have $(gN)^{\alpha} = \left ( (gN)^{-\alpha} \right )^{-1} = (g^{-\alpha}N)^{-1} = (g^{-\alpha})^{-1}N = g^{\alpha}N$.
\end{proof}

\begin{problem}[3.1.5]
Use the preceding exercise to prove that the order of the element $gN$ in $G/N$ is $n$, where $n$ is the smallest positive integer such that $g^n \in N$ (and $gN$ has infinite order if no such positive integer exists). Give an example to show that the order of $gN$ in $G/N$ may be strictly smaller than the order of $g$ in $G$.
\end{problem}
\begin{proof}
Let $n$ be as defined. We know $g^n \in N$ which means $g^n N = N$. Then using Problem~\ref{normalpowers}, $N = g^nN = (gN)^n$. Since $n$ is the smallest positive integer such that this is true, we must have $|gN| = n$. If no such $n$ exists, then $g^n \notin N$ for all positive $n$. Therefore $g^nN \neq N$ for all positive $n$ and thus $|gN| = \infty$.

As an example, let $G = D_8$ and $N = \langle r \rangle$. Then $|r| = 4$ in $G$, and $|rN| = 1$.
\end{proof}

\begin{problem}[3.1.16]
Let $G$ be a group, let $N$ be a normal subgroup of $G$ and let $\overline{G} = G/N$. Prove that if $G = \langle x, y \rangle$ then $\overline{G} = \langle \overline{x}, \overline{y}\rangle$. Prove more generally that if $G = \langle S \rangle$ for any subset $S$ of $G$, then $\overline{G} = \langle \overline{S} \rangle$.
\end{problem}
\begin{proof}
Let $S = \{a_1, \dots , a_n\}$ be a subset of $G$ such that $G = \langle S \rangle$. Let $\overline{x} \in \overline{G}$. Since $G = \langle S \rangle$ we can write $\overline{x} = \overline{a_1^{i_1}a_2^{i_2} \dots a_n^{i_n}}$. Using the generalized associative principle this reduces to $\overline{a_1^{i_1}} \overline{a_2^{i_2}} \dots \overline{a_n^{i_n}}$. Thus any element of $\overline{G}$ can be written as a product of powers of elements in $\overline{S}$. Therefore $\overline{G} = \langle \overline{S} \rangle$. In particular, if $S = \{x, y\}$ then $G = \langle x, y \rangle$ and $\overline{G} = \langle \overline{x}, \overline{y} \rangle$.
\end{proof}

\begin{problem}[3.1.17]
Let $G$ be the dihedral group of order $16$:
\[
G = \langle r, s \mid r^8 = s^2 = 1, rs = sr^{-1} \rangle
\]
and let $\overline{G} = G/\langle r^4 \rangle$ be the quotient of $G$ by the subgroup generated by $r^4$ (this subgroup is the center of $G$, hence is normal).\\
(a) Show that the order of $\overline{G}$ is $8$.\\
(b) Exhibit each element of $\overline{G}$ in the form $\overline{s}^a \overline{r}^b$, for some integers $a$ and $b$.\\
(c) Find the order of each of the elements of $\overline{G}$ exhibited in (b).\\
(d) Write each of the following elements of $\overline{G}$ in the form $\overline{s}^a \overline{r}^b$, for some integers $a$ and $b$ as in (b): $\overline{rs}$, $\overline{sr^{-2}s}$, $\overline{s^{-1}r^{-1}sr}$.\\
(e) Prove that $\overline{H} = \langle \overline{s}, \overline{r}^2 \rangle$ is a normal subgroup of $\overline{G}$ and $\overline{H}$ is isomorphic to the Klein 4-group. Describe the isomorphism type of the complete preimage of $\overline{H}$ in $G$.\\
(f) Find the center of $\overline{G}$ and describe the isomorphism type of $\overline{G}/Z(\overline{G})$.
\end{problem}
\begin{proof}
(a) From Lagrange's Theorem, $|\overline{G}| = |G|/|\langle r \rangle| = 16/2 = 8$.

(b) We have
\begin{align*}
\overline{G}
&= \{x\langle r^4 \rangle \mid x \in G\} \\
&= \{\langle r^4 \rangle, r\langle r^4 \rangle, r^2\langle r^4 \rangle, r^3\langle r^4 \rangle, s\langle r^4 \rangle, sr\langle r^4 \rangle, sr^2\langle r^4 \rangle, sr^3\langle r^4 \rangle\} \\
&= \{\langle r^4 \rangle, r\langle r^4 \rangle, (r\langle r^4 \rangle)^2, (r\langle r^4 \rangle)^3, s\langle r^4 \rangle, s\langle r^4 \rangle r\langle r^4 \rangle, s\langle r^4 \rangle (r\langle r^4 \rangle)^2, s\langle r^4 \rangle (r\langle r^4 \rangle)^3\} \\
&= \{\overline{1}, \overline{r}, \overline{r}^2, \overline{r}^3, \overline{s}, \overline{s}\overline{r}, \overline{s}\overline{r}^2, \overline{s}\overline{r}^3\}
\end{align*}

(c) Using Problem~\ref{normalpowers} and the fact that $\langle r^4 \rangle = \{1, r^4\}$, we have $|\overline{1}| = 1$, $|\overline{r}| = 4$, $|\overline{r}^2| = 2$, $|\overline{r}^3| = 4$, $|\overline{s}| = 2$, $|\overline{s}\overline{r}| = 2$, $|\overline{s}\overline{r}^2| = 2$, $|\overline{s}\overline{r}^3| = 2$.

(d) We have $\overline{rs} = \overline{r}\overline{s}$, $\overline{sr^{-2}s} = \overline{r^2s^2} = \overline{r}^2$ and $\overline{s^{-1}r^{-1}sr} = \overline{sr^{-1}sr} = \overline{rssr} = \overline{r}^2$.

(e) Note that since $\langle r^4 \rangle \subseteq H$ we have $\langle r^4 \rangle \leq H$ and so the Third Isomorphism Theorem applies. That is, $H/\langle r \rangle \normal G/\langle r \rangle$. Now note that $\overline{H} = \{\overline{s}, \overline{r}^2, \overline{s}\overline{r}^2, \overline{1}\}$. From part (c) we know that each of the nonidentity elements has order $2$. Furthermore, $\overline{s} \cdot \overline{r}^2 = \overline{s}\overline{r}^2$, $\overline{r}^2 \cdot \overline{s} = \overline{s}\overline{r}^{-2} = \overline{s}\overline{r}^6 = \overline{s}\overline{r}^2$, $\overline{s} \cdot \overline{s}\overline{r}^2 = \overline{r}^2$, $\overline{s}\overline{r^2} \cdot \overline{s} = \overline{r}^{-2}\overline{s}^2 = \overline{r}^2$, $\overline{r}^2 \cdot \overline{s}\overline{r}^2 = \overline{s}\overline{r}^{-2}\overline{r}^2 = \overline{s}$ and $\overline{s}\overline{r}^2 \cdot \overline{r}^2 = \overline{s}\overline{r}^4 = \overline{s}$. We've shown that all the relations hold for $\overline{H}$ being isomorphic the the Klein 4-group.

The preimage of $\overline{H}$ in $G$ is $\{1, r^4, s, sr^4, r^2, r^6, sr^2, sr^6\}$. Renaming $r^2$ as $r$ we see that $r^4 = s^2 = 1$ and $rs = sr^{-1}$. Thus the preimage of $\overline{H}$ in $G$ is isomorphic to $D_8$.

(f) We know $\overline{r},\overline{s} \notin Z(\overline{G})$ since $\overline{s}\overline{r} = \overline{r}^{-1}\overline{s}$. The same applies to $\overline{r}^3$. Multiplying $s$ by $sr^i$ results in $r^i$ and so these elements are not in $Z(\overline{G})$ either. This only leaves $\overline{r}^2$, which obviously commutes with $\overline{r}$ and $\overline{r}^3$. Now consider $\overline{r}^2(\overline{s}\overline{r}^i) = \overline{s}\overline{r}^{-2+i} = (\overline{s}\overline{r}^i)\overline{r}^2$. Thus $Z(\overline{G}) = \{\overline{1},\overline{r}^2\}$. We also have $\overline{G}/Z(\overline{G}) \cong V_4$. This can be seen by noticing $\overline{\overline{r}}^2 = \overline{\overline{r}^2} = \overline{\overline{1}}$, $\overline{\overline{s}}^2 = \overline{\overline{1}}$ and $\overline{\overline{sr}}^2 = \overline{\overline{sr}^2} = \overline{\overline{1}}$, and all the elements commute with each other.
\end{proof}

\begin{problem}[3.1.21]
Let $G = Z_4 \times Z_4$ be given in terms of the following generators and relations:
\[
G = \langle x, y \mid x^4 = y^4 = 1, xy = yx \rangle.
\]
Let $\overline{G} = G/\langle x^2y^2 \rangle$ (note that every subgroup of the abelian group $G$ is normal).\\
(a) Show that the order of $\overline{G}$ is $8$.\\
(b) Exhibit each element of $\overline{G}$ in the form $\overline{x}^a\overline{y}^b$, for some integers $a$ and $b$.\\
(c) Find the order of each of the elements of $\overline{G}$ exhibited in (b).\\
(d) Prove that $\overline{G} \cong Z_4 \times Z_2$.
\end{problem}
\begin{proof}
(a) Let $N = \langle x^2y^2 \rangle$. From Lagrange's Theorem we know $|\overline{G}| = |G|/|N| = 16/2 = 8$.\\

(b) We have
\[
\overline{G} = \{N, xN, x^2N, x^3N, yN, yxN, yx^2N, yx^3N\} = \{\overline{1}, \overline{x}, \overline{x}^2, \overline{x}^3, \overline{y}, \overline{y}\overline{x}, \overline{y}\overline{x}^2, \overline{y}\overline{x}^3\}.
\]

(c) We have $|\overline{1}| = 1$, $|\overline{x}| = 4$, $|\overline{x}^2| = 2$, $|\overline{x}^3| = 4$, $|\overline{y}| = 4$, $|\overline{y}\overline{x}| = 2$, $|\overline{y}\overline{x}^2| = 4$, $|\overline{y}\overline{x}^3| = 4$.

(d) Let $Z_4 \times Z_2 = \langle a, b \mid a^2 = b^4 = 1, ab = ba \rangle$. Let $\phi : \overline{G} \rightarrow Z_4 \times Z_2$ be a function such that $\phi(\overline{x}) = b$ and $\phi(\overline{y}\overline{x}^3) = a$. Note that $|\overline{x}| = |b| = 4$ and $|\overline{yx^3}| = |a| = 2$. Furthermore, $\overline{G} = \langle \overline{x}, \overline{y}\overline{x}^3 \rangle$. To see this, note that $\overline{x}^i\overline{y}^j = (\overline{y}\overline{x}^3)^j (\overline{x})^{-3j+i}$. Since the generators of $\overline{G}$ are mapped to the generators of $Z_4 \times Z_2$ and these groups have the same order, we see $\phi$ preserves the group structure. Injectivity and surjectivity also follow from this fact and we see that $\overline{G} \cong Z_4 \times Z_2$.
\end{proof}

\begin{problem}[3.1.24]
Prove that if $N \normal G$ and $H$ is any subgroup of $G$ then $N \cap H \normal H$.
\end{problem}
\begin{proof}
Let $h \in H$ and let $x \in N \cap H$. Then we have $hxh^{-1} \in N$ since $N \normal G$ and $hxh^{-1} \in H$ since $H \leq G$. Therefore $h(N \cap H)h^{-1} \subseteq N \cap H$ for all $h \in H$. Thus $(N \cap H) \normal H$.
\end{proof}

\begin{problem}[3.1.31]
Prove that if $H \leq G$ and $N$ is a normal subgroup of $H$ then $H \leq N_G(N)$. Deduce that $N_G(N)$ is the largest subgroup of $G$ in which $N$ is normal (i.e., is the join of all subgroups $H$ for which $N \normal H$).
\end{problem}
\begin{proof}
Since $N \normal H$, for all $h \in H$ we have $hNh^{-1} = N$. But then $H \leq \{g \in G \mid gNg^{-1} = N\} = N_G(N)$. Since this fact is true for any subgroup $H$ for which $N \normal H$, we see that $N_G(N)$ is the join of all such subgroups.
\end{proof}

\begin{problem}[3.1.36]
\label{cycliccenter}
Prove that if $G/Z(G)$ is cyclic then $G$ is abelian.
\end{problem}
\begin{proof}
Assume that $G/Z(G)$ is cyclic with generator $xZ(G)$. The left cosets of $G/Z(G)$ partition $G$, so for $u \in G$, we know $u \in (xZ(G))^a = x^aZ(G)$ for some integer $a$. But this means we can write $u = x^az$ for $z \in Z(G)$. Now take $u,v \in G$. Since $Z(G)$ is the set of elements of $G$ which commute with every element of $G$, we can write $uv = (x^az_1)(x^bz_2) = x^ax^bz_2z_1 = x^{a+b}z_2z_1 = x^bz_2x^az_1 = vu$.
\end{proof}

\begin{problem}[3.1.37]
Let $A$ and $B$ be groups. Show that $\{(a,1) \mid a \in A\}$ is a normal subgroup of $A \times B$ and the quotient of $A \times B$ by the is subgroup is isomorphic to $B$.
\end{problem}
\begin{proof}
Let $N = \{(a,1) \mid a \in A\}$. Let $(x,y) \in A \times B$ and consider $(x,y)(a,1)(x^{-1},y^{-1}) = (xax^{-1},yy^{-1}) = (xax^{-1},1)$. Thus $(x,y)N(x,y)^{-1} \subseteq N$ for all $(x,y) \in A \times B$. Thus $N \normal A \times B$. Now consider the function $\varphi : A \times B/N \to B$ such that $\phi((a,b)N) = b$. This function is injective, since $(a_1,b_1)N \neq (a_2,b_2)N$ implies $(a_2^{-1}a_1, b_2^{-1}b_1) \notin N$. Thus $b_2^{-1}b_1 \neq 1$ and $b_1 \neq b_2$. The map is clearly surjective since given $b \in B$ any element of the form $(a,b)N$ will map to it. Suppose we have $(a_1,b_1)N = (a_2,b_2)N$. Then $(a_2^{-1}a_1, b_2^{-1}b_1) \in N$. But this means $b_2^{-1}b_1 = 1$ and $b_1 = b_2$. Thus $\varphi$ is well defined. Finally, note $\varphi((a_1,b_1)N(a_2,b_2)N) = \varphi((a_1a_2,b_1b_2)N) = b_1b_2 = \varphi((a_1,b_1)N)\varphi((a_2,b_2)N)$. Thus $\varphi$ is an isomorphism and $A \times B/N \cong B$.
\end{proof}

\begin{problem}[3.1.41]
Let $G$ be a group. Prove that $N = \langle x^{-1}y^{-1}xy \mid x,y \in G \rangle$ is a normal subgroup of $G$ and $G/N$ is abelian ($N$ is called the \emph{commutator subgroup} of $G$).
\end{problem}
\begin{proof}
First note that $(x^{-1}y^{-1}xy){-1} = y^{-1}x^{-1}yx$. For $g \in G$ we have $g(x^{-1}y^{-1}xy)g^{-1} = gx^{-1}g^{-1}gy^{-1}g^{-1}gxg^{-1}gyg^{-1} = (gxg^{-1})^{-1}(gyg^{-1})^{-1}(gxg^{-1})(gyg^{-1})$. Thus, conjugation of a single commutator results in another commutator. Now suppose $z = (x_1^{-1}y_1^{-1}x_1y_1) \dots (x_n^{-1}y_n^{-1}x_ny_n)$ is the product of commutators. Then we have $gzg^{-1} = g(x_1^{-1}y_1^{-1}x_1y_1)g^{-1} \dots g(x_n^{-1}y_n^{-1}x_ny_n)g^{-1}$ and from the above result, we know this is then the product of commutators. This is then extended to the case where each commutator is raised to a power. For positive powers, the exact same argument holds. For negative powers, first separate the power into an inverse taken to a positive power, then use the first result of the proof. The conjugation is then a product of commutators. Therefore $gNg^{-1} \subseteq N$ for all $g \in G$ and thus $N \normal G$.

Now let $aN, bN \in G/N$. Then $aNbN = abN = \{abx \mid x \in N\}$. Now consider some element of $abN$, $ab(x^{-1}y^{-1}xy)$. We can write this as $ba(a^{-1}b^{-1}ab)(x^{-1}y^{-1}xy)$. Thus for each element of $abN$ we can find an equivalent element in $baN$ and vice versa. Therefore $abN = baN$ which means $aNbN = bNaN$ and $G/N$ is abelian.
\end{proof}

\begin{problem}[3.2.4]
Show that if $|G| = pq$ for primes $p$ and $q$ (not necessarily distinct) then either $G$ is abelian or $Z(G) = 1$.
\end{problem}
\begin{proof}
We know $|Z(G)| \mid |G|$. Assuming that $Z(G) \neq 1$, without loss of generality we either have $|Z(G)| = p$ or $|Z(G)| = pq$. In the later case we're done since $G = Z(G)$ which is abelian. In the former case, let $x \in G \backslash Z(G)$. Then $|y| = q$ and so $G = Z(G) \cup \langle y \rangle$. Since $Z(G)$ commutes with everything and $\langle y \rangle$ is abelian, $G$ must be abelian.
\end{proof}

\begin{problem}[3.2.11]
Let $H \leq K \leq G$. Prove that $|G : H| = |G : K| \cdot |K : H|$.
\end{problem}
\begin{proof}
Note that $|K : H|$ is the number of left cosets of $H$ in $K$. Also, $|G : K|$ is the number of left cosets of $K$ in $G$. That is, for each coset $K$ in $G$, we can further partition this coset into $|K : H|$ cosets of $H$ in $G$. Since there are $|G : K|$ of these partitions, and this gives all left cosets of $H$ in $G$, this gives $|G : K| \cdot |K : H| = |G : H|$.
\end{proof}

\begin{problem}[3.2.19]
Prove that if $N$ is a normal subgroup of the finite group $G$ and $(|N|, |G : N|) = 1$ then $N$ is the unique subgroup of $G$ of order $|N|$.
\end{problem}
\begin{proof}
Note that $G/N$ partitions $G$ into $|G : N|$ left cosets each with $|N|$ elements. But since $|N|$ and $|G : N| = |G|/|N|$ are relatively prime, there's only one way to do this.
\end{proof}

\begin{problem}[4.1.1]
Let $G$ act on the set $A$. Prove that if $a,b \in A$ and $b = g \cdot a$ for some $g \in G$, then $G_b = gG_ag^{-1}$ ($G_a$ is the stabilizer of $a$). Deduce that if $G$ acts transitively on $A$ then the kernel of the action is $\bigcap_{g \in G} gG_ag^{-1}$.
\end{problem}
\begin{proof}
Let $x \in G_b$. Then $x \cdot b = b = g \cdot a$. Therefore $g \cdot a = x \cdot b = x (g \cdot a)$ and so $a = g^{-1}g \cdot a = g^{-1}xg \cdot a$. Thus $x \in gG_ag^{-1}$. If $G$ acts transitively on $A$ then $b = g \cdot a$ for all $b \in A$ and some $g \in G$. Then for all $b \in A$ we have $G_b = gG_ag^{-1}$ for some $g \in G$. But we know the kernel of the action is $\bigcap_{b \in A} G_b = \bigcap_{g \in G} gG_ag^{-1}$.
\end{proof}

\begin{problem}[4.2.8]
Prove that if $H$ has finite index $n$ then there is a normal subgroup $K$ of $G$ with $K \leq H$ and $|G : K| \leq n!$.
\end{problem}
\begin{proof}
Let $G$ act on the set $A$ of left cosets of $H$ by left multiplication. Let $\pi_H$ be the permutation representation afforded by this action. Then we know $\ker \pi_H = \bigcap_{x \in G} xHx^{-1}$. Let $K = \ker \pi_H$. Then we know $K$ is normal and contained in $H$. Furthermore, since $|G : H| = n$, $\pi_H(G) \leq S_n$ and by the first isomorphism theorem $G/K \leq S_n$. Therefore $|G : K| \leq n!$.
\end{proof}

\begin{problem}[4.2.9]
Prove that if $p$ is prime and $G$ is a group of order $p^{\alpha}$ for some $\alpha \in \mathbb{Z}^+$, then every subgroup of index $p$ is normal in $G$. Deduce that every group of order $p^2$ has a normal subgroup of order $p$.
\end{problem}
\begin{proof}
We know that if $q$ is the smallest prime dividing $|G|$ then any subgroup of order $q$ is normal in $G$. But since the only prime which divides $|G|$ is $p$, we know that $p$ is the smallest prime dividing $|G|$ and hence every subgroup of order $p$ is normal in $G$. Suppose $|G| = p^2$ then there exists $x \in G$ such that $x \neq 1$. Since $\langle x \rangle \mid |G|$ we know $\langle x \rangle = p$. But then $|G : \langle x \rangle | = p$ as well as so $\langle x \rangle \normal G$.
\end{proof}

\begin{problem}[4.3.7]
\label{partitions}
For $n = 3$, $4$, $6$ and $7$ make lists of the partitions of $n$ and give representatives for the corresponding conjugacy classes of $S_n$.
\end{problem}

For $n=3$ the partitions are $(1,,1,1)$, $(1,2)$ and $(3)$ and they have representatives $(1)$, $(1 \; 2)$ and $(1 \; 2 \; 3)$.

For $n=4$ the partitions are $(1,1,1,1)$, $(1,1,2)$, $(1,3)$, $(4)$ and $(2,2)$ and they have representatives $(1)$, $(1 \; 2)$, $(1 \; 2 \; 3)$, $(1 \; 2 \; 3 \; 4)$ and $(1 \; 2)(3 \; 4)$.

For $n=6$ the partitions are $(1,1,1,1,1,1)$, $(1,1,1,1,2)$, $(1,1,1,3)$, $(1,1,4)$, $(1,5)$, $(6)$, $(1,1,2,2)$, $(1,2,3)$, $(2,2,2)$, $(2,4)$ and $(3,3)$ and they have representatives $(1)$, $(1 \; 2)$, $(1 \; 2 \; 3)$, $(1 \; 2 \; 3 \; 4)$, $(1 \; 2 \; 3 \; 4 \; 5)$, $(1 \; 2 \; 3 \; 4 \; 5 \; 6)$, $(1 \; 2)(3 \; 4)$, $(1 \; 2)(3 \; 4)(5 \; 6)$, $(1 \; 2)(3 \; 4 \; 5 \; 6)$ and $(1 \; 2 \; 3)(4 \; 5 \; 6)$.

For $n=7$ the partitions are $(1,1,1,1,1,1,1)$, $(1,1,1,1,1,2)$, $(1,1,1,1,3)$, $(1,1,1,4)$, $(1,1,5)$, $(1,6)$, $(7)$, $(1,1,1,2,2)$, $(1,1,2,3)$, $(1,2,2,2)$, $(1,3,3)$, $(1,2,4)$, $(2,5)$, and $(3,4)$ and they have representatives $(1)$, $(1 \; 2)$, $(1 \; 2 \; 3)$, $(1 \; 2 \; 3 \; 4)$, $(1 \; 2 \; 3 \; 4 \; 5)$, $(1 \; 2 \; 3 \; 4 \; 5 \; 6)$, $(1 \; 2 \; 3 \; 4 \; 5 \; 6 \; 7)$, $(1 \; 2)(3 \; 4)$, $(1 \; 2)(3 \; 4 \; 5)$, $(1 \; 2)(3 \; 4)(5 \; 6)$, $(1 \; 2 \; 3)(4 \; 5 \; 6)$, $(1 \; 2)(3 \; 4 \; 5 \; 6)$, $(1 \; 2)(3 \; 4 \; 5 \; 6 \; 7)$ and $(1 \; 2 \; 3)(4 \; 5 \; 6 \; 7)$.

\begin{problem}[4.3.26]
Let $G$ be a transitive permutation group on the finite set $A$ with $|A| > 1$. Show that there is some $\sigma \in G$ such that $\sigma(a) \neq a$ for all $a \in A$ (such an element $\sigma$ is called \emph{fixed point free}).
\end{problem}
\begin{proof}
Let $a \in A$. We know that $|\{\sigma(a) \mid \sigma \in G\}| = |G : G_a| = |G|/|G_a|$. But since $G$ acts transitively on $A$ we know that $A = \{\sigma(a) \mid \sigma \in G\}$. Therefore $|G|/|A| = |G_a|$. This is true for all $a \in A$. Furthermore, since $|A| > 1$ we know that $|G_a| < |G|$ for each $a \in A$. Thus there exists $\sigma \in G$ for each $a \in A$ such that $\sigma(a) \neq a$. Then taking the union $U = \bigcup_{a \in A} G_a$ we can find $\sigma \in G \backslash U$ which doesn't fix any $a \in A$.
\end{proof}

\begin{problem}[4.3.29]
Let $p$ be a prime and let $G$ be a group of order $p^{\alpha}$. Prove that $G$ has a subgroup of order $p^{\beta}$, for every $\beta$ with $0 \leq \beta \leq \alpha$.
\end{problem}
\begin{proof}
For the base case $\alpha = 0$ the problem is trivial since $|G| = 1$. Assume the statement is true for groups of order $\alpha$ and suppose that $|G| = p^{\alpha + 1}$. Then we know $Z(G) \neq 1$ which means $|Z(G)| = p^{\gamma}$ where $1 < \gamma \leq \alpha + 1$. Since $Z(G)$ is abelian, we know there exists $x \in Z(G)$ such that $|x| = p$. Now consider $\overline{G} = |G|/\langle x \rangle$. Note that $\overline{G} \cong H$ for some $H \leq G$. But also $|\overline{G}| = |H| = p^{\alpha}$. Therefore $H$ has subgroups of order $p^{\beta}$ for each $0 \leq \beta \leq \alpha$ and therefore $G$ does as well.
\end{proof}

\begin{problem}
Write the class equation for $A_4$.
\end{problem}
\begin{proof}
From Problem~\ref{partitions} we know that representatives of the cycles types of even permutations of $S_4$ can be taken to be $(1)$, $(1 \; 2 \; 3)$ and $(1 \; 2)(3 \; 4)$. Furthermore we know that
\[
C_{S_4}((1 \; 2 \; 3)) = \{(1 \; 2 \; 3)^{i} \tau \mid \text{$i = 0$, $1$ or $2$ and $\tau \in S_{4-3}$}\} = \langle (1 \; 2 \; 3) \rangle
\]
which directly implies $C_{A_4}((1 \; 2 \; 3)) = \langle (1 \; 2 \; 3) \rangle$. This group has order $3$ and index $4$. Since there are $8$ $3$-cycles in $A_4$, and four of them are in the conjugacy class of $(1 \; 2 \; 3)$, there must be another $3$-cycle not in this class. Using the same logic as above, we see that the centralizer of this $3$-cycle is also of order $3$ and so it has index $4$. Finally, note that $\langle (1 \; 2)(3 \; 4), (1 \; 3)(2 \; 4) \rangle \cong V_4$ and so all of these elements commute with each other. Since these are the only elements of $A_4$ with this cycle type, we see that $|C_{A_4}((1 \; 2)(3 \; 4))| = 4$ and has index $3$. Therefore, the class equation for $A_4$ is
\[
|A_4| = |Z(A_4)| + |A_4 : C_{A_4} ((1 \; 2 \; 3))| + |A_4 : C_{A_4} ((1 \; 3 \; 2))| + |A_4 : C_{A_4}((1 \; 2)(3 \; 4))| = 1 + 4 + 4 + 3 = 12.
\]
\end{proof}

\end{document}