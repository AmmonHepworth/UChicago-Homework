\documentclass{article}
\usepackage{amsmath,amsthm,amssymb,amsfonts,fullpage,fancyhdr}

\pagestyle{fancy}
\renewcommand{\headheight}{50pt}
\renewcommand{\footskip}{40pt}
\renewcommand{\textheight}{590pt}
\renewcommand{\headrulewidth}{0pt}

\newtheorem{problem}{Problem}

\newcommand{\normal}{\unlhd}
\newcommand{\aut}{\textup{Aut}}

\begin{document}

\rhead{Kris Harper\\MATH 25700\\November 13, 2009\\}
\chead{Homework 6\\}

\begin{problem}[4.6.2]
Find all normal subgroups of $S_n$ for all $n \geq 5$.
\end{problem}
\begin{proof}
Since $A_n$ is simple for $n \geq 5$, we see that no proper nontrivial subgroup of $A_n$ is normal in $S_n$. Therefore the only possible proper nontrivial normal subgroup of $S_n$ is $A_n$, which is indeed normal since it has index $2$. Thus $1$, $A_n$, and $S_n$ are the only normal subgroups of $S_n$ for $n \geq 5$.
\end{proof}

\begin{problem}[4.6.4]
Prove that $A_n$ is generated by the set of all $3$-cycles for $n \geq 3$.
\end{problem}
\begin{proof}
First note that any pair of transpositions can be written as a product of $3$-cycles. If $a \neq c$ and $b \neq d$ then $(ab)(cd) = (acb)(acd)$ and in the case $a = c$, $(ab)(cd) = (adb)$ (note that if $a = c$ and $b=d$ then $(ab)(cd) = 1$). Since any element of $S_n$ can be written as the product of transpositions, and $A_n$ is the collection of even permutations, any element $x \in A_n$ can be written as an even number of transpositions. Then we can pair these up and write $x$ as a product of $3$-cycles. This shows that $A_n$ is generated by $3$-cycles.
\end{proof}

\begin{problem}[5.1.4]
Let $A$ and $B$ be finite groups and let $p$ be a prime. Prove that any Sylow $p$-subgroup of $A \times B$ is of the form $P \times Q$ where $P \in Syl_p(A)$ and $Q \in Syl_p(B)$. Prove that $n_p(A \times B) = n_p(A)n_p(B)$. Generalize both of these results to a direct product of any finite number of finite groups (so that the number of Sylow $p$-subgroups of a direct product is the product of the numbers of Sylow $p$-subgroups of the factors).
\end{problem}
\begin{proof}
Let $|A| = p^am$ and $|B| = p^bn$ with $p \nmid m$ and $p \nmid n$ so that $|A \times B| = p^{a+b}mn$ where $p \nmid mn$. Suppose $R \in Syl_p(A \times B)$. Then $R \leq \{(a,b) \mid a \in P, b \in Q\}$ for $P \leq A$ and $Q \leq B$. That is, if we consider the coordinates of $R$ corresponding to $A$ and $B$ separately, these elements form subgroups of $A$ and $B$ respectively, although we are not assuming that $R$ is the entire direct product $P \times Q$. Note that $|R| = p^{a+b}$ which means $p^{a+b} \leq |P||Q|$. Since $P \leq A$ and $a$ is maximal for $A$, then $|P| = p^a$. Likewise $|Q| = p^b$. Thus $P \in Syl_p(A)$ and $Q \in Syl_p(B)$ and we must have $R = P \times Q$. Furthermore, this shows that if $P' \in Syl_p(A)$ and $Q' \in Syl_p(B)$ then $P' \times Q' \in Syl_p(A \times B)$. Therefore, $n_p(A \times B) \leq n_p(A)n_p(B)$ by the first statement and $n_p(A)n_p(B) \leq n_p(A \times B)$ by the second. Thus they must be equal.

To generalize to a finite product of finite groups, we use induction on the number of groups. The $n = 1$ case is trivial, and the inductive step has been done above by letting $A$ be a direct product of $n-1$ finite groups.
\end{proof}

\begin{problem}[5.1.5]
Exhibit a nonnormal subgroup of $Q_8 \times Z_4$ (note that every subgroup of each factor is normal).
\end{problem}
\begin{proof}
Consider the group $H = \langle (i, x) \rangle$. Then $(j, 1)H(j, 1)^{-1}$ contains the element $(jij^{-1}, x) = (-i, x)$ which isn't in $H$ (the only element of $H$ with $-i$ in the first coordinate has $x^3$ in the second coordinate). Thus $H \ntrianglelefteq Q_8 \times Z_4$.
\end{proof}

\begin{problem}[5.1.10]
Let $p$ be a prime. Let $A$ and $B$ be two cyclic groups of order $p$ with generators $x$ and $y$ respectively. Set $E = A \times B$ so that $E$ is the elementary abelian group of order $p^2$: $E_{p^2}$. Prove that the distinct subgroups of $E$ of order $p$ are
\[
\text{$\langle x \rangle$, $\langle xy \rangle$, $\langle xy^2 \rangle$, \dots , $\langle xy^{p-1} \rangle$, $\langle y \rangle$}
\]
(note there are $p+1$ of them).
\end{problem}
\begin{proof}
A subgroup of order $p$ must be generated by some element of $E$. We show that a given element $x^iy^j \in E$ is in one of the enumerated subgroups. This is equivalent to finding $k$ such that $(xy^k)^i = x^iy^{ik} = x^iy^j$. That is, finding $0 \leq k \leq p-1$ such that $ik \equiv j \pmod{p}$. Since $i$ and $k$ are necessarily relatively prime to $p$, such a $k$ must exist. Therefore $x^iy^j \in \langle xy^k \rangle$ for some $k$ and is thus in one of the enumerated subgroups. Since $y$ has order $p$, it's clear that $y^k$ gives distinct values for all $0 \leq k \leq p-1$. Thus, the elements $x$, $xy$, ..., $xy^{p-1}$, $y$ are all distinct elements of $E$ and each generates a distinct subgroup of order $p$. Since there are $p+1$ of these elements and there cannot be more than $p+1$ subgroups of order $p$ in $E$, these must be exactly the groups of order $p$.
\end{proof}

\begin{problem}[5.1.11]
Let $p$ be a prime and let $n \in \mathbb{Z}^+$. Find a formula for the number of subgroups of order $p$ in the elementary abelian group $E_{p^n}$.
\end{problem}
\begin{proof}
Note that $|E_{p^n}| = p^n$ and every nonidentity element has order $p$. Thus, there are $p^n - 1$ elements of order $p$ and each of these generates a subgroup of order $p$. Each of these subgroups have trivial intersection since they are all distinct and every nonidentity element is a generator. Then there are $p-1$ elements of order $p$ in each subgroup, so there are $(p^n-1)/(p-1)$ subgroups of order $p$.
\end{proof}

\begin{problem}[5.1.14]
\label{fractions}
Let $G = A_1 \times A_2 \times \dots \times A_n$ and for each $i$ let $B_i$ be a normal subgroup of $A_i$. Prove that $B_1 \times B_2 \times \dots \times B_n \normal G$ and that
\[
(A_1 \times A_2 \times \dots \times A_n)/(B_1 \times B_2 \times \dots \times B_n) \cong (A_1/B_1) \times (A_2/B_2) \times \dots \times (A_n/B_n).
\]
\end{problem}
\begin{proof}
Let $H = B_1 \times B_2 \times \dots \times B_n$ and $K = (A_1/B_1) \times (A_2/B_2) \times \dots \times (A_n/B_n)$. Let $a = (a_1, \dots , a_n) \in G$ and note that since $B_i \normal A_i$ we have $a_iB_ia_i^{-1} = B_i$. Thus
\[
aHa^{-1} = (a_1, \dots , a_n)(B_1 \times \dots \times B_n)(a_1^{-1}, \dots , a_n^{-1}) = a_1B_1a_1^{-1} \times \dots \times a_nB_na_n^{-1} = B_1 \times \dots \times B_n = H
\]
and $H \normal G$. Now define $\varphi : G \to K$ by $\varphi((a_1, \dots , a_n)) = (a_1B_1, \dots , a_nB_n)$. Note that for $(a_1, \dots , a_n), (b_1, \dots , b_n) \in G$ we have
\begin{align*}
\varphi((a_1, \dots , a_n)(b_1, \dots , b_n))
&= \varphi((a_1b_1, \dots , a_nb_n))\\
&= (a_1b_1B_1, \dots , a_nb_nB_n)\\
&= (a_1B_1b_1B_1, \dots , a_nB_nb_nB_n)\\
&= (a_1B_1, \dots , a_nB_n)(b_1B_n, \dots , b_nB_n)\\
&= \varphi((a_1, \dots , a_n))\varphi((b_1, \dots , b_n))
\end{align*}
and thus $\varphi$ is a homomorphism. Also note that if $(a_1B_1, \dots , a_nB_n) \in K$, then $\varphi((a_1, \dots a_n)) = (a_1B_1, \dots , a_nB_n)$ and so $\varphi(G) = K$. Furthermore, if $\varphi((a_1, \dots , a_n)) = (B_1, \dots , B_n)$ then $a_iB_i = B_i$ and so necessarily $a_i \in B_i$. Thus $(a_1, \dots, a_n) \in H$. And if $(a_1, \dots , a_n) \in H$ then $\varphi((a_1, \dots , a_n)) = (a_1B_1, \dots , a_nB_n) = (B_1, \dots B_n)$ since $a_i \in B_i$ implies $a_iB_i = B_i$. Therefore $\ker \varphi = H$. From the first isomorphism theorem, we now have $G/H \cong K$ and this concludes the proof.
\end{proof}

\begin{problem}[5.2.7]
\label{pthpower}
Let $p$ be a prime and let $A = \langle x_1 \rangle \times \langle x_2 \rangle \times \dots \times \langle x_n \rangle$ be an abelian $p$-group, where $|x_i| = p^{\alpha_i} > 1$ for all $i$. Define the $p^{\textup{th}}$-\emph{power map}
\[
\text{$\varphi : A \to A$ by $x \mapsto x^p$.}
\]
(a) Prove that $\varphi$ is a homomorphism.\\
(b) Describe the image and kernel of $\varphi$ in terms of the given generators.\\
(c) Prove both $\ker \varphi$ and $A/\textup{im} \varphi$ have rank $n$ (i.e., have the same rank as $A$) and prove these groups are both isomorphic to the elementary abelian group, $E_{p^n}$, of order $p^n$.
\end{problem}
\begin{proof}
(a) For $x_1^{a_1} \cdots x_n^{a_n}$ and $x_1^{b_1} \cdot x_n^{a_n}$ elements of $A$ we have
\begin{align*}
\varphi(x_1^{a_1} \cdots x_n^{a_n}x_1^{b_1} \cdots x_n^{b_n})
&= \varphi(x_1^{a_1+b_1} \cdots x_n^{a_n+b_n})\\
&= (x_1^{a_1+b_1} \cdots x_n^{a_n+b_n})^p\\
&= x_1^{pa_1+pb_1} \cdots x_n^{pa_n+pb_n}\\
&= x_1^{pa_1} \cdots x_n^{pa_n}x_1^{pb_1} \cdots x_n^{pb_n}\\
&= (x_1^{a_1} \cdots x_n^{a_n})^p(x_1^{b_1} \cdots x_n^{b_n})^p\\
&= \varphi(x_1^{a_1} \cdots x_n^{a_n})\varphi(x_1^{b_1} \cdots x_n^{a_n}).
\end{align*}

(b) Note that in each coordinate the elements which map to $1$ under $\varphi$ are those of the form $x^{kp^{\alpha_i-1}}$. We therefore have
\[
\ker \varphi = \prod_{i = 1}^n \{x_i^{kp^{\alpha_i-1}} \mid 0 \leq k \leq p-1\}.
\]
Since there are $p$ choices for $k$ in each of these components ($p^{\alpha_i} > 1$ by assumption), we find that $\ker \varphi = E_{p^n}$, or more explicitly,
\[
\ker \varphi = \langle x_1 \rangle/Z_{p^{\alpha_1-1}} \times \dots \times \langle x_n \rangle/Z_{p^{\alpha_n-1}}.
\]
Now since $\varphi$ is a homomorphism by (a), from the first isomorphism theorem we know that $\varphi(A) \cong A/\ker \varphi = A/E_{p^n}$. In particular, each component is equal to $\langle x_i \rangle /Z_p$ using Problem~\ref{fractions}.

(c) We showed in part (b) that $\ker \varphi \cong E_{p^n}$. Now consider
\[
A/\varphi(G) = A/(A/E_{p^n}) \cong \langle x_1 \rangle/(\langle x_1 \rangle/Z_p) \times \dots \times \langle x_n \rangle/(\langle x_n \rangle/Z_p).
\]
Note that from Lagrange's Theorem, we know that each component has order $p$ (again, $\alpha_i > 1$ by assumption), and is thus isomorphic to $Z_p$. Therefore $A/\varphi(G) \cong E_{p^n}$. Since each of $\ker \varphi$ and $A/\varphi(G)$ are isomorphic to $E_{p^n}$, we have shown that they each have rank $n$.
\end{proof}

\begin{problem}[5.2.8]
Let $A$ be a finite abelian group (written multiplicatively) and let $p$ be a prime. Let
\[
\text{$A^p = \{a^p \mid a \in A\}$ and $A_p = \{x \mid x^p = 1\}$}
\]
(so $A^p$ and $A_p$ are the image and kernel of the $p^{\textup{th}}$-power map, respectively).\\
(a) Prove that $A/A^p \cong A_p$.\\
(b) Prove that the number of subgroups of $A$ of order $p$ equals the number of of subgroups of $A$ of index $p$.
\end{problem}
\begin{proof}
(a) Let $A = Z_{n_1} \times \dots \times Z_{n_t}$ and let $\varphi$ be the $p^{\text{th}}$ power map. If $p \nmid n_i$ then $Z_{n_i}$ has no elements of order $p$. Thus the kernel of $\varphi$ in $Z_{n_i}$ is trivial and this map is injective. Therefore $Z_{n_i}^p \cong Z_{n_i}$. On the other hand, if $p \mid n_i$ then $n_i = p^{\alpha_i}m_i$ where $p \nmid m_i$. The kernel of $\varphi$ in $Z_{n_i}$ in this case is all elements of the form $x^{p^{k\alpha_i-1}m}$ where $0 \leq k \leq p-1$ which is thus $Z_p$. Therefore, as in Problem~\ref{pthpower}, we find that $A_p = E_{p^s}$ where $s \leq t$ and $t-s$ is the number of $n_i$ which don't have $p$ as a factor. Now using the first isomorphism theorem we have $A^p \cong A/E_{p^s}$ and using Problem~\ref{fractions} we have
\begin{align*}
A/A^p
&= A/(A/E_{p^s})\\
&= Z_{n_1} \times \dots \times Z_{n_t}/(Z_{n_1} \times \dots \times Z_{n_t}/Z_p \times \dots \times Z_p)\\
&\cong Z_{n_1}/(Z_{n_1} \times \dots \times Z_{n_t}/Z_p \times \dots \times Z_p) \times \dots \times Z_{n_t}/(Z_{n_1} \times \dots \times Z_{n_t}/Z_p \times \dots \times Z_p)\\
&\cong Z_{n_1}/(Z_{n_1}/Z_p) \times \dots \times Z_{n_t}/(Z_{n_t}/Z_p).
\end{align*}
Note that we've written this product so that for $n_i$ with $p \nmid n_i$, a $1$ appears in the product $E_{p^s}$. That is, $E_{p^s}$ in this case is the product of $s$ copies of $Z_p$ along with $t-s$ trivial groups in $i$th place if $p \nmid n_i$. Now using Lagrange's Theorem, each of the groups in the product has order $p$ or $1$, with $t-s$ groups of order $1$, and is thus isomorphic to $Z_p$ which shows that $A/A^p \cong E_{n^s} \cong A_p$.

(b) Note that the number of elements of order $p$ is precisely the number of elements in $A_p$ (minus the identity) as these  elements get mapped to $1$ when raised to the $p^{\text{th}}$ power and $p$ is prime. Each generates a subgroup of order $p$, and each of these subgroups trivially intersect. There are then $p-1$ distinct elements contributed from each subgroup, so the total number of subgroups of order $p$ is
\[
\frac{|A_p| - 1}{p-1} = \frac{p^s-1}{p-1}.
\]
Now we consider groups of index $p$. Each element of $A/A^p$ corresponds to a group of index $p$. This can be seen by noting that $A^p \cong A/A_p$. This counts the trivial group as well, and so there are $|A/A^p|$ groups of index $p$ subgroups, and for each one there are $p-1$ different elements which give the same group. Thus there are
\[
\frac{|A/A^p|-1}{p-1} = \frac{p^s-1}{p-1}
\]
groups of index $p$.
\end{proof}

\begin{problem}[5.2.10]
Let $n$ and $k$ be positive integers and let $A$ be the free abelian group of rank $n$ (written additively). Prove that $A/kA$ is isomorphic to the direct product of $n$ copies of $\mathbb{Z}/k\mathbb{Z}$ (here $kA = \{ka \mid a \in A\}$).
\end{problem}
\begin{proof}
Using Problem~\ref{fractions} it suffices to prove that $k\mathbb{Z} \normal \mathbb{Z}$ for each $k$. But $\mathbb{Z}$ is abelian, so every subgroup is normal.
\end{proof}

\end{document}