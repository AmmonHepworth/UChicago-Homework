\documentclass{article}
\usepackage{amsmath,amsthm,amssymb,amsfonts,fullpage,fancyhdr}

\pagestyle{fancy}
\renewcommand{\headheight}{50pt}
\renewcommand{\footskip}{40pt}
\renewcommand{\textheight}{590pt}
\renewcommand{\headrulewidth}{0pt}

\newtheorem{problem}{Problem}

\newcommand{\normal}{\unlhd}
\newcommand{\aut}{\textup{Aut}}
\newcommand{\inner}{\textup{Inn}}

\begin{document}

\rhead{Kris Harper\\MATH 25700\\November 6, 2009\\}
\chead{Homework 5\\}

\begin{problem}[4.4.1]
If $\sigma \in \aut (G)$ and $\varphi_g$ is conjugation by $g$, prove $\sigma \varphi_g \sigma^{-1} = \varphi_{\sigma(g)}$. Deduce that $\inner (G) \normal \aut (G)$. (The group $\aut (G) / \inner (G)$ is called the \emph{outer automorphism group} of $G$.)
\end{problem}
\begin{proof}
Let $x \in G$. Then
\[
\sigma \varphi_g \sigma^{-1}(x) = \sigma(\varphi_g(\sigma^{-1}(x))) = \sigma(g\sigma^{-1}(x)g^{-1}) = \sigma(g)\sigma(\sigma^{-1}(x))\sigma(g^{-1}) = \sigma(g)x\sigma(g)^{-1} = \varphi_{\sigma(g)}.
\]
Since $\varphi_g, \varphi_{\sigma(g)} \in \inner (G)$ and $\sigma \in \aut (G)$, we see that $\sigma \inner(G) \sigma^{-1} \subseteq \inner (G)$ for all $\sigma \in \aut (G)$. Therefore $\inner(G) \normal \aut (G)$.
\end{proof}

\begin{problem}[4.4.3]
Prove that under any automorphism of $D_8$, $r$ has at most $2$ possible images and $s$ has has at most $4$ possible images. Deduce that $|\aut (D_8)| \leq 8$.
\end{problem}
\begin{proof}
Note that $|sr^i| = 2$ since $(sr^i)(sr^i) = s^2r^{-i}r^i = 1$. Furthermore, $|r^2| = 2$ and $|r^3| = 4$. Since automorphisms preserve order, $r$ must be mapped to either $r$ or $r^3$. Furthermore, $s$ can't be mapped to $r^2$ because then the image of $sr$ is the same as the image of $rs$. Since automorphisms are homomorphisms, this is impossible. Therefore $s$ can be mapped to one of $sr^i$ for the four possible values of $i$. Since $r$ and $s$ are the two generators of $D_8$, there are at most $2 \cdot 4 = 8$ possible automorphisms of $D_8$.
\end{proof}

\begin{problem}[4.4.6]
Prove that characteristic subgroups are normal. Give an example of a normal subgroup that is not characteristic.
\end{problem}
\begin{proof}
Let $H$ be a characteristic subgroup and let $\varphi_g$ be conjugation by $g \in G$. We know that $\varphi_g$ is an automorphism and so $gHg^{-1} = \varphi_g(H) = H$. Since this is true for all $g \in G$, we see that $H \normal G$. Consider the Klein 4-group $V_4$ with generators $a,b,c$. Then $\langle a \rangle$ is normal, since $V_4$ is abelian, but an automorphism which maps $a$ to $b$, $b$ to $c$ and $c$ to $a$ won't fix $\langle a \rangle$. Thus $\langle a \rangle$ is not characteristic.
\end{proof}

\begin{problem}[4.4.7]
\label{unique}
If $H$ is the unique subgroup of a given order in a group $G$ prove $H$ is characteristic in $G$.
\end{problem}
\begin{proof}
Let $\varphi \in \aut (G)$ and let $a, b \in H$. Then $\varphi(a)\varphi(b)^{-1} = \varphi(ab^{-1}) \in \varphi(H)$. Therefore $\varphi(H) \leq G$ and so it's order is preserved. Since this is true for any subgroup of $G$, we know that if $H$ has unique order among subgroups of $G$, then it is preserved by any automorphism of $G$. Therefore $H$ char $G$.
\end{proof}

\begin{problem}[4.4.8]
\label{normaltransitive}
Let $G$ be a group with subgroups $H$ and $K$ with $H \leq K$.\\
(a) Prove that if $H$ is characteristic in $K$ and $K$ is normal in $G$ then $H$ is normal in $G$.\\
(b) Prove that if $H$ is characteristic in $K$ and $K$ is characteristic in $G$ then $H$ is characteristic in $G$. Use this to prove that the Klein 4-group $V_4$ is characteristic in $S_4$.\\
(c) Give an example to show that if $H$ is normal in $K$ and $K$ is characteristic in $G$ then $H$ need not be normal in $G$.
\end{problem}
\begin{proof}
(a) Suppose $H$ char $K$ and $K \normal G$. Then for all $g \in G$, $gKg^{-1} = K$. But this is an automorphism of $K$ and so $gHg^{-1} = H$ since $H$ char $K$. Therefore $H \normal G$.

(b) Let $\varphi \in \aut (G)$. Then $\varphi(K) = K$ and so $\varphi \in \aut (K)$. But then $\varphi (H) = H$ since $H$ char $K$. Therefore $H$ char $G$. To show that $V_4$ char $S_4$ note that $V_4$ char $A_4$ by Problem~\ref{unique} since it is the only subgroup of order $4$ in $A_4$. Likewise, $A_4$ char $S_4$ since it the unique subgroup of order $12$ in $S_4$. Thus $V_4$ char $S_4$.

(c) Let $G = S_4$, $K = \langle (1 \; 2)(3 \; 4), (1 \; 3)(2 \; 4) \rangle$ and $H = \langle (1 \; 2)(3 \; 4) \rangle$. Then $H \normal K$ since $K \cong V_4$ and so $K$ is abelian. Also $K$ char $G$ using part (b). But then $H$ is not normal in $G$ as can be seen by conjugating by $(1 \; 2 \; 3)$.
\end{proof}

\begin{problem}[4.4.15]
Prove that each of the following (multiplicative) groups is cyclic: $(\mathbb{Z}/5\mathbb{Z})^{\times}$, $(\mathbb{Z}/9\mathbb{Z})^{\times}$ and $(\mathbb{Z}/18\mathbb{Z})^{\times}$.
\end{problem}
\begin{proof}
We know that $(\mathbb{Z}/5\mathbb{Z})^{\times} \cong \aut (\mathbb{Z}/5\mathbb{Z})$. Let $\Psi : (\mathbb{Z}/5\mathbb{Z})^{\times} \to \aut (\mathbb{Z}/5\mathbb{Z})$ be an isomorphism such that $\Psi(a) = \psi_a$ where $\psi_a : \mathbb{Z}/5\mathbb{Z} \to \mathbb{Z}/5\mathbb{Z}$ is an isomorphism such that $\psi_a(x) = x^a$ for a generator $x$. Note that since $\psi_a$ is an isomorphism, $x^a$ must be a generator for $\mathbb{Z}/5\mathbb{Z}$ and so $(a,4) = 1$. Such an $a$ must exist in $(\mathbb{Z}/5\mathbb{Z})^{\times}$ because $\Psi$ is an isomorphism. Now consider $\psi_b \in \aut(\mathbb{Z}/5\mathbb{Z})$. We know $\psi_b(x) = x^b = (x^a)^b = (\psi_a(x))^b$. Therefore $\aut(\mathbb{Z}/5\mathbb{Z})$ is cyclic and so $(\mathbb{Z}/5\mathbb{Z})^{\times}$ is cyclic as well. A similar argument holds for $(\mathbb{Z}/9\mathbb{Z})^{\times}$ and $(\mathbb{Z}/18\mathbb{Z})^{\times}$.
\end{proof}

\begin{problem}[4.4.16]
Prove that $(\mathbb{Z}/24\mathbb{Z})^{\times}$ is an elementary abelian group of order $8$.
\end{problem}
\begin{proof}
We know $|(\mathbb{Z}/24\mathbb{Z})^{\times}| = 8$ since $\varphi(24) = 8$. Furthermore, $\overline{1}^2 = \overline{5}^2 = \overline{7}^2 = \overline{11}^2 = \overline{13}^2 = \overline{17}^2 = \overline{19}^2 = \overline{23}^2 = \overline{1}$. Therefore since $(\mathbb{Z}/24\mathbb{Z})^{\times}$ has order $2^3$ and each element applied to itself $2$ times is the identity, it must be an elementary abelian group.
\end{proof}

\begin{problem}[4.4.20]
For any finite group $P$ let $d(P)$ be the minimum number of generators of $P$ (so, for example, $d(P) = 1$ if an only if $p$ is a nontrivial cyclic group and $d(Q_8) = 2$). Let $m(P)$ be the maximum of the integers $d(A)$ as $A$ runs over all \emph{abelian} subgroups of $P$ (so, for example, $m(Q_8) = 1$ and $m(D_8) = 2$). Define
\[
J(P) = \langle A \mid \text{$A$ is an abelian subgroup of $P$ with $d(A) = m(P)$} \rangle.
\]
($J(P)$ is called the \emph{Thompson subgroup} of $P$.)\\
(a) Prove that $J(P)$ is a characteristic subgroup of $P$.\\
(b) For each of the following groups $P$ list all abelian subgroups $A$ of $P$ that satisfy $d(A) = m(P)$: $Q_8$, $D_8$, $D_{16}$ and $QD_{16}$ (where $QD_{16}$ is the quasidihedral group of order $16$).\\
(c) Show that $J(Q_8) = Q_8$, $J(D_8) = D_8$, $J(D_{16}) = D_{16}$ and $J(QD_{16})$ is a dihedral subgroup of order $8$ in $QD_{16}$.\\
(d) Prove that if $Q \leq P$ and $J(P)$ is a subgroup of $Q$ then $J(P) = J(Q)$. Deduce that if $P$ is a subgroup (not necessarily normal) of the finite group $G$ and $J(P)$ is contained in some subgroup $Q$ of $P$ such that $Q \normal G$, then $J(P) \normal G$.
\end{problem}
\begin{proof}
(a) Let $\varphi$ be an automorphism of $P$. We know that for each abelian subgroup $A \leq P$, $\varphi$ must map $A$ to some abelian subgroup $\varphi(A)$ of the same order. Furthermore, suppose $d(A) = k$ and $d(\varphi(A)) = k'$. Then it must be the case that $k = k'$. If one were less than the other then we could use $\varphi$ or $\varphi^{-1}$ and to find a smaller set of generators for $A$ or $\varphi(A)$. Therefore for each abelian subgroup $A \leq P$ we have $d(A) = d(\varphi(A))$. But this directly implies that $m(P) = m(\varphi(P))$. If $x \in J(P)$, then $x$ is a product of elements of $A \leq P$ where $d(A) = m(P)$. Then $\varphi(x)$ can be written as a product of images under $\varphi$ of elements of these sets, and therefore $\varphi(x) \in \langle \varphi(A) \mid \text{$\varphi(A)$ is abelian and $d(\varphi(A)) = m(P)$} \rangle$. But we've just shown this set is precisely $J(P)$. Therefore $J(P)$ char $P$.

(b) For $Q_8$ the subgroups are $\langle i \rangle$, $\langle j \rangle$, $\langle k \rangle$, $\langle -1 \rangle$ and $\langle 1 \rangle$. For $D_8$ the subgroups are $\langle s, r^2 \rangle$ and $\langle rs, r^2 \rangle$. For $D_{16}$ the subgroups are $\langle sr^2, r^4 \rangle$, $\langle s, r^4 \rangle$, $\langle sr^3, r^4 \rangle$ and $\langle sr^5, r^4 \rangle$. For $QD_{16}$ the subgroups are $\langle a^4, x \rangle$ and $\langle a^4, a^2x \rangle$.

(c) Every subgroup of $Q_8$ is used to generate $J(Q_8)$. In particular, the cyclic group generated by every element is a generator for $J(Q_8)$. Thus $J(Q_8) = Q_8$. In $J(D_8)$, note that $s$ and $rs$ are both generators by part (b). But then $(rs)s = r$ is in $J(D_8)$ and so both generators of $D_8$ are in $J(D_8)$. Therefore the groups are equal. The case for $J(D_{16})$ is similar since $s$, $sr^5$ and $r^4$ are all generators of $J(D_{16})$. The group $J(QD_{16})$ is generated by $a^4$, $x$, and $a^2x$. Multiplying $a^2x$ and $x$, it's easy to get $a^4$, so we have $J(QD_{16}) = \langle x, a^2 \rangle$. Then these elements have order $2$ and $4$ respectively, so they can be mapped to $s$ and $r$ in $D_8$. In particular $x(a^2)^i = (a^2)^{-i}x$. This shows that $J(QD_{16}) \cong D_8$ in $QD_{16}$.

(d) Since $Q \leq P$ it's certainly the case that $J(Q) \subseteq J(P)$. Let $x \in J(P)$. Then $x \in Q$ and $x$ is the product of elements from abelian subgroups $A$ of $P$ such that $d(A) = m(P)$. But note that since $J(P) \leq Q$, each of these subgroups $A$ is a subgroup of $Q$ as well. Therefore $x \in J(Q)$ and so $J(P) \subseteq J(P)$. We've shown both inclusions so $J(P) = J(Q)$. If $P \leq G$ and $J(P) \leq Q$ where $Q \normal G$, then $J(P) = J(Q)$. From part (a) we know that $J(Q)$ char $Q$, and from Problem~\ref{normaltransitive} we know this means $J(Q) \normal G$. But then $J(P) \normal G$.
\end{proof}

\begin{problem}[4.5.1]
Prove that if $P \in Syl_p (G)$ and $H$ is a subgroup of $G$ containing $P$ then $P \in Syl_p(H)$. Give an example to show that, in general, a Sylow $p$-subgroup of a subgroup of $G$ need not be a Sylow $p$-subgroup of $G$.
\end{problem}
\begin{proof}
Note that $|G| = p^{\alpha}m$ with $p \nmid m$ and by Lagrange's Theorem, $|H| = p^{\beta}k$ where $0 \leq \beta \leq \alpha$, $p \nmid k$ and $k \leq m$. Since $P \leq H$ we know $p^{\alpha} \mid |H|$ and thus $\beta = \alpha$. Therefore $|H| = p^{\alpha}k$ where $p \nmid k$. Hence $P \in Syl_p(H)$.

As an example, $A_4$ has the unique Sylow 2-subgroup $\langle (12)(34), (13(24) \rangle$ with order $2^2$, but $|S_4| = 2^3 \cdot 3$ and so this is not a Sylow 2-subgroup in $S_4$.
\end{proof}

\begin{problem}[4.5.13]
\label{order56}
Prove that a group of order $56$ has a normal Sylow $p$-subgroup for some prime $p$ dividing its order.
\end{problem}
\begin{proof}
Note that $56 = 2^3 \cdot 7$ and the Sylow divisibility and congruence rules dictate that either $n_2 = 1$ or $n_2 = 3$ and either $n_7 = 1$ or $n_7 = 8$. Suppose that $n_2 \neq 1$ and $n_7 \neq 1$. Then there are $8$ subgroups of $G$ of order $7$ and since distinct Sylow $p$-subgroups intersect only at the identity, there are $8 \cdot 6 = 48$ elements of $G$ with order $7$. Similarly, there $3$ subgroups of order $8$ which means at least $7 + 1 = 8$ distinct elements of order $2$, $4$ or $8$. But $48 + 8 + 1 = 57 > 56$ which is a contradiction. Therefore either $n_2 = 1$ or $n_7 = 1$ which implies the Sylow $p$-subgroup corresponding to this prime is normal in $G$.
\end{proof}

\begin{problem}[4.5.14]
Prove that a group of order $312$ has a normal Sylow $p$-subgroup for some prime $p$ dividing its order.
\end{problem}
\begin{proof}
Note that $312 = 2^3 \cdot 3 \cdot 13$. But from the Sylow divisibility rules $n_{13} = 1 + 13k$ for some $k$ and $n_{13} \mid 24$. This forces $k = 0$ so $n_{13} = 1$ which directly implies $G$ has a Sylow $13$-subgroup which is normal in $G$.
\end{proof}

\begin{problem}[4.5.16]
\label{pqr}
Let $|G| = pqr$, where $p$, $q$ and $r$ are primes with $p < q < r$. Prove that $G$ has a normal Sylow subgroup for either $p$, $q$ or $r$.
\end{problem}
\begin{proof}
First, consider all the elements of order $p$. There are at least $n_p$ of these, and for each Sylow $p$-subgroup $P$ there are $(p-1)$ automorphisms of $P$, that is $(p-1)$ elements of $P$ with order $p$. Therefore, there are $n_p(p-1)$ elements in $G$ with order $p$. The same can be said for $q$ and $r$ as well. Since the two sets of elements of order $q$ and order $r$ are disjoint except for the identity, we have
\[
n_q(q-1) + n_r(r-1) \leq pqr -1.
\]
Now we can use Sylow divisibility conditions on $n_q$ and $n_r$. Since $n_q \mid pr$, we have one of $n_q = 1$, $n_q = p$, $n_q = r$ or $n_q = pr$. We also know that $n_q \equiv 1 \pmod{q}$ so if $n_q \neq 1$, $n_q > q$ and we must have $n_q \geq r$. Likewise, either $n_r = 1$, $n_r = p$, $n_r = q$ or $n_r = pq$. Using the fact that $n_r \equiv 1 \pmod{r}$, if $n_r \neq 1$ then $n_r > r$ and so $n_r = pq$. Now, assume to the contrary that $n_q \neq 1$ and $n_r \neq 1$. Then we use the fact that $p$, $q$ and $r$ are primes and $q$ is between $p$ and $r$. We have
\begin{align*}
n_q(q-1) + n_r(r-1)
&\geq r(q-1) + pq(r-1)\\
&= r(q-1) + pqr - pq\\
&= pqr - pq + p(q-1) + (r-p)(q-1)\\
&\geq pqr - pq + p(q-1) + 2(q-1)\\
&> pqr - pq + p(q-1) + p\\
&= pqr - pq + pq\\
&= pqr.
\end{align*}
This contradicts our original statement about $n_q$ and $n_r$ and thus $n_q = 1$ or $n_r = 1$. Therefore either a Sylow $q$-subgroup or Sylow $r$-subgroup is one of unique order and is therefore normal in $G$.
\end{proof}

\begin{problem}[4.5.22]
Prove that if $|G| = 132$ then $G$ is not simple.
\end{problem}
\begin{proof}
Note that $132 = 2^2 \cdot 3 \cdot 11$. Using the Sylow divisibility rules we know $n_{11} = 1$ or $n_{11} = 12$, $n_3 = 1$, $n_3 = 4$ or $n_3 = 22$ and $n_2 = 1$, $n_2 = 3$, $n_2 = 11$ or $n_2 = 33$. Suppose that $n_{11} \neq 1$. Then $n_{11} = 12$ and since distinct Sylow $p$-subgroups intersect only in the identity, there are $12 \cdot 10 = 120$ elements of $G$ with order $11$. Now suppose that $n_3 \neq 1$. If $n_3 = 22$ then there are $22 \cdot 2 = 44$ elements of order $3$. This is a contradiction since $120 + 44 = 164 > 132$. Therefore $n_3 = 4$ which adds $4 \cdot 2 = 8$ elements of order $3$. Now suppose that $n_2 \neq 1$ and that $n_2 = 3$. Then there are $3$ subgroups of order $4$ which adds $3 + 1 = 4$ elements of order $2$ or order $4$. But now $G$ has at least $120 + 8 + 4 + 1 = 133 > 132$ distinct elements which is a contradiction. A similar contradiction arises if $n_2 = 11$ or $n_2 = 33$. Therefore at least one of $n_{11}$, $n_3$ or $n_2$ must be $1$ which implies the existence of a normal Sylow $p$-subgroup of $G$. Thus $G$ is not simple.
\end{proof}

\begin{problem}
If $G$ is a nonabelian simple group of order $< 100$, prove that $G \cong A_5$.
\end{problem}
\begin{proof}
We proceed by eliminating all possible orders but $60$. First eliminate all the prime and prime squared orders. Now eliminate orders of the form $pq$ for primes $p$ and $q$ with $p < q$. Eliminate orders of the form $p^2q$ for $p \neq q$. Also eliminate the order $30$ and $56$ by Problem~\ref{order56}. This leaves the following possible orders:
\[
\text{$8$, $16$, $24$, $27$, $32$, $36$, $40$, $42$, $48$, $54$, $60$, $64$, $66$, $70$, $72$, $78$, $80$, $81$, $84$, $88$, $90$, $96$.}
\]
Now consider orders of the form $p^{\alpha}m$ where $m < p$. Since $n_p = 1 + kp$, but $n_p \mid m$, the only possibility for $k$ is $k = 0$. Therefore $n_p = 1$ and these groups are not simple. Using this and Problem~\ref{pqr} we're left with the following possibilities:
\[
\text{$24$, $36$, $40$, $48$, $60$, $72$, $80$, $84$, $90$, $96$.}
\]
In a group of order $40$ we find that $n_5 = 1$ and in a group of $84$ we find that $n_7 = 1$ by Sylow divisibility conditions. Also in a group of order $80$ we have $n_5 = 1$ or $n_5 = 16$ and $n_2 = 1$ or $n_2 = 5$. If this group is to be simple, we need $n_5 = 16$, which gives $64$ elements of order $5$, and $n_2 = 5$, which gives $16 + 1 = 17$ elements of order other than $5$. But this is now $81$ distinct elements, a contradiction. This leaves us with the following possible orders:
\[
\text{$24$, $36$, $48$, $60$, $72$, $90$, $96$.}
\]
In a group $G$ of order $24$, we find $n_2 = 1$ or $3$. Assuming that $G$ is simple, $n_2 = 3$. Now let $G$ act on the Sylow $2$-subgroups of $G$ by conjugation. Then this defines a homomorphism $\varphi : G \to S_3$. But since $|G| = 24 > 3! = |S_3|$ we see that $\varphi$ has a nontrivial kernel and this kernel gives a nontrivial normal subgroup of $G$. Therefore $G$ is not simple. The same argument holds for a group of order $48$ or $96$ since in those cases $n_2 = 1$ or $n_2 = 3$ as well. In the case of orders $36$ or $72$ we find that $n_3 = 1$ or $n_3 = 4$ and a similar argument holds since $|S_4| = 24 < 36 < 72$.

This only leaves the possible orders as $60$ or $90$. Suppose $|G| = 90$. Then $n_5 = 1$ or $n_5 = 6$ and if $G$ is to be simple, we need $n_5 = 6$. This gives $24$ elements of order $5$. Furthermore, $n_3 = 1$ or $n_3 = 10$. Suppose $P,Q \in Syl_3(G)$ such that $P \cap Q = R$ with $|R| = 3$. Note here that we're assuming $P \neq Q$, but that $P$ and $Q$ are not disjoint. Lagrange's Theorem ensures $|R| = 3$. Since $|P| = 9 = 3^2$, $P$ is abelian and so $R \normal P$. Thus, $P \leq N = N_G(R)$ and the same is true for $Q$. Thus we know that $|N| = 18$, $|N| = 45$ or $|N| = 90$. In the third case, we must have $R \normal G$ and in the second case $|G : N| = 2$ so $N \normal G$. In the final case $|G : N| = 5$ and since $90 \nmid 5!$, $G$ cannot be simple. Thus we must have $P \cap Q = 1$ for all Sylow $3$-subgroups. This gives $8 \cdot 10 = 80$ elements of order $3$ or $9$. But then there are $24 + 80$ nonidentity elements of $G$ which is a contradiction. This shows that $G$ is not simple which leaves the only simple nonabelian order less than $100$ as $60$.
\end{proof}

\begin{problem}[4.5.30]
How many elements of order $7$ must there be in a simple group of order $168$.
\end{problem}
\begin{proof}
Let $G$ be such a group. Note that $168 = 2^3 \cdot 3 \cdot 7$. The Sylow divisibility rules dictate that $n_7 = 1$ or $n_7 = 8$. But since $G$ is simple, $n_7 \neq 1$ and so there are $8$ subgroups or order $7$. Each of pair of these intersects only in the identity and so there are $8 \cdot 6 = 48$ elements of order $7$.
\end{proof}

\begin{problem}[4.5.31]
For $p = 2$, $3$ and $5$ find $n_p(A_5)$ and $n_p(S_5)$.
\end{problem}
\begin{proof}
From the Sylow divisibility rules on $A_5$ we know $n_5 = 1$ or $n_5 = 6$, $n_3 = 1$, $n_3 = 4$ or $n_3 = 10$ and $n_2 = 1$, $n_2 = 3$, $n_2 = 5$ or $n_2 = 15$. Note that since $A_5$ is simple, we can conclude $n_5 = 6$. Also, since $\binom{5}{3} = 10$, there are at least $10$ distinct $3$-cycles in $A_5$ each which generate a subgroup of order $3$. Therefore $n_3 = 10$. Finally, there are $\binom{5}{4} = 5$ copies of $V_4$ formed by taking double transpositions on four of the five elements. So $n_5 \geq 5$. But note that each of the elements in these copies of $V_4$ is distinct since they're formed by taking one element and replacing it with a different element of $\{1, 2, 3, 4, 5\}$. But now we have $4 \cdot 6 + 2 \cdot 10 + 3 \cdot 5 + 1 = 60$ distinct elements. Thus $n_2 = 5$.

From the Sylow divisibility rules on $S_5$ we know $n_5 = 1$ or $n_5 = 6$, $n_3 = 1$, $n_3 = 4$, $n_3 = 10$ or $n_3 = 40$ and $n_2 = 1$, $n_2 = 3$, $n_2 = 5$ or $n_2 = 15$. Since $A_5 \leq S_5$ we know $n_5 = 6$, $n_3 \geq 5$ and $n_2 \geq 5$. Now note that in addition to the Klein $4$-groups in $A_5$ we also have at least one group of order $4$ generated by a four cycle in $S_5$. Therefore $n_2 = 15$. Finally, note that a group of order $3$ must be cyclic, and thus generated by an element of order $3$. Since this is necessarily a $3$ cycle, all $10$ subgroups of order $3$ are in $A_5$ and so $n_3 = 10$.
\end{proof}

\begin{problem}[4.5.32]
Let $P$ be a Sylow $p$-subgroup of $H$ and let $H$ be a subgroup of $K$. If $P \normal H$ and $H \normal K$, prove that $P$ is normal in $K$. Deduce that if $P \in Syl_p(G)$ and $H = N_G(P)$, then $N_G(H) = H$ (in words: \emph{normalizers of Sylow $p$-subgroups are self-normalizing}).
\end{problem}
\begin{proof}
Note that since $P \normal H$ we also know $P$ char $H$. Then by Problem~\ref{normaltransitive} we know $P \normal K$. If $P \in Syl_p(G)$ and $H = N_G(P)$ then $P \normal H$ and $H \normal N_G(H)$. Therefore $P \normal N_G(H)$. But $H$ is the largest subgroup of $G$ which contains $P$ as a normal subgroup and therefore $N_G(H) = H$.
\end{proof}

\begin{problem}[4.5.33]
Let $P$ be a normal Sylow $p$-subgroup of $G$ and let $H$ be any subgroup of $G$. Prove that $P \cap H$ is the unique Sylow $p$-subgroup of $H$.
\end{problem}
\begin{proof}
Assume that $P \cap H$ is not a Sylow $p$-subgroup. Then some subgroup $K \leq P \cap H$ has order $p^{\alpha}$ where $\alpha$ is maximal for $H$. But then $K \leq gPg^{-} = P$ since $P$ is normal. But then $K \leq H \cap P$, which is a contradiction. To show uniqueness let $h \in H$ and consider $h(P \cap H)h^{-1} = \{hkh^{-1} \mid k \in P \cap H\} = \{hkh^{-1} \mid k \in P\} \cap \{hkh^{-1} \mid k \in H\} = hPh^{-1} \cap hHh^{-1} = P \cap H$ since $P \normal G$. Therefore $P \cap H \normal H$ and so $P \cap H$ is the unique Sylow $p$-subgroup of $H$.
\end{proof}

\begin{problem}[4.5.34]
Let $P \in Syl_p(G)$ and assume $N \normal G$. Use the conjugacy part of Sylow's Theorem to prove that $P \cap N$ is a Sylow $p$-subgroup of $N$. Deduce that $PN/N$ is a Sylow $p$-subgroup of $G/N$ (note that this may also be done by the Second Isomorphism Theorem).
\end{problem}
\begin{proof}
Suppose that $P \cap N$ is not a Sylow $p$-subgroup of $N$. Then there exists some $K \leq N$ such that $|K| = p^{\alpha}$ and $\alpha$ is maximal for $N$. Then $K$ is a $p$-subgroup so $K \leq gPg^{-1}$ and $g^{-1}Kg \leq P$. But $N$ is normal so $g^{-1}Kg \leq g^{-1}Ng = N$. Thus $g^{-1}Kg \leq P \cap N$ and since conjugation is an automorphism $|g^{-1}Kg| = p^{\alpha}$. But this is a contradiction and so $P \cap N$ is a Sylow $p$ subgroup of $N$.

Let $|G| = p^am$ and $|N| = p^bk$ Use the formula $|PN/N| = |PN|/|N| = |P||N|/(|P \cap N||N|) = |P|/|P \cap N| = p^a/p^b = p^{a-b}$. But also $|G/N| = |G|/|N| = (m/k)p^{a-b}$ and we're done.
\end{proof}

\begin{problem}[4.5.37]
Let $R$ be a normal $p$-subgroup of $G$ (not necessarily a Sylow subgroup).\\
(a) Prove that $R$ is contained in every Sylow $p$-subgroup of $G$.\\
(b) If $S$ is another normal $p$-subgroup of $G$, prove that $RS$ is also a normal $p$-subgroup of $G$.\\
(c) The subgroup $O_p(G)$ is defined to be the group generated by all normal $p$-subgroups of $G$. Prove that $O_p(G)$ is the unique largest normal $p$-subgroup of $G$ and $O_p(G)$ equals the intersection of all Sylow $p$-subgroups of $G$.\\
(d) Let $\overline{G} = G/O_p(G)$. Prove that $O_p(\overline{G}) = \overline{1}$ (i.e., $\overline{G}$ has no nontrivial normal $p$-subgroup).
\end{problem}
\begin{proof}
(a) Let $P \in Syl_p(G)$. We know that since $R$ is a $p$-subgroup of $G$, there exists $g \in G$ such that $R \leq gPg^{-1}$. But this is the same as saying $gRg^{-1} \leq P$. Since $R \normal G$ we know $R \leq P$ for each $P \in Syl_p(G)$.

(b) Let $g \in G$. Then $gRSg^{-1} = \{gqg^{-1} \mid q \in RS\} = \{grsg^{-1} \mid r \in R, s \in S\} = \{grg^{-1}gsg^{-1} \mid r \in R, s \in S\} = gRg^{-1}gSg^{-1} = RS$ since $R$ and $S$ are normal in $G$. To see that $RS$ is a $p$-subgroup let $r \in R$ and $s \in S$. Then $|r| = p^a$ and $|s| = p^b$. But then $|rs| = p^{a+b}$ and we're done.
\end{proof}

\begin{problem}[4.5.39]
Show that the subgroup of strictly upper triangular matrices in $GL_n(\mathbb{F}_p)$ is a Sylow $p$-subgroup of this finite group.
\end{problem}
\begin{proof}
We know $|GL_n(\mathbb{F}_p)| = (p^n - 1)(p^n - p) \dots (p^n-p^{n-1})$. Then the smallest power in the expansion will be the product of all the second powers in the binomials, that is $p^{0 + 1 + \dots + n-1} = p^{n(n-1)/2}$. Since this power of $p$ is common to every term in the expansion, but no higher power is, a Sylow $p$-subgroup of $GL_n(\mathbb{F}_p)$ must have order $p^{n(n-1)/2}$. But then note that in any matrix, there are precisely $n(n-1)/2$ elements above the diagonal. Since there are $p$ choices for each element, this gives the result.
\end{proof}

\begin{problem}[4.5.40]
Prove that the number of Sylow $p$-subgroups of $GL_2(\mathbb{F}_p)$ is $p+1$.
\end{problem}
\begin{proof}
Note that $|GL_2(\mathbb{F}_p)| = (p^2-1)(p^2-p) = p(p+1)(p-1)^2$. Therefore $n_p = 1$, $n_p = p+1$ or $n_p > p+1$. But note that
\[
\left \langle \left (
\begin{array}{cc}
1 & 1\\
0 & 1
\end{array}
\right ) \right \rangle
\]
has order $p$ and is thus a Sylow $p$-subgroup of $GL_2(\mathbb{F}_p)$. But in addition
\[
\left \langle \left (
\begin{array}{cc}
1 & 0\\
1 & 1
\end{array}
\right ) \right \rangle
\]
has order $p$ as well. Since these two groups are distinct, we must have $n_p > 1$. Let $A$ be the set of strictly upper triangular matrices (i.e., the first group mentioned above). Then note that any (nonzero) upper triangular matrix will conjugate this group. Since there are $p$ choices for the off diagonal element and $p-1$ nonzero choices for each diagonal element, we see that $|N_{Gl_2}(A)| = p(p-1)^2$. From Sylow's theorem, this directly shows that $n_p \leq p+1$ from which we conclude that $n_p = p+1$.
\end{proof}

\begin{problem}[4.5.44]
Let $p$ be the smallest prime dividing the order of the finite group $G$. If $P \in Syl_p(G)$ and $P$ is cyclic prove that $N_G(P) = C_G(P)$.
\end{problem}
\begin{proof}
Suppose that $|G| = p^{\alpha}m$. We know $N_G(P)/C_G(P)$ is isomorphic to a subgroup of $\aut(P)$. Also note that $|\aut(P)| = \varphi(\alpha) = p^{\alpha-1}(p-1)$. Since $P$ is cyclic, $P \leq C_G(P)$ and thus $|N_G(P)/C_G(P)|$ contains no powers of $P$, as $P$ is a Sylow $p$-subgroup. Therefore $|N_G(P)/C_G(P)| \mid p-1$ and since $p$ is the smallest prime dividing the order of $G$, every divisor of this number is greater than $p-1$. Therefore $N_G(P)/C_G(P) = 1$ and $N_G(P) = C_G(P)$.
\end{proof}

\begin{problem}[4.5.50]
Prove that if $U$ and $W$ are normal subsets of a Sylow $p$-subgroup $P$ of $G$ then $U$ is conjugate to $W$ in $G$ if and only if $U$ is conjugate to $W$ in $N_G(P)$. Deduce that two elements in the center of $P$ are conjugate in $G$ if and only if they are conjugate in $N_G(P)$. (A subset $U$ of $P$ is normal in $P$ if $N_P(U) = P$.)
\end{problem}
\begin{proof}
We see that if $U$ is conjugate to $W$ in $N_G(P)$ then $U$ is certainly conjugate to $W$ in $G$. To prove the other direction let $g \in G$ such that $gUg^{-1} = W$. Note that we may assume $g \notin N_G(P)$ since otherwise, we'd be done. Furthermore, since conjugation is an automorphism, $gPg^{-1} = Q$ for some Sylow $p$-subgroup $Q$. Also note that $N_Q(gUg^{-1}) = N_Q(W) = Q$ since normality is preserved by automorphisms. Consider $N_G(W)$. This is a subgroup of $G$ and we now know, $N_G(W)$ contains both $P$ and $Q$. Furthermore, these are Sylow $p$-subgroups of $N_G(W)$ which means there exists $h \in N_G(W)$ such that $P = hQh^{-1}$. But then $P = (gh)P(gh)^{-1}$ which means $gh \in N_G(P)$. We've chosen $h \in N_G(W)$ so $hWh^{-1} = W$ but then $(gh)U(gh)^{-1} = W$. This concludes the proof.
\end{proof}

\begin{problem}
Prove if $p$ is prime and $G$ is any group of order $2p$, then $G$ must have a subgroup of order $p$, which is normal in $G$.
\end{problem}
\begin{proof}
Since $p \mid |G|$, $G$ has an element of order $p$. But then $|G : \langle p \rangle| = 2$ and thus $\langle p \rangle \normal G$.
\end{proof}

\begin{problem}
Suppose $G$ has order $pq$, where $p$ and $q$ are distinct primes. If $G$ has a normal subgroup of order $p$ and a normal subgroup of order $q$, prove $G$ is cyclic.
\end{problem}
\begin{proof}
Let $P \normal G$ and $Q \normal G$ with $|P| = p$ and $|Q| = q$. Note that $P$ and $Q$ are cyclic subgroups of $G$, so let $P = \langle x \rangle$ and $Q = \langle y \rangle$. Now since $P$ and $Q$ are both normal in $G$, we know $PQ$ is a subgroup and for any two powers of $x$ and $y$ we have $x^iy^j = y^jx^i$. Since all powers of $x$ and $y$ commute, we have $(xy)^{pq} = x^{pq}y^{pq} = 1$. Suppose there exists some integer $a < pq$ such that $(xy)^a = 1$. Then $x^ay^a = 1$. Note that since $(xy)^{pq} = 1$, we must have $a \mid pq$. Therefore $a = p$ or $a = q$. But since $p$ and $q$ are distinct, at least one of $x^a$ or $y^a$ will not be the identity. Therefore $|xy| = pq$ and $G$ is cyclic.
\end{proof}

\begin{problem}
Suppose $p$ and $q$ are distinct primes and $q$ divides $(p � 1)$. Show there exists a non-abelian group of order $pq$ and any two such non-abelian groups are isomorphic.
\end{problem}
\begin{proof}
Consider $Z_p = \langle x \rangle$ and $Z_q = \langle y \rangle$. We wish to define a relation $yxy^{-1} = x^c$ which will force our group to have order $21$. To do this, note that if we apply the conjugation $yxy^{-1} = x^c$ $q$ times we arrive at $y^qxy^{-q} = x^{c^q}$. Since $|y| = q$, we want to find $c$ such that $c^q \equiv 1 \pmod{p}$. We know that $qk = p-1$ for some $k$ and $\aut (Z_p) \cong (\mathbb{Z}/p\mathbb{Z})^{\times}$. This shows the automorphism of $Z_p$ which takes a generator $x$ to $x^k$ will have order $q$. So now define $yxy^{-1} = x^k$ for a generator $y$ of $Z_q$. Let $G = \langle x, y \mid x^p = y^q = 1, yxy^{-1} = x^k \rangle$. This group has order $pq$ by construction.

Furthermore, we know that we must have the rule $yxy^{-1} = x^c$ for some $c$. There may be more than one choice for $c$. Namely, any $c$ for which $p \mid c^q - 1$ will work. But for any two of these groups the order of these automorphisms in $Z_p$ is always $q$. We can use the fact that $\aut(Z_p)$ is cyclic to conclude that the two groups must be isomorphic.
\end{proof}

\end{document}