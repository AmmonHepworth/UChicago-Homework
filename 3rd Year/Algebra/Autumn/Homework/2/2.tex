\documentclass{article}
\usepackage{amsmath,amsthm,amssymb,amsfonts,fullpage,fancyhdr}

\pagestyle{fancy}
\renewcommand{\headheight}{50pt}
\renewcommand{\footskip}{40pt}
\renewcommand{\textheight}{590pt}
\renewcommand{\headrulewidth}{0pt}

\newtheorem{problem}{Problem}

\begin{document}

\rhead{Kris Harper\\MATH 25700\\October 9, 2009\\}
\chead{Homework 2\\}

\begin{problem}[0.3.6]
Prove that the squares of the elements in $\mathbb{Z}/4\mathbb{Z}$ are just $\overline{0}$ and $\overline{1}$.
\end{problem}
\begin{proof}
We have
\[
\overline{0}^2 = \overline{0}\overline{0} = \overline{0 \cdot 0} = \overline{0}
\]
\[
\overline{1}^2 = \overline{1}\overline{1} = \overline{1 \cdot 1} = \overline{1}
\]
\[
\overline{2}^2 = \overline{2}\overline{2} = \overline{2 \cdot 2} = \overline{4} = \overline{0}
\]
\[
\overline{3}^2 = \overline{3}\overline{3} = \overline{3 \cdot 3} = \overline{9} = \overline{1}.
\]
\end{proof}

\begin{problem}[0.3.12]
Let $n \in \mathbb{Z}$, $n > 1$, and let $a \in \mathbb{Z}$ with $1 \leq a \leq n$. Prove if $a$ and $n$ are not relatively prime, there exists an integer $b$ with $1 \leq b < n$ such that $ab \equiv 0 \pmod{n}$ and deduce that there cannot be an integer $c$ such that $ac \equiv 1 \pmod{n}$.
\end{problem}
\begin{proof}
Since $(a,n) \neq 1$ we know there exists $d \neq 1$ such that $d \mid a$ and $d \mid n$. Thus there exist positive integers $c$ and $b$ such that $a = cd$ and $n = bd$. But then $cn = ab$. Since $1 < d$ we know that $b < n$ Thus $n \mid ab - 0$ so $ab \equiv 0 \pmod{n}$. Suppose there exists some other integer $c$ such that $ac \equiv 1 \pmod{n}$. Then since $1 \leq b < n$ we have $abc \equiv b \pmod {n}$. But this is impossible since we already know $ab \equiv 0 \pmod{n}$. Thus there exists no such integer $c$.
\end{proof}

\begin{problem}[0.3.13]
Let $n \in \mathbb{Z}$, $n > 1$, and let $a \in \mathbb{Z}$ with $1 \leq a \leq n$. Prove if $a$ and $n$ are relatively prime, there exists an integer $c$ such that $ac \equiv n \pmod{n}$.
\end{problem}
\begin{proof}
From the Euclidean algorithm we know we can express $(a,n) = 1 = ac + nm$ for integers $x$ and $y$. Thus $n(-m) = ac - 1$ and $n \mid (ac - 1)$. Therefore $ac \equiv 1 \pmod{n}$.
\end{proof}

\begin{problem}[0.3.14]
Conclude from the previous two exercises that $(\mathbb{Z}/n\mathbb{Z})^{\times}$ is the set of elements $\overline{a}$ of $\mathbb{Z}/n\mathbb{Z}$ with $(a,n) = 1$ and hence prove Proposition 4. Verify this directly in the case $n = 12$.
\end{problem}
\begin{proof}
By definition
\[
(\mathbb{Z}/n\mathbb{Z})^{\times} = \{\overline{a} \in \mathbb{Z}/n\mathbb{Z} \mid \text{there exists } \overline{c} \in \mathbb{Z}/n\mathbb{Z} \text{ with } \overline{a}\overline{c} = \overline{1}\}.
\]
Let $\overline{a} \in \mathbb{Z}/n\mathbb{Z}$ such that $(a,n) = 1$. Then we know from Problem 2 that there exists $c \in C$ such that $ac \equiv 1 \pmod{n}$ and so $n \mid (ac - 1)$. Therefore $\overline{1} = \overline{ac} = \overline{a}\overline{c}$ and thus $\overline{a} \in (\mathbb{Z}/n\mathbb{Z})^{\times}$. On the other hand, let $\overline{b} \in (\mathbb{Z}/n\mathbb{Z})^{\times}$. Then there exists $\overline{c} \in \mathbb{Z}/n\mathbb{Z}$ such that $\overline{b}\overline{c} = \overline{1}$. But then $\overline{bc} = \overline{1}$ so $n \mid (bc - 1)$. Thus $bc \equiv 1 \pmod{n}$. But from Problem 1 we conclude that $(b,n) = 1$. Thus, the two sets are equal as both inclusions have been shown.
\end{proof}

\begin{problem}[0.3.15]
For each of the following pairs of integers $a$ and $n$, show that $a$ is relatively prime to $n$ and determine the multiplicative inverse of $\overline{a}$ in $\mathbb{Z}/n\mathbb{Z}$.\\
(a) $a = 13$, $n = 20$.
\end{problem}
\begin{proof}
Using the Euclidean algorithm:
\[
20 = 13 + 7\\
13 = 7 + 6\\
7 = 6 + 1\\
6 = 6
\]
So $(a,n) = 1$. Working backwards we have
\[
1 = (2)20 + (-3)13.
\]
Thus $\overline{13}^{-1} = \overline{-3} = \overline{17}$.
\end{proof}

\begin{problem}[1.1.1]
Determine which of the following operations are associative:\\
(a) The operation $\star$ on $\mathbb{Z}$ defined by $a \star b = a - b$.\\
(d) The operation $\star$ on $\mathbb{Z} \times \mathbb{Z}$ defined by $(a,b) \star (c,d) = (ad + bc, bd)$.\\
(e) The operation $\star$ on $\mathbb{Q} \backslash \{0\}$ defined by $a \star b = \frac{a}{b}$.
\end{problem}
\begin{proof}
(a) Here $\star$ is not associative. As a counterexample take $3, 4, 5 \in \mathbb{Z}$. Then $3 - (4 - 5) = 3 - (-1) = 4$ but $(3-4) - 5 = (-1) - 5 = -6$.

(d) This simply addition in $\mathbb{Q}$. Take $(a_1, b_1), (a_2,b_2), (a_3,b_3) \in \mathbb{Z} \times \mathbb{Z}$. We have
\begin{align*}
(a_1,b_1) \star ((a_2,b_2) \star (a_3,b_3))
&= (a_1,b_1) \star (a_2b_3 + b_2a_3,b_2b_3) \\
&= (a_1b_2b_3 + b_1(a_2b_3 + b_2a_3), b_1b_2b_3) \\
&= (a_1b_2b_3 + b_1a_2b_3 + b_1b_2a_3, b_1b_2b_3) \\
&= (b_3(a_1b_2 + b_1a_2) + b_1b_2a_3, b_1b_2b_3) \\
&= (a_1b_2 + b_1a_2, b_1b_2) \star (a_3,b_3) \\
&= ((a_1,b_1) \star (a_2,b_2)) \star (a_3,b_3).
\end{align*}

(e) This is not associative. Take $1/2, 1/3, 1/6 \in \mathbb{Q}$. Then
\[
\left ( \frac{1}{2} \star \frac{1}{3} \right ) \star \frac{1}{6} = \frac{3}{2} \star \frac{1}{6} = 9.
\]
But
\[
\frac{1}{2} \star \left ( \frac{1}{3} \star \frac{1}{6} \right ) = \frac{1}{2} \star 2 = \frac{1}{4}.
\]
\end{proof}

\begin{problem}[1.1.2]
Determine which of the binary operations in the preceding exercise are commutative.
\end{problem}
\begin{proof}
(a) This is not commutative. Take $1, 0 \in \mathbb{Z}$. Then $1 \star 0 = 1 - 0 = 1$ but $0 \star 1 = 0 - 1 = -1$.

(d) This is commutative as addition in $\mathbb{Q}$ is commutative. Let $(a_1, b_1), (a_2, b_2) \in \mathbb{Z} \times \mathbb{Z}$. Then
\[
(a_1, b_1) \star (a_2, b_2) = (a_1b_2 + b_1a_2, b_1b_2) = (a_2b_1 + b_2a_1, b_2b_1) = (a_2, b_2) \star (a_1, b_1).
\]

(e) This is not commutative. Take $1$ and $2$. Then $1 \star 2 = 1/2$ but $2 \star 1 = 2/1$.
\end{proof}

\begin{problem}[1.1.6]
Determine which of the following sets are groups under addition:\\
(a) The set of rational numbers (including $0 = 0/1$) in lowest terms whose denominators are odd.\\
(b) The set of rational numbers in lowest terms whose denominators are even together with $0$.\\
(c) The est of rational numbers of absolute value $< 1$.\\
(d) The set of rational numbers of absolute value $\geq 1$ together with $0$.\\
(e) The set of rational numbers with denominators equal to $1$ or $2$.\\
(f) The set of rational numbers with denominators equal to $1$, $2$ or $3$.
\end{problem}
\begin{proof}
(a) This is a group under addition. Since $\mathbb{Q}$ is already a group under addition, we just need to show that this set is a subgroup. Take $a/b$ and $c/d$ with $b$ and $d$ odd. Then $a/b + c/d = (ad + bc)/bd$. Since $b$ and $d$ are both odd, $bd$ is odd. If $ad + bc$ happens to contain a factor of $b$ or $d$, then reducing the fraction still results in a fraction with odd denominator since odd numbers have no even factors. We know $0 = 0/1$ is in the group. And note that inverses are in the group as well since $(a/b)^{-1} = -a/b$ which has odd denominator.

(b) This is not closed under addition. Consider $1/6 + 1/6 = 2/6 = 1/3$.

(c) This is not closed under addition. Consider $2/3 + 2/3 = 4/3$ and $|4/3| \geq 1$.

(d) This is not closed under addition. Consider $1 + -1/2 = 1/2$ and $|1/2| < 1$.

(e) This is a group under addition. Again, we must show that this is a subgroup. Take $a/b$ and $c/d$ with $b$ and $d$ equal to $1$ or $2$. Then $a/b + c/d = (ad + bc)/bd$. In the case that $b=1$ or $d=1$ we see that the sum of $a/b$ and $c/d$ still has denominator $1$ or $2$. In the case that $b = d = 2$, note that we can undistribute $2$ from the numerator, resulting in a denominator of either $1$ or $2$. Of course $0 = 0/1$ is in the group and inverses are as well since $(a/b)^{-1} = -a/b$.

(f) This is not closed under addition. Consider $1/2 + 1/3 = 5/6$.
\end{proof}

\begin{problem}[1.1.9]
Let $G = \{a + b \sqrt{2} \in \mathbb{R} \mid a, b \in \mathbb{Q}\}$.\\
(a) Prove that $G$ is a group under addition.\\
(b) Prove that the nonzero elements of $G$ are a group under multiplication.
\end{problem}
\begin{proof}
(a) Let $a + b \sqrt{2}$ and $c + d \sqrt{2}$ be two elements of $G$. Then using commutativity, associativity and distributivity of the reals we have
\[
(a + b \sqrt{2}) + (c + d \sqrt{2}) = (a + c) + (b + d)\sqrt{2}.
\]
Since $\mathbb{Q}$ is closed under addition this expression is also in $G$. Thus $G$ is closed under addition. We use $0$ for the identity of $G$. To find $(a + b \sqrt{2})^{-1}$ take $(-a - b \sqrt{2})$ so that
\[
(a + b \sqrt{2}) + (-a - b \sqrt{2}) = (a - a) + (b - b)\sqrt{2} = 0 + 0 = 0.
\]
Associativity follows from the associativity of the reals.

(b)  Let $a + b \sqrt{2}$ and $c + d \sqrt{2}$ be two elements of $G$. Then using commutativity, associativity and distributivity of the reals we have
\[
(a + b \sqrt{2})(c + d \sqrt{2}) = (ac + 2bd) + (bc + ad)\sqrt{2}.
\]
Thus $G$ is closed under multiplication. We use $1$ for the identity element of $G$. To find $(a + b \sqrt{2})^{-1}$ take
\[
(a + b \sqrt{2})^{-1} = \frac{-a}{2b^2 - a^2} + \frac{b}{2b^2 - a^2} \sqrt{2}.
\]
It's clear that when these are multiplied as above we obtain $1 + 0 \sqrt{2} = 1$. Once again, associativity follows from that of the reals.
\end{proof}

\begin{problem}[1.1.10]
Prove that a finite group is abelian if and only if it's group table is a symmetric matrix.
\end{problem}
\begin{proof}
Let $G$ be a group and let $M$ be its group table. Suppose that $G$ is abelian. The $ij$ element of $M$ is $g_ig_j$. Since $G$ is abelian, $g_ig_j = g_jg_i$. But this is the $ji$ element of $M$. Thus $M$ is symmetric. Conversely, suppose $M$ is symmetric. Then $M_{ij} = M_{ji}$ for all $i$ and $j$. But $g_ig_j = M_{ij} = M_{ji} = g_jg_i$. Thus $G$ must be abelian.
\end{proof}

\begin{problem}[1.1.11]
Find the orders of each element of the additive group $\mathbb{Z}/12\mathbb{Z}$.
\end{problem}

$|\overline{0}| = 1$, $|\overline{1}| = 12$, $|\overline{2}| = 6$, $|\overline{3}| = 4$, $|\overline{4}| = 3$, $|\overline{5}| = 12$, $|\overline{6}| = 2$, $|\overline{7}| = 12$, $|\overline{8}| = 3$, $|\overline{9}| = 4$, $|\overline{10}| = 5$, $|\overline{11}| = 12$.


\begin{problem}[1.1.13]
Find the orders of the following elements of the additive group $\mathbb{Z}/36\mathbb{Z}$: $\overline{1}$, $\overline{2}$, $\overline{6}$, $\overline{9}$, $\overline{10}$, $\overline{12}$, $\overline{-1}$, $\overline{-10}$, $\overline{-18}$.
\end{problem}

$|\overline{1}| = 36$, $|\overline{2}| = 18$, $|\overline{6}| = 6$, $|\overline{9}| = 4$, $|\overline{10}| = 18$, $|\overline{12}| = 3$, $|\overline{-1}| = 36$, $|\overline{-10}| = 18$, $|\overline{-18}| = 2$.

\begin{problem}[1.1.16]
Let $x$ be an element of $G$. Prove that $x^2 = 1$ if an only if $|x|$ is either $1$ or $2$.
\end{problem}
\begin{proof}
Suppose that $x^2 = 1$. Note that $|x|$ cannot be greater than $2$, because order is defined to be the least positive integer $n$ such that $x^n = 1$. If $|x| = n$ then $n$ can be no larger than $2$. But also, if $|x| = 1$ then $x = 1$ so $x^2 = 1^2 = 1$. Thus $|x|$ is either $1$ or $2$. Conversely, suppose $|x|$ is either $1$ or $2$. We've already seen that if $|x| = 1$ then $x^2 = 1$. If $|x| = 2$ then by definition $x^2 = 1$.
\end{proof}

\begin{problem}[1.1.17]
\label{inversepowers}
Let $x$ be an element of $G$. Prove that if $|x| = n$ for some positive integer $n$ then $x^{-1} = x^{n-1}$.
\end{problem}
\begin{proof}
Since $|x| = n$ we have $x^n = 1$. We can write this as $1 = x \cdot x \cdot \dots \cdot x$ where there are $n$ $x$s being multiplied. Now multiply both sides on the right by $x^{-1}$. We then have $x^{-1} = x \cdot x \cdot \dots \cdot x \cdot x^{-1} = x \cdot x \cdot \dots \cdot x = x^{n-1}$.
\end{proof}

\begin{problem}[1.1.18]
Let $x$ and $y$ be elements of $G$. Prove that $xy = yx$ if and only if $y^{-1}xy = x$ if and only if $x^{-1}y^{-1}xy = 1$.
\end{problem}
\begin{proof}
Suppose $xy = yx$. Multiplying both sides on the left by $y^{-1}$ gives $y^{-1}xy = y^{-1}yx = 1 \cdot x = x$. Now suppose $y^{-1}xy = x$. Multiplying both sides on the left again by $x^{-1}$ gives $x^{-1}y^{-1}xy = x^{-1}x = 1$. Finally assume $x^{-1}y^{-1}xy = x^{-1}x = 1$. Multiply both sides on the left by $yx$ to get $yx = yxx^{-1}y^{-1}xy = yy^{-1}xy = xy$.
\end{proof}

\begin{problem}[1.1.19]
\label{powers}
Let $x \in G$ and let $a, b \in \mathbb{Z}^+$.\\
(a) Prove that $x^{a+b} = x^ax^b$ and $(x^a)^b = x^{ab}$.\\
(b) Prove that $(x^a)^{-1} = x^{-a}$.\\
(c) Establish part (a) for arbitrary integers $a$ and $b$ (positive, negative or zero).
\end{problem}
\begin{proof}
(a) By definition we have $x^{a+b} = x \cdot x \cdot \dots \cdot x$ where there are $a + b$ $x$s being multiplied. By the generalized associative law, this is independent of bracketing, so group the first $a$ $x$s. This is now $x^a \cdot x \cdot x \cdot \dots \cdot x$ where there are now $b$ $x$s being multiplied. But then this is just $x^ax^b$.

For the second part, fix $a \in \mathbb{Z}^+$. Note that for $b = 1$ we have $(x^a)^1 = x^a = x^{a \cdot 1}$. Assume the statement holds for some $b \in \mathbb{Z}^+$. Consider $(x^a)^{b+1}$. From above we know this is $(x^a)^b(x^a)$. Inductively this is $(x^{ab})x^a$ and from above again this is $x^{ab+a} = x^{a(b+1)}$. By induction, the statement holds.

(b) Consider the product $x^a x^{-a}$. Note that for $a = 1$ we have $xx^{-1} = 1$. Thus $(x^1)^{-1} = x^{-1}$. Now suppose the statement holds for some $a \in \mathbb{Z}^+$. Consider $x^{a+1}x^{-(a+1)}$. By definition this is $x \cdot x \cdot \dots \cdot x \cdot x^{-1} \cdot x^{-1} \cdot \dots \cdot x^{-1}$ where there are $a+1$ $x$s and $a+1$ $x^{-1}$s. Ignoring the first and last terms, we see that the remaining product is simply $x^ax^{-a}$ which we inductively know is $1$. Thus, the product simplifies to $xx^{-1} = 1$. Therefore $x^{a+1}x^{-(a+1)} = 1$. The statement is thus proved by induction.

(c) First assume $a$ and $b$ are both negative. Note that $-(a+b)$ is positive. Then
\[
x^{b+a} = \left ( x^{-(a+b)} \right )^{-1} = \left (x^{(-a) + (-b)} \right )^{-1} = \left (x^{-a}x^{-b} \right )^{-1} = \left (x^{-b} \right )^{-1} \left (x^{-a} \right )^{-1} = x^bx^a.
\]
Commuting $a+b$ reverses the final order. To show the product rule we need the fact that $(x^{-1})^a = x^{-a}$ for positive $a$. To see this note that $(x^{-1})^a$ is simply $x^{-1}$ product $a$ times. But this is exactly $x^{-a}$. Now for $a < 0$ and $b < 0$ we have
\[
\left (x^a \right )^b = \left ( \left ( \left ( x^{-a} \right )^{-1} \right )^{-b} \right )^{-1} = \left ( \left (x^{-a} \right )^b \right )^{-1} = (x^{-a})^{-b} = x^{(-a)(-b)} = x^{ab}.
\]
In the case that $a = 0$ we have $x^{0 + b} = x^b = 1 \cdot x^b = x^0x^b$. Also $(x^0)^b = 1^b = 1 = x^0 = x^{0 \cdot b}$. A similar argument holds for $b = 0$. Now suppose that $a < 0$, $b > 0$ and $a+b > 0$. From part (b) we know that $x^ax^{-a} = x^{-a}x^a = 1$. Then we have
\[
x^{b+a} = x^{b+a}(x^{-a}x^a) = x^{b+a-a}x^a = x^bx^a.
\]
On the other hand, if $a+b < 0$ then we have
\[
\left (x^{b+a} \right )^{-1} = x^{-(a+b)} = x^{(-a)+(-b)}(x^{b}x^{-b}) = x^{(-a)+(-b)+b}x^{-b} = x^{-a}x^{-b} = \left (x^a \right )^{-1} \left (x^b \right )^{-1} = \left (x^bx^a \right)^{-1}.
\]
Multiplying by $x^{b+a}$ on the left and $x^bx^a$ on the right gives the result. A similar argument holds for $a > 0$ and $b < 0$.

For the second part of (a) assume first that $a < 0$ and $b > 0$. Then we have
\[
\left (x^a \right )^b = \left ( \left (x^{-a} \right )^{-1} \right )^b = \left ( \left (x^{-1} \right )^{-a} \right )^b = \left (x^{-1} \right )^{-ab} = x^{ab}.
\]
Finally, if $a > 0$ and $b < 0$ then we have
\[
\left (x^a \right )^b = \left ( \left (x^a \right )^{-b} \right )^{-1} = \left (x^{-ab} \right )^{-1} = x^{ab}.
\]
\end{proof}

\begin{problem}[1.1.20]
For $x$ an element in $G$ show that $x$ and $x^{-1}$ have the same order.
\end{problem}
\begin{proof}
Let $|x| = n$. Then $x^n = 1$. Multiply both sides by $x^{-n}$ to obtain $(x^{-1})^n = x^{-n} = x^n x^{-n} = x^{n-n} = 1$. Thus $|x^{-1}| \leq n$. Suppose that $|x{-1}| = m < n$. Then $(x^{-1})^m = 1$. This is $x^{-m} = 1$. Multiplying both sides by $x^m$ gives a contradiction since $n$ is the least positive integer for which $x^n = 1$. This also shows that if $x$ has infinite order but $x^{-1}$ has finite order, we arrive a contradiction. Thus $|x| = |x^{-1}|$.
\end{proof}

\begin{problem}[1.1.22]
If $x$ and $g$ are elements of the group $G$, prove that $|x| = |g^{-1}xg|$. Deduce that $|ab| = |ba|$ for all $a,b \in G$.
\end{problem}
\begin{proof}
Let $|x| = n$. Then $x^n = 1$. Now multiply on the left by $g^{-n}$ and on the right by $g^n$. We have
\[
(g^{-1} x g)^n = (g^{-1})^n x^n g^n = g^{-n} x^n g^n = g^{-n}g^n = g^{n - n} = 1.
\]
Thus $|g^{-1}xg| \leq n$. Suppose that $|g^{-1}xg| = m < n$. Then we have $g^{-m}x^mg^m = 1$. Multiplying by $g^m$ on the left and $g^{-m}$ on the right gives the same contradiction as above. Once again, if $|x|$ is infinite, this shows that $|g^{-1}xg|$ cannot be finite. Therefore $|x| = |g^{-1}xg|$. If $|ab| = n$ then $a^nb^n = 1$ and so $1 = b^{-n}a^{-n} = ((ba)^{-1})^n$. But we know that elements and their inverses have the same order. If $ab$ has infinite order then a similar argument as above will produce a contradiction if $ba$ doesn't have infinite order. Thus $|ab| = |ba|$.
\end{proof}

\begin{problem}[1.1.24]
\label{commute}
If $a$ and $b$ are \emph{commuting} elements of $G$, prove that $(ab)^n = a^nb^n$ for all $n \in \mathbb{Z}$.
\end{problem}
\begin{proof}
We know that $(ab)^0 = 1 = a^0 b^0$. Now suppose that $(ab)^na = a^{n+1} b^n$ for some $n \in \mathbb{Z}_0^+$. Using Problem~\ref{powers} and the commutativity of $a$ and $b$ we have
\[
(ab)^{n+1}a = (ab)^n(ab)a = (a^nb^n)(ab)a = (a^{n+1}b^n)(b)a = a^{n+1}(b^{n+1})a = a^{n+1}b^{n+1}a.
\]
After canceling $a$, the statement then holds for all nonnegative $n$ by induction. Now consider $n \in \mathbb{Z}^+$. We have
\[
(ab)^{-n} = \left ( (ab)^n \right )^{-1} = \left ( (ab)^{-1} \right )^{n} = (b^{-1}a^{-1})^n = (a^{-1}b^{-1})^n = (a^{-1})^n(b^{-1})^n = a^{-n}b^{-n}.
\]
\end{proof}

\begin{problem}[1.1.25]
Prove that if $x^2 = 1$ for all $x \in G$ then $G$ is abelian.
\end{problem}
\begin{proof}
Let $x, y \in G$. We have $x^2y^2 = (xy)^2 = 1$. Multiply on the left by $x^{-1}$ and on the right by $y^{-1}$ to get $xy = x^{-1}y^{-1} = (yx)^{-1}$. But note that since $(yx)^2 = 1$ we can multiply by $(yx)^{-1}$ to get $yx = (yx)^{-1}$. Thus $xy = yx$.
\end{proof}

\begin{problem}[1.1.26]
Assume $H$ is a nonempty subset of $(G, \star)$ which is closed under the binary operation on $G$ and is closed under inverses, i.e., for all $h$ and $k \in H$, $hk$ and $h^{-1} \in H$. Prove that $H$ is a group under the operation $\star$ restricted to $H$.
\end{problem}
\begin{proof}
Take $i,j,k \in H$. Then we have $i \star (j \star k) = (i \star j) \star k$ since these are also elements of $G$ where the associativity law holds. Note that every element in $H$ has an inverse by assumption, and the operation $\star$ is closed by assumption. We need only show that $H$ has an identity element $e$. But $e \in H$ since $h^{-1} \in H$ for all $h \in H$ and we know $hh^{-1} = e$. This $e$ is thus the same identity as for $G$ and so it works as an identity for $\star$ for all of $H$.
\end{proof}

\begin{problem}[1.1.28]
Let $(A, \star)$ and $(B, \diamond)$ be groups and let $A \times B$ be their direct product. Verifty all the group axioms for $A \times B$:\\
(a) Prove that the associative law holds: For all $(a_i, b_i) \in A \times B$, $i = 1,2,3$ $(a_1,b_1) [(a_2,b_2)(a_3,b_3)] = [(a_1,b_1)(a_2,b_2)](a_3,b_3)$.\\
(b) Prove that $(1, 1)$ is the identity of $A \times B$.\\
(c) Prove that the inverse of $(a,b)$ is $(a^{-1}, b^{-1})$.
\end{problem}
\begin{proof}
(a) We have
\begin{align*}
(a_1,b_1) [(a_2,b_2)(a_3,b_3)]
&= (a_1,b_1)(a_2 \star a_3, b_2 \diamond b_3) \\
&= (a_1 \star (a_2 \star a_3), b_1 \diamond (b_2 \diamond b_3)) \\
&= ((a_1 \star a_2) \star a_3, (b_1 \diamond b_2) \diamond b_3) \\
&= (a_1 \star a_2, b_1 \diamond b_2)(a_3, b_3) \\
&= ((a_1, b_1)(a_2,b_2))(a_3, b_3).
\end{align*}

(b) We have $(a,b)(1,1) = (a \star 1, b \diamond 1) = (a, b)$.

(c) We have $(a,b)(a^{-1},b^{-1}) = (a \star a^{-1}, b \diamond b^{-1}) = (1,1)$.
\end{proof}

\begin{problem}[1.1.29]
Prove that $A \times B$ is an abelian group if and only if $A$ and $B$ are abelian.
\end{problem}
\begin{proof}
Suppose that $A \times B$ abelian. Then
\[
(a_1 \star a_2, b_1 \diamond b_2) = (a_1,b_1)(a_2,b_2) = (a_2,b_2)(a_1,b_1) = (a_2 \star a_1, b_2, \diamond b_1).
\]
Equating components we have $a_1 \star a_2 = a_2 \star a_1$ and $b_1 \diamond b_2 = b_2 \diamond b_1$. Conversely, to show $A$ and $B$ abelian implies $A \times B$ is abelian, we reverse the equalities. That is
\[
(a_1,b_1)(a_2,b_2) = (a_1 \star a_2, b_1 \diamond b_2) = (a_2 \star a_1, b_2, \diamond b_1) = (a_2,b_2)(a_1,b_1).
\]
This shows $A \times B$ is abelian.
\end{proof}

\begin{problem}[1.1.30]
Prove that the elements $(a,1)$ and $(1,b)$ of $A \times B$ commute and deduce that the order of $(a,b)$ is the least common multiple of $|a|$ and $|b|$.
\end{problem}
\begin{proof}
We have $(a,1)(1,b) = (a \star 1, 1 \diamond b) = (1 \star a, b \diamond 1) = (1,b)(a,1)$. Note that $(a,1)(1,b) = (a,b)$. From Problem~\ref{commute} we know $(a,b)^n = ((a,1)(1,b))^n = (a,1)^n(1,b)^n$. Thus $|(a,b)$ must be the least common multiple of $(a,1)$ and $(1,b)$ as this is the smallest positive integer $n$ for which both $(a,1)^n = 1$ and $(1,b)^n = 1$.
\end{proof}

\begin{problem}[1.2.4]
If $n = 2k$ is even and $n \geq 4$ show that $z = r^k$ is an element of order $2$ which commutes with all elements of $D_{2n}$. Show also that $z$ is the only nonidentity element of $D_{2n}$ which commutes with all elements of $D_{2n}$.
\end{problem}
\begin{proof}
To show that $|z| = 2$ we note that $r^i \neq r^j$ for all $i \neq j$ provided $0 \leq i < n$ and $0 \leq j < n$. Also, we know $r^0 = 1$ so $|r^k| \neq 1$. But $(r^k)^2 = r^2k = r^n = 1$ from Problem~\ref{powers} and so $|z| = 2$. Note this also implies that $z^{-1} = z$ since $z^2 = 1$ and so $z^{-1}z^2 = z = 1 \cdot z^{-1} = z^{-1}$. To see that $z$ commutes with every element of $D_{2n}$ note that $zr^{i} = r^kr^i = r^{k+i} = r^{i+k} = r^ir^k = r^iz$. Likewise, $zs = r^ks = sr^{-k} = sr^k = sz$. All powers of $s$ are either $s$ or $1$ so, we've shown that $z$ commutes with all powers of $r$ and of $s$. Now we take $zr^is = zr^i(z^{-1}z)s = (zr^i)z^{-1}(zs) = (r^iz)z^{-1}(sz) = r^i(zz^{-1}sz = r^isz$. Therefore $z$ commutes with all elements of $D_{2n}$. Now take $r^i$ such that $r^i \neq r^k$ and $r^i \neq 1$. Then $r^is = sr^{-i}$. For $r^i$ to be commutative, we need $r^i = r^{-i}$, which would also imply $r^{2i} = (r^i)^2 = 1$. But we know this isn't the case since $i \nmid n$ by assumption. This shows that $s$ and any power of $r$ besides powers of $z$ are not commutative. Finally, we consider $sr^i$. This element doesn't commute with $r$ since $sr^ir = sr^{i+1} = r^{-i-1}s \neq rsr^i$.
\end{proof}

\begin{problem}[1.2.8]
Find the order of the cyclic subgroup of $D_{2n}$ generated by $r$.
\end{problem}
\begin{proof}
We know that $|r| = n$. Furthermore, we know that $1, r, r^2, \dots, r^{n-1}$ are all distinct. Finally, note that $(r^i)^{-1} = r^{n-i}$ as shown in Problem~\ref{inversepowers}. These facts together show that $|\langle r \rangle | = n$.

\begin{problem}[1.3.4]
\label{symmetricorders}
Compute the order of each of the elements in the following groups:\\
(a) $S_3$.\\
(b) $S_4$.
\end{problem}

(a) $|(1)| = 1$, $|(12)| = 2$, $|(23)| = 2$, $|(13)| = 2$, $|(123)| = 3$, $|(132)| = 3$.

(b) $|(1)| = 1$, $|(12)| = 2$, $|(13)| = 2$, $|(14)| = 2$, $|(23)| = 2$, $|(24)| = 2$, $|(34)| = 2$, $|(123)| = 3$, $|(132)| = 3$, $|(124)| = 3$, $|(142)| = 3$, $|(134)| = 3$, $|(143)| = 3$, $|(234)| = 3$, $|(243)| = 3$, $|(12)(34)| = 2$, $|(13)(24)| = 2$, $|(14)(23)| = 2$, $|(1234)| = 2$, $|(1243)| = 2$, $|(1324)| = 4$, $|(1342)| = 4$, $|(1423)| = 4$, $|(1432)| = 4$.

\begin{problem}[1.3.10]
\label{cyclepowers}
Prove that if $\sigma$ is the $m$-cycle $(a_1 a_2 \dots a_m)$, then for all $i \in \{1, 2, \dots , m\}$, $\sigma^i(a_k) = a_{k+i}$, where $k+i$ is replaced by its least positive residue mod $m$. Deduce that $|\sigma| = m$.
\end{problem}
\begin{proof}
Note that for $i = 1$ we have $\sigma(a_k) = a_{k+1}$ by definition of a cycle. Suppose the statement is true for some $i \in \mathbb{Z}^+$. Consider $\sigma^{i+1} (a_k) = \sigma^i(\sigma(a_k)) = \sigma^i(a_{k+1}) = a_{k+i+1}$. Thus, the statement is true for all positive $i$, in particular $i \in \{1, 2, \dots , m\}$. Putting in $i = m$ we have $\sigma^m(a_k) = a_{k+m} = a_k$. Since $m$ is the smallest positive integer for which this holds, we see that $|\sigma| = m$.
\end{proof}

\begin{problem}[1.3.11]
Let $\sigma$ be the $m$-cycle $(1 2 \dots m)$. Show that $\sigma^i$ is also an $m$-cycle if and only if $i$ is relatively prime to $m$.
\end{problem}
\begin{proof}
Suppose that $(i,m) \neq 1$. Then there exists $d < m$ such that $i = jd$ and $m = nd$. This implies $jm = in$. This means that $\sigma^i$ will have cycles of length $n$. To see this, note that by Problem~\ref{cyclepowers} the first cycle in $\sigma_i$ will be $(1 \; 1 + i \; 1 + 2i \dots 1 + n(i-1))$. The next cycle will continue in the same fashion. Since $n \mid ni$ we know $1 + ni \equiv 1 \pmod{m}$ and so the cycle must end with $1 + n(i-1)$. But since $n < m$ because $d > 1$, we know this is not an $m$-cycle.

Conversely, suppose that $(i,m) = 1$. Again from Problem~\ref{cyclepowers} we know $\sigma^i$ starts with $(1 \; 1 + i \; 1 + 2i \dots 1 + n(i-1))$ where $n$ is the smallest positive integer such that $1 + ni \equiv 1 \pmod{m}$. Since $(i,m) = 1$ we know that $n = m$ and so this cycle is an $m$-cycle.
\end{proof}

\begin{problem}[1.3.14]
Let $p$ be a prime. Show that an element has order $p$ in $S_n$ if and only if its cycle decomposition is a product of commuting $p$-cycles. Show by an explicit example that this need be the case if $p$ is not prime.
\end{problem}
\begin{proof}
Suppose that $\sigma \in S_n$ has order $p$. Then $\sigma^p = 1$. If we write $\sigma = (a_1 \dots a_{m_1})\dots(a_{m_{k-1}+1} \dots a_{m_k})$ then since all these cycles are disjoint we have
\[
1 = \sigma^p = ((a_1, \dots a_{m_1})\dots(a_{m_{k-1}+1} \dots a_{m_k}))^p = (a_1 \dots a_{m_1})^p\dots(a_{m_{k-1}+1} \dots a_{m_k})^p.
\]
But since each of these terms are disjoint, each one must evaluate to the identity. Since $p$ is prime, $p$ cannot be the product of any of the lengths of the cycles. Thus, each must be a $p$-cycle. Conversely, suppose that $\sigma$ is a product of commuting $p$-cycles. Then $\sigma = (a_1 \dots a_{m_1})\dots(a_{m_{k-1}+1} \dots a_{m_k})$. Raising this to the $p$th power by Problem~\ref{commute} we have
\[
\sigma^p = ((a_1 \dots a_{m_1})\dots(a_{m_{k-1}+1} \dots a_{m_k}))^p = (a_1 \dots a_{m_1})^p\dots(a_{m_{k-1}+1} \dots a_{m_k})^p = 1.
\]
Since each cycle is a $p$-cycle we know no smaller integer would produce the same result. Thus $|\sigma| = p$.

As an example, let $p = 6$ and take the cycle $\sigma = (1 \; 2 \; 3)(4 \; 5)$. We can see $|\sigma| = 6$, but this isn't a decomposition of $6$-cycles.
\end{proof}

\begin{problem}[1.3.15]
Prove that the order of an element in $S_n$ equals the least common multiple of the cycles in its cycle decomposition.
\end{problem}
\begin{proof}
A cycle decomposition of an element in $S_n$ is a product of disjoint cycles. Since the cycles are disjoint, they commute with each other. Let $\sigma = (a_1 \dots a_{m_1})\dots(a_{m_{k-1}+1} \dots a_{m_k})$ be an element of $S_n$. We know that each term in the decomposition has order $m_i$, that is, the order is equal to the number of elements in the cycle. This follows from Problem~\ref{cyclepowers}. Let $n$ be the least common multiple of all the $m_i$. Then by Problem~\ref{commute} we know
\[
\sigma^n = (a_1 \dots a_{m_1})^n\dots(a_{m_{k-1}+1} \dots a_{m_k})^n = ((a_1 \dots a_{m_1})\dots(a_{m_{k-1}+1} \dots a_{m_k}))^n = 1.
\]
Since $n$ is the least integer with this property, we know $|\sigma| = n$.
\end{proof}

\begin{problem}[1.5.1]
Compute the order of each of the elements in $Q_8$.
\end{problem}

$|1| = 1$, $|-1| = 2$, $|i| = 4$, $|-i| = 4$, $|j| = 4$, $|-j| = 4$, $|k| = 4$, $|-k| = 4$.

\begin{problem}[1.6.1]
\label{homopowers}
Let $\phi : G \to H$ be a homomorphism.\\
(a) Prove that $\varphi(x^n) = \varphi(x)^n$ for all $n \in \mathbb{Z}^+$.\\
(b) Do part (a) for $n = -1$ and deduce that $\varphi(x^n) = \varphi(x)^n$ for all $n \in \mathbb{Z}$.
\end{problem}
\begin{proof}
(a) For $n = 1$ we're done. Suppose the statement holds for some $n \in \mathbb{N}$. Then using Problem~\ref{powers} we have
\[
\varphi(x^{n+1}) = \varphi(x^nx) = \varphi(x^n) \varphi(x) = \varphi(x)^n \varphi(x) = \varphi(x)^{n+1}.
\]

(b) We have $\varphi(e) = \varphi(xx^{-1}) = \varphi(x)\varphi(x^{-1})$. Since inverses are preserved and inverses are unique, we must have $\varphi(x^{-1}) = \varphi(x)^{-1}$. Now if $n < 0$ we again use Problem~\ref{powers} to obtain
\[
\varphi(x^n) = \varphi((x^{-n})^{-1}) = \varphi(x^{-n})^{-1} = (\varphi(x)^{-n})^{-1} = \varphi(x)^{-n}.
\]
\end{proof}

\begin{problem}[1.6.3]
If $\varphi : G \to H$ is an isomorphism, prove that $G$ is abelian if and only if $H$ is abelian. If $\varphi : G \to H$ is a homomorphism, what additional conditions on $\phi$ (if any) are sufficient to ensure that if $G$ is abelian, the so is $H$?
\end{problem}
\begin{proof}
Suppose that $G$ is abelian. For $u,v \in H$, since $\varphi$ is surjective, there exists $x,y \in G$ such that $\varphi(x) = u$ and $\varphi(v) = y$. Then
\[
uv = \varphi(x)\varphi(y) = \varphi(xy) = \varphi(yx) = \varphi(y)\varphi(x) = vu.
\]
Conversely, supposing $H$ is abelian, for $x,y \in G$ there exists elements $u,v \in H$ such that $\varphi^{-1}(u) = x$ and $\varphi^{-1}(v) = y$. Then
\[
xy = \varphi^{-1}(u)\varphi^{-1}(v) = \varphi^{-1}(v)\varphi^{-1}(u) = yx.
\]
If $\varphi : G \to H$ is a homomorphism, with $G$ abelian, then provided $\varphi$ is surjective, $H$ will be abelian as well. The reason for this can be seen in the first part of the above proof.
\end{proof}

\begin{problem}[1.6.9]
Prove that $D_{24}$ and $S_4$ are not isomorphic.
\end{problem}
As we saw in Problem~\ref{homopowers}, an isomorphism will preserve powers between elements of groups, which in particular means that if two groups are isomorphic, each group must have the same number of elements of a specific order. We know that in $D_{24}$, $|r| = 24$. But as Problem~\ref{symmetricorders} showed, there is no such element in $S_4$. Thus, the two groups are not isomorphic.
\end{proof}

\begin{problem}[1.6.13]
Let $G$ and $H$ be groups and let $\varphi : G \to H$ be a homomorphism. Prove that the image of $\varphi$, $\varphi(G)$, is a subgroup of $H$. Prove that if $\phi$ is injective then $G \cong \phi(G)$.
\end{problem}
\begin{proof}
We know $\varphi$ preserves the identity of $G$ so $1_H \in \varphi(G)$. Take $x,y \in \varphi(G)$. Then there exists $u,v \in G$ such that $\varphi(u) = x$ and $\varphi(v) = y$. Then $\varphi(uv) = \varphi(u)\varphi(v) = xy$ and so $xy \in \varphi(G)$. Also note that from Problem~\ref{homopowers} $\varphi(u^{-1}) = \varphi(u)^{-1} = x^{-1}$. Thus $\varphi(G)$ is closed under multiplication and inverses and so it's a subgroup of $H$. Assume now that $\phi$ is injective. We must show that $\varphi : G \to \varphi(G)$ is surjective. But this is clearly true since $y \in \varphi(G)$ only if there exists $x \in G$ such that $\varphi(x) = y$. Thus $\varphi : G \to \varphi(G)$ is a bijection and a homomorphism and so $G \cong \varphi(G)$.
\end{proof}

\begin{problem}[1.6.14]
Let $G$ and $H$ be groups and let $\varphi : G \to H$ be a homomorphism. Define the \emph{kernel} of $\varphi$ to be $\{g \in G \mid \varphi(g) = 1_H\}$. Prove that the kernel of $\varphi$ is a subgroup of $G$. Prove that $\varphi$ is injective if and only if the kernel of $\varphi$ is the identity subgroup of $G$.
\end{problem}
\begin{proof}
We know that $e \in \ker \varphi$. Let $x, y \in \ker \varphi$. Then $\varphi(x) = 1_H = \varphi(y)$ and so $\varphi(xy) = \varphi(x)\varphi(y) = 1_H$. Also from Problem~\ref{homopowers} we know $\varphi(x^{-1}) = \varphi(x)^{-1} = 1_H^{-1} = 1_H$. Thus $\ker \varphi$ is closed under multiplication and inverses, so it's a subgroup of $G$.

Assume that $\varphi$ is injective. Then for all $x,y \in \ker \varphi$ with $x \neq y$ we have $\varphi(x) \neq \varphi(y)$. But this limits $|\ker \varphi| = 1$ and since $1_G \in \ker \varphi$, it must be the identity subgroup of $G$. Conversely, suppose that $\ker \varphi$ is the identity subgroup of $G$. Suppose we have $\varphi(x) = \varphi(y)$ for $x,y \in G$. Then $1_H = \varphi(x)(\varphi(y))^{-1} = \varphi(x)\varphi(y^{-1}) = \varphi(xy^{-1})$. Since identities are preserved, $xy^{-1} = 1_G$ and so $x = y$. Therefore $\varphi$ is injective.
\end{proof}

\begin{problem}[1.6.16]
Let $A$ and $B$ be groups and let $G$ be their direct product, $A \times B$. Prove that the maps $\pi_1 : G \to A$ and $\pi_2 : G \to B$ defined by $\pi((a,b)) = a$ and $\pi_2((a,b)) = b$ are homomorphisms and find their kernels.
\end{problem}
\begin{proof}
Let $(a,b),(c,d) \in G$. We have
\[
\pi_1((a,b)(c,d)) = \pi_1((ac,bd)) = ac = \pi_1((a,b))\pi((c,d)).
\]
A similar argument holds for $\pi_2$. Consider an element $(a,b) \in G$ such that $\pi_1(a,b) = 1$. This means $a = 1$. On the other hand, consider $\pi_1(1,b) = 1$. Therefore $\ker \pi_1 = \{(a,b) \mid a = 1\}$ and $\ker \pi_2 = \{(a,b) \mid b = 1\}$ by a similar proof.
\end{proof}

\begin{problem}[1.6.17]
Let $G$ be any group. Prove that the map from $G$ to itself defined by $g \mapsto g^{-1}$ is a homomorphism if and only if $G$ is abelian.
\end{problem}
\begin{proof}
Suppose that $\varphi : G \to G$ such that $\varphi(g) = g^{-1}$ is a homomorphism. Then for $x,y \in G$ we have
\[
yx = (y^{-1})^{-1}(x^{-1})^{-1} = (x^{-1}y^{-1})^{-1} = \varphi(x^{-1}y^{-1}) = \varphi(x^{-1})\varphi(y^{-1}) = xy.
\]
Conversely, suppose that $G$ is abelian. Then we have
\[
\varphi(xy) = (xy)^{-1} = y^{-1}x^{-1} = x^{-1}y^{-1} = \varphi(x)\varphi(y).
\]
\end{proof}

\begin{problem}[1.7.4]
Let $G$ be a group acting on a set $A$ and fix some $a \in A$. Show that the following sets are subgroups of $G$:\\
(a) The kernel of the action.\\
(b) $\{g \in G \mid ga = a\}$ --- this subgroup is called the \emph{stabilizer} of $a$ in $G$.
\end{problem}
\begin{proof}
(a) Let $b \in A$. We see that the identity element $e$ of $G$ is in the kernel of the action because $1.b = b$ by definition. Let $x$ and $y$ be in the kernel. Then $y.b = b$ so $x(y.b) = x.b$. But $x.b = b$ and thus $(xy).b = b$. Therefore $xy$ is in the kernel. Also since $x.b = b$ we can multiply by $x^{-1}$ to obtain $b = 1.b = (x^{-1}x).b = x^{-1}(x.b) = x^{-1}.b$. Thus the kernel is closed under the action and inverses so it's a subgroup of $G$.

(b) This proof is exactly the same as for part (a).
\end{proof}

\begin{problem}[1.7.6]
Prove that a group $G$ acts faithfully on a set $A$ if and only if the kernel of the action is the set consisting only of the identity.
\end{problem}
\begin{proof}
Suppose $G$ acts faithfully on $A$. Take $x,y$ in the kernel such that $x \neq y$. Then we know $x.a \neq y.a$ for all $a \in A$. But this limits the size of the kernel to $1$ and we know the identity is in the kernel. Conversely, suppose the kernel is just the identity. Then take $\varphi(x) = \sigma_x$ and $\varphi(y) = \sigma_y$ in $S_A$ such that $\sigma_x = \sigma_y$. Then
\[
\sigma_1 = \varphi(x)\varphi(y)^{-1} = \varphi(x)\varphi(y^{-1}) = \varphi(xy^{-1}).
\]
Since identities are preserved in homomorphisms we have $xy^{-1} = 1$ and $x = y$. Therefore two permutations which are the same always arise from the same elements of $G$ and the action is faithful.
\end{proof}

\begin{problem}[1.7.13]
Find the kernel of the left regular action.
\end{problem}
\begin{proof}
This is the set of all $x \in G$ such that $xy = y$ for all $y \in G$. Multiplying by $y^{-1}$ on the right gives $x = 1$. Thus the kernel is the identity group.
\end{proof}

\begin{problem}[1.7.14]
Let $G$ be a group and let $A = G$. Show that if $G$ is non-abelian then the maps defined by $g.a = ag$ for all $g,a \in G$ do \emph{not} satisfy the axioms of a (left) group action of $G$ on itself.
\end{problem}
\begin{proof}
Fix $a \in G$ and take $x,y \in G$. Then we have $x.a = ax$, but $(yx).a = y.(x.a) = y.ax = yax$. Since $G$ is non-abelian this violates the group action axioms.
\end{proof}

\begin{problem}[1.7.16]
\label{conj}
Let $G$ be any group and let $A = G$. Show that the maps defined by $g.a = gag^{-1}$ for all $g,a \in G$ \emph{do} satisfy the axioms of a (left) group action.
\end{problem}
\begin{proof}
Fix $a \in G$ and let $x,y \in G$. We have
\[
x.(y.a) = x.(yay^{-1}) = x(yay^{-1})x^{-1} = xyay^{-1}x^{-1} = xya(xy)^{-1} = (xy).a.
\]
Also $1.a = 1a1^{-1} = a$.
\end{proof}

\begin{problem}[1.7.17]
Let $G$ be a group and let $G$ act on itself by left conjugation, so each $g \in G$ maps $G$ to $G$ by
\[
x \mapsto gxg^{-1}.
\]
For fixed $g \in G$, prove that conjugation by $g$ is an isomorphism from $G$ to onto itself. Deduce that $x$ and $gxg^{-1}$ have the same order for all $x \in G$ and that for any subset $A$ of $G$, $|A| = |gAg^{-1}|$.
\end{problem}
\begin{proof}
Problem~\ref{conj} shows that conjugation is an action. This means that $\sigma_g (x) = gxg^{-1}$ is a permutation of $G$. But permutations are isomorphisms. The fact that $|x| = |gxg^{-1}|$ follows from Problem~\ref{homopowers} as powers and identities are preserved through isomorphisms. If $A \subseteq G$ then conjugation maps each $a \in A$ to a distinct element $gag^{-1}$ in $gAg^{-1}$. Thus $|A| = |gAg^{-1}|$.
\end{proof}

\end{document}