\documentclass{article}
\usepackage{amsmath,amsthm,amssymb,amsfonts,fullpage,fancyhdr}

\pagestyle{fancy}
\renewcommand{\headheight}{50pt}
\renewcommand{\footskip}{10pt}
\renewcommand{\textheight}{609pt}
\renewcommand{\headrulewidth}{0pt}

\newcommand{\gal}{\textup{Gal}}

\newtheorem{problem}{Problem}

\begin{document}

\rhead{Kris Harper\\MATH 25900\\May 14, 2010\\}
\chead{Homework 5\\}

\begin{problem}[14.5.3]
Determine the quadratic equation satisfied by the period $\alpha = \zeta_5 + \zeta_5^{-1}$ of the $5^{\textup{th}}$ root of unity $\zeta_5$. Determine the quadratic equation satisfied by $\zeta_5$ over $\mathbb{Q}(\alpha)$ and use this to explicitly solve for the $5^{\textup{th}}$ root of unity.
\end{problem}
\begin{proof}
Let $\zeta = \zeta_5$. Note that $\alpha^2 + \alpha - 1 = (\zeta + \zeta^{-1})^2 + \zeta + \zeta^{-1} - 1 = \zeta^2 + 2 + \zeta^{-2} + \zeta + \zeta^{-1} - 1 = \zeta^4 + \zeta^3 + \zeta^2 + \zeta + 1 = 0$. Now note that $\zeta^2 - \alpha \zeta + 1 = \zeta^2 - \zeta^2 - 1 + 1 = 0$ so $\zeta$ satisfies $x^2 -\alpha x + 1$. Now note that $\alpha = (-1 \pm \sqrt{1 + 4})/2 = (-1 \pm \sqrt{5})/2$. Then
\[
\zeta = \frac{\alpha \pm \sqrt{\alpha^2 - 4}}{2} =  \frac{1}{2} \left ( \frac{-1 \pm \sqrt{5}}{2} \pm \sqrt{ \left (\frac{-1 \pm \sqrt{5}}{2} \right )^2 - 4} \right) = \frac{1}{4} \left (-1 \pm \sqrt{5} + i\sqrt{2(5 \pm \sqrt{5})} \right ).
\]
\end{proof}

\begin{problem}[14.5.4]
Let $\sigma_a \in \gal(\mathbb{Q}(\zeta_n)/\mathbb{Q})$ denote the automorphism of the cyclotomic field of $n^{\textup{th}}$ roots of unity which maps $\zeta_n$ to $\zeta_n^a$ where $a$ is relatively prime to $n$ and $\zeta_n$ is a primitive $n^{\textup{th}}$ root of unity. Show that $\sigma_a(\zeta) = \zeta^a$ for \emph{every} $n^{\textup{th}}$ root of unity.
\end{problem}
\begin{proof}
Let $\zeta$ be an $n^{\textup{th}}$ root of unity. Then we know $\zeta = \zeta_n^b$ for some integer $b$ with $(b,n) = 1$. Now $\sigma_a(\zeta) = \sigma_a(\zeta_n^b) = \sigma(\zeta_n)^b = (\zeta_n^a)^b = (\zeta_n^b)^a = \zeta^a$.
\end{proof}

\begin{problem}[14.5.6]
Let $\zeta_n$ denote a primitive $n^{\textup{th}}$ root of unity and let $K = \mathbb{Q}(\zeta_n)$ be the associated cyclotomic field. Let $a$ denote the trace of $\zeta_n$ from $K$ to $\mathbb{Q}$. Prove that $a = 1$ if $n = 1$, $a = 0$ if $n$ is divisible by the square of a prime, and $a = (-1)^r$ if $n$ is the product of $r$ distinct primes.
\end{problem}
\begin{proof}
Note that $a$ is simply the sum of the primitive $n^{\textup{th}}$ roots of unity. Let $f(n)$ be the sum of the primitive $n^{\textup{th}}$ roots of unity and let $g(n)$ be the sum of the $n^{\textup{th}}$ roots of unity. Note that $g(1) = 1$ and $g(0) = 0$ which can be seen by looking at the roots of unity in the complex plane and pairing them up on opposite sides of the real and imaginary axes. If we group the roots of unity by the divisors of $n$ we see that $g(n) = \sum_{d \mid n} f(d)$. Then M\"{o}bius inversion tells us that $f(n) = \sum_{d \mid n} \mu(d) g(d/n) = \mu(n)$.
\end{proof}

\begin{problem}[14.6.2]
Determine the Galois groups of the following polynomials:\\
(a) $x^3-x^2-4$
(b) $x^3-2x+4$
(c) $x^3-x+1$
(d) $x^3+x^2-2x-1$.
\end{problem}
\begin{proof}
(a) This factors as $(x-2)(x^2+x+2)$ and the quadratic term is irreducible using the quadratic formula. The Galois group is thus of order $2$.

(b) This factors as $(x+2)(x^2-2x+2)$ and the quadratic term is irreducible using the quadratic formula. The Galois group is thus of order $2$.

(c) This is irreducible using the rational root theorem since $\pm 1$ are not roots. The discriminant $0^2(-1)^2 - 4(-1)^3 - 4(0)^3(1) - 27(1)^2 + 18(0)(-1)(1) = 4 - 27 = -23$ is not a square so the Galois group is $S_3$.

(d) This is irreducible using the rational root theorem since $\pm 1$ are not roots. The discriminant $(1)^2(-2)^2 - 4(-2)^3 - 4(1)^3(-1) - 27(-1)^2 + 18(1)(-2)(-1) = 4 + 32 + 4 - 27 + 36 = 49$ is a square so the Galois group is $A_3$.
\end{proof}

\begin{problem}[14.6.3]
Prove for any $a,b \in \mathbb{F}_{p^n}$ that if $x^3 + ax + b$ is irreducible then $-4a^3-27b^2$ is a square in $\mathbb{F}_{p^n}$.
\end{problem}
\begin{proof}
If $x^3 + ax + b$ is irreducible and degree $3$ we know that its Galois group is either $A_3$ or $S_3$. But it's over a finite field so it must have cyclic Galois group, namely $A_3$. This means the discriminant is a square. The discriminant is $0^2a^2 - 4a^3 - 4(0)^3b - 27b^2 + 18(0)ab = -4a^3 - 27b^2$.
\end{proof}

\begin{problem}[14.6.6]
Determine the Galois group of $x^4 + 3x^3 - 3x - 2$.
\end{problem}
\begin{proof}
Putting in $\pm 1$ gives $-1$ and putting in $\pm 2$ gives $32$ and $-4$ so there is no linear factor by the rational root test. Suppose $x^4 + 3x^3 - 3x - 2 = (x^2 + bx + c)(x^2 + ex + f) = x^4 + (b+e)x^3 + (c + f + be)x^2 + (bf+ce)x + cf$. Then $b + e = 3$, $c + f + be = 0$, $bf + ce = -3$ and $cf = -2$. This means $c = \pm 1$ and $f = \mp 2$ so that $c + f = \pm 1 \mp 2 = \pm 1$ and $0 = c + f + be = \pm 1 + be$. Thus $be = \pm 1$ so $b = e = \pm 1$. But $b + e = 3$, a contradiction. Therefore $x^4 + 3x^3 - 3x - 2$ is irreducible.

In this case $p = 1/8(-3(3)^2 + 8(0)) = -27/8$, $q = 1/8(3^3 - 4(3)(0) + 8(-3)) = 3/8$ and $r = 1/256(-3(3)^4 + 16(3)^2(0) - 64(3)(-3) + 256(-2)) = -179/256$. This gives a resultant cubic of $h(x) = x^3 + 27/4 x^2 + 227/16 x + 9/64 = 1/64(64x^3 + 432x^2 + 908x + 9)$. Putting in $\pm 1$, $\pm 3$ and $\pm 9$ gives $1413$, $-531$, $8349$, $-555$, $89829$ and $-19827$ respectively so this is irreducible by the rational root test. Also $h(x)$ has discriminant $((27/4)^2 (227/16)^2 - 4(227/16)^3 - 4(27/4)^3(9/64) - 27(9/64)^2 + 18(27/4)(227/16)(9/64) = -2183$ which is not a square. Thus this polynomial has Galois group $S_4$.
\end{proof}

\begin{problem}[14.6.7]
Determine the Galois group of $x^4 + 2x^2 + x + 3$.
\end{problem}
\begin{proof}
Reducing modulo $2$ gives $x^4 + x + 1$ which is irreducible over $\mathbb{F}_2$ thus also over $\mathbb{Q}$. Since there is no cubic term we easily compute that $p = 2$, $q = 1$ and $r = 3$ giving a resolvent cubic of $h(x) = x^3 - 4x^2 - 8x + 1$. This is irreducible by the rational root test since $\pm 1$ gives no zeros. The discriminant of $h(x)$ is $(-4)^2(-8)^2 - 4(-8)^3 - 4(-4)^3(1) - 27(1)^3 + 18(-4)(-8)(1) = 3877$ which is not a square. Thus the Galois group of this polynomial is $S_4$.
\end{proof}

\begin{problem}[14.6.8]
Determine the Galois group of $x^4 + 8x + 12$.
\end{problem}
\begin{proof}
Putting in the values $\pm 1$, $\pm 2$, $\pm 3$, $\pm 4$, $\pm 6$ and $12$ gives $21$, $5$, $44$, $12$, $117$, $69$, $300$, $236$, $1356$, $1260$, $20844$ and $20652$ so there is no linear factor by the rational root test. Suppose now $x^4 + 8x + 12 = (x^2 + bx + c)(x^2 + ex + f) = x^4 + (b+e)x^3 + (c + f + be)x^2 + (bf+ce)x + cf$ so that $b + e = c + f + be = 0$, $bf + ce = 8$ and $cf = 12$. Then $b = -e$ so $c + f = b^2$. Given that $cf = 12$, and they add to a nonnegative number, the possibilities for $c$ and $f$ are $1$, $2$, $3$, $4$, $6$ and $12$. But $1 + 12 = 13$, $2 + 6 = 8$ and $3 + 4 = 7$ none of which are square numbers. This is a contradiction, so $x^4 + 8x + 12$ must be irreducible.

We easily see that $p = 0$, $q = 8$ and $r = 12$ giving a resultant cubic of $h(x) = x^3 - 48x + 64$. Putting in $\pm 1$, $\pm 2$, $\pm 4$, $\pm 8$, $\pm 16$, $\pm 32$, $\pm 64$ gives $17$, $111$, $-24$, $152$, $-64$, $192$, $192$, $-64$, $3392$, $-3264$, $31296$, $-31168$, $259136$ and $-259008$ showing that $h(x)$ is irreducible. The discriminant of $h(x)$ is $(0)^2(-48)^2 - 4(-48)^3 - 4(0)^3(64) - 27(64)^2 + 18(0)(-48)(64) = 331776 = 576^2$. Since $h(x)$ is irreducible and the discriminant is a square we see that the Galois group is $A_4$.
\end{proof}

\begin{problem}[14.6.11]
Let $F$ be an extension of $\mathbb{Q}$ of degree $4$ that is not Galois over $\mathbb{Q}$. Prove that the Galois closure of $F$ has Galois group either $S_4$, $A_4$ or the dihedral group $D_8$ of order $8$. Prove the the Galois group is dihedral if and only if $F$ contains a quadratic extension of $\mathbb{Q}$.
\end{problem}
\begin{proof}
Since $F$ is a degree $4$ extension, it's generated by a root $\alpha$ of some fourth degree polynomial $p(x) \in \mathbb{Q}[x]$. But since $F$ is not Galois, it's not a splitting field for $p(x)$ so in $F$ we must have either $p(x) = (x-\alpha)q(x)$ or $(x-\alpha)(x-\beta)q(x)$ where $q(x)$ is either an irreducible cubic or an irreducible quadratic in $F[x]$. The Galois group is now determined by the extension of $L/F$ where $L$ is the galois closure of $F$, namely, the splitting field for $p(x)$. If $[L:F] = 6$ then $[L:\mathbb{Q}] = 24$ and the Galois group is $S_4$. If $[L:F] = 3$ then $[L:\mathbb{Q}] = 12$ and the Galois group is $A_4$. These cover all the possibilities for the first case since then $q(x)$ is an irreducible cubic over $F$ so the extension must be $3$ or $6$. In the second case we must have $[L:F] = 2$ so $[L:\mathbb{Q}] = 8$ and $D_8$ is the Galois group.

Thus if $F$ contains a quadratic extension of $\mathbb{Q}$ then $\mathbb{Q}(\sqrt{D}) \subseteq F$ for some squarefree element of $\mathbb{Q}$. But this implies $p(x) = (x \pm \sqrt{D})q(x)$ over $F$ for a quadratic $q(x)$. Thus the Galois group must be $D_8$ as above. Conversely, if the Galois group is $D_8$, then $p(x)$ must split as $(x-\alpha)(x-\beta)q(x)$ over $F$. But since $\alpha, \beta \notin \mathbb{Q}$ this simplifies to saying that $F$ contains a quadratic extension of $\mathbb{Q}$.
\end{proof}

\begin{problem}[14.6.13]
(a) Let $\pm \alpha$, $\pm \beta$ denote the roots of the polynomial $f(x) = x^4 + ax^2 + b \in \mathbb{Z}[x]$. Prove that $f(x)$ is irreducible if and only if $\alpha^2$, $\alpha \pm \beta$ are not elements of $\mathbb{Q}$.\\
(b) Suppose $f(x)$ is irreducible and let $G$ be the Galois group of $f(x)$. Prove that\\
(i) $G \cong V$, the Klein $4$-group, if and only if $b$ is a square in $\mathbb{Q}$ if and only if $\alpha \beta \in \mathbb{Q}$ is rational.\\
(ii) $G \cong C$, the cyclic group of order $4$, if and only if $b(a^2 - 4b)$ is a square in $\mathbb{Q}$ if and only if $\mathbb{Q}(\alpha \beta) = \mathbb{Q}(\alpha^2)$.\\
(iii) $G \cong D_8$, the dihedral group of order $8$, if and only if $b$ and $b(a^2 - 4b)$ are not squares in $\mathbb{Q}$ if and only if $\alpha \beta \notin \mathbb{Q}(\alpha^2)$.
\end{problem}
\begin{proof}
(a) Suppose that $\alpha^2, \alpha \pm \beta \notin \mathbb{Q}$. Then $\alpha = 1/2(\alpha + \beta + \alpha - \beta)$ and $\beta = 1/2(\alpha + \beta - (\alpha - \beta))$ are not in $\mathbb{Q}$ either. Thus $f(x)$ cannot factor as $(x-a')(x^3 + b'x^2 + c'x + d')$ because $a' \notin \mathbb{Q}$.

So suppose $x^4 + ax^2 + b = (x^2 + cx + d)(x^2 + ex + f) = x^4 + (c + e)x^3 + (d + f + ce)x^2 + (cf + de)x + df$. This gives $c + e = cf + de = 0$ and $df = b$ so that $c = -e$ and $c(f-d) = 0$. Suppose first $c = 0$. The roots to $x^2 + cx + d$ are $(1/2)(-c \pm \sqrt{c^2 - 4d})$. Without loss of generality we can assume $\alpha = (1/2)(-c + \sqrt{c^2 - 4d})$ so that $\alpha^2 = (1/4)(c^2 - 2c\sqrt{c^2 - 4d} + c^2 - 4d)$. But if $c = 0$ this reduces to $-d$ showing that $-d \notin \mathbb{Q}$, a contradiction. On the other hand, suppose $f-d = 0$ so that $f = d$. Since $c = -e$ we can then express all four roots as $(1/2)(\pm c \pm \sqrt{c^2 - 4d})$. But then $\alpha - \beta = 0$, a contradiction. Therefore $f(x)$ doesn't split into a linear factor and a cubic or into two quadratics, so it must be irreducible.

Conversely, suppose $f(x)$ is irreducible. Then we know $\pm \alpha$ and $\pm \beta$ are not elements of $\mathbb{Q}$. Note that $x^4 + ax^2 + b = (x^2 - (1/2)(-a + \sqrt{a^2 - 4b}))(x^2 - (1/2)(-a - \sqrt{a^2 - 4b}))$. This gives the four solutions
\[
\pm \sqrt{\frac{-a \pm \sqrt{a^2 - 4b}}{2}}.
\]
Without loss of generality take $\alpha$ to be the solution with two $+$ signs. Then $\alpha^2 = (1/2)(-a + \sqrt{a^2 - 4b})$. Since $f(x)$ is irreducible and of degree $4$ we know $a^2 - 4b \neq 0$ and squarefree. Thus $\alpha^2 \notin \mathbb{Q}$. Similarly, $b$ is nonzero (otherwise $f(x)$ is reducible) so none of the roots are $0$ and $\alpha \pm \beta \notin \mathbb{Q}$ either.

(b) (i) The resolvent cubic for $f(x)$ is $h(x) = x^3 - 2ax^2 + (a^2-4bx) = x(x^2 - 2ax + a^2 - 4b)$. The solutions to the quadratic term are $(1/2)(2a \pm \sqrt{4a^2 - 4(a^2-4b)}) = (a \pm \sqrt{a^2 - (a^2 - 4b)}) = a \pm 2\sqrt{b}$. Note $G \cong V$ if and only if $h(x)$ splits into linear factors, which is true if and only if $\sqrt{b} \in \mathbb{Q}$ so $b$ is a square. Note also that if $G \cong V$ then every element of $G$ has order $2$ so we must have $\alpha \beta \in \mathbb{Q}$. Conversely, if $\alpha \beta \in \mathbb{Q}$ then $[\mathbb{Q}(\alpha, \beta) : \mathbb{Q}] \leq 4$ so $G$ must be isomorphic to $V$ since given the factorization of $h(x)$ it cannot be isomorphic to $C$.

(ii) Note that the discriminant $D = 16a^4b - 128a^2b^2 + 256b^3 = 16b(a^2-4b)^2$. Note that $G \cong C$ if and only if $f(x)$ is reducible over $\mathbb{Q}(\sqrt{D})$ since $h(x)$ has a linear factor. But given the value of $D$, this is true if and only if $b(a^2-4b)$ is a square. Additionally, we know that if $G \cong C$ then $\alpha \beta \notin \mathbb{Q}$ but since $[\mathbb{Q}(\alpha \beta) : \mathbb{Q}] = 4$ we see that $\alpha \beta \in \mathbb{Q}(\alpha^2)$. The converse is true by the same degree considerations.

(iii) This follows from the other two parts since the only possibilities for $G$ are $V$, $C$ and $D_8$ since $h(x)$ factors. Thus, $G \cong D_8$ if and only if $G$ is not $V$ and $G$ is not $C$ if and only if $b$ and $b(a^2-4b)$ are not squares if and only if $\alpha \beta \notin \mathbb{Q}(\alpha^2)$.
\end{proof}

\end{document}