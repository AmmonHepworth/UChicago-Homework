\documentclass{article}
\usepackage{amsmath,amsthm,amssymb,amsfonts,fullpage,fancyhdr}

\pagestyle{fancy}
\renewcommand{\headheight}{50pt}
\renewcommand{\footskip}{10pt}
\renewcommand{\textheight}{609pt}
\renewcommand{\headrulewidth}{0pt}

\newcommand{\gal}{\textup{Gal}}

\newtheorem{problem}{Problem}

\begin{document}

\rhead{Kris Harper\\MATH 25900\\May 7, 2010\\}
\chead{Homework 4\\}

\begin{problem}[14.3.3]
Prove that an algebraically closed field must be infinite.
\end{problem}
\begin{proof}
Let $F = \{a_1, \dots a_n\}$ be a finite field. Consider $p(x) = (x-a_1) \dots (x-a_n) + 1$. This clearly has no roots in $F$ so $F$ cannot be algebraically closed.
\end{proof}

\begin{problem}[14.3.5]
Exhibit an explicit isomorphism between the splitting fields of $x^3-x+1$ and $x^3-x-1$ over $\mathbb{F}_3$.
\end{problem}
\begin{proof}
Let $K$ and $L$ be splitting fields for $x^3-x+1$ and $x^3-x-1$ respectively. Let $\alpha$, $\beta$ and $\gamma$ be the roots of $x^3-x+1$ in $K$. Then note that $\alpha^3 - \alpha = -1$ so $(-\alpha)^3 + \alpha = 1$ and $-\alpha$ is a root for $x^3-x-1$ in $L$. Thus there is an isomorphism from $K$ to $L$ fixing $\mathbb{F}_3$ and taking $\alpha$, $\beta$ and $\gamma$ to $-\alpha$, $-\beta$ and $-\gamma$.
\end{proof}

\begin{problem}[14.4.2]
Find a primitive generator for $\mathbb{Q}(\sqrt{2}, \sqrt{3}, \sqrt{5})$ over $\mathbb{Q}$.
\end{problem}
\begin{proof}
Note that $\gal(\mathbb{Q}(\sqrt{2}, \sqrt{3}, \sqrt{5})/\mathbb{Q})$ is generated by $\rho$, $\sigma$ and $\tau$ where $\rho : \sqrt{2} \mapsto -\sqrt{2}$, $\sigma : \sqrt{3} \mapsto -\sqrt{3}$ and $\tau : \sqrt{5} \mapsto -\sqrt{5}$. But then none of $\rho$, $\sigma$ or $\tau$ or their products will fix $\sqrt{2} + \sqrt{3} + \sqrt{5}$. Thus only the trivial automorphism fixes it so $\mathbb{Q}(\sqrt{2}, \sqrt{3}, \sqrt{5}) \subseteq \mathbb{Q}(\sqrt{2} + \sqrt{3} + \sqrt{5})$. The other inclusion is obvious so $\sqrt{2} + \sqrt{3} + \sqrt{5}$ is a generator.
\end{proof}

\begin{problem}[14.4.6]
Prove that $\mathbb{F}_p(x,y)/\mathbb{F}_p(x^p,y^p)$ is not a simple extension by explicitly exhibiting an infinite number of intermediate subfields.
\end{problem}
\begin{proof}
Let $F = \mathbb{F}_p(x^p,x^p)$ and consider the extensions $F(x+y^k)$ where $p \nmid k$. Clearly each of these contains $F$. Suppose $F(x+y^k) = F(x+y^j)$. Then we have the element $y^k-y^j$ in both of these fields. Note that if $k \neq j$, $F(y^k-y^j)$ is a degree $p$ extension and it contains $F(y)$, also a degree $p$ extension. Thus $F(x+y^k)$ contains the element $y$ and therefore also $x$. Then $F(x+y^k) = F(x,y)$ which we know can't be true by degree considerations. Thus any two $F(x+y^k)$ and $F(x+y^j)$ are distinct.
\end{proof}

\end{document}