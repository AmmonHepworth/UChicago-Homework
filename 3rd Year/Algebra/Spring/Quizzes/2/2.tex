\documentclass{article}
\usepackage{amsmath,amsthm,amssymb,amsfonts,fullpage,fancyhdr,longtable}

\pagestyle{fancy}
\renewcommand{\headheight}{50pt}
\renewcommand{\footskip}{10pt}
\renewcommand{\textheight}{600pt}
\renewcommand{\headrulewidth}{0pt}

\newtheorem{problem}{Problem}

\begin{document}

\rhead{Kris Harper\\MATH 25900\\April 14, 2010\\}
\chead{Quiz 2\\}

\begin{problem}
Prove no finite field is algebraically closed.
\end{problem}
\begin{proof}
Let $F$ be a finite field with $q$ elements $\{a_1, \dots , a_q\}$. Let $n > 1$ and let $p(x) \in F[x]$ be a monic polynomial with degree $n$. For each $a_i \in F$ let $p_i(x) = p(x) + a_i$. Then we have a collection of $q$ polynomials in $F[x]$ each identical except for a distinct constant term. If $F$ is algebraically closed, then each of these polynomials has a root in $F$ and for $i \neq j$, a root of $p_i(x)$ cannot be a root of $p_j(x)$ since they differ by $a_i-a_j$. Thus we have $q$ distinct roots and $q$ elements of $F$. The only way this can happen is if $p_i(x) = (x-a_j)^n$ where $a_j$ is the root for $p_i(x)$. But this is impossible since $n$ is greater than $1$ and all the $p_i(x)$ are identical except for their constant terms. Thus, for example, the $n-1$ term of each $p_i(x)$ is different in this case, contrary to our assumption. Therefore there must be at least one $p_i(x)$ which has no root in $F$.
\end{proof}

\begin{problem}
An algebraic number $a$ is said to be an algebraic integer if it satisfies an equation of the form
\[
a^m + \alpha_1 a^{m-1} + \dots + \alpha_m = 0
\]
where $\alpha_1, \dots , \alpha_m$ are integers.\\
(a) If $a$ is any algebraic number, prove that there is a positive integer $n$ such that $na$ is an algebraic integer.\\
(b) If $a$ is an algebraic integer and $m$ is an ordinary integer, prove that $a+m$ is an algebraic integer.
\end{problem}
\begin{proof}
(a) Since $a$ is algebraic there exists some polynomial $p(x) \in \mathbb{Q}[x]$ such that $a^m + \alpha_1 a^{m-1} + \dots + \alpha_m = 0$ and $\alpha_i \in \mathbb{Q}$. Now find the least common multiple of the $\alpha_i$ and call it $n$. Multiply our polynomial by $n$ so we have $na^m + \beta_1 a^{m-1} + \dots + \beta_m = 0$ where $\beta_i \in \mathbb{Z}$. Finally, multiply both sides by $n^{m-1}$ so we have $n^m a^m + \beta_1 n^{m-1}a^{m-1} + \beta_2 n^{m-1} a^{m-2} + \dots + \beta_m n^{m-1} = 0$. We can now pass the appropriate exponent of $n$ inside each exponent of $a$ for every term which results in the equation $(na)^m + \beta_1 (na)^{m-1} + \beta_2 n (na)^{m-2} + \dots + \beta_{m-1} n^{m-2} (na) + \beta_m n^{m-1} = 0$. Since each $\beta_in^{i-1}$ is an integer we see that $na$ is an algebraic integer.

(b) Let $a$ and $b$ be any two algebraic integers. Then we can write $a^m = -\alpha_1 a^{m-1} - \dots - \alpha_m$ and $a^{m+1} = -\alpha_1 a^m - \dots - \alpha_m a$. Substituting the above equation in for $a^m$ we see that $a^{m+1}$ can be expressed as a polynomial with integer coefficients. Similarly, any power of $a$ can be expressed as a linear combination of the elements $1, a, \dots , a^{m-1}$. Likewise any power of $b$ can be expressed as a linear combination of $1, b, \dots , b^{n-1}$. Thus any polynomial in $a$ and $b$ can be expressed as a linear combination of the $mn$ elements $a^ib^j$ for $0 \leq i \leq m-1$ and $0 \leq j \leq n-1$ with integer coefficients. Thus we can write $a + b = c_1a^0b^0 + c_2a^1b^0 + \dots + c_{mn}a^{m-1}b^{m-1}$. We can now multiply both sides of this equation successively by $a^ib^j$ for $0 \leq i \leq m-1$, $0 \leq j \leq n-1$ and rewrite the righthand side as a linear combination of powers of $a$ and $b$ with integer coefficients. We then obtain $mn$ equations
\begin{align*}
(a+b)a^0b^0 &= c_1'a^0b^0 + c_2'a^1b^0 + \dots + c_{mn}'a^{m-1}b^{n-1}\\
(a+b)a^1b^0 &= c_1''a^0b^0 + c_2''a^1b^0 + \dots + c_{mn}''a^{m-1}b^{n-1}\\
&\vdots \\
(a+b)a^{m-1}b^{n-1} &= c_1^{(mn)}a^0b^0 + c_2^{(mn)}a^1b^0 + \dots + c_{mn}^{(mn)}a^{m-1}b^{n-1}.\\
\end{align*}
Subtracting the left hand side we see that these have a nontrivial solution and so the corresponding matrix must have determinant zero. That is
\[
\det \left (
\begin{array}{cccc}
c_1'-(a+b) & c_2' & \cdots & c_{mn}'\\
c_1'' & c_2'' - (a+b) & \cdots & c_{mn}''\\
\vdots & \vdots && \vdots\\
c_1^{(mn)} & c_2^{(mn)} & \cdots & c_{mn}^{(mn)} - (a+b)
\end{array}
\right )
= 0.
\]
This determinant is a polynomial in $a+b$ with integer coefficients and leading coefficient $\pm 1$ so $a+b$ must be an algebraic integer. But now it's easy to see that any integer $b$ is an algebraic integer because it's the solution to $x-b$.
\end{proof}

\end{document}