\documentclass{article}
\usepackage{amsmath,amsthm,amsfonts,amssymb,fullpage}

\newtheorem{problem}{Problem}

\begin{document}

\begin{flushright}
Kris Harper\\

MATH 27000\\

December 1, 2009
\end{flushright}

\begin{center}
Homework 8
\end{center}

\begin{problem}
(a) Let $u$ be real harmonic. Show that $u^2$ is subharmonic.\\
(b) Let $u$ be real harmonic, $u = u(x,y)$. Show that
\[
(\textup{grad} u)^2 = (\textup{grad} u) \cdot (\textup{grad}u)
\]
is subharmonic.\\
(c) Show that the function $u(x,y) = x^2 + y^2 - 1$ is subharmonic.\\
(d) Let $u_1$, $u_2$ be subharmonic, and $c_1$, $c_2$ positive numbers. Show that $c_1 u_1 + c_2 u_2$ is subharmonic.
\end{problem}
\begin{proof}
We have
\[
\frac{\partial u^2}{\partial x} = 2 u \frac{\partial u}{\partial x}
\]
and
\[
\frac{\partial u^2}{\partial x^2} = 2 \frac{\partial^2 u}{\partial x^2} u + 2 \left ( \frac{\partial u}{\partial x} \right )^2.
\]
Likewise
\[
\frac{\partial u^2}{\partial y^2} = 2 \frac{\partial^2 u}{\partial y^2} u + 2 \left ( \frac{\partial u}{\partial y} \right )^2.
\]
Adding these equations and noting that $\Delta u = 0$ we obtain
\[
\Delta u^2 = \frac{\partial u^2}{\partial x^2} + \frac{\partial u^2}{\partial x^2} = 2 \left ( \frac{\partial u}{\partial x} \right )^2 + 2 \left ( \frac{\partial u}{\partial x} \right )^2 \geq 0.
\]

(b) Let $v = (\textup{grad} u)^2$. Note that
\[
v = \left ( \frac{\partial u}{\partial x} \right )^2 + \left ( \frac{\partial u}{\partial y} \right )^2.
\]
Differentiating twice we find
\[
\frac{\partial^2 v}{\partial x^2} = 2 \left ( \frac{\partial^3 u}{\partial x^3} \frac{\partial u}{\partial x} + \left ( \frac{\partial^2 u}{\partial x^2} \right )^2 + \frac{\partial^3 u}{\partial y^2 \partial x} \frac{\partial u}{\partial x} + \left ( \frac{\partial^2 u}{\partial x \partial y} \right )^2 \right )^2
\]
and a similar expression for $y$. Adding these two equations and again using the fact that $\Delta u = 0$, we find that $\Delta v \geq 0$ as in part (a).

(c) It's immediate that $\Delta u = 4 \geq 0$.

(d) Since differentiation is linear, we have $\Delta (c_1 u_1 + c_2 u_2) = c_1 \Delta u_1 + c_2 \Delta u_2$. Since $c_1$ and $c_2$ are positive and $u_1$ and $u_2$ are subharmonic, it's immediate that this sum is greater than or equal to $0$.
\end{proof}

\begin{problem}
Suppose that $\varphi$ is defined on an open set $U$ and is subharmonic on $U$. Prove the maximum principal, that no point $a \in U$ can be a strict maximum for $\varphi$, i.e. that for every disk of radius $r$ centered at $a$ with $r$ sufficiently small, we have
\[
\begin{tabular}{ccc}
$\varphi(a) \leq \max \varphi(z)$ & for & $|z-a| = r$.
\end{tabular}
\]
\end{problem}
\begin{proof}
Let $a \in U$ and choose $r$ small enough that $\overline{D}_r(a) \subseteq U$. Now since $\varphi$ is continuous on the circle $\partial \overline{D}_r(a)$ which is compact, it obtains a maximum on that set. Therefore $\varphi(a + re^{i \theta}) \leq \max_{z \in \partial \overline{D}_r(a)} \varphi(z)$ for every $\theta$. Now integrate over $\theta$ from $0$ to $2 \pi$.
\[
\varphi(a) \leq \int_0^{2\pi} \varphi(a + re^{i \theta}) \frac{d \theta}{2\pi} \leq \max_{z \in \partial \overline{D}_r(a)} \varphi(z).
\]
This is precisely the statement.
\end{proof}

\begin{problem}
\label{boundarymax}
Let $\varphi$ be subharmonic on an open set $U$. Assume that the closure $\overline{U}$ is compact, and that $\varphi$ extends to a continuous function on $\overline{U}$. Show that a maximum for $\varphi$ occurs on the boundary.
\end{problem}
\begin{proof}
Since $\varphi$ is continuous and $\overline{U}$ is compact, we know $\varphi$ obtains a maximum on $\overline{U}$. Suppose this maximum is at $a$ in the interior of $U$. Choose $r$ small enough so that $\overline{D}_r(a) \subseteq U$. Define a function $f$ on $\partial \overline{D}_r(a)$ as $f(\theta) = \varphi(a) - \varphi(a + re^{i \theta})$. Note that $f \geq 0$. Suppose there exists $0 \leq \theta_0 \leq 2 \pi$ such that $f(\theta_0) > 0$. Then
\[
\int_0^{2 \pi} f(\theta) d \theta > 0
\]
since $f$ is continuous. But then
\[
\varphi(a) \leq \int_0^{2 \pi} \varphi(a + r e^{i \theta} \frac{d \theta}{2 \pi} < \varphi (a)
\]
and so $f$ must be constantly $0$. Therefore $\varphi$ is locally constant.

For each connected open set $V \subseteq U$ we see that $\varphi(z) = \varphi(a)$ for $z \in V$. Suppose that $V$ is the largest such connected open set and suppose that $\partial V \nsubseteq U$. Then there exists $z \in \partial V$ with $z \in U$ and so there exists some $D_{\varepsilon}(z) \subseteq U$. But then $V \subseteq D_{\varepsilon}(z)$ is open, connected and contained in $U$ which contradicts the maximality of $V$. Thus $\partial V \subseteq \partial U$. Now using continuity, it must be that $\varphi(z) = \varphi(a)$ for $z \in \partial V$ and thus there exists $z \in \partial U$ such that $\varphi(z) = \varphi(a)$. Therefore, $\varphi$ attains a maximum on $\partial U$.
\end{proof}

\begin{problem}
Let $U$ be a bounded open set. Let $u$, $v$ be continuous functions on $\overline{U}$ such that $U$ is harmonic on $U$, $v$ is subharmonic on $U$ and $u = v$ on the boundary of $U$. Show that $v \leq u$ on $U$. Thus a subharmonic function lies below the harmonic function having the same boundary value, whence its name.
\end{problem}
\begin{proof}
The function $v-u$ is subharmonic by linearity of differentiations. That is
\[
\Delta (v-u) = \Delta v - \Delta u = \Delta v \leq 0.
\]
Note that on $\partial U$ we have $v-u \leq 0$ and so using Problem~\ref{boundarymax} we must have $v-u \leq 0$ on all of $U$. Thus $v \leq u$ on all of $U$.
\end{proof}

\begin{problem}
\label{inequality}
Define
\[
P_{R,r}(\theta) = \frac{1}{2 \pi} \frac{R^2 - r^2}{R^2 - 2Rr \cos \theta + r^2}
\]
for $0 \leq r < R$. Prove the inequalities
\[
\frac{R-r}{R+r} \leq 2 \pi P_{R,r} (\theta - \varphi) \leq \frac{R+r}{R-r}
\]
for $0 \leq r < R$.
\end{problem}
\begin{proof}
Note that
\[
-2rR \leq -2rR\cos(\theta-\varphi) \leq 2rR
\]
which means
\[
(R-r)^2 \leq R^2 - 2rR\cos(\theta-\varphi) + r^2 \leq (R+r)^2.
\]
Now using $0 \leq r < R$ we have
\[
\frac{R^2-r^2}{(R+r)^2} \leq 2 \pi P_{R,r} (\theta - \varphi) \leq \frac{R^2 - r^2}{(R-r)^2}
\]
Now expand $R^2 - r^2 = (R+r)(R-r)$ to obtain
\[
\frac{R-r}{R+r} \leq 2 \pi P_{R,r} (\theta - \varphi) \leq \frac{R+r}{R-r}.
\]
\end{proof}

\begin{problem}
Let $f$ be analytic on the closed disk $\overline{D}(\alpha, R)$ and let $u = \textup{Re} (f)$. Assume that $u \geq 0$. Show that for $0 \leq r < R$ we have
\[
\frac{R-r}{R+r} u(\alpha) \leq u(\alpha + re^{i \theta}) \leq \frac{R+r}{R-r} u(\alpha).
\]
\end{problem}
\begin{proof}
We can assume $\alpha = 0$ by applying a translation of the disk. Note that $u$ is harmonic because it is the real part of an analytic function and therefore
\[
u(re^{i \varphi}) = \int_0^{2 \pi} u(r3^{i \theta})P_r(\theta-\varphi)d\theta.
\]
Now use Problem~\ref{inequality} and the fact that $u \geq 0$ to obtain
\[
\int_{0}^{2 \pi} u(re^{i\theta}) \frac{R-r}{R+r}\frac{d\theta}{2 \pi} \leq u(re^{i \varphi}) \leq \int_0^{2 \pi} u(re^{i \theta}) \frac{R+r}{R-r} \frac{d \theta}{2 \pi}.
\]
But since
\[
u(0) = \int_0^{2 \pi} u(re^{i \theta}) \frac{d \theta}{2 \pi}
\]
the result follows. Shifting back by $\alpha$ will finish the general case.
\end{proof}

\begin{problem}
Prove that if $v(z) = \textup{Im} \left ( \left ( \frac{1+z}{1-z} \right )^2 \right )$, then $v$ is harmonic on $\mathbb{D}$ and $\lim_{r \uparrow 1} v(re^{i \theta}) = 0$ for all $\theta \in [0, 2 \pi)$. Why does this not contradict the maximum principal?
\end{problem}
\begin{proof}
Let $z = x+ iy$. Then
\begin{align*}
\textup{Im} \left ( \left ( \frac{1+z}{1-z} \right )^2 \right )
&= \textup{Im} \left ( \left ( \frac{x+iy+1}{-x-iy+1} \right )^2 \right )\\
&= \textup{Im} \left ( \frac{1 + 2x + x^2 + 2iy + 2ixy - y^2}{1 - 2x + x^2 - 2iy + 2ixy - y^2} \right )\\
&= \textup{Im} \left ( \frac{1 + 2x + x^2 + 2iy + 2ixy - y^2}{1 - 2x + x^2 - 2iy + 2ixy - y^2} \frac{1 - 2x + x^2 + 2iy - 2ixy - y^2}{1 - 2x + x^2 + 2iy - 2ixy - y^2} \right )\\
&= \textup{Im} \left ( \frac{1 - 2x^2 + x^4 + 4iy - 4ix^2y - 6y^2 + 2x^2y^2 - 4iy^3 + y^4}{1 - 4x + 6x^2 - 4x^3 + x^4 + 2y^2 - 4xy^2 + 2x^2y^2 + y^4} \right )\\
&= \frac{1 - 2x^2 + x^4 - 6y^2 + 2x^2y^2 + y^4}{1 - 4x + 6x^2 - 4x^3 + x^4 + 2y^2 - 4xy^2 + 2x^2y^2 + y^4}.
\end{align*}
But also
\begin{align*}
\frac{\partial^2v}{\partial x^2}
&=\frac{\partial^2}{\partial x^2} \frac{1 - 2x^2 + x^4 - 6y^2 + 2x^2y^2 + y^4}{1 - 4x + 6x^2 - 4x^3 + x^4 + 2y^2 - 4xy^2 + 2x^2y^2 + y^4}\\
&= \frac{8(2 - 2x^4 + x^5 - 16y^2 + 6y^4 + x^2(8 - 12y^2) - 2x^3 (1 + y^2) + x(-7 + 30y^2 - 3y^4))}{(1 - 2x + x^2 + y^2)^4}\\
&= -\frac{\partial^2}{\partial y^2} \frac{1 - 2x^2 + x^4 - 6y^2 + 2x^2y^2 + y^4}{1 - 4x + 6x^2 - 4x^3 + x^4 + 2y^2 - 4xy^2 + 2x^2y^2 + y^4}\\
&= -\frac{\partial^2v}{\partial y^2}.
\end{align*}
Thus $v(z)$ is harmonic.

Note that $\frac{1+z}{1-z}$ maps $\mathbb{D}$ to the right half plane. Thus $\left (\frac{1+z}{1-z} \right )^2$ maps $\mathbb{D}$ to $\mathbb{C}$ without the negative imaginary axis and $v$ maps $\mathbb{D}$ to the upper half plane. Furthermore, note that $\frac{1+z}{1-z}$ takes $\partial \mathbb{D}$ to the imaginary axis and squaring this line results in the negative real axis. The imaginary part of this is obviously $0$ which shows why $\lim_{r \uparrow 1} v(re^{i \theta}) = 0$ for all $\theta \in [0, 2 \pi)$. This doesn't contradict the maximum principle because $v(z)$ is not continuous on $\partial \mathbb{D}$ at $1$.
\end{proof}

\end{document}