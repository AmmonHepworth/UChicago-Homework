\documentclass{article}
\usepackage{amsmath,amsthm,amsfonts,amssymb,fullpage}

\newtheorem{problem}{Problem}

\begin{document}

\begin{flushright}
Kris Harper\\

MATH 24200\\

April 19, 2010
\end{flushright}

\begin{center}
Homework 2
\end{center}

\begin{problem}
Show that there are infinitely many primes congruent to $-1$ modulo $6$.
\end{problem}
\begin{proof}
Suppose there are finitely many such, $p_1, \dots , p_n$ and let $m = 3p_1 \dots p_n + 2$. Then $m \equiv -1 \pmod {6}$. Assume that $m$ is not prime so that $m = q_1 \dots q_k$ with $q_i$ prime. Note that $3 \nmid m$ since $m \equiv 2 \pmod{3}$ and no $q_i$ is congruent to $-1$ modulo $6$ because $m$ is not a multiple of any $p_i$. Then every $q_i$ is congruent to $1$ modulo $6$ so $m \equiv 1 \pmod{6}$, a contradiction. Thus there must be infinitely many primes congruent to $-1$ modulo $6$.
\end{proof}

\begin{problem}
Show that the equation $3x^2 + 2 = y^2$ has no integer solutions.
\end{problem}
\begin{proof}
If we reduce modulo $3$ we're left with $2 \equiv y^2 \pmod{3}$. But $1^2 \equiv 2^2 \equiv 1 \pmod{2}$ so there are no integers $y$ such that $y^2 \equiv 2 \pmod{3}$.
\end{proof}

\begin{problem}
\label{icong}
Extend the notion of congruence to the ring $\mathbb{Z}[i]$ and prove that $a + bi$ is always congruent to $0$ or $1$ modulo $1 + i$.
\end{problem}
\begin{proof}
The definition will be the same as that of $\mathbb{Z}$. Namely, $a + bi \equiv c + di \pmod{m+ni}$ if there exists $e + fi \in \mathbb{Z}[i]$ such that $(e+fi)(m+ni) = (a + bi) - (c + di)$.

Now note that $(e + fi)(1 + i) = (e-f) + (e + f)i$. Without loss of generality assume $a < b$. Let $k = 2((b-a)/2 - [(b-a)/2])$ so that $k$ is $0$ if $b-a$ is even and $1$ if $b-a$ is odd. Now let $f = [b-a/2]$ and $e = a + f + k$. Then $e + f = b$ and $e - f$ is either $a$ or $a + 1$ depending on $k$. Thus $(e + fi)(i+1) = (a + bi)$ or $(e + fi)(i+1) = (a + bi) - 1$ so that $a + bi \equiv 0 \pmod{1 + i}$ or $a + bi \equiv 1 \pmod{1 + i}$.
\end{proof}

\begin{problem}
\label{omegacong}
Extend the notion of congruence to $\mathbb{Z}[\omega]$ and prove that $a + b \omega$ is always congruent to either $-1$, $1$ or $0$ modulo $1 - \omega$.
\end{problem}
\begin{proof}
The definition of congruence is exactly like that in Problem~\ref{icong}. Let $a + b \omega \in \mathbb{Z}[\omega]$ and suppose $(e + f \omega)(1 - \omega) = (a + b \omega) - (c + d \omega)$. Note that $(e + f \omega)(1 - \omega) = (e + f) + (2f - e) \omega$. Choose $f = [(a+b)/3]$ and $e = a - f + k$ where $k = 0$ if $a + b \equiv 0 \pmod{3}$, $k = 1$ if $a+b \equiv 1 \pmod{3}$ and $k = -1$ if $a + b \equiv -1 \pmod{3}$. Then $b = 2f - e$ and $a = e + f$ or $a = e + f \pm 1$ depending on the value of $k$. Thus $(e + f)(1 - \omega) = a + b \omega$ or $(e + f)(1 - \omega) = (a + b \omega) \pm 1$. Thus $a + b \omega$ is congruent to either $0$, $1$ or $-1$ modulo $1 - \omega$.
\end{proof}

\begin{problem}
\label{cube}
Let $\lambda = 1 - \omega \in \mathbb{Z}[\omega]$. If $ \alpha \in \mathbb{Z}[\omega]$ and $\alpha \equiv 1 \pmod{\lambda}$, prove that $\alpha^3 \equiv 1 \pmod{9}$.
\end{problem}
\begin{proof}
We can write $\alpha = 1 + \beta \lambda$ so that $\alpha^3 = 1 + 3\beta \lambda + 3\beta^2\lambda^2 + \beta^3 \lambda^3$. Note also that $-\omega^2\lambda^2 = -\omega^2(1-\omega)^2 = -\omega^2(1 - 2 \omega + \omega^2) = -\omega^2(-3\omega) = 3$. Putting this in the first equation we have $\alpha^3 = 1 - \beta \omega^2 \lambda^3 - \beta^2\omega^2\lambda^4 + \beta^3 \lambda^3$. Reducing modulo $\lambda^4$ we have $\alpha^3 \equiv 1 + (\beta^3 - \beta \omega^2)\lambda^3 \pmod{\lambda^4}$. But note that $\lambda \mid \beta^3 - \beta \omega^2$ so this equation reduces to $\alpha^3 \equiv 1 \pmod{\lambda^4}$. By the above statement $9 \mid \lambda^4$ so we're done.
\end{proof}

\begin{problem}
Use Exercise 25 to show that if $\xi, \eta, \zeta \in \mathbb{Z}[\omega]$ are not zero and $\xi^3 + \eta^3 + \zeta^3 = 0$ then $\lambda$ divides at least one of the elements $\xi$, $\eta$, $\zeta$.
\end{problem}
\begin{proof}
From Problem~\ref{omegacong} we know that each of $\xi$, $\eta$ and $\zeta$ is congruent to either $-1$, $1$ or $0$ modulo $\lambda$. Moreover, from Problem~\ref{cube} we know that if any one of $\xi$, $\eta$ or $\zeta$ is congruent to $1$ modulo $\lambda$ then it's cube also congruent to $1$ modulo $9$. Therefore it's not possible that all three of these are congruent to $1$ modulo $\lambda$ because otherwise $\xi^3 + \eta^3 + \zeta^3 \equiv 3 \pmod{9}$. It's also not possible that any two of these are congruent to $1$ modulo $\lambda$ since then the congruence of the cubic sum would be at least $1$. So at most $1$ of $\xi$, $\eta$ or $\zeta$ can be congruent to $1$ modulo $\lambda$. Since the congruences must sum to $0$ we can either have one value congruent to $1$, one value congruent to $-1$ and one value congruent to $0$, or all three congruent to $0$. In either case we have at least one value congruent to $0$ which means $\lambda$ divides at least one value.
\end{proof}

\begin{problem}
Suppose that $a$ is a primitive root modulo $p^n$, $p$ and odd prime. Show that $a$ is a primitive root modulo $p$.
\end{problem}
\begin{proof}
Let $n$ be the order of $a$ modulo $p$. We can write $a^n = 1 + mp$ for some $m$. Then $a^{np^{n-1}} = (1 + mp)^{p^{n-1}} = 1 + p^{n-1}mp + \cdots$ where we've expanded using the binomial theorem. Reducing this equation modulo $p^n$ we have $a^{np^{n-1}} \equiv 1 \pmod{p^n}$. Since $a$ is a primitive root modulo $p^n$ we must have $np^{n-1} = \phi(p^n) = p^{n-1}(p-1)$. Therefore $n = p-1$ and $a$ is a primitive root modulo $p$.
\end{proof}

\begin{problem}
\label{even}
Consider a prime $p$ of the form $4t + 1$. Show that $a$ is a primitive root modulo $p$ iff $-a$ is a primitive root modulo $p$.
\end{problem}
\begin{proof}
If $a$ is a primitive root modulo $p$ then $a$ has order $\phi(p) = 4t$ modulo $p$. Let $n$ be the order of $-a$ modulo $p$ and suppose $n < 4t$. Note $(-a)^n \equiv (-1)^n a^n \equiv 1 \pmod{p}$. Thus $n$ must be odd otherwise $(-1)^n = 1$. But then $(-a)^{2n} \equiv (-1)^{2n} a^{2n} \equiv a^{2n} \equiv 1 \pmod{p}$. But since $n$ is odd $4 \nmid 2n$ and since $a$ has order $4t$ modulo $p$ we have a contradiction. Thus the order of $-a$ modulo $p$ is $4t$ and $-a$ is a primitive root modulo $p$.

Now suppose $-a$ is a primitive root modulo $p$ so that the order of $-a$ modulo $p$ is $4t$. Let $n$ be the order of $a$ modulo $p$. Then $a^n \equiv 1 \pmod{p}$ and $1 \equiv a^{2n} \equiv (-1)^{2n} a^{2n} \equiv (-a)^{2n} \pmod{p}$. Thus $4t \mid 2n$ and $n$ must be even. But then $a^n \equiv (-1)^n a^n \equiv (-a)^n \pmod{p}$ and we see that $n = 4t$ is the order of $a$ modulo $p$ so $a$ is a primitive root modulo $p$.
\end{proof}

\begin{problem}
Consider a prime $p$ of the form $4t + 3$. Show that $a$ is a primitive root modulo $p$ iff $-a$ has order $(p-1)/2$.
\end{problem}
\begin{proof}
Suppose $a$ is a primitive root modulo $p$ so that the order of $a$ modulo $p$ is $\phi(p) = 4t+2 = 2(2t+1)$. Let $n$ be the order of $-a$ modulo $p$ and suppose $n < 4t+2$. Then $(-a)^n \equiv (-1)^n a^n \equiv 1 \pmod{t}$ so that $n$ must be odd otherwise $(-1)^n = 1$. Then $(-a)^{2n} \equiv a^{2n} \equiv 1 \pmod{p}$ so $4t+2 \mid 2n$ and $2t + 1 \mid n$. This shows that either $n = 2t+1$ or $n = 4t+2$ and since $n$ is odd we must have $n = 2t+1$ so that the order of $-a$ modulo $p$ is $2t+1 = (p-1)/2$.

Now suppose $-a$ has order $(p-1)/2 = 2t+1$ modulo $p$ so that $(-a)^{2t+1} \equiv 1 \pmod{p}$. Let $n$ be the order of $a$ modulo $p$. Then $(-a)^{4t+2} \equiv (-1)^{2(2t+1)} a^{4t+2} \equiv 1 \pmod{p}$ so that $4t+2 \mid n$. But $\phi(p) = 4t+2$ so we must have $n = 4t+2$ and $a$ is a primitive root modulo $p$.
\end{proof}

\begin{problem}
Show that the product of all the primitive roots modulo $p$ is congruent to $(-1)^{\phi(p-1)}$ modulo $p$.
\end{problem}
\begin{proof}
Let $a$ be a primitive root modulo $p$. There are $\phi(p-1)$ primitive roots modulo $p$ and note that every primitive root modulo $p$ can be expressed as some power of $a$, $a^{n_i}$. Furthermore we must have $(n_i, p-1) = 1$ otherwise these can't be primitive roots. Note that for every integer $n$, the sum of all integers $t$ with $1 \leq t \leq n$ and $(t,n) = 1$ is $\frac{1}{2} n \phi(n)$. Thus the product of all the primitive roots is
\[
a^{\sum_{i=1}^{\phi(p-1)} n_i} = a^{\frac{1}{2}(p-1)\phi(p-1)}.
\]
But note that $a^{(p-1)/2} \equiv -1 \pmod{p}$ so this reduces to $(-1)^{\phi(p-1)}$ modulo $p$ as desired.
\end{proof}

\begin{problem}
Let $K$ be a field and $G \subseteq K^*$ a finite subgroup of the multiplicative group of $K$. Extend the arguments used in the proof of Theorem 1 to show that $G$ is cyclic.
\end{problem}
\begin{proof}
Let $|G| = n$ and for $d \mid n$ define $\psi(d)$ to be the number of elements in $G$ of order $d$. In any finite commutative group the set of elements $x$ such that $x^d = 1$ form a subgroup. Moreover the order of this subgroup must be at least $d$. Thus $\sum_{c \mid d} \psi(c) \leq d$. If we apply M\"{o}bius inversion we have $\psi(d) \geq \sum_{c \mid d} \mu(c) d/c = \phi(d)$. In particular $\psi(n) \geq \phi(n)$ which for $n > 1$ is greater than $1$. Thus there is at least one element of $G$ which generates the entire group and thus $G$ is cyclic.
\end{proof}

\end{document}