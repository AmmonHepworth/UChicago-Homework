\documentclass{article}
\usepackage{amsmath,amsthm,amsfonts,amssymb,fullpage}

\newtheorem{problem}{Problem}

\begin{document}

\begin{flushright}
Kris Harper\\

MATH 24200\\

April 28, 2010
\end{flushright}

\begin{center}
Homework 3
\end{center}

\begin{problem}
If $p = 2^n + 1$ is a Fermat prime, show that $3$ is a primitive root modulo $p$.
\end{problem}
\begin{proof}
Suppose $3$ is not a primitive root modulo $p$. Then $3^{(p-1)/2}$ is not equivalent to  $-1$ modulo $p$. But then $3$ is a square modulo $p$. Since $p = 4t + 1$ we know there exists an integer $a$ such that $-3 \equiv a^2 \pmod{p}$. Now consider the equation $2u \equiv -1 + a \pmod{p}$. We have $4u^2 \equiv a^2 - 2a + 1 \equiv -2a-2 \pmod{p}$ and $4u^3 \equiv (-a - 1)(a-1) \equiv -a^2 + 1 \equiv 4 \pmod{p}$ so $u^3 \equiv 1 \pmod{p}$. But then $u$ has order $3$ modulo $p$ which implies $p \equiv 1 \pmod{3}$. This is a contradiction and so $3$ must be a primitive root modulo $p$.
\end{proof}

\begin{problem}
\label{7th}
Use the fact that $2$ is a primitive root modulo $29$ to find the seven solutions to $x^7 \equiv 1 \pmod{29}$.
\end{problem}
\begin{proof}
Note that $a^7 \equiv a^{\phi(29)/4} \equiv 1 \pmod{29}$ if and only if there exists $x$ such that $x^4 \equiv a \pmod{29}$. Since $2$ is a primitive root modulo $29$, all the solutions of this can be found by raising looking at multiples of $2^4$. Note that $2^4 \equiv 16 \pmod{29}$, $(2^2)^4 = 16(2^4) \equiv 24 \pmod{29}$, $(2^3)^4 = 24(2^4) \equiv 7 \pmod{29}$, $(2^4)^4 = 7(2^4) \equiv 25 \pmod{29}$, $(2^5)^4 = 25(2^4) \equiv 23 \pmod{29}$, $(2^6)^4 = 23(2^4) \equiv 20 \pmod{29}$. Thus the seven solutions are $1$, $7$, $16$, $20$, $23$, $24$ and $25$.
\end{proof}

\begin{problem}
Solve the congruence $1 + x + x^2 + \dots + x^6 \equiv 0 \pmod{29}$.
\end{problem}
\begin{proof}
Thus the the $7$th degree cyclotomic polynomial. The solutions to it are the nontrivial solutions to $x^7 \equiv 1 \pmod{7}$. By Problem~\ref{7th} we know the solutions are $7$, $16$, $20$, $23$, $24$ and $25$.
\end{proof}

\begin{problem}
Use Gauss' lemma to determine $\left ( \frac{5}{7} \right )$, $\left ( \frac{3}{11} \right )$, $\left ( \frac{6}{13} \right )$, and $\left ( \frac{-1}{p} \right )$.
\end{problem}
\begin{proof}
We know $(7-1)/2 = 3$ and $5$, $10$ and $15$ reduce to $-2$, $3$ and $1$ modulo $7$ so $\left ( \frac{5}{7} \right ) = -1$.

We know $(11-1)/2 = 5$ and $3$, $6$, $9$, $12$ and $15$ reduce to $3$, $-5$, $-2$, $1$ and $4$ modulo $11$ so $\left ( \frac{3}{11} \right ) = (-1)^2 = 1$.

We know $(13-1)/2 = 6$ and $6$, $12$, $18$, $24$, $30$ and $36$ reduce to $6$, $-1$, $5$, $-2$, $6$ and $-3$ modulo $13$ so $\left ( \frac{6}{13} \right ) = (-1)^3 = -1$.

Now we need to consider $-1$ times the values $\{1, 2, \dots , (p-1)/2\}$. But clearly all of these are going to be in the set of least residues mod $p$ and they will all be negative. Thus $\left ( \frac{-1}{p} \right ) = (-1)^{(p-1)/2}$.
\end{proof}

\begin{problem}
Show that the number of solutions to $x^2 \equiv a \pmod{p}$ is given by $1 + (a/p)$.
\end{problem}
\begin{proof}
If $p \mid a$ then $(a/p) = 0$ and $x = 0$ is the only solution. If $a$ is not a quadratic residue modulo $p$ then there are no solutions and $1 + (a/p) = 0$. If $a$ is a quadratic residue modulo $p$ then there exists $x$ such that $x^2 \equiv a \pmod{p}$. Note that $-x$ is clearly also a solution. But we know that there are exactly $(2,\phi(2)) = 2$ solutions so there are $2 = 1 + (a/p)$ solutions.
\end{proof}

\begin{problem}
Prove that $\sum_{a=1}^{p-1} (a/p) = 0$.
\end{problem}
\begin{proof}
We know there are as many residues as nonresidues modulo $p$. Since these have Legendre symbols $1$ and $-1$ respectively, their sum must be $0$.
\end{proof}

\begin{problem}
Suppose that $p \equiv 3 \pmod{4}$ and that $q = 2p + 1$ is also a prime. Prove that $2^p - 1$ is not prime. One must assume that $p > 3$.
\end{problem}
\begin{proof}
Since $p \equiv 3 \pmod{4}$ and $q = 2p + 1$ we see that $q \equiv 7 \pmod{8}$ so $(2/q) = 1$. Thus there exists $m$ such that $m^2 \equiv 2 \pmod{q}$. Then $2^p \equiv 2^{(q-1)/2} \equiv m^{q-1} \equiv 1 \pmod{q}$. Thus $q \mid 2^p - 1$. If $p > 3$ then $2^p - 1 > 2p + 1$ so $q$ is not the only factor and $2^p - 1$ is not prime.
\end{proof}

\begin{problem}
Let $f(x) \in \mathbb{Z}[x]$. We say that a prime $p$ divides $f(x)$ if there is an integer $n$ such that $p \mid f(n)$. Describe the prime divisors of $x^2 + 1$ and $x^2 -2$.
\end{problem}
\begin{proof}
A prime $p$ is a prime divisor of $x^2 + 1$ if there exists $n$ such that $p \mid (n^2 + 1)$. But this simply means $n^2 \equiv -1 \pmod{p}$ so $(-1/p) = 1$. Thus $p$ divides $x^2 + 1$ if and only if $(-1)^{(p-1)/2} = 1$ or $p \equiv 1 \pmod{4}$. Likewise, $p \mid x^2 - 2$ if and only if $(2/p) = 1$, or $p$ is congruent to $1$ or $-1$ modulo $8$.
\end{proof}

\begin{problem}
Show that any prime divisor of $x^4 - x^2 + 1$ is congruent to $1$ modulo $12$.
\end{problem}
\begin{proof}
Suppose that $p$ is a prime divisor of $x^4 - x^2 + 1$. Then $x^4 - x^2 + 1 \equiv 0 \pmod{p}$ so $4x^4 + 4x^2 + 4 \equiv = (2x^2 - 1)^2 + 3 \equiv 0 \pmod{p}$ and $(2x^2 - 1)^2 \equiv -3 \pmod{p}$. Likewise $x^4 - 2x^2 + 1 \equiv (x^2 - 1)^2 \equiv -x^2 \pmod{p}$. From this we know $1 = (-3/p) = (-1/p)(3/p) = (-1)^{(p-1)/2} (3/p)$. So either $(-1)^{(p-1)/2} = -1$ and $(3/p) = -1$ or $(-1)^{(p-1)/2} = 1$ and $(3/p) = 1$. But note that $1 = (-x^2/p) = (-1/p)(x/p)(x/p) = (-1/p)(x/p)^2 = (-1/p)$. Thus $(-1/p) = 1$ and therefore $(3/p) = 1$ as well. From quadratic reciprocity and the fact that $(p-1)/2 \equiv 0 \pmod{2}$, we know $(3/p) = (p/3) = 1$. But $1$ is the only nontrivial quadratic residue modulo $3$ so it follows that $p \equiv 1 \pmod{3}$ and $p \equiv 1 \pmod{4}$ so $p \equiv 1 \pmod{12}$.
\end{proof}

\begin{problem}
Use the fact that $U(\mathbb{Z}/p\mathbb{Z})$ is cyclic to give a direct proof that $(-3/p) = 1$ when $p \equiv 1 \pmod{3}$.
\end{problem}
\begin{proof}
Since $p \equiv 1 \pmod{3}$ we know $p = 3t + 1$ and $\phi(p) = 3t$. Since $U(\mathbb{Z}/p\mathbb{Z})$ is cyclic there exists some element $\rho$ which generates a subgroup of order $3$, i.e., that has order $3$. We also have $\rho^2 + \rho + 1 = \rho^3 + \rho^2 + \rho = \rho(\rho^2 + \rho + 1)$. Since $\rho \neq 1$ it must be the case that $\rho^2 + \rho + 1 = 0$. Thus $4\rho^2 + 4\rho + 4 = 0$ and $-3 = 4\rho^2 + 4\rho + 1 = (2\rho +1)^2$. Thus $(-3/p) = 1$.
\end{proof}

\begin{problem}
Using quadratic reciprocity find the primes for which $7$ is a quadratic residue. Do the same for $15$.
\end{problem}
\begin{proof}
We wish to solve $(7/p) = 1$ for $p$. By quadratic reciprocity and the fact that $7 \equiv 3 \pmod{4}$ we know $(7/p) = -(p/7)$. By a simple calculation, we see the quadratic residues modulo $7$ are $1$, $2$ and $4$. Thus we need $p \equiv 3 \pmod{7}$, $p \equiv 5 \pmod{7}$ or $p \equiv 6 \pmod{7}$. These are the primes for which $7$ is a quadratic residue.

We have precisely the same setup as before since $15 \equiv 3 \pmod{4}$. Thus $(15/p) = -(p/15)$. Another quick check shows that $1$, $4$, $6$, $9$, and $10$ are quadratic residues modulo $15$. Thus $p \equiv 2$, $p \equiv 7$, $p \equiv 8$, $p \equiv 11$, $p \equiv 12$, $p \equiv 13$ and $p \equiv 14$ are values of $p$ such that $15$ is a quadratic residue modulo $p$.
\end{proof}

\end{document}