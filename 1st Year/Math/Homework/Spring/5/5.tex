\documentclass{article}
\usepackage{amsmath,amsthm,amsfonts,amssymb,fullpage}

\begin{document}
\begin{flushright}
Kris Harper

MATH 16300

Mikl\'{o}s Ab\'{e}rt

May 27, 2008
\end{flushright}

\begin{center}
Homework 5
\end{center}

\begin{flushleft}

\textbf{Exercise 1}
\textsl{Show that for all $\alpha \in K$ and $v \in V$ we have
\[
\alpha \cdot_V 0 = 0 \cdot_V v = 0
\]
and if $\alpha \cdot_V v = 0$ then $\alpha = 0$ or $v = 0$.}
\begin{proof}
Let $\alpha \in K$ and $v \in V$. Then we have $0 +_V 0 = 0$ using V2 so $\alpha \cdot_V (0 +_V 0) = \alpha \cdot_V 0$. Then using V5 we have $(\alpha \cdot_V 0) +_V (\alpha \cdot_V 0) = \alpha \cdot_V 0$ and so
\[
\alpha \cdot_V 0 = \alpha \cdot_V 0 +_V 0 = \alpha \cdot_V 0 +_V (-\alpha \cdot_V 0) = 0.
\]
Similarly, $0 +_K 0 = 0$ using $V1$ and so $(0 +_K 0) \cdot_V v = 0 \cdot_V v$. Then using $V4$ we have $(0 \cdot_V v) +_V (0 \cdot_V v) = 0 \cdot_V v$ and so
\[
0 \cdot_V v = 0 \cdot_V +_V 0 = 0 \cdot_V v +_V (-0 \cdot_V v) = 0.
\]
\end{proof}

\textbf{Exercise 5}
\textsl{$X$ is linearly dependent if and only if there exists $x \in X$ that depends on $X \backslash x$.}
\begin{proof}
Suppose that $X$ is linearly dependent. Then there exist $a_1, \dots , a_n$ and $v_1, \dots , v_n$ with some $a_j \neq 0$ and $v_i \neq v_j$ such that
\[
\sum_{i=1}^n a_i v_i = 0.
\]
Then
\[
\sum_{i=1}^{j-1} a_i v_i + \sum_{i=j+1}^{n} a_i v_i = -a_j v_j
\]
and since $a_j \neq 0$ we have
\[
v_j = \sum_{i=1}^{j-1} \frac{a_i}{a_j} v_i + \sum_{i=j+1}^{n} \frac{a_i}{a_j} v_i.
\]
Since $v_j \in X$ and the right hand side of this is a linear combination of distinct elements of $X$, we have $v_j$  depends on $X \backslash x$.\newline

Conversely suppose that there exists $x \in X$ that depends on $X \backslash x$. Then
\[
x = \sum_{i=1}^{n} a_i v_i
\]
for $v_i \in X \backslash x$. Note that if $v_i = v_j$ for some $i, j$ then we can write $a_i v_i + a_j v_j = (a_i + a_j) v_i$ so that each term in $\sum_{i=1}^{n} a_i v_i$ is distinct. Then we have
\[
\sum_{i=1}^{n} a_i v_i - x = 0
\]
and since $v_i \neq x$ for all $i$, each term is distinct and the coefficient for $x$ is not $0$ so we have a non-trivial linear combination of distinct elements of $X$ which equals $0$ so $X$ is linear dependent.
\end{proof}

\textbf{Exercise 6}
\textsl{Let $X$ be a linearly independent subset. Then $v \in V$ is depends on $X$ if and only if $X \cup \{v\}$ is linearly dependent.}
\begin{proof}
Let $v$ depend on $X$. Then
\[
v = \sum_{i=1}^n a_i v_i
\]
for $v_i \in X$. Note that since $X$ is linearly independent every non-trivial linear combination of elements of $X$ is nonzero. Thus it's not the case that $v = v_i$ for some $i$. If $v = a_j v_j$ for some $j$ and $a_j \neq 1$ then we can write $v - a_j v_j = (1 - a_j) v = \sum_{i=1}^{j-1} a_i v_i + \sum_{i=j+1}^n a_i v_i$. Thus we can assume that $v \neq a_j v_j$. Then
\[
\sum_{i=1}^n a_i v_i - v = 0
\]
and since $a_i \neq a_j$ for all $i, j$ and $v \neq v_i$ for all $i$ we have a linear combination of distinct elements of $X \cup \{v\}$ which equals $0$. Thus $X \cup \{v\}$ is linearly dependent.\newline

Conversely assume that $X \cup \{v\}$ is linearly dependent. Then there exists distinct elements of $X \cup \{v\}$, $v_1, \dots , v_n$, and $a_1, \dots , a_n$ with some $a_j \neq 0$ such that
\[
\sum_{i=1}^n a_i v_i = 0.
\]
Since $X$ is linearly independent, every non-trivial linear combination of elements of $X$ is nonzero. Thus we must have $v = v_j$ for some $j$ and $a_j \neq 0$. Then we have
\[
-(\sum_{i=1}^{j-1} \frac{a_i}{a_j} v_i + \sum_{i=j+1}^n \frac{a_i}{a_j} v_i) = v
\]
and so $v$ depends on $X$ since $v_i \neq v_j$ for all $i, j$.
\end{proof}

\textbf{Exercise 8}
\textsl{The intersection of an arbitrary number of subspaces is a subspace.}
\begin{proof}
Let $S = \{U_1, U_2, \dots\}$ be a set of subspaces and consider
\[
T = \bigcap_{U_i \in S} U_i.
\]
Let $u,v \in T$ and let $a \in K$. Then we have $u,v \in U_i$ for all $i$ and so $u+v \in U_i$ for all $i$ which means $u + v \in T$. Also $au \in U_i$ for all $i$ and so $au \in T$. Thus $T$ is a subspace.
\end{proof}

\textbf{Exercise 10}
\textsl{For an subset $X \subseteq V$ we have
\[
\langle X \rangle = \left \{ \sum_{i=1}^n a_i v_i \mid \textup{$n \in \mathbb{N}$, $a_i \in K$, $v_i \in X$} \right \}.
\]}
\begin{proof}
Let
\[
x \in \left \{ \sum_{i=1}^n a_i v_i \mid \text{$n \in \mathbb{N}$, $a_i \in K$, $v_i \in X$} \right \}.
\]
Then $x$ is a linear combination of elements of $X$. But then for all subspaces $U$ such that $X \subseteq U$ we have $x \in U$ since $U$ is closed under addition and scalar multiplication. Thus
\[
\left \{ \sum_{i=1}^n a_i v_i \mid \text{$n \in \mathbb{N}$, $a_i \in K$, $v_i \in X$} \right \} \subseteq \bigcap_{\substack{\text{$U$ is a subspace in $V$} \\ X \subseteq U}} U = \langle X \rangle.
\]
Now let
\[
x \in \bigcap_{\substack{\text{$U$ is a subspace in $V$} \\ X \subseteq U}} U.
\]
Then since all subspaces $U$ are closed under addition and scalar multiplication we know $x$ is some linear combination of elements which are in $U$ for all $X \subseteq U$. But since this is true for all $U$ such that $X \subseteq U$, we must have
\[
x \in \left \{ \sum_{i=1}^n a_i v_i \mid \text{$n \in \mathbb{N}$, $a_i \in K$, $v_i \in X$} \right \}
\]
and so
\[
\left \{ \sum_{i=1}^n a_i v_i \mid \text{$n \in \mathbb{N}$, $a_i \in K$, $v_i \in X$} \right \} = \langle X \rangle.
\]
\end{proof}

\textbf{Exercise 13}
\textsl{A subset $X \subseteq V$ is a basis if and only if every element of $V$ can be obtained as a unique linear combination of elements of $X$.}
\begin{proof}
Suppose $X \subseteq V$ is a basis. Then
\[
V = \langle X \rangle = \left \{ \sum_{i=1}^n a_i v_i \mid \text{$n \in \mathbb{N}$, $a_i \in K$, $v_i \in X$} \right \}
\]
which means that each element, $v$, of $V$ is some linear combination of elements of $X$. But then $X \cup \{v\}$ is linearly dependent and so the linear combination must be unique. Conversely suppose that for all $v \in V$ we have $v$ is a unique linear combination of elements in $X$. Then
\[
v \in \left \{ \sum_{i=1}^n a_i v_i \mid \text{$n \in \mathbb{N}$, $a_i \in K$, $v_i \in X$} \right \} = \langle X \rangle.
\]
Since this is true for all $v \in V$ we have $V \subseteq \langle X \rangle$. But $X \subseteq V$ and so we must have $V = \langle X \rangle$ so $X$ is a basis for $V$.
\end{proof}

\textbf{Exercise 14}
\textsl{The following are equivalent for a subset $X \subseteq V$:\\
1) $X$ is a basis;\\
2) $X$ is a maximal independent subset;\\
3) $X$ is a minimal spanning subset.}

\textbf{Exercise 15}
\textsl{Let $X \subseteq V$ be a finite linearly independent subset and let $Y \subseteq V$ be a finite spanning subset. Then $|X| \leq |Y|$.}

\textbf{Exercise 16}
\textsl{If $V$ is finitely generated then it has a basis and every basis has the same number of elements.}
\begin{proof}
Suppose that $V$ is finitely generated by $X$. Then we have
\[
\langle X \rangle = \left \{ \sum_{i=1}^n a_i v_i \mid \text{$n \in \mathbb{N}$, $a_i \in K$, $v_i \in X$} \right \} = V
\]
and so every element in $V$ is linear combination of elements of $X$ which means that a subset of $X$ serves as a basis for $V$. Also, if two basis, $X$ and $Y$, have different numbers of elements then we can use $X$ to write one element of $Y$ as a linear combination of the other elements of $Y$ which means $Y$ is not linearly independent.
\end{proof}

\textbf{Exercise 18}
\textsl{Assume that $V$ is finitely generated. Then any linearly independent subset of $V$ can be extended to be a basis and any spanning subset of $V$ contains a basis.}
\begin{proof}
Let $V$ be finitely generated by $X$. Exercise 16 shows that a spanning subset of $V$ contains a basis since a spanning subset of $V$ will be linear combinations of elements of $X$. A linearly independent subset of $V$, $Y$ can be made a basis by adding the vectors of $X \backslash Y$ to $Y$.
\end{proof}

\end{flushleft}
\end{document}