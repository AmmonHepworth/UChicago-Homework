~\documentclass{article}
\usepackage{amsmath,amsthm,amsfonts,fullpage}

\begin{document}
\begin{flushright}
Kris Harper

MATH 16100

Mikl\'{o}s Ab\'{e}rt

October 16, 2007
\end{flushright}

\begin{center}
Homework 3
\end{center}

\begin{flushleft}

1. \textsl{Let $f$ and $g$ be defined by
\[
f(x) = x^3 + 1 \; (x \in \mathbb{R}) \text{ and } g(x) = \frac{1}{1-x} \; (x \in \mathbb{R}\backslash \{1\})
\]
Find the functions
\begin{itemize}{}
\item[a)] $f \circ g$
\item[b)] $g \circ f$
\item[c)] $f^{-1}$
\item[d)] $f^{-1} \circ g \circ f$
\end{itemize}}

\begin{itemize}
\item[a)]
\begin{align*}
f \circ g &= f(g(x)) = f\left( \frac{1}{1-x} \right) = \left(\frac{1}{1-x} \right)^3 + 1 = \frac{1 + \left(1-x\right)^3}{\left(1-x\right)^3} = \frac{-x^3 + 3x^2 - 3x + 2}{(-1)(x-1)^3}\\
													&= \frac{(2-x)(x^2-x+1)}{(-1)(x-1)^3} = \frac{(x-2)(x^2-x+1)}{(x-1)^3} \; (x \in \mathbb{R}\backslash \{1\})
\end{align*}\newline

\item[b)]
\[
g \circ f = g(f(x)) = g(x^3 + 1) = \frac{1}{1-(x^3 + 1)} = \frac{1}{-x^3} = \; (x \in \mathbb{R}\backslash \{0\})
\]

\item[c)]
We have $f(x) = x^3 + 1$ and so $x^3 = f(x) - 1$ which means $x = (f(x) - 1)^{\frac{1}{3}}$. We now let $x = f^{-1}(x)$ and $f(x) = x$ so we have
\[
f^{-1}(x) = (x-1)^{\frac{1}{3}} \; (x \in \mathbb{R})
\]

\item[d)]
\[
f^{-1} \circ g \circ f = f^{-1}(g(f(x))) = f^{-1}\left(\frac{1}{-x^3}\right) = \left( \frac{1}{-x^3} - 1 \right)^{\frac{1}{3}} = \left( \frac{1 + x^3}{-x^3} \right)^{\frac{1}{3}} = \frac{(1+x^3)^{\frac{1}{3}}}{-x} \; (x \in \mathbb{R}\backslash \{0\})
\]
\end{itemize}

2. \textsl{Negate the following sentences:
\begin{itemize}
\item[a)] Every horse is red.
\item[b)] In every war there is a hero who does not die.
\item[c)] If it is raining outside then frogs laugh.
\item[d)] If my grandmother had wheels, she would be a carriage.
\item[e)] For every horse that jumps higher than the holy rabbit, there exists a lion that runs faster than the ugly goat and wants to eat that horse.
\end{itemize}}

\begin{itemize}
\item[a)] There exists a horse which is not red.
\item[b)] There is a war in which every hero dies.
\item[c)] It is raining outside and frogs do not laugh.
\item[d)] My grandmother has wheels and she is not a carriage.
\item[e)] There exists a horse that does not jump higher than the holy rabbit such that all lions do not run faster than the ugly goat or do not want to eat that horse.
\end{itemize}

3. Show that for all $n \in \mathbb{N}$ we have
\[
\sum_{k=0}^{n} \binom{n}{k} \binom{n}{n-k} = \binom{2n}{n}
\]
\begin{proof}
We use induction on $n$. We first show that for $n=1$ we have $\sum_{k=0}^1 \binom{1}{k} \binom{1}{1-k} = \binom{1}{0} \binom{1}{1} + \binom{1}{1} \binom{1}{0} = 2 = \binom{2}{1}$. We now assume that for some $j \in \mathbb{N}$ the statement holds and we show that this implies it is true for $j+1$. We see that
\begin{align*}
\sum_{k=0}^{j+1} \binom{j+1}{k} \binom{j+1}{j+1-k} &= \sum_{k=0}^{j+1} \left( \frac{(j+1)!}{k! (j+1-k)!} \right) \left( \frac{(j+1)!}{(j+1-k)! (j+1 - (j+1-k))!} \right) \\
&= \sum_{k=0}^{j+1} \binom{j+1}{k}^2 \\
&= \sum_{k=0}^{j+1} \left( \binom{j}{k} + \binom{j}{k-1} \right)^2\\
&= \sum_{k=0}^{j+1} \left( \binom{j}{k}^2 + 2\binom{j}{k} \binom{j}{k-1} + \binom{j}{k-1}^2 \right) \\
&= \sum_{k=0}^{j} \binom{j}{k}^2 + \sum_{k=0}^{j+1} \binom{j}{k-1}^2 + 2\sum_{k=0}^{j+1} \binom{j}{k} \binom{j}{k-1} + \binom{j}{j+1}^2 \\
&= \sum_{k=0}^{j} \binom{j}{k}^2 + \sum_{k=0}^{j} \binom{j}{k}^2 + 2\sum_{k=0}^{j}\binom{j}{k} \binom{j}{k-1} + \binom{j}{-1}^2 + \binom{j}{j+1} \binom{j}{j} \\
&= \binom{2j}{j} + \binom{2j}{j} + 2\binom{2j}{j+1} \\
&= 2 \left( \binom{2j}{j} + \binom{2j}{j+1} \right) \\
&= 2 \binom{2j+1}{j+1} \\
&= \left( \frac{2j+2}{j+1} \right) \left( \frac{(2j+1)!}{(j+1)!j!} \right) \\
&= \left( \frac{(2(j+1))!}{(j+1)! (2(j+1)-(j+1))!} \right) \\
&= \binom{2(j+1)}{j+1} \\
\end{align*}
Since the statement is true for $n=1$ and since when it is true for $j \in \mathbb{N}$ it is also true for $j+1$, we see that it must be true for all $n$.
\end{proof}

4. \textsl{Prove that every natural number is either $1$ or a product of primes.}
\begin{proof}
Suppose that there exists a natural number which is not $1$ and is not a product of primes. By the Well Ordering Principle, there exists a least natural number $n$ which has this property. Then $n$ is not prime and so $n = ab$ for some $a,b \in \mathbb{N}$. But then $\frac{n}{a} = b$ and $\frac{n}{b} = a$ and since $a > 1$ and $b > 1$ we have $a<n$ and $b<n$. But then $a$ and $b$ must be a product of primes since $a$ and $b$ are less than $n$, but $n$ is the least natural number which is not a product of primes. But then $ab$ is a product of primes and $ab=n$. This is a contradiction.
\end{proof}

5. \textsl{Show that this decomposition is unique.}
\begin{proof}
Let $S$ be the set of natural numbers such that each element of $S$ does not have a unique decomposition. Then by the Well Ordering Principle there exists a least element $k$ of $S$. By definition $k=a_1a_2a_3 \cdots a_n$ and $k=b_1b_2b_3 \cdots b_m$ for primes $a_i,b_i$ and $n,m \in \mathbb{N}$. But the fraction $(a_1a_2 \dots a_n) / (b_1b_2 \cdots b_m)$ must equal $k/k=1$ and so there must exists some $a_i$ and $b_j$ such that $a_i=b_j$. But then consider $k/a_i$. We see that $k/a_i=a_1a_2 \cdots a_{i-1}a_{i+1} \cdots a_n$ but since $a_i = b_j$ we have $k/a_i = b_1b_2 \cdots b_{j-1}b_{j+1} \cdots b_m$. Thus, $k/a_i$ has two different factorizations, but since $a_i > 1$ we have $k/a_i < k$ which is a contradiction and so $S$ must be empty.
\end{proof}

6. \textsl{Continue the following sequence: $7,4,27,3,12,5...$}\newline

We see that if the terms are numbered $a_1,a_2,\dots ,a_n$ then the positive difference between $a_n$ for an odd $n$ and $a_{n+2}$ is five times the value of $a_{n+1}$. In other words if $n$ is odd then $| a_n - a_{n+2} | = 5 a_{n+1}$. By this logic the next term, $a_7$, would be $12 + 5a_6 = 12 + 25 = 37$.





\end{flushleft}
\end{document}