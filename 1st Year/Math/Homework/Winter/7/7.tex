\documentclass{article}
\usepackage{amsmath,amsthm,amsfonts,amssymb,fullpage}

\begin{document}
\begin{flushright}
Kris Harper

MATH 16200

Mikl\'{o}s Ab\'{e}rt

February 26, 2008
\end{flushright}

\begin{center}
Homework 7
\end{center}

\begin{flushleft}

\textbf{Exercise 1}
\textsl{Is there a continuous real function $f \; : \; \mathbb{R} \rightarrow \mathbb{R}$ that takes on every real number an even number of times?}\newline

No.
\begin{proof}
Suppose there is such a function $f$. Then $f(0)$ takes on $n=2k$ values for some $k \in \mathbb{N}$.  We know that $f$ cannot take on every real number zero times because it has to map every real number to something. Thus, if $k=0$ then we can change our zero axis to be some other number $r$ such that $f(r)$ takes on $2k$ values and $k \neq 0$. Therefore, we can effectively $f(0)$ takes on $n=2k$ values and $k \neq 0$. Let these values be $x_1, x_2, \dots x_n$. Assume that $f(x) \geq 0$ for $x \in [x_1;x_n]$. Consider the closed intervals $[x_1;x_2], [x_2;x_3], \dots [x_{n-1};x_n]$. For each interval $[x_i;x_{i+1}]$ there exists $c_i$ such that $f(c_i) \geq f(x)$ for $x \in [x_i ; x_{i-1}]$. Note that if there exists only one $c_i$ for each interval then every real number between $0$ and $f(c_i)$ is taken on twice in that interval. If there exists more than one $c_i$ for an interval then there exists a minimum between each pair of consecutive $c_i$  which may or may not be taken on an even number of times, but every other number between $0$ and $f(c_i)$ is. There exists a maximum value which will taken on either an odd or an even number of times. If it's taken on an odd number of times then $f$ will have to take it on again before $x_1$ or after $x_n$. Assume that it takes it on again before $x_1$ then $f$ won't take it on again after $x_n$ so there exists some number between this maximum and $0$ which will be taken on an odd number of times. If $f$ takes on this maximum an even number of times then there exists some minimum in some interval $[x_i ; x_{i+1}]$ where $f$ takes on the maximum twice. This value will be taken on an odd number of times. This is a contradiction.
\end{proof}

\textbf{Exercise 2}
\textsl{$\sum_{n=1}^{\infty} \frac{1}{n(n+1)}=1$.}
\begin{proof}
Consider the partial sum
\begin{align*}
s_n & = \sum_{k=1}^{n} \frac{1}{k(k+1)} \\
	& = \sum_{k=1}^{n} \frac{k+1}{k(k+1)} - \frac{k}{k(k+1)} \\
	& = \sum_{k=1}^{n} \frac{1}{k} - \frac{1}{k+1} \\
	& = (\frac{1}{1} - \frac{1}{2}) + (\frac{1}{2}-\frac{1}{3}) + (\frac{1}{3} - \frac{1}{4}) + \dots + (\frac{1}{n} - \frac{1}{n+1}) \\
	& = 1 - \frac{1}{n+1}.
\end{align*}
Then take
\[
\lim_{n \rightarrow \infty} s_n = \lim_{n \rightarrow \infty} 1 - \lim_{n \rightarrow \infty} \frac{1}{n+1} = 1 + 0 = 1.
\]
\end{proof}

\textbf{Exercise 3}
\textsl{$\sum_{n=1}^{\infty} \frac{1}{n^2}$ is convergent.}
\begin{proof}
Note that for $n \in \mathbb{N}$ we have $n > n-1$ so $n^2 > n(n-1)$ and $\frac{1}{n^2} < \frac{1}{n(n-1)}$. Also note that
\[
\sum_{k = 1}^{\infty} \frac{1}{k(k+1)} = \frac{1}{2} + \sum_{k=2}^{\infty} \frac{1}{k(k+1)} = \sum_{k=2}^{\infty} \frac{1}{k(k-1)}.
\]
And since $\frac{1}{n^2} < \frac{1}{n(n-1)}$ we have
\[
\sum_{k=2}^{n} \frac{1}{k^2} < \sum_{k=2}^{n} \frac{1}{k(k-1)} < \sum_{k=2}^{\infty} \frac{1}{k(k-1)} = \sum_{k=1}^{\infty} \frac{1}{k(k+1)} = 1.
\]
and so
\[
1 + \sum_{k=2}^{n} \frac{1}{k^2} = \sum_{k=1}^{n} \frac{1}{k^2} < 2.
\]
But the partial sums of this series are an increasing sequence because $\frac{1}{n^2} > 0$ for all $n \in \mathbb{N}$. Then the partial sums are an increasing bounded sequence so they converge.
\end{proof}

\textbf{Exercise 4}
\textsl{$\sum_{n=1}^{\infty} \frac{1}{n^3}$ is convergent.}
\begin{proof}
We have $n \geq 1$ for $n \in \mathbb{N}$ and so $n^3 \geq n^2$. Then $\frac{1}{n^3} \leq \frac{1}{n^2}$ which means
\[
\sum_{k=1}^{n} \frac{1}{k^3} \leq \sum_{k=1}^{n} \frac{1}{k^2} < 2.
\]
So again we have the partial sums as a bounded increasing sequence which means they converge.
\end{proof}

\textbf{Exercise 5}
\textsl{A frog takes off? his slippers and starts to jump around on the plane. At every step, he jumps twice as far as his last jump (he chooses the direction). What is the minimal number of jumps that can get him back to his slippers?}
\begin{proof}
He won't get back to his slippers. Suppose he takes off his slippers at the origin and that his first jump has distance $l$. Assume that there exists $n$ jumps which will get him back to the origin. Then his last jump is equal to $(2^n)l$. Note that from the Triangle Inequality we know that his displacement from the origin is always less than or equal to his displacement had he gone in a single direction. That is, after $n-1$ jumps, he can be no farther than $l \cdot \sum_{k=0}^{n-1} 2^k$ away from the origin. But then using induction we can show $2^n = \sum_{k=0}^{n-1} 2^k + 1$. We see that for $n=1$ we have $2 = 2^0 + 1 = 2$. Assume that for $n \in \mathbb{N}$ we have $2^n = \sum_{k=0}^{n-1} 2^k + 1$. Then $2^{n+1} = 2^n + 2^n = \sum_{k=0}^{n-1} 2^k + 2^n + 1 = \sum_{k=0}^{n} 2^k + 1$ as desired. So we have his final jump will be farther than his maximal distance from the origin before this jump which is a contradiction.
\end{proof}

\textbf{Exercise 6}
\textsl{What is the moral of the prisoner's dilemma?}\newline

Don't be too confident in matters of life and death.

\end{flushleft}
\end{document}