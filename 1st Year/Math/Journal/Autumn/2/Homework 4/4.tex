\documentclass{article}
\usepackage{amsmath,amsthm,amsfonts,amssymb,fullpage}

\begin{document}
\begin{flushleft}

\Large

Sheet 4: Revenge of $\mathbb{Q}$\newline

\normalsize

Let $\mathbb{Z}$ denote the integers. Let
\[
P = \{(a,b) \mid a,b \in \mathbb{Z}, b \neq 0\}
\]
and let the relation $\sim$ be defined on $P$ by
\[
(a_1,b_1) \sim (a_2,b_2) \text{ if } a_1b_2=a_2b_1
\]

\textbf{Theorem 1}
\textsl{$\sim$ is an equivalence relation on $P$.}
\begin{proof}
Let $(a,b) \in P$. Then $ab=ab$ and so $(a,b) \sim (a,b)$. Hence, reflexivity applies to $\sim$. Now let $(a_1,b_1), (a_2,b_2) \in P$ such that $(a_1,b_1) \sim (a_2,b_2)$. Then $a_1b_2=a_2b_1$ and so $a_2b_1=a_1b_2$. Thus $(a_2,b_2) \sim (a_1,b_1)$ and so symmetry holds for $\sim$. Now suppose $(a_1,b_1), (a_2,b_2), (a_3,b_3) \in P$ such that $(a_1,b_1) \sim (a_2,b_2)$ and $(a_2,b_2) \sim (a_3,b_3)$. Then $a_1b_2=a_2b_1$ and $a_2b_3=a_3b_2$. Multiplying the first equation by $b_3$ we have $a_1b_2b_3=a_2b_1b_3$. But then since $a_2b_3=a_3b_2$ we have $a_1b_2b_3=a_3b_1b_2$ and dividing by $b_2 \neq 0$ we have $a_1b_3=a_3b_1$. Therefore $(a_1,b_1) \sim (a_3,b_3)$ implying transitivity and since all three conditions have been met, $\sim$ is an equivalence relation on $P$.
\end{proof}

Now let $\mathbb{Q}$ denote the set of $\sim$-equivalence classes of $P$. We now define two operators, $+$ and $\cdot$ as follows. For $X,Y \in \mathbb{Q}$ let $(a_1,b_2) \in X$ and $(a_2,b_2) \in Y$. Let
\[
X + Y = \overline{(a_1b_2 + a_2b_1,b_1b_2)}
\]
and let
\[
X \cdot Y = \overline{(a_1a_2,b_1b_2)}.
\]
We now show that these definitions are well-defined.\newline

\textbf{Theorem 2}
\textsl{If $(a_1,b_1) \sim (c_1,d_1)$ and $(a_2,b_2) \sim (c_2,d_2)$ then
\[
(a_1b_2+a_2b_1,b_1b_2) \sim (c_1d_2+c_2d_1,d_1d_2)
\]
and
\[
(a_1a_2,b_1b_2) \sim (c_1c_2,d_1d_2).
\]}
\begin{proof}
Let $(a_1,b_1) \sim (c_1,d_1)$ and $(a_2,b_2) \sim (c_2,d_2)$. Then we have $a_1d_1=b_1c_1$ and $a_2d_2=b_2c_2$. We multiply the first equation by $b_2d_2$ so we have $a_1b_2d_1d_2 = b_1b_2c_1d_2$ and we multiply the second equation by $b_1d_1$ so we have $a_2b_1d_1d_2 = b_1b_2c_2d_1$. Now we add the two new equations together so we have $a_1b_2d_1d_2 + a_2b_1d_1d_2 = b_1b_2c_1d_2 + b_1b_2c_2d_1$ and so $(a_1b_2+a_2b_1)d_1d_2 = (c_1d_2 + c_2d_1)b_1b_2$ which implies $(a_1b_2+a_2b_1,b_1b_2) \sim (c_1d_2+c_2d_1,d_1d_2)$. Similarly, if we multiply $a_1d_1=b_1c_1$ and $a_2d_2=b_2c_2$ together we have $a_1a_2d_1d_2=b_1b_2c_1c_2$ and so $(a_1a_2,b_1b_2) \sim (c_1c_2,d_1d_2)$.
\end{proof}

\textbf{Theorem 3 (Associativity of Addition)}
\textsl{For all $p,q,r \in \mathbb{Q}$ we have $(p+q)+r = p+(q+r)$.}
\begin{proof}
Let $p,q,r \in \mathbb{Q}$ such that $(p_1,p_2) \in p$, $(q_1,q_2) \in q$ and $(r_1,r_2) \in r$. Then we see that
\begin{align*}
(p+q)+r&=\left(\overline{(p_1,p_2)}+\overline{(q_1,q_2)}\right)+\overline{(r_1,r_2)} \\
		&=\overline{(p_1q_2+p_2q_1,p_2q_2)}+\overline{(r_1,r_2)} \\
		&=\overline{((p_1q_2+p_2q_1)r_2+p_2q_2r_1,p_2q_2r_2)} \\
		&=\overline{(p_1q_2r_2+p_2q_1r_2+p_2q_2r_1,p_2q_2r_2)} \\
		&=\overline{((q_1r_2+q_2r_1)p_2+p_1q_2r_2,p_2q_2r_2)} \\
		&=p+\overline{(q_1r_2+q_2r_1,q_2r_2)} \\
		&=p+(q+r).
\end{align*}
\end{proof}

\textbf{Theorem 4 (Commutativity of Addition)}
\textsl{For all $p,q \in \mathbb{Q}$ we have $p+q=q+p$.}
\begin{proof}
Let $p,q \in \mathbb{Q}$ such that $(p_1,p_2) \in p$ and $(q_1,q_2) \in q$. Then we have $p+q=\overline{(p_1,p_2)}+\overline{(q_1,q_2)}=\overline{(p_1q_2+p_2q_1,p_2q_2)}=\overline{(q_1p_2+q_2p_1,q_2p_2)}=\overline{(q_1,q_2)}+\overline{(p_1,p_2)}=q+p$.
\end{proof}

\textbf{Theorem 5 (Additive Identity)}
\textsl{There exists an $n \in \mathbb{Q}$ such that for all $p \in \mathbb{Q}$ we have $n+p=p$. Show that $n$ is unique.}
\begin{proof}
We see that if we let $n \in \mathbb{Q}$ such that $n = \overline{(0,1)}$ and if we let $(p_1,p_2) \in p$ for some $p \in \mathbb{Q}$ then we have $n+p=\overline{(0,1)}+\overline{(p_1,p_2)}=\overline{((0)p_2+(1)p_1,(1)p_2)}=\overline{(p_1,p_2)}=p$. Now suppose there exist two additive identities such that for all $p \in \mathbb{Q}$ we have $n_1+p=p$ and $n_2+p=p$. Then we have $n_2=n_1+n_2 = n_2+n_1=n_1$ and so $n_1=n_2$. Thus, the additive identity is unique.
\end{proof}

From now on we will call the additive identity $0$.\newline

\textbf{Theorem 6 (Additive Inverse)}
\textsl{For all $p \in \mathbb{Q}$ there exists $q \in \mathbb{Q}$ such that $p+q=0$. Show that $q$ is unique.}
\begin{proof}
Let $p \in \mathbb{Q}$ such that $(p_1,p_2) \in p$. Then we choose $q=\overline{(-p_1,p_2)}$ for $q \in \mathbb{Q}$. Then we have $p+q=\overline{(p_1,p_2)}+\overline{(-p_1,p_2)}=\overline{(p_1p_2+-p_1p_2,p_2p_2)}=\overline{(0,p_2p_2)}=\overline{(0,1)}=0$ since $(0)p_2p_2=(0)(1)$. Now suppose there exist two additive inverses so that $p+n_1=0$ and $p+n_2=0$. Then we have $p+n_1=p+n_2$ and adding $\overline{(-p_1,p_2)}$ to both sides we have
\[
\overline{(-p_1,p_2)}+\overline{(p_1,p_2)}+n_1=\overline{(-p_1p_2+p_1p_2,p_2p_2)}+n_1=0+n_1=n_1
\]
on the left and
\[
\overline{(-p_1,p_2)}+\overline{(p_1,p_2)}+n_2=\overline{(-p_1p_2+p_1p_2,p_2p_2)}+n_2=0+n_2=n_2
\]
on the right. So $n_1=n_2$ and the additive inverse is unique.
\end{proof}

From now on we will call the additive inverse for $p$, $-p$.\newline

\textbf{Theorem 7 (Associativity of Multiplication)}
\textsl{For all $p,q,r \in \mathbb{Q}$ we have $(p \cdot q) \cdot r = p \cdot (q \cdot r)$.}
\begin{proof}
Let $p,q,r \in \mathbb{Q}$ such that $(p_1,p_2) \in p$, $(q_1,q_2) \in q$ and $(r_1,r_2) \in r$. Then we have $(p \cdot q) \cdot r=\left(\overline{(p_1,p_2)} \cdot \overline{(q_1,q_2)}\right) \cdot \overline{(r_1,r_2)}=\overline{(p_1q_1,p_2q_2)} \cdot \overline{(r_1,r_2)}=\overline{(p_1q_1r_1,p_2q_2r_2)}=p \cdot \overline{(q_1r_1,q_2r_2)}=p \cdot (q \cdot r)$.
\end{proof}

\textbf{Theorem 8 (Commutativity of Multiplication)}
\textsl{For all $p,q \in \mathbb{Q}$ we have $p \cdot q = q \cdot p$.}
\begin{proof}
Let $p,q \in \mathbb{Q}$ such that $(p_1,p_2) \in p$ and $(q_1,q_2) \in q$. Then $p \cdot q = \overline{(p_1,p_2)} \cdot \overline{(q_1,q_2)} = \overline{(p_1q_1,p_2q_2)} = \overline{(q_1p_1,q_2p_2)} = \overline{(q_1,q_2)} \cdot \overline{(p_1,p_2)} = q \cdot p$.
\end{proof}

\textbf{Theorem 9 (Multiplicative Identity)}
\textsl{There exists $e \in \mathbb{Q}$ such that for all $p \in \mathbb{Q}$ we have $e \cdot p=p$.}
\begin{proof}
Let $p \in \mathbb{Q}$ such that $(p_1,p_2) \in p$ and let $e \in \mathbb{Q}$ such that $e = \overline{(1,1)}$. Then we have $e \cdot p = \overline{(1,1)} \cdot \overline{(p_1,p_2)} = \overline{(p_1(1),p_2(1))} = p$. Suppose there exist two multiplicative identities $e_1$ and $e_2$ such that for all $p \in \mathbb{Q}$, $e_1 \cdot p = p$ and $e_2 \cdot p = p$. Then we have $e_1 = e_2 \cdot e_1$ and $e_2 = e_1 \cdot e_2 = e_2 \cdot e_1$. So we have $e_1 = e_2$ and so the multiplicative identity is unique.
\end{proof}

From now on we will call the multiplicative identity $1$.\newline

\textbf{Theorem 10 (Multiplicative Inverse)}
\textsl{For all $p \in \mathbb{Q}$ with $p \neq 0$ there exists $q \in \mathbb{Q}$ such that $p \cdot q=1$.}
\begin{proof}
Let $p \in \mathbb{Q}$ such that $(p_1,p_2) \in p$ and since $p_1 \neq 0$ let $q \in \mathbb{Q}$ such that $(p_2,p_1) \in q$. Then we see that $p \cdot q = \overline{(p_1,p_2)} \cdot \overline{(p_2,p_1)} = \overline{(p_1p_2,p_1p_2)}=\overline{(1,1)}=1$. Now suppose there are two multiplicative inverses for some $p \in \mathbb{Q}$ such that $p \cdot q_1=1$ and $p \cdot q_2=1$. Then, multiplying both equations by $\overline{(p_2,p_1)}$, we have $q_1=\overline{(1,1)} \cdot q_1=\overline{(p_1p_2,p_1p_2)} \cdot q_1=\overline{(p_2,p_1)} \cdot \overline{(p_1,p_2)} \cdot q_1=\overline{(p_2,p_1)} \cdot \overline{(p_1,p_2)} \cdot q_2=\overline{(p_1p_2,p_1p_2)} \cdot q_2=\overline{(1,1)} \cdot q_2=q_2$. So the multiplicative inverse is unique.
\end{proof}

From now on we will call the multiplicative inverse for $p$, $p^{-1}$.\newline

\textbf{Theorem 11 (Distributivity)}
\textsl{For all $p,q,r \in \mathbb{Q}$ we have $p \cdot (q+r)=p \cdot q + p \cdot r$.}
\begin{proof}
Let $p,q,r \in \mathbb{Q}$ such that $(p_1,p_2) \in p$, $(q_1,q_2) \in q$ and $(r_1,r_2) \in r$. Then we have
\begin{align*}
p \cdot (q+r) &= \overline{(p_1,p_2)} \cdot \left(\overline{(q_1,q_2)} + \overline{(r_1,r_2)}\right) \\
			   &= \overline{(p_1,p_2)} \cdot \overline{(q_1r_2+q_2r_1,q_2r_2)}\\
			   &= \overline{(p_1q_1r_2+p_1q_2r_1,p_2q_2r_2)} \\
			   &= \overline{(p_1q_1r_2+p_1q_2r_1,p_2q_2r_2)} \cdot \overline{(p_2,p_2)} \\
			   &= \overline{(p_1p_2q_1r_2+p_1p_2q_2r_1,p_2p_2q_2r_2)} \\
			   &= \overline{(p_1q_1,p_2q_2)} + \overline{(p_1r_1,p_2r_2)} \\
			   &= \overline{(p_1,p_2)} \cdot \overline{(q_1,q_2)} + \overline{(p_1,p_2)} \cdot \overline{(r_1,r_2)} \\
			   &= p \cdot q + p \cdot r. \\
\end{align*}
\end{proof}

\textbf{Theorem 12}
\textsl{The function $f\; : \; \mathbb{Z} \rightarrow \mathbb{Q}$ where $f(n)=\overline{(n,1)}$ is injective.}
\begin{proof}
Let $a,b \in \mathbb{Z}$ such that $f(a)=f(b)$. Then we have $\overline{(a,1)}=\overline{(b,1)}$ and so $(a,1) \sim (b,1)$ which implies $a=b$.
\end{proof}

\textbf{Theorem 13}
\textsl{For all $m,n \in \mathbb{Z}$ we have
\[
f(m+n)=f(m)+f(n) \text{ and } f(mn)=f(m) \cdot f(n).
\]}
\begin{proof}
Let $m,n \in \mathbb{Z}$. Then we have $f(m+n)=\overline{(m+n,1)}=\overline{(m(1)+n(1),(1)(1))}=\overline{(m,1)}+\overline{(n,1)}=f(m)+f(n)$. Additionally we see that $f(mn)=\overline{(mn,(1)(1))}=\overline{(m,1)} \cdot \overline{(n,1)}=f(m) \cdot f(n)$.
\end{proof}

\textbf{Theorem 14}
\textsl{For every rational number $r \in \mathbb{Q}$ there exist $m,n \in \mathbb{Z}$ such that $n \neq 0$ and $r=mn^{-1}$.}
\begin{proof}
Let $r \in \mathbb{Q}$ such that $(m,n) \in r$ (since $r$ is nonempty). Then we see $m,n \in \mathbb{Z}$. Thus we can write $m=\overline{(m,1)}$ and $n=\overline{(n,1)}$. And so $n^{-1}=\overline{(1,n)}$ since $n \neq 0$ and we have $m \cdot n^{-1} = \overline{(m,1)} \cdot \overline{(1,n)} = \overline{(m,n)} = r$.
\end{proof}

\textbf{Lemma 15}
\textsl{Any element in $\mathbb{Q}$ can be written as $\overline{(a,b)}$ with $b>0$.}
\begin{proof}
Let $\overline{(a,b)} \in \mathbb{Q}$. There are two cases:\newline

\textsl{Case 1:} If $b>0$ then we are done.\newline

\textsl{Case 2:} If $b<0$ then we have $a(-b)=-ab=(-a)b$ and so $(a,b) \sim (-a,-b)$. Thus $\overline{(a,b)} = \overline{(-a,-b)}$ and $-b>0$.
\end{proof}

We now define a relation $<$ on $\mathbb{Q}$. For $p,q \in \mathbb{Q}$ let $(a_1,b_1) \in p$ such that $b_1>0$ and let $(a_2,b_2) \in q$ such that $b_2>0$. Then we define
\[
p<q \text{ if } a_1b_2<a_2b_1
\]

\textbf{Theorem 16}
\textsl{Show that $<$ is a well-defined relation on $\mathbb{Q}$.}
\begin{proof}
Let $\overline{(a_1,b_1)},\overline{(a_2,b_2)},\overline{(c_1,d_1)},\overline{(c_2,d_2)} \in \mathbb{Q}$ such that $\overline{(a_1,b_1)} < \overline{(a_2,b_2)}$ and $(a_1,b_1) \sim (c_1,d_1)$ and $(a_2,b_2) \sim (c_2,d_2)$. We take $b_1$, $b_2$, $d_1$ and $d_2$ to all be greater than $0$ by Lemma 15. Then we have $a_1b_2 < a_2b_1$ and so $a_1b_2d_1d_2 < a_2b_1d_1d_2$. But we also know that $a_1d_1=b_1c_1$ and $a_2d_2=b_2c_2$. Making the appropriate substitutions we see $b_1b_2c_1d_2 < b_1b_2c_2d_1$. Since $b_1b_2>0$ we have $c_1d_2<c_2d_1$ and so $\overline{(c_1,c_2)} < \overline{(d_1,d_2)}$. This shows that $<$ is well-defined.
\end{proof}

\textbf{Theorem 17}
\textsl{The relation $<$ is an ordering on $\mathbb{Q}$.}
\begin{proof}
Let $p,q,r \in \mathbb{Q}$ such that $(p_1,p_2) \in p$, $(q_1,q_2) \in q$ and $(r_1,r_2) \in r$. By Lemma 15 we let $p_2$, $q_2$ and $r_2$ all be greater than $0$. If $p \neq q$ then we see that $(p_1,p_2) \nsim (q_1,q_2)$ and so $p_1q_2 \neq p_2q_1$. Then we have have either $p_1q_2<p_2q_1$ and so $p<q$ or $p_2q_1<p_1q_2$ and so $q<p$. Secondly if $p<q$ then we have $p_1q_2<p_2q_1$ and so $p_1q_2 \neq p_2q_1$. Therefore $(p_1,p_2) \nsim (q_1,q_2)$. Thus $p \neq q$. Finally, if $p<q$ and $q<r$ then $p_1q_2<p_2q_1$ and $q_1r_2<q_2r_1$. Multiplying the first inequality by $r_2$ and the second by $p_2$ we have $p_1q_2r_2<p_2q_1r_2$ and $p_2q_1r_2<p_2q_2r_1$ since $p_2>0$ and $r_2>0$. This implies $p_1q_2r_2<p_2q_2r_1$ and since $q_2>0$ we have $p_1r_2<p_2r_1$ and so $p<r$. Since all three conditions are satisfied, we see that $<$ is an ordering on $\mathbb{Q}$.
\end{proof}

\textbf{Exercise 18}
\textsl{Is $(\mathbb{Q},<)$ a model of $C$? That is, which axioms does it satisfy?}
\begin{proof}
Since the integers are a subset of $\mathbb{Q}$ and there exists at least one integer and since we showed that $<$ was and ordering on $\mathbb{Q}$, we see that axioms 1 and 2 are satisfied. Theorem 20 shows that there is no last point of $\mathbb{Q}$. To show that there is no first point we use a similar argument. Let $\overline{(a,b)} \in \mathbb{Q}$ such that $b>0$. We consider three cases:\newline

\textsl{Case 1:} Let $a>0$. Then $a(1)>(0)b$ and so $\overline{(a,b)} > \overline{(0,1)} = 0$.\newline

\textsl{Case 2:} Let $a<0$. Then since $b>0$, $a>ab-b$ which means $\overline{(a,b)} > \overline{(a-1,1)} = a-1$.\newline

\textsl{Case 3:} Let $a=0$ then $\overline{(a,b)} = \overline{(0,b)} = 0$ and since $-1<0$ we see $\overline{(a,b)}>\overline{(-1,1)}=-1$.\newline

So we see that for any element of $\mathbb{Q}$ there is always an element greater than it and an element less than it which means it can have no first or last point and so it satisfies axiom 3.
\end{proof}

\textbf{Theorem 19}
\textsl{For every $p,q \in \mathbb{Q}$ such that $p<q$ there exists $r \in \mathbb{Q}$ such that $p<r<q$.}
\begin{proof}
Let $p,q,r \in \mathbb{Q}$ such that $(p_1,p_2) \in p$, $(q_1,q_2) \in q$ and $r = \overline{(p_1q_2+p_2q_1,2p_2q_2)}$. Let $p<q$ and by Lemma 15 let $p_2>0$ and $q_2>0$. Then we have $p_1q_2<p_2q_1$ and so $p_1p_2q_2<p_2p_2q_1$ which implies $2p_1p_2q_2<p_1p_2q_2+p_2p_2q_1$. We see that this implies $\overline{(p_1,p_2)} < \overline{(p_1q_2+p_2q_1,2p_2q_2)}$ which means $p<r$. Similarly, we have $p_1q_2<p_2q_1$ which means $p_1q_2q_2<p_2q_1q_2$ and $p_1q_2q_2+p_2q_1q_2<2p_2q_1q_2$. This implies $\overline{(p_1q_2+p_2q_1,2p_2q_2)}<\overline{(q_1,q_2)}$ which means $r<q$. Thus $p<r<q$.
\end{proof}

\textbf{Theorem 20 (Archimedean Property)}
\textsl{For every $p \in \mathbb{Q}$ there exists $n \in \mathbb{Z}$ such that $p<n$.}
\begin{proof}
Let $p \in \mathbb{Q}$ such that $(a,b) \in p$. Let $b>0$ by Lemma 15. We have to consider three cases:\newline

\textsl{Case 1:} Let $a>0$. Then $a<ab+b$ and so $\overline{(a,b)} < \overline{(a+1,1)} = a+1$.\newline

\textsl{Case 2:} Let $a<0$. Then $a(1)<b(0)$ and so $\overline{(a,b)} < \overline{(0,1)} = 0$.\newline

\textsl{Case 3:} Let $a=0$. Then $\overline{(a,b)}=\overline{(0,b)}=\overline{(0,1)}=0$ and since $0<1$ we see $\overline{(a,b)} < \overline{(1,1)} = 1$.
\end{proof}

\end{flushleft}
\end{document}