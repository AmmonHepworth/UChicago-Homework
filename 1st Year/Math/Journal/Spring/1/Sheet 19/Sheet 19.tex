\documentclass{article}
\usepackage{amsmath,amsthm,amsfonts,amssymb,fullpage}

\begin{document}

\begin{flushright}
Kris Harper

MATH 16300

Mikl\'{o}s Ab\'{e}rt

April 15, 2008
\end{flushright}

\begin{flushleft}

\Large

Sheet 19: Polynomials\newline

\normalsize

\textbf{Definition 1}
\textsl{A real polynomial of degree $n$ is a function of the form
\[
p(x) = a_n x^n + a_{n-1} x^{n-1} + \dots + a_1 x + a_0
\]
where $a_i \in \mathbb{R}$ $(0 \leq i \leq n)$ and $a_n \neq 0$. If $p(x) = 0$ then we define the degree $\deg p = - \infty$. The set of real polynomials is denoted by $\mathbb{R}[x]$.}\newline

\textbf{Theorem 2}
\textsl{For all $p,q \in \mathbb{R}[x]$ we have
\[
\deg(p+q) \leq \max(\deg(p), \deg(q))
\]
and
\[
\deg(pq) = \deg(p) + \deg(q)
\]}
\begin{proof}
Let $p(x) = \sum_{i=0}^n a_i x^i$ and $q(x) = \sum_{i=0}^m b_i x^i$. Then
\[
p+q(x) = p(x) + q(x) = \left ( \sum_{i=0}^n a_i x^i \right ) + \left ( \sum_{i=0}^m b_i x^i \right )
\]
and so $\deg(p+q) = \max(n,m) = \max(\deg(p), \deg(q))$. Also
\[
pq(x) = p(x)q(x) = \left ( \sum_{i=0}^n a_i x^i \right ) \left ( \sum_{i=0}^m b_i x^i \right )
\]
and so using the product of powers $\deg pq = n+m = \deg(p) + \deg(q)$.
\end{proof}

\textbf{Theorem 3 (Division Remainder)}
\textsl{Let $a,b \in \mathbb{R}[x]$ be polynomials with $b \neq 0$. Then there exists unique $q,r \in \mathbb{R}[x]$ such that
\[
a = bq + r
\]
and
\[
\deg r < \deg b.
\]}
\begin{proof}
To show existence consider the set $S = \{a-bc \mid c \in \mathbb{R}[x]\}$. Suppose that for all $r \in S$, $\deg(r) \geq \deg(b)$. Choose $p \in S$ such that $\deg(p)$ is the minimum degree of all elements of $S$ using the Well Ordering Principle. Note that $p=a-bc$ for some $c \in \mathbb{R}[x]$. Now let $q = p-bd$ for some $d \in \mathbb{R}[x]$. Then $q = a-bc-bd=a-b(c+d)$ and so $q \in S$. Thus $\deg(q) \geq \deg(p)$. But then if $p(x) = \sum_{i=0}^n a_i x^i$ and $b(x) = \sum_{i=0}^m b_i x^i$ then consider $d = (a_n/b_m) x^{(n-m)}$. Then $\deg(bd) = n$ and so $\deg(q) < \deg(p)$ since $q = p-bd$. This is a contradiction and so there exists $r \in S$ such that $\deg(r) < \deg(b)$.\newline

For uniqueness suppose that there exists $q,q',r,r'$ with $q \neq q'$ and $r \neq r'$ such that $a=bq+r$, $a=bq'+r'$, $\deg(r) < b$ and $\deg(r') < b$. Then $bq+r = bq'+r'$ and $b(q-q') = r'-r$. Note that since $q \neq q'$ and $r \neq r'$, $\deg(q-q') \geq 0$ and $\deg(r-r') \geq 0$. But then using Theorem 2 we have $\deg(r-r') < b$ and $\deg(b(q-q')) = \deg(b) + \deg(q-q') \geq \deg(b)$ (19.2). This is a contradiction and so $q=q'$ and $r=r'$ which means $q$ and $r$ are unique.
\end{proof}

\textbf{Definition 4}
\textsl{We call $r$ the remainder of $a$ divided by $b$.}\newline

\textbf{Exercise 5}
\textsl{Divide $x^3 + 4$ by $2x^2 - 1$ with remainder. Also $x^4-1$ by $x^2 - 1$.}\newline

$x^3+4$ divided by $2x^2-1$ is $x/2$ with $x/2+4$ as a remainder because $x^3+4 = x^3 + x/2 - x/2 + 4 = (2x^2-1)(x/2) + x/2+4$. Also $(x^2-1)(x^2+1) = x^4-1$ so $x^4-1$ divided by $x^2-1$ is $x^2+1$ with no remainder.\newline

\textbf{Definition 6}
\textsl{A real number $\alpha$ is a root of $p(x) \in \mathbb{R}[x]$ if $p(\alpha) = 0$.}\newline

\textbf{Theorem 7}
\textsl{Let $p,q \in \mathbb{R}[x]$. Then $\alpha$ is a root of $pq$ if and only if $\alpha$ is a root of $p$ or $q$.}
\begin{proof}
Let $p(x) = \sum_{i=0}^n a_i x^i$ and $q(x) = \sum_{i=0}^m b_i x^i$ and suppose that $\alpha$ is a root of $p$ or $q$. Without loss of generality suppose that $\alpha$ is a root of $p$. Then $\sum_{i=0}^n a_i \alpha^i = 0$ and so
\[
pq(\alpha) = p(\alpha) q(\alpha) = \left ( \sum_{i=0}^n a_i \alpha^i \right ) \left( \sum_{i=1}^m b_i \alpha^i \right ) = 0 \cdot \left ( \sum_{i=1}^m b_i \alpha^i \right ) = 0
\]
which means $\alpha$ is a root of $pq$. For the converse we use the contrapositive. Suppose that $\alpha$ is not a root of $p$ and $q$. Then $p(\alpha) \neq 0$ and $q(\alpha) \neq 0$. But then $pq(\alpha) = p(\alpha) q(\alpha) \neq 0$.
\end{proof}

\textbf{Theorem 8}
\textsl{Let $p \in \mathbb{R}[x]$. Then $\alpha$ is a root of $p$ if and only if $p = (x-\alpha)q$ for some $q \in \mathbb{R}[x]$.}
\begin{proof}
Suppose that $p = (x - \alpha)q$ for some $q \in \mathbb{R}[x]$. Then $p(\alpha) = (\alpha-\alpha)q = 0$ and so $\alpha$ is a root of $p$. Conversely suppose that $\alpha$ is a root of $p$. From Theorem 3 we know that $p = (x-\alpha)q + r$ for $q,r \in \mathbb{R}[x]$ and $\deg (r) = 0$ (19.3). Thus $r$ is a constant and since $\alpha$ is a root of $p$ we have $p(\alpha) = (\alpha - \alpha)q + r = r = 0$. Thus $p = (x-\alpha)q$ for some $q \in \mathbb{R}[x]$.
\end{proof}

\textbf{Theorem 9}
\textsl{Let $p \in \mathbb{R}[x]$ be a nonzero polynomial of degree $n$. Then $p$ has at most $n$ roots.}
\begin{proof}
Suppose that $\deg(p) = n$ and $p$ has $m$ distinct roots with $m>n$. Let the $m$ roots be $\alpha_1, \alpha_2, \dots ,\alpha_m$. From Theorem 8 we know that $p = (x-\alpha_1)q_1$ for some $q \in \mathbb{R}[x]$ (19.8). From Theorem 7 we know that since $\alpha_2$ is a root of $p$ it is a root of $(x-\alpha_1)$ or $q$ (19.7). Since $\alpha_2-\alpha_1 \neq 0$, $\alpha_2$ is a root of $q_1$. This $q_1 = (x-\alpha_2)q_2$ and $p = (x-\alpha_1)(x-\alpha_2)q_2$ (19.8). We can continue in this process $m$ times until we have
\[
p = \prod_{i=1}^m (x-\alpha_i) q_m.
\]
But then $\deg(p) = m \neq n$ which is a contradiction.
\end{proof}

\textbf{Theorem 10}
\textsl{For every even $n$ there exists a real polynomial of degree $n$ with no roots. Every real polynomial of odd degree has a root.}
\begin{proof}
Let $n$ be even. Consider the polynomial $p(x)=x^n+1$. Since $n$ is even, $n=2k$ for some $k \in \mathbb{N}$. Then $p(x)=x^{2k}+1=(x^k)^2+1$. But then $p(x) > 0$ for all $x \in \mathbb{R}$ and so $p(x)$ has no roots.\newline

Now let $p$ be a polynomial of degree $n$ with $n$ odd such that $p(x) = \sum_{i=0}^n a_i x^i$. Suppose that $a_n > 0$. We know $\lim_{x \rightarrow \infty} p(x)/(a_n x^n) = 1$. Let $\varepsilon = 1/2$. Then there exists $m \in \mathbb{R}$ such that for all $x > m$ we have $|p(x)/(a_n x^n) - 1| < 1/2$. Thus there exists $x_1>0$ such that $1/2 < p(x_1)/(a_n x_1^n)$. Since $x_1, a_n > 0$ and $n$ is odd we have $0 < (a_n x_1^n)/2 < p(x_1)$. Thus $p(x_1)$ is positive. Similarly take $\lim_{x \rightarrow -\infty} p(x)/(a_n x^n) = 1$ and let $\varepsilon = 1/2$. Then there exists $m \in \mathbb{R}$ such that for all $x<m$ we have $|p(x)/(a_n x^n)-1| < 1/2$. Then there exists $x_2 < 0$ such that $1/2 < p(x)/(a_n x^n)$. But since $x_2 < 0$ and $a_n > 0$ we have $a_n x^n < 0$ so then $p(x) < (a_n x^n)/2 < 0$. Thus $p(x_2) < 0$. Therefore there exist $x_1,x_2 \in \mathbb{R}$ with $p(x_2) < 0$ and $p(x_1) > 0$ so there must exist $c \in (x_2;x_1)$ with $p(c) = 0$ by the Intermediate Value Theorem. A very similar proof holds if $a_n < 0$ where the limits give values of opposite signs as in this proof.
\end{proof}

\textbf{Theorem 11 (Lagrange Interpolation)}
\textsl{Let $a_1 < a_2 < \dots < a_n$ and $b_1, b_2, \dots , b_n$ be real numbers. Then there exists a polynomial $p(x)$ of degree at most $n-1$ such that
\[
p(a_i) = b_i \; (1 \leq i \leq n).
\]}
\begin{proof}
Consider the polynomial
\[
p(x) = \sum_{i=1}^n b_i \prod_{j=1, j \neq i}^n \frac{(x-a_j)}{(a_i-a_j)}.
\]
Note that
\[
p(a_k) = \sum_{i=1}^n b_i \prod_{j=1, j \neq i}^n \frac{(a_k-a_j)}{(a_i-a_j)} = b_k \prod_{j=1, j \neq k}^n \frac{(a_k-a_j)}{(a_k-a_j)} = b_k.
\]
\end{proof}

\textbf{Exercise 12}
\textsl{Is this polynomial unique?}\newline

Yes.
\begin{proof}
Let $a_1 < a_2 < \dots < a_n$ and $b_1, b_2, \dots , b_n$ be real numbers. Consider two polynomials $f(x)$ and $g(x)$ such that $f(a_i) = b_i$ and $g(a_i) = b_i$ $(1 \leq i \leq n)$. Then consider $h(x) = f(x) - g(x)$. We see $h(x) = 0$ for each $a_i$ and so $h$ has $n$ roots. But then $n \leq \deg(h) \leq \max(\deg(p), \deg(q))$ (19.2, 19.9). Thus $\deg(p)$ or $\deg(q)$ is greater than or equal to $n$ which means there exists only one such polynomial with degree less than $n$.
\end{proof}

\textbf{Theorem 13}
\textsl{Let $p$ be a real polynomial which maps rationals to rationals. Then all the coefficients of $p$ are rational.}
\begin{proof}
Take $n+1$ rational points $a_1 < a_2 < \dots < a_{n+1}$ and their images $p(a_1) = b_1, p_(a_2) = b_2, \dots , p(a_{n+1}) = b_{n+1}$. From Theorem 11 we know that there exists a polynomial of degree $n$
\[
p'(x) = \sum_{i=1}^n b_i \prod_{j=1, j \neq i}^n \frac{(x-a_j)}{(a_i-a_j)}
\]
such that $p'(a_i) = b_i$ $(1 \leq i \leq n+1)$ (19.11). Note that the coefficients of $p'$ are all rational because $a_i,b_i \in \mathbb{Q}$ $(1 \leq i \leq n)$. From Exercise 12 we know that this polynomial is unique and so $p=p'$ (19.12). Thus $p$ has all rational coefficients.
\end{proof}

\end{flushleft}
\end{document}