\documentclass{article}
\usepackage{amsmath,amsthm,amsfonts,amssymb,fullpage}

\begin{document}
\begin{flushleft}

\Large

Sheet 24: Planar Graphs\newline

\normalsize

\textbf{Definition 1}
\textsl{A planar graph is a graph that can be drawn on the plane so that no edges intersect.}\newline

\textbf{Exercise 2}
\textsl{Find a graph that is not planar.}
\begin{proof}
Take the complete graph defined by five vertices in the shape of a regular pentagon.

\textbf{Exercise 3}
\textsl{A planar graph is a tree if and only if it has one face.}
\begin{proof}
If a graph is a tree then there are no cycles and so there are no enclosed faces. This leaves the only face to be the infinite face. Conversely suppose that a planar graph only has the infinite face. Then it has no cycles.

\textbf{Theorem 4}
\textsl{Let $V$ denote the set of vertices, $E$ denote the set of edges and $F$ denote the set of faces. Let $v = |V|$, $e = |E|$ and $f = |F|$. Then in a connected planar graph on at least $3$ vertices, we have $3f \leq 2e$.}

\textbf{Theorem 5 (Euler's Formula)}
\textsl{In a connected planar graph we have 