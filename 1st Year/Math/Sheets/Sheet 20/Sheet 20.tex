\documentclass{article}
\usepackage{amsmath,amsthm,amsfonts,amssymb,fullpage}

\begin{document}
\begin{flushleft}

\Large

Sheet 20: Modulo\newline

\normalsize

\textbf{Theorem 1}
\textsl{Let $a \equiv b \pmod{n}$ if $n \mid b-a$. Then $\equiv$ is an equivalence relation.}
\begin{proof}
Let $a$, $b$ and $c$ be integers. Note that $n \mid a-a$ because $0 \cdot n = 0 = a-a$ so $a \equiv a \pmod{n}$. Let $a \equiv b \pmod {n}$. Then there exists $k \in \mathbb{Z}$ such that $kn = b-a$ and so $-kn = a-b$. Since $-k \in \mathbb{Z}$ we have $n \mid a-b$ so $b \equiv a \pmod{n}$. Now let $a \equiv b \pmod{n}$ and $b \equiv c \pmod{n}$. Then there exists $k, l \in \mathbb{Z}$ such that $nk = b-a$ and $nl = c-b$. Then $n(l+k) = b-a+c-b = c-a$ so $n \mid c-a$. Thus $a \equiv c \pmod{n}$. Hence we have shown reflexivity, symmetry and transitivity so $\equiv$ is an equivalence relation.
\end{proof}

\textbf{Definition 2}
\textsl{The equivalence classes of integers under this relation are called residue classes modulo $n$. We denote it by $Z_n$.}\newline

\textbf{Theorem 3}
\textsl{There are exactly $n$ residue classes modulo $n$.}\newline

We first prove a lemma showing that every $x \in \mathbb{Z}$ can be written as $x=an+b$ where $b \in \{0, 1, \dots , n\}$.

\begin{proof}
Let $x \in \mathbb{N} \cup \{0\}$ and let $S=\{0, 1, \dots , n\}$. Then let $T=\{b \in \mathbb{N} \cup \{0\} \mid \text{there exists } a \in \mathbb{Z} \text{ such that } x = an+b\}$. Then we see that $T \neq \emptyset$ since $x = n(0) + x$ and $x \in \mathbb{N} \cup \{0\}$ and $0 \in \mathbb{Z}$. Then we see there exists a least element $m$ of $T$ and so $x=an+m$ for some $a \in \mathbb{Z}$. If $m \in S$ then we are done. If $m \notin S$ then $m > n$ and so $m - n > 0$. Therefore we can write $x=n(a+1)+(m-n)$ and so $(m-n) \in T$. But $m-n<m$ and since $m$ is the least element of $T$ this is a contradiction so $m \in S$. Therefore every $x \in \mathbb{N} \cup \{0\}$ can be written as $an+b$ for some $a \in \mathbb{Z}$ and $b \in S$. We now consider the case where $x \in \mathbb{Z} \backslash (\mathbb{N} \cup \{0\})$. We see that $-x = -an-b = n(-a-1) + (-b+n)$. But if $b \neq 0$ then $-b+n \in S$ and if $b=0$ then $x=an$ and so $-x=a(-n)$ and so we see that for $x \in \mathbb{Z}$ we can write $x=an+b$ for $n \in \mathbb{Z}$ and $b \in S$.
\end{proof}

Now we prove the original result.

\begin{proof}
Let $x \in \mathbb{Z}$ and let $S = \{0, 1, \dots , n\}$. Then we see that $x = an+b$ and $x-b=an$ for some $a \in \mathbb{Z}$ and $b \in S$. But then $x \equiv a \pmod{n}$ and so $x \in \overline{b}$. Since there are only $n$ possible values for $b$, we see that there are at most $n$ equivalence classes. If we choose two elements $p,q \in S$ such that $p \neq q$ then without loss of generality we can assume $p>q$ and so $(p-q) \in S$. But then $p-q \neq an$ for some $a \in \mathbb{Z}$ and so $p$ is not equivalent to $q$ modulo $n$ and $\overline{p} \neq \overline{q}$. So no two equivalence classes are the same. Additionally, for every $p \in S$ we see that $p = n(0) + p$ and since $0 \in \mathbb{Z}$ and $p \in S$, we see every element of $p$ is in an equivalence class. So we see that there are at least $n$ and at most $n$ equivalence classes so there must be exactly $n$ equivalence classes.
\end{proof}


\textbf{Definition 4}
\textsl{For $a,b \in Z_n$ let $x \in a$, $y \in b$ and let
\[
a+b = \overline{(x+y)}
\]
\[
a \cdot b = \overline{(x \cdot y)}
\]
where $\overline{z}$ denotes the residue class of $z \in \mathbb{Z}$.}\newline

\textbf{Theorem 5}
\textsl{The operations $+$ and $\cdot$ are well-defined on $Z_n$. Also $(Z_n, +, \cdot)$ is a ring.}
\begin{proof}
Let $a_1, b_1, a_2, b_2 \in Z_n$ such that $a_1 = b_1$ and $a_2 = b_2$. Let $x_1 \in a_1$, $y_1 \in b_1$, $x_2 \in a_2$ and $y_2 \in b_2$. Then
\[
a_1 + a_2 = \overline{x_1 + x_2} = \overline{y_1 + y_2} = b_1 + b_2
\]
and
\[
a_1 \cdot a_2 = \overline{x_1 \cdot x_2} = \overline{y_1 \cdot y_2} = b_1 \cdot b_2
\]
so $+$ and $\cdot$ are well defined. Now let $a_3 \in Z_n$ such that $x_3 \in a_3$. Note that
\[
a_1 + a_2 = \overline{x_1 + x_2} = \overline{x_2 + x_1} = a_2 + a_1
\]
and
\[
(a_1 + a_2) + a_3 = \overline{x_1 + x_2} + \overline{x_3} = \overline{x_1 + x_2 + x_3} = \overline{x_1} + \overline{x_2 + x_3} = a_1 + (a_2 + a_3).
\]
Also let $0 = \overline{0}$ so we have
\[
a_1 + 0 = \overline{x_1 + 0} = \overline{x_1} = a_1
\]
and let $-a_1 = \overline{-x_1}$ so we have
\[
a_1 + -a_1 = \overline{x_1 + -x_1} = \overline{0} = 0.
\]
Now note that
\[
a_1 \cdot a_2 = \overline{x_1 \cdot x_2} = \overline{x_2 \cdot x_1} = a_2 \cdot a_1
\]
and
\[
(a_1 \cdot a_2) \cdot a_3 = \overline{x_1 \cdot x_2} \cdot \overline{x_3} = \overline{x_1 \cdot x_2 \cdot x_3} = \overline{x_1} \cdot \overline{x_2 \cdot x_3} = a_1 \cdot (a_2 \cdot a_3).
\]
Now let $1 = \overline{1}$ so we have
\[
a_1 \cdot 1 = \overline{x_1 \cdot 1} = \overline{x_1} = a_1.
\]
Finally we have
\begin{align*}
a_1 \cdot (a_2 + a_3) &= \overline{x_1} \cdot \overline{x_2 + x_3} \\
	&= \overline{x_1 \cdot x_2 + x_1 \cdot x_3} \\
	&= \overline{x_1 \cdot x_2} + \overline{x_1 \cdot x_3} \\
	&= a_1 \cdot a_2 + a_1 \cdot a_3.
\end{align*}
So we've show additive commutativity, associativity, identity, inverse, multiplicative commutativity, associativity, identity and also distributivity so $Z_n$ is a ring.
\end{proof}

\textbf{Exercise 6}
\textsl{Solve the following congruencies:\\
1) $2x+1 \equiv 3 \pmod{5}$; \\
2) $x^2 \equiv 1 \pmod{17}$; \\
3) $2x \equiv 5 \pmod{8}$; \\
4) $3x \equiv 3 \pmod{6}$.}\newline

\textbf{Definition 7}
\textsl{Let $R$ be a ring. An element $0 \neq a \in R$ is a zero divisor if there exists $0 \neq b \in R$ with $ab=0$.}\newline

\textbf{Exercise 8}
\textsl{What are the zero divisors modulo $6$, $7$ and $12$?}\newline

The zero divisors of $6$ are $2$ and $3$. Since $7$ is prime is has no zero divisors. The zero divisors of $12$ are $2$, $3$, $4$ and $6$.\newline

\textbf{Lemma 9}
\textsl{Let $0 \neq a \in R$ be a non-zero-divisor. Then $ax=ay$ implies $x=y$.}
\begin{proof}
We have $a(x-y) = ax-ay = 0$. But $a$ is not a zero divisor so for all $0 \neq b \in R$ we have $ab \neq 0$. Therefore $(x-y) = 0$ and so $x=y$.
\end{proof}

\textbf{Theorem 10}
\textsl{Let $R$ be a finite ring. Then $0 \neq a \in R$ has a multiplicative inverse if and only if $a$ is not a zero divisor.}
\begin{proof}
Suppose that $a$ is not a zero divisor. Then for all $0 \neq b \in R$ we have $ab \neq 0$. Multiply $a$ by every element of $R$ which is not a zero divisor. Note that Lemma 9 implies that this is an injective process and so it must return every element which is not a zero divisor. But $1$ is not a zero divisor and so there must exist $b \in R$ such that $ab = 1$.\newline

Conversely assume that $0 \neq a \in R$ has a multiplicative inverse, $b$. Then $ab = 1$. If $a$ is a zero divisor then there exists $0 \neq c \in R$ such that $ac = 0$. Then $a(b+c) = ab + ac = 1$ but then $b+c$ is a multiplicative inverse of $a$ and since multiplicative inverses are unique, $b+c = b$ and $c \neq 0$. This is a contradiction and so $a$ is a zero divisor.
\end{proof}

\textbf{Definition 11}
\textsl{For a prime $p$ let $\mathbb{F}_p = Z_p$.}\newline

\textbf{Theorem 12}
\textsl{For a prime $p$ every nonzero element of $\mathbb{F}_p$ is invertible.}
\begin{proof}
Let $0 \neq a \in \mathbb{F}_p$. Suppose there exists $0 \neq b \in \mathbb{F}_p$ such that $ab = 0$. Then $p \mid ab$ and so there exists $c \in \mathbb{Z}$ such that $pc = ab$. But then because of unique factorization we have $p \mid a$ or $p \mid b$. Thus either $a = 0$ or $b = 0$ which is a contradiction. Thus $a$ is not a zero divisor and so it has a multiplicative inverse (20.10).
\end{proof}

\textbf{Theorem 13 (Wilson's Theorem)}
\textsl{Let $p$ be a prime. Then
\[
(p-1)! \equiv -1 \pmod{p}.
\]}
\begin{proof}
Since $p$ is prime, every term in $(p-2)!$ is invertible in the field $\mathbb{F}_p$ (20.12). Note that $1$ has its own inverse and for $p > 2$ we have $p-3$ terms with inverses in the product $(p-2)!$ that aren't $1$. Each of these pairs will multiply to $1$ and $1$ will multiply with that and so we're left with just $p-1 \equiv -1 \pmod{p}$.
\end{proof}

\textbf{Theorem 14}
\textsl{For all $a,b \in \mathbf{F}_p$ we have
\[
(a+b)^p = a^p+b^p.
\]}
\begin{proof}
We have
\[
(a+b)^p = \sum_{k=1}^p \binom{p}{k} a^k b^{p-k} = \sum_{k=1}^p \frac{p!}{k!(p-k)!} a^k b^{p-k}
\]
and each term of this sum will be $0$ unless $k = 0$ or $k = p$ because of the $p!$ term. Thus we have
\[
(a+b)^p = a^p + b^p.
\]
\end{proof}

\textbf{Theorem 15 (Fermat's Little Theorem)}
\textsl{Let $p$ be a prime and let $a$ be an integer. Then
\[
a^p \equiv a \pmod{p}.
\]}
\begin{proof}
Note that if $p \mid a$ then we are done so assume that $a$ is not a multiple of $p$. Consider the product
\[
a^{p-1} (p-1)! \equiv \prod_{i=1}^{p-1} ia \equiv (p-1)! \pmod{p}
\]
since $a$ is not a zero divisor using the same injective logic as in Theorem 10 (20.10, 20.12). But then we have
\[
a^{p-1} \equiv 1 \pmod {p}
\]
and so
\[
a^p = a
\]
since $a$ is not a zero divisor.
\end{proof}

\textbf{Corollary 16}
\textsl{Let $p$ be a prime and let $a$ be an integer not divisible by $p$. Then
\[
a^{p-1} \equiv 1 \pmod{p}.
\]}
\begin{proof}
This follows from Theorem 15 (20.15).
\end{proof}

\textbf{Theorem 17}
\textsl{Let $R$ be a finite ring and let $a \in R$ be invertible. Then there exists a natural number $k$ with $a^k = 1$.}
\begin{proof}
Note that $R$ is a finite ring and so there must exist $k,l \in \mathbb{N}$ with $k \neq l$ such that $a^k = a^l$. Without loss of generality assume that $k > l$. But then $k-l \in \mathbb{N}$ and since $a$ is invertible, it's not a zero divisor (20.10). Then $a^l$ is not a zero divisor as well. Thus $a^l = a^k = a^k a^{-l} a^l = a^{k-l} a^l$ implies $a^{k-l} = 1$ (20.9).
\end{proof}

\textbf{Definition 18}
\textsl{The minimal $n$ with the above property is called the multiplicative order of $a$. We denote it by $o(a)$.}

\textbf{Theorem 19}
\textsl{Let $0 \neq a \in \mathbb{F}_p$. Then $o(a)$ divides $p-1$.}
\begin{proof}
Note that $a^{o(a)} = a^{p-1}$ and since $o(a) \leq p-1$ by definition we have $a^{\frac{p-1}{o(a)}} = 1$ so $o(a) \mid p-1$.
\end{proof}

\textbf{Theorem 20}
\textsl{Let $a$ be an integer and let $n$ be a natural number. Then the following are equivalent:\\
1) $a$ is relatively prime to $n$; \\
2) $a$ is invertible modulo $n$; \\
3) There exist integers $x,y$ with $ax+ny=1$.}
\begin{proof}
Let $a$ be relatively prime to $n$. Then $a$ and $n$ share no common factors and so for all $0 \neq b \in Z_n$ we have $ab \neq 0$. Thus $a$ is not a zero divisor and so it must be invertible modulo $n$ (20.10). Now assume that $a$ is invertible modulo $n$. Then there exists $x$ such that $ax = 1 \pmod{n}$ which means that there exists $y \in \mathbb{Z}$ such that $ny = 1 - ax$ and so $ax + ny = 1$. Finally assume that there exists integers $x$ and $y$ such that $ax + ny = 1$. Then $ny = 1 - ax$ and since $ny$ and $ax$ differ by a factor of $1$ they share no common factors and so $n$ and $a$ are relatively prime.
\end{proof}

\textbf{Definition 21 (Euler's Totient Function)}
\textsl{For a natural number $n$ let $U(n)$ denote the set of invertible elements of $Z_n$. Let $\phi(n)$ be the size of $U(n)$.}\newline

\textbf{Exercise 22}
\textsl{Find a formula for $\phi(n)$.}\newline

\textbf{Lemma 23}
\textsl{If $a,b \in U(n)$ then $ab \in U(n)$.}
\begin{proof}
Since $a,b \in U(n)$ there exist $a^{-1}, b^{-1} \in Z_n$. Then take $a^{-1}b^{-1} \in Z_n$ and note that $ab \cdot a^{-1}b^{-1} = 1$. Thus $ab \in U(n)$.
\end{proof}

\textbf{Theorem 24}
\textsl{Let $0 \neq a \in Z_n$ be invertible. Then $o(a)$ divides $\phi(n)$.}

\textbf{Theorem 25 (Euler's Theorem)}
\textsl{Let $n$ be a natural number and let $a$ be an integer relatively prime to $n$. Then
\[
a^{\phi(n)} \equiv 1 \pmod{n}.
\]}
\begin{proof}
Since $a$ is relatively prime to $n$ we have $a$ is invertible modulo $n$ (20.20). Then since $o(a) \mid \phi(n)$ we have
\[
a^{\phi(n)} \equiv a^{o(a)} \equiv 1 \pmod{p}
\]
using Theorem 24 (20.24).
\end{proof}

\textbf{Definition 26}
\textsl{Complex numbers are $\mathbb{R}[x]$ modulo $x^2 + 1$.}

\end{flushleft}
\end{document}