\documentclass{article}
\usepackage{amsmath,amsthm,amsfonts,amssymb,fullpage}

\begin{document}
\begin{flushleft}

\Large

Sheet 28: Primes\newline

\normalsize

\textbf{Lemma 1}
\textsl{Let $N > 2$ be an integer. We have
\[
\sum_{i=1}^N \frac{1}{i} > \log (N).
\]}
\begin{proof}
Let $P = \{1, 2, 3, \dots , N\}$ be a partition of $[1;N]$ and $f = 1/x$. Note that
\[
\log (N) = \int_1^N \frac{1}{t} dt = \inf \{U(f, P) \mid \text{$P$ is a partition of $[1;N]$}\} < U(f,P) = \sum_{i=1}^N \frac{1}{i}.
\]
\end{proof}

\textbf{Lemma 2}
\textsl{If $n > 1$ then
\[
\sum_{i=0}^N \frac{1}{n^i} < 1 + \frac{1}{n-1}.
\]}
\begin{proof}
Note that
\begin{align*}
\sum_{i=0}^N \frac{1}{n^i} &= \frac{\sum_{i=0}^N n^i}{n^N} \\
					 &= \frac{(n-1) \sum_{i=0}^N n^i}{(n-1) n^N} \\
					 &= \frac{n^{N+1} - 1}{(n-1) n^N} \\
					 &< \frac{n^{N+1}}{(n-1) n^N} \\
					 &= \frac{n}{n-1} \\
					 &= 1 + \frac{1}{n-1}.
\end{align*}
\end{proof}

\textbf{Lemma 3}
\textsl{If $n \geq 2$ then
\[
\frac{1}{n-1} \leq \frac{2}{n}.
\]}
\begin{proof}
Note that since $n \geq 2$ we have $n \leq 2n - 2$ which gives
\[
\frac{1}{n-1} \leq \frac{2}{n}.
\]
\end{proof}

\textbf{Lemma 4}
\textsl{If $x > 0$ then
\[
\log (1+x) < x.
\]}
\begin{proof}
This follows from Theorem 16 on Sheet 26 (26.16).
\end{proof}

\textbf{Lemma 5}
\textsl{Let $p_1, \dots , p_k$ be the positive primes less than or equal to $N$. We have
\[
\prod_{i=1}^k \sum_{j=0}^N \frac{1}{p_i^j} = \left ( 1 + \frac{1}{p_1} + \dots + \frac{1}{p_1^N} \right ) \dots \left ( 1 + \frac{1}{p_k} + \dots + \frac{1}{p_k^N} \right ) > \sum_{i=i}^N \frac{1}{i}.
\]}
\begin{proof}
Note that for each $n \leq N$ there exists a unique prime factorization
\[
n = p_{n_1}^{a_1}p_{n_2}^{a_2} \dots p_{n_j}^{a_j}
\]
where $0 \leq n_i \leq k$ and $0 \leq a_i \leq N$ for all $i$. But then we know that $1/n$ will be in the product
\[
\prod_{i=1}^k \sum_{j=0}^N \frac{1}{p_i^j}
\]
since this will contain the reciprocals of all possible combinations of products of primes less than or equal to $N$ raised to powers less than or equal to $N$. Note also that since $N > 2$ there must be a term in the product whose reciprocal is greater than $N$. Thus we have the strict inequality
\[
\prod_{i=1}^k \sum_{j=0}^N \frac{1}{p_i^j} > \sum_{i=i}^N \frac{1}{i}.
\]
\end{proof}

\textbf{Theorem 6}
\textsl{We have
\[
\sum_{i=1}^k \frac{1}{p_i} > \frac{1}{2} \log ( \log ( N ) ).
\]}
\begin{proof}
We have
\begin{align*}
\frac{1}{2} \log (\log(N)) &< \frac{1}{2} \log \left ( \sum_{i=1}^N \frac{1}{i} \right ) \\
					&< \frac{1}{2} \log \left ( \prod_{i=1}^k \sum_{j=0}^N \frac{1}{p_i^j} \right ) \\
					&< \frac{1}{2} \log \left ( \prod_{i=1}^k \left ( 1 + \frac{1}{p_i-1} \right ) \right ) \\
					&= \frac{1}{2} \sum_{i=1}^k \log \left ( 1 + \frac{1}{p_i-1} \right ) \\
					&< \frac{1}{2} \sum_{i=1}^k \frac{1}{p_i-1} \\
					&\leq \frac{1}{2} \sum_{i=1}^k \frac{2}{p_i} \\
					&= \sum_{i=1}^k \frac{1}{p_i}
\end{align*}
from Lemmas 1, 2, 3, 4 and 5 (28.1, 28.2, 28.3, 28.4, 28.5).
\end{proof}

\textbf{Corollary 7}
\textsl{We have
\[
\sum_{\textup{$p$ is a prime}}^{\infty} \frac{1}{p}
\]
is divergent.}
\begin{proof}
Note that from Theorem 6 we have the partial sums of
\[
\sum_{\textup{$p$ is a prime}}^{\infty} \frac{1}{p}
\]
are unbounded. Thus
\[
\sum_{\textup{$p$ is a prime}}^{\infty} \frac{1}{p}
\]
is divergent (13.15).
\end{proof}

\end{flushleft}
\end{document}