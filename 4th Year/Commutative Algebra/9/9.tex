\documentclass{article}
\usepackage{amsmath,amsthm,amsfonts,amssymb,fullpage,multirow}

\newcommand{\spec}{\textup{Spec}}
\newcommand{\Ht}{\textup{ht}\;}
\newcommand{\Max}{\textup{Max}}
\newcommand{\Gal}{\textup{Gal}}

\newtheorem{problem}{Problem}

\begin{document}

\begin{flushright}
Kris Harper\\

MATH 26800\\

March 8, 2011
\end{flushright}

\begin{center}
Homework 9
\end{center}

\begin{problem}
Let $(A,M)$ a Noetherian local ring of dimension $n$. Show that $\dim_{A/M}(M/M^2) \geq n$. (Here $\dim_{A/M}(M/M^2)$ denotes the dimension of the $A/M$-vector space $M/M^2$).
\end{problem}
\begin{proof}
Let $x_1, \dots , x_m$ be a basis of $M/M^2$ over $A/M$. Since $A$ is Noetherian, $M$ is a finitely generated $A$-module. Since $A$ is local, $M$ is contained in the Jacobson radical of $A$. So we may apply Nakayama's Lemma to conclude that $m$ preimages of the $x_i$ generate $M$. But since $A$ is local, $\Ht M = \dim(A) = n$ and by the principal ideal theorem, $n = \Ht M \leq m$.
\end{proof}

\noindent
A Noetherian local ring $(A,M)$ of dimension $n$ is called a regular local ring if $\dim_{A/M}(M/M^2) = n$.

\begin{problem}
\label{generators}
Show that a Noetherian local ring $(A,M)$ is regular if and only if $M$ is generated by $n = \dim(A)$ elements.
\end{problem}
\begin{proof}
If $M$ is generated by $x_1, \dots , x_n$, then $M/M^2$ is generated by $x_1 + M^2, \dots , x_n + M^2$ by simply taking an element $y + M^2$ and writing $y$ as a linear combination of the $x_i$. On the other hand, if $x_1 + M^2, \dots , x_n + M^2$ is a basis for $M/M^2$ over $A/M$, then take $N = (x_1, \dots , x_n) \subseteq M$. Then $M \subseteq N + M^2$, so $M/N \subseteq M (M/N)$. Now apply Nakayama's Lemma to get $M/N = 0$ and $M = N$.
\end{proof}

\begin{problem}
\label{findx}
Let $(A,M)$ be a regular local ring of dimension $n \geq 1$. Show that there exists an $x \in M$, $x \notin M^2$ and $x$ not in any minimal prime ideal $P$ with $\dim(A/P) = n$. Further show that for any such choice of $x$, $A/Ax = \overline{A}$ is a regular local ring of dimension $n-1$.
\end{problem}
\begin{proof}
Since $\dim(A) = \Ht M \geq 1$, we know $M \neq 0$. Thus $M \neq M^2$ by Nakayama's Lemma. Suppose $M$ is contained in the union of $M^2$ and the minimal primes of $A$. Since $A$ is Noetherian, this is a finite union only (possibly) one ideal of which is not prime. By the prime avoidance lemma, $M$ is contained in one of these ideals. But $M \nsubseteq M^2$, and $M$ cannot be minimal since $\Ht M = 1$. Thus we can find $x \in M$ which is not in $M^2$ and not in any minimal prime of $A$. Now note that $\overline{M} = M/Ax$ is a maximal ideal in $\overline{A}$ which must contain any other ideal, since $M$ contains all ideals in $A$. Thus $\overline{A}$ is a local ring.

Since $x \notin M^2$, we can pick it to be in a basis for the space $M/M^2$ over $A/M$. Then $M/M^2$ has basis $(\overline{x}, \overline{x_2}, \dots , \overline{x_n})$. Using Nakayama's Lemma, the preimages $(x, x_1, \dots , x_n)$ generate $M$ in $A$. But then $(x_2 + Ax, \dots , x_n + Ax)$ generate $\overline{M}$. Therefore $\Ht \overline{M} = \dim(\overline{A}) \leq n-1$ by the principal ideal theorem. 

This shows that $\overline{A}$ is a regular ring since $\overline{M}$ is generated by $n-1 = \dim(\overline{A})$ elements. Problem~\ref{generators} now shows that $\overline{A}$ is regular. Now note that since $x$ is not in any minimal prime, we can pick a minimal prime $P$ and consider $A/P$. This is a local integral domain with $\overline{x} = x + P$ not in the maximal ideal. By a previous problem, and since $P$ is minimal, we know $\dim(A/Ax) = \dim((A/P)/(A/P)\overline{x}) = \dim(A/P) - 1 = n - 1$.
\end{proof}

\begin{problem}
Show that regular local ring $(A,M)$ is an integral domain.
\end{problem}
\begin{proof}
Induct on $n$. If $n = 0$ then $\Ht M = 0$ so $M$ is minimal and $A$ is a field, thus an integral domain. Suppose $n \geq n$. Using Problem~\ref{findx}, we can find $x \in M$ with $\dim(A/Ax) = n-1$, so by the inductive hypothesis, this is an integral domain. Therefore $Ax$ is a prime ideal. Since $x$ is not in any minimal prime of $A$, $Ax$ cannot be a minimal prime. Therefore $Ax$ strictly contains some minimal prime $P \in \spec(A)$. Pick $y \in P$, and since $P \subsetneq Ax$, $y = ax$ for some $a \in A$. Since $x \notin P$, we have $a \in P$, so $P = Px$. By Nakayama's Lemma, $P = 0$. Since $0$ is a prime ideal in $A$, $A$ must be an integral domain.
\end{proof}

\begin{problem}
(a) Let $A$ be a UFD. Show that $A$ is an integrally closed integral domain.\\
(b) Let $A$ be a UFD with $2 \in A^*$. Let $d \in A$, $d = p_1 \cdots p_r$, $r > 0$, $p_i$ prime elements of $A$ such that $Ap_i \neq Ap_j$ if $i \neq j$. Let $B$ be a ring containing $A$. Suppose $B = A[\alpha]$, $\alpha^2 = d$. Let $K$, $L$ be the fields of fractions of $A$ and $B$ respectively. Show that $L/K$ is an algebraic extension of degree $2$. Further, show that $B$ is an integrally closed domain. (We recall that an integral domain $A$ with field of fractions $K$ is an \emph{integrally closed domain} if $x \in K$, integral over $A$ implies $x \in A$).
\end{problem}
\begin{proof}
(a) Let $K$ be the field of fractions of $A$ and let $a_n(r/s)^n + \cdots + a_0 = 0$ be a polynomial with coefficients in $A$ where we take $a_n = 1$. Assume that the factorizations of $r,s \in A$ into irreducibles are distinct. Then multiply by $s^n$ to get $r^n + \cdots + a_0s^n = 0$ and subtract to get $r^n = -s(-a_{n-1}r^{n-1} + \cdots + a_0s^{n-1})$. Since $s$ and $r$ share no irreducible factors, we must have $s \in A^*$, which means $r/s = rs^{-1} \in A$. Thus $A$ is integrally closed.

(b) Note that $x^2 - d$ is an irreducible polynomial since $d = p_1 \cdots p_r$ is square free (since $Ap_i \neq Ap_j$ if $i \neq j$). Then $K[x]/(x^2 - d)$ is a field, and since $B = A[\alpha]$ and $A$ is an integral domain, we have $L = K[x]/(x^2 - d)$ using the map $x \mapsto \alpha$. Then $L$ is an algebraic extension of $\deg(x^2 - d) = 2$.

To show that $B$ is integrally closed, we can take any element $a + b\alpha \in L$ with $a,b \in K$ as an arbitrary element of $L$ which is integral over $B$. Let $(a + b\alpha)^n + c_{n-1}(a + b\alpha)^{n-1} + \cdots + c_0 = 0$ be a monic polynomial that $a + b\alpha$ satisfies with $c_i \in B$. We can write $c_i = a_i + b_i\alpha$ with $a_i, b_i \in A$. Separate this polynomial into it's $\alpha$ and non $\alpha$ parts so we have
\[
(a + b\alpha)^n + a_{n-1}(a + b\alpha)^{n-1} + \cdots + a_0 = b_{n-1}\alpha (a + b\alpha)^{n-1} + \cdots + b_0\alpha.
\]
Now square both sides and note that since the left side is purely $\alpha$, it becomes purely in $A$ since $\alpha^2 \in A$. Moving the terms back to the right gives a polynomial in $A[x]$ which $a + b\alpha$ satisfies. Thus $a + b\alpha$ is integral over $A$.

Note that $\alpha$ is a degree $2$ algebraic element over $K$, so there exists some order $2$ element $\sigma \in \Gal(K(\alpha)/K)$ with $\sigma : \alpha \to -\alpha$. But then if $a + b \alpha$ is integral over $A$, then apply $\sigma$ to any monic polynomial that $a + b\alpha$ satisfies, and $a - b\alpha$ will satisfy the same polynomial. So $a - b\alpha$ is integral over $A$ as well.

Note that $A$ is integrally closed since $A$ is a UFD. Then note that $a + b\alpha + a - b\alpha = 2a$ is in $A$, since the sum of two integral elements is integral. Since $2 \in A^*$, we have $a \in A$. Similarly, $(a + b\alpha)(a - b\alpha) = a^2 - (b\alpha)^2 = a^2 - b^2d$ is in $A$ since it's the product of two integral elements. Since $a \in A$, $b^2d \in A$. Since $b \in K$, we can write $b^2 = b_1^2/b_2^2$ for $b_1,b_2 \in A$ with $b_1$ and $b_2$ are relatively prime. But then since $b^2d \in A$, we must have $b_2^2 \mid d$, but $d$ is squarefree, as noted earlier. Thus $b_2^2$ is a unit, and $b^2 = ub_1^2$ for some $u \in A^*$. Because $A$ is a UFD, $b = b_1$ is in $A$. Therefore $a,b \in A$, so $a + b\alpha \in B$ and $B$ is integrally closed.
\end{proof}

\noindent
Let $B$ be a ring and $A \subseteq B$ a subring. Recall that $A' = \{b \in B \mid \text{$b$ is integral over A}\}$ is a subring of $B$ called the \emph{integral closure} of $A$ in $B$.

\begin{problem}
With $A$, $B$ and $A'$ as above, let $S \subseteq A$ be a multiplicatively closed subset. Show that $S^{-1}A'$ is the integral closure of $S^{-1}A$ in $S^{-1}B$.
\end{problem}
\begin{proof}
Since localization preserves integrality, and $A'$ is integral over $A$, we know $S^{-1}A'$ is integral over $S^{-1}A$. On the other hand, if $b/s \in S^{-1}B$ is integral over $S^{-1}A$ then there exists a polynomial $(b/s)^n + (a_{n-1}/s_{n-1})(b/s)^{n-1} + \cdots + a_0/s_0 = 0$ with coefficients in $S^{-1}A$. Let $t = s_0 \cdots s_{n-1}$ and multiply by $(st)^n$. We now have an polynomial of which $bt$ is a root. Therefore $bt \in A'$ and $b/s = bt/st$ is in $S^{-1}A'$. This gives the reverse inclusion.
\end{proof}

\begin{problem}
Let $B$ be integral over its subring $A$.\\
(a) Let $P \in \spec(A)$. Show that $PB \cap A = P$.\\
(b) Let $I \subseteq A$ be an ideal. Show that $IB \cap A \subseteq \sqrt{I}$. Thus deduce that if $I = \sqrt{I}$, then $IB \cap A = I$.\\
(c) $I \subsetneq A$ an ideal implies $IB \neq B$.\\
(d) $B^* \cap A = A^*$.
\end{problem}
\begin{proof}
(a) This follows immediately from part (b) since for a prime ideal $\sqrt{P} = P$.

(b) Take $x \in IB \cap A$. Since $x \in IB$, $x = ab$ for $a \in I$ and $b \in B$. Since $B$ is integral over $A$, $b$ satisfies $b^n + a_{n-1}b^{n-1} + \cdots + a_0 = 0$ with $a_i \in A$. Now multiply both sides by $a^n$ to get $(ab)^n + aa_{n-1}(ab)^{n-1} + \cdots + a^na_0 = 0$. Note that $a^ia_{n-i} \in I$ since $a \in I$. In particular, $a^na_0 \in I$, so we must also have $(ab)^n + aa_{n-1}(ab)^{n-1} + \cdots + a^{n-1}a_1(ab) \in I$ (since this is $-a^na_0 \in I$). Now note that since $ab \in IB \cap A$, $ab \in A$ so $a^{n-1}a_1(ab) \in I$ as well. Thus the sum of the first $n-1$ terms is in $I$. Inductively, we see that $(ab)^n \in I$, so $ab \in \sqrt{I}$. Thus $IB \cap A \subseteq \sqrt{I}$. If $\sqrt{I} = I$, then $IB \cap A \subseteq I$. But note that $I \subseteq A$ and $I \subseteq IB$, so $I \subseteq IB \cap A$ and $IB \cap A = I$.

(c) Suppose that $IB = B$. Then by part (b) we have $A = B \cap A = IB \cap A \subseteq \sqrt{I}$. But clearly $\sqrt{I} \subseteq A$, so then $A = I$.

(d) We definitely have $A^* \subseteq B^*$ so $A^* \subseteq B^* \cap A$. Conversely, let $u \in B^* \cap A$. Then $u^{-1} \in B$ so we have $(u^{-1})^n + a_{n-1}(u^{-1})^{n-1} + \cdots + a_0 = 0$. Multiply both sides by $u^{n-1}$ to get $u^{-1} = -(a_{n-1} + \cdots + a_0u^{n-1})$. Since $u \in A$ and $a_i \in A$, we have $u^{-1} \in A$. Then we also have $u^{-1} \in A^*$.
\end{proof}

\begin{problem}
Let $K$ be a field and $B = K[X,Y]/(Y)(Y+1,X) = K[x,y]$, $x$, $y$ images of $X$, $Y$ in $B$. Let $A = K[x] \subseteq B$. Show that $B$ is integral over $A$. Let $Q = Bx + B(y + 1)$. Compute $\Ht Q$ and $\Ht Q \cap A$ and show that $\Ht Q < \Ht Q \cap A$.
\end{problem}
\begin{proof}
Note that $A = K[x] \subseteq K[x,y] = B$ so $B = A[y]$. Note also that $y = \overline{Y} = Y + (Y)(Y+1,X)$. Then $y^2 + y = \overline{Y}^2 + \overline{Y} = (Y^2 + Y) + (Y)(Y+1,X) = 0$ since $Y^2 + Y \in (Y)(Y+1,X)$. Therefore $y$ is integral over $A$ since $B$ is generated by $A$ and $y$ and all of these elements are integral over $A$, we must have $B$ is integral over $A$.

Suppose that $P \subsetneq Q$ is a prime contained in $Q$. Then $0 \in P$, and $0 = xy = y(y+1)$, so either $x \in P$ or $y \in P$. But $y \notin Q$ since $y+1 \in Q$ and then $1$ would be in $Q$. So $x \in P$ and likewise $y+1 \in P$. Therefore $P \supseteq Q$ and $P = Q$. So $Q$ is minimal. Note that $Q$ is prime since $B/Q \cong K$. We can see this by noting that any term with $x$ goes to zero in the quotient and any term with $y$ goes to some constant in $K$. Therefore $Q$ is a minimal prime and $\Ht Q = 0$. Note now that $Q \cap A = Bx \cap A = Ax$. But in $A$, $0$ is prime so $0 \subseteq Ax$ is a chain of length one, and by the principal ideal theorem we know $\Ht Q \cap A \leq 1$. Thus $\Ht Q = 0 < 1 = \Ht Q \cap A$.
\end{proof}

\begin{problem}
Let $B$ be integral over its subring $A$. Let $P \in \spec(A)$. Show that $B_P = S^{-1}B$, ($S = A \backslash P$) is a local ring if and only if there exists exactly one prime ideal $Q \subseteq B$ such that $Q \cap A = P$. Is this true if $B$ is \emph{not} integral over $A$. Justify your answer.
\end{problem}
\begin{proof}
Suppose there exists exactly one prime $Q \subseteq B$ with $Q \cap A = P$. Note that $\spec(B_P) = \{S^{-1}R \mid R \in \spec(B), R \cap S = \emptyset\}$. But note for a prime $R \in \spec(B)$, $R \cap S = R \cap (A \backslash P)$ so if $R \cap S = \emptyset$ then $R \cap A \subseteq P$. We have $Q \cap A = P$, so $Q$ is maximal with respect to these primes. This follows because if we consider any $S^{-1}R$, we must have $S^{-1}R \subseteq S^{-1}Q$.  Then since $Q$ is unique in this respect, $B_P$ must be a local ring.

Conversely, suppose $B_P$ is a local ring. Then there exists $Q \in \spec(B)$ such that $Q \cap S = \emptyset$ and $Q$ is maximal with respect to $R \in \spec(B)$ with $R \cap A \subseteq P$. Using the going up theorem, we know there exists a prime $R \in \spec(B)$ with $R \cap A = P$. Then we must have $R = Q$ so that $Q \cap A = P$, as desired.

The statement is not true if $B$ is not integral over $A$. Take $B = \mathbb{R}$ and $A = \mathbb{Z}$. Take $P = 2\mathbb{Z}$. Then $B_P = B$ since $B$ already contains the field of fractions for $A$. So $B_P$ is a field and thus a local ring. But since the only prime of $B$ is $0$, there are no primes $Q \subseteq B$ with $Q \cap A = P$.
\end{proof}

\begin{problem}
\label{minors}
Let $R$ be a Noetherian ring with $R \neq 0$. Let $A = R[x_{ij}]$, $1 \leq i \leq m$, $1 \leq j \leq n$ be a polynomial ring in $mn$ variables $x_{ij}$ over $R$. Let $\alpha = (x_{ij})$ be the $m \times n$ matrix with entries $x_{ij}$. Let $P \in \spec(A)$ be a minimal prime ideal of $V(I_r(\alpha))$, where $r \leq \min(m,n)$ and $I_r(\alpha)$ is the ideal of $A$ generated by all $r \times r$ minors of $\alpha$. Show that $\Ht P = (m-r+1)(n-r+1)$. Thus $\Ht I_r(\alpha) = (m-r+1)(n-r+1)$.
\end{problem}
\begin{proof}
We use induction on $r$. In the case $r = 1$ we have $I_1(\alpha) = (x_{11}, \dots , x_{mn})$. By the principal ideal theorem, we know $\Ht I \leq mn$, and thus $\Ht P \leq mn$. To construct a chain of $mn$ primes in $P$, we use induction on $mn$. In the base case, we have $R[x]$ and a prime $P \subseteq R[x]$. There are two possibilities for $P$. Either $P \cap R = Q[x]$, $Q \in \spec(R)$, or $P \cap R \neq Q[x]$, $Q \in \spec(R)$. The first case is impossible since $x \in P$. In the second case we know that $\Ht P = \Ht Q + 1 \geq 1$. The general case follows by letting $R = R[x_{ij}]$, $1 \leq i \leq n$, $1 \leq j \leq m-1$ and $x = x_{mn}$. Thus $\Ht P = mn$ and $\Ht I_1(\alpha) = mn$ as well. So assume $r > 1$ and that the statement is true for $r - 1$.

We know already that $\Ht P \leq (m-r+1)(n-r+1)$, so it just remains to show $\Ht P \geq (m-r+1)(n-r+1)$. Since $r > 1$, $m-r+1 < m$ and $n-r+1<n$ so $\Ht P \leq (m-r+1)(n-r+1) < mn$. Thus at least one $x_{ij} \notin P$, otherwise the ideal $(x_{11}, \dots , x_{mn}) \subseteq P$, in which case $\Ht P = mn$ by the base case. Note that interchanging rows and columns of $\alpha$ will only affect the minors by multiplication by a unit, so we may assume $x_{ij} = x_{11}$, with $x_{11} \notin P$.

Now define $A' = A[x_{11}^{-1}]$ and $P' = PA' = S^{-1}P$ with $S = \{x_{11}^n \mid n \geq 0\}$ and $A' = S^{-1}A$. Let $I_r'(\alpha)$ be the ideal generated by $I_r(\alpha)$ in $A'$, that is, $I_r'(\alpha) = S^{-1}I_r(\alpha)$. Now perform a single row reduction as follows. Subtract $x_{11}^{-1}x_{ij}$ times the first column of $\alpha$ from the $j^{\textup{th}}$ column for $j \geq 2$. Multiply the first row by $x_{11}^{-1}$ to get $(1, 0, \dots , 0)$ in the first row. Subtract $x_{i1}$ times the first row from each each $i^{\textup{th}}$ row for $i \geq 2$. This now transforms $\alpha$ into
\[
\left (
\begin{array}{cccc}
1 & 0 & \cdots & 0\\
0 & \multicolumn{3}{c}{\multirow{4}{*}{$\alpha'$}}\\
\vdots &\\
0 &
\end{array}
\right )
\]
where $\alpha' = (x_{ij}') = (x_{ij} - x_{11}^{-1}x_{ij}x_{11})$, $2 \leq i \leq m$ and $2 \leq j \leq n$. Since addition of a scaled row to another doesn't change the determinant, we now have $I_r'(\alpha) I_{r-1}'(\alpha')$, the ideal generated by all $(r-1) \times (r-1)$ minors of $\alpha'$. Define $R' = R[x_{11}, x_{11}', x_{12}, \dots , x_{1n}, x_{21}, x_{31}, \dots , x_{n1}]$. Then we see that $A' = A[x_{11}^{-1}] = R'[x_{ij}]$, $2 \leq i \leq m$, $2 \leq j \leq n$. This is the same (through an invertible transformation of variables) as $R'[x_{ij}']$, $2 \leq i \leq m$, $2 \leq j \leq n$. This ring is a polynomial ring in $(m-1)(n-1)$ variables over $R'$. We can now use the induction hypothesis to note that
\[
\Ht I_r'(\alpha) = \Ht I_{r-1}'(\alpha') = ((m-1) - (r-1) + 1)((n-1) - (r-1) + 1) = (m-r+1)(n-r+1).
\]
Since $x_{11} \notin P$, $x_{11} \notin I_r(\alpha)$, so $P'$ contains $I_r'(\alpha)$ and is minimal over $I_r'(\alpha)$. Therefore $\Ht P' = (m-r+1)(n-r+1) = \Ht P$ since $P \cap S = \emptyset$. Thus $\Ht I_r(\alpha) = (m-r+1)(n-r+1)$ and so we're done by induction.
\end{proof}

\begin{problem}
\label{oneprime}
Let $K$ be a field and $A = K[x_1, \dots , x_n]$. Let $I \subsetneq A$ be an ideal. Show that $\Ht I = 1$ if and only if $I = Jf$ for some $f \in A$, $f \notin K$ and $J$ some ideal of $A$.
\end{problem}
\begin{proof}
Suppose $I = Jf$ for $f \in A \backslash K$ and $J$ an ideal of $A$. Then $I \subseteq Af$ which has height less than or equal to $1$ by the principal ideal theorem. The $I$ has height at most one, and since $f \neq 0$, $\Ht I = 1$. Conversely, suppose $\Ht I = 1$. Then $I$ is contained in some height one prime $P$. Since $A$ is a UFD, we know $P$ is a principal ideal, so $P = Af$ for some prime $f$. Therefore $x \in I$ means $x = af$ for some $a \in A$. So define $J = \{a \in A \mid af \in I\}$. Then $I = Jf$, so if we can show $J$ is an ideal, then we're done. Take $a, b \in J$ so $af, bf \in I$. Then $af + bf = (a+b)f$ is in $I$, so $a + b \in J$. Similarly, $afbf = (abf)f$ is in $I$, so $abf \in J$. Note that $abf \in I$ since $bf \in I$, so $ab \in J$. Finally, take $c \in A$, then $caf = c(af)$ is in $I$ since $af \in I$. Thus $ca \in J$ and $J$ is an ideal.
\end{proof}

\begin{problem}
Let $A = K[x,y,z,t]$. Let $I \subseteq A$ be the ideal generated by $xt-yz$, $y^2-xz$, $z^2-yt$. Compute the height of $I$.
\end{problem}
\begin{proof}
Consider the matrix
\[
\alpha =
\left (
\begin{array}{ccc}
t & z & y\\
z & y & x
\end{array}
\right ).
\]
The minors of this matrix are $ty - z^2$, $xz - y^2$ and $xt-yz$. Note that up to multiplication by a unit, these are precisely the generators of $I$, so $I = I_2(\alpha)$. Then we know $\Ht I \leq (2 - 2 + 1)(3 - 2 + 1) = 2$. Now note that there is no $f \in A$ with $f$ dividing each generator of $I$. Then by Problem~\ref{oneprime}, we know $\Ht I \neq 1$, so $\Ht I = 2$ (since $I \neq 0$).
\end{proof}

\begin{problem}
Let $A$ be a finitely generated algebra over a field $K$. Suppose $A$ is a finite dimensional $K$-vector space of dimension $n$. Show that $|\Max(A)| \leq n$.
\end{problem}
\begin{proof}
We induct on $n$. In the case $n = 1$ we have $A = K$, $\Max(A) = 0$ and so $|\Max(A)| \leq 1$. Suppose the inequality holds for $1 \leq j \leq n-1$. Every ideal of $A$ is a $K$-submodule and therefore a vector space over $K$ with finite dimension. Let $I$ be the proper nonzero ideal of $A$ with the least dimension (assuming $\Max(A) \neq \emptyset$, such an ideal exists). Then $I$ must be minimal over $0$. Let $\dim(I) = k < n$.

Now note that $I$ is maximal in $K+I$, so $K+I/I$ is a field and $A/I$ is a finitely dimensional vector space over $K+I/I$ with dimension $n - k \geq 1$. By the inductive hypothesis there exist at most $j \leq n-k$ maximal ideals $M_1, \dots , M_j$ of $A/I$, so there are at most $j$ ideals in $A$ containing $I$. Pick $M, N \in \Max(A) \backslash \{M_1, \dots , M_j\}$. Then $I \nsubseteq M$ so $I \cap M \subsetneq I$. But $I$ is minimal so $I \cap M = 0$. Similarly $I \cap N = 0$. Thus $I + M \supsetneq M$ so $I + M = A$. Now we have $N = AN = (M+I)N = MN + IN = MN$ since $IN \subseteq I \cap N = 0$. Thus $N = MN \subseteq M \cap N \subseteq M$. If we interchange $M$ and $N$ we get $M \subseteq N$ so $M = N$. Thus there are at most $j + 1 \leq n-k+1 \leq n$ maximal ideals of $A$ (since $0 < k < n$).
\end{proof}

\end{document}
