\documentclass{article}
\usepackage{amsmath,amssymb,amsfonts,amsthm,fullpage}

\newtheorem{problem}{Problem}
\newtheorem{lemma}{Lemma}
\newtheorem{**}{** Problem}

\begin{document}
\begin{flushright}
Kris Harper\\

MATH 20700\\

October 13, 2008
\end{flushright}

\begin{center}
Homework 2
\end{center}

\begin{flushleft}

\begin{problem}
1) What is the negation of ``$P(b)$, for all $b \in B$''? What about the negation of ``$P(b)$, for some $b \in B$''?\\
2) State $\overline{2}$ and $\overline{3}$ for the equivalence relation axioms (non-symmetry and non-transitivity). How is non-symmetry different from antisymmetry?\\
3) Show that the axioms for an equivalence relation are completely independent.\newline
\end{problem}

1) The negation of ``$P(b)$, for all $b \in B$'' is ``$\overline{P}(b)$ for some $b \in B$''. The negation of ``$P(b)$ for some $b \in B$'' is ``$\overline{P}(b)$ for all $b \in B$.''\newline

2) Non-symmetry is stated as, ``there exists $a,b \in A$ such that $a \sim b$ but $b \nsim a$.'' Non-transitivity is stated as ``there exists $a,b,c \in A$ such that if $a \sim b$ and $b \sim c$ then $a \nsim c$.'' Antisymmetry is stated as ``for all $a,b \in A$, if $a \sim b$ and $b \sim a$ then $a = b$.''\newline

3)\begin{proof}
The following relations on the set $\{a,b,c\}$ satisfy each of the axioms they are assigned to:\newline

$\{1, 2, 3\}$: $\{(a,a), (a,b), (a,c), (b,a), (b,b), (b,c), (c,a), (c,b), (c,c)\}$\newline

$\{\overline{1}, 2, 3\}$: $\{(b,b), (c,c)\}$\newline

$\{1, \overline{2}, 3\}$: $\{(a,a), (b,b), (c,c), (a,b), (c,a), (c,b)\}$\newline

$\{1, 2, \overline{3}\}$: $\{(a,a), (b,b), (c,c), (a,b), (b,a), (b,c), (c,b)\}$\newline

$\{\overline{1}, \overline{2}, 3\}$: $\{(b,b), (c,c), (a,b), (c,a), (c,b)\}$\newline

$\{\overline{1}, 2, \overline{3}\}$: $\{(b,b), (c,c), (a,b), (b,a), (b,c), (c,b)\}$\newline

$\{1, \overline{2}, \overline{3}\}$: $\{(a,a), (b,b), (c,c), (a,b), (b,c)\}$\newline

$\{\overline{1}, \overline{2}, \overline{3}\}$: $\{(b,b), (c,c), (a,b), (b,c)\}$
\end{proof}

\begin{**}
Show that the group axioms are completely independent.
\end{**}
Let $(G, \circ)$ be a group where $G = \{a,b,c\}$. Enumerate the group axioms as follows:\newline

1) $\circ$ is associative.\newline

2) There exists an identity element in $G$.\newline

3) $G$ is solvable.\newline

The following multiplication tables show how $\circ$ works on $G$ such that the respective axioms are satisfied. When composing two elements the left element is taken from the vertical column and the right element is taken from the horizontal column.\newline

\begin{tabular}{ccc}
$\{1,2,3\}$:

\begin{tabular}{|c|c|c|c|}
\hline
$\times$ & $a$ & $b$ & $c$\\
\hline
$a$ & $a$ & $b$ & $c$\\
\hline
$b$ & $b$ & $c$ & $a$\\
\hline
$c$ & $c$ & $a$ & $b$\\
\hline
\end{tabular}
&
$\{\overline{1},2,3\}$:

\begin{tabular}{|c|c|c|c|}
\hline
$\times$ & $a$ & $b$ & $c$\\
\hline
$a$ & $a$ & $b$ & $c$\\
\hline
$b$ & $c$ & $a$ & $b$\\
\hline
$c$ & $b$ & $c$ & $a$\\
\hline
\end{tabular}
&
$\{1,2,\overline{3}\}$:

\begin{tabular}{|c|c|c|c|}
\hline
$\times$ & $a$ & $b$ & $c$\\
\hline
$a$ & $a$ & $b$ & $c$\\
\hline
$b$ & $b$ & $c$ & $b$\\
\hline
$c$ & $c$ & $b$ & $c$\\
\hline
\end{tabular}\\

&&\\

$\{\overline{1}, \overline{2}, 3\}$:

\begin{tabular}{|c|c|c|c|}
\hline
$\times$ & $a$ & $b$ & $c$\\
\hline
$a$ & $b$ & $b$ & $b$\\
\hline
$b$ & $c$ & $c$ & $c$\\
\hline
$c$ & $a$ & $a$ & $a$\\
\hline
\end{tabular}
&

$\{\overline{1},2,\overline{3}\}$:

\begin{tabular}{|c|c|c|c|}
\hline
$\times$ & $a$ & $b$ & $c$\\
\hline
$a$ & $a$ & $b$ & $c$\\
\hline
$b$ & $b$ & $c$ & $a$\\
\hline
$c$ & $b$ & $a$ & $b$\\
\hline
\end{tabular}
&

$\{1,\overline{2},\overline{3}\}$:

\begin{tabular}{|c|c|c|c|}
\hline
$\times$ & $a$ & $b$ & $c$\\
\hline
$a$ & $a$ & $a$ & $a$\\
\hline
$b$ & $a$ & $a$ & $a$\\
\hline
$c$ & $a$ & $a$ & $a$\\
\hline
\end{tabular}\\

&&\\

$\{\overline{1},\overline{2},\overline{3}\}$:

\begin{tabular}{|c|c|c|c|}
\hline
$\times$ & $a$ & $b$ & $c$\\
\hline
$a$ & $a$ & $c$ & $c$\\
\hline
$b$ & $c$ & $c$ & $a$\\
\hline
$c$ & $c$ & $a$ & $b$\\
\hline
\end{tabular}
&
&
\end{tabular}

The set of axioms $\{1, \overline{2}, 3\}$ is satisfied by the natural numbers under addition.

\begin{**}
For a ring, $R$, with $a,b,c \in R$ show\\
1) If $a+b = a+c$ then $b=c$.\\
2) $a \cdot 0 = 0 \cdot a = 0$.
\end{**}
\begin{proof}
1) Let $a+b = a+c$. Add the additive inverse of $a$ to both sides so that we have
\[
b = 0+b = ((-a) + a) + b = (-a) + (a + b) = (-a) + (a+c) = ((-a) + a) + c = 0+c = c.
\]
2) Note that $0$ is the additive identity, so $0 + 0 = 0$. Then multiply both sides by $a$ so we have $a \cdot (0 + 0) = a \cdot 0$ and distributing we have $a \cdot 0 + a \cdot 0 = a \cdot 0$. Now add the additive inverse of $a \cdot 0$ to both sides so we have
\[
a \cdot 0 = 0 + a \cdot 0 = (-(a \cdot 0) + a \cdot 0) + a \cdot 0 = -(a \cdot 0) + (a \cdot 0 + a \cdot 0) = -(a \cdot 0) + a \cdot 0 = 0.
\]
\end{proof}

\begin{**}
Let $R$ be a commutative ring with $1$. Show that $(R[x], +, \cdot)$ is a commutative ring with $1$.
\end{**}
\begin{proof}
Let $(a_n), (b_n), (c_n) \in R[x]$. Then we have
\[
(a_n) + ((b_n) + (c_n)) = (a_n) + (b_n + c_n) = (a_n + (b_n + c_n)) = ((a_n + b_n) + c_n) = (a_n + b_n) + (c_n) = ((a_n) + (b_n)) + (c_n)
\]
so $R[x]$ is associative under addition. Also
\[
(a_n) + (b_n) = (a_n + b_n) = (b_n + a_n) = (b_n) + (a_n)
\]
so $R[x]$ is commutative under addition. If we let $(0_n) = (d_n)$ such that $d_n = 0$ for all $n$, then we have
\[
(0_n) + (a_n) = (0_n + a_n) = (a_n)
\]
for all $(a_n) \in R[x]$. Thus $(0_n)$ is the additive identity of $R[x]$. Then we see that for $(a_n), (b_n) \in R[x]$ we have
\[
(b_n - a_n) + (a_n) = (b_n - a_n + a_n) = (b_n)
\]
so $R[x]$ is solvable. Hence $(R[x], +)$ is an abelian group. Now we consider multiplication in $R[x]$. For $(a_n), (b_n), (c_n) \in R[x]$ we have
\begin{align*}
(a_n) \cdot ((b_n) \cdot (c_n)) &= (a_n) \cdot \left ( \left ( \sum_{i=0}^{n} b_i c_{n-i} \right )_n \right )\\
&= \left ( \left ( \sum_{j=0}^{n} a_j \sum_{i=0}^{n-j} b_i c_{n-i} \right )_n \right )\\
&= \left ( \left ( \sum_{j=0}^{n} \sum_{i=0}^{n-j} a_j b_i c_{n-i} \right )_n \right )\\
&= \left ( \left ( \sum_{j=0}^{n} a_j b_{n-j} \sum_{i=0}^{n} c_i \right )_n \right )\\
&= \left ( \left ( \sum_{j=0}^{n} a_j b_{n-j} \right )_n \right ) \cdot (c_n)\\
&= ((a_n) \cdot (b_n)) \cdot (c_n)
\end{align*}
so $R[x]$ is associative under addition. Consider
\[
(a_n) \cdot (b_n) = \left ( \left ( \sum_{i=0}^{n} a_i b_{n-i} \right )_n \right ) = \left ( \left ( \sum_{i=0}^{n} a_{n-i} b_i \right )_n \right ) = \left ( \left ( \sum_{i=0}^{n} b_i a_{n-i} \right )_n \right ) = (b_n) \cdot (a_n)
\]
which shows $R[x]$ is commutative under multiplication. Let $(1_n)$ be the sequence for which $1_0 = 1$ and $1_n = 0$ for all $n \neq 0$. Then for all $(a_n) \in R[x]$ we have
\[
(a_n) \cdot (1_n) = \left ( \left ( \sum_{i=0}^{n} a_n b_{n-i} \right )_n \right ) = (a_n \cdot 1) = (a_n)
\]
which means that $(1_n)$ is the identity for $R[x]$. Finally for $(a_n), (b_n), (c_n) \in R[x]$ we have
\begin{align*}
(a_n) \cdot ((b_n) + (c_n))
&= (a_n) \cdot (b_n + c_n)\\
&= \left ( \left ( \sum_{i=0}^{n} a_n (b_{n-i} + c_{n-i}) \right )_n \right )\\
&= \left ( \left ( \sum_{i=0}^{n} a_n b_{n-i} \right )_n \right ) + \left ( \left ( \sum_{j=0}^{n} a_j c_{n-j} \right )_n \right )\\
&= (a_n) \cdot (b_n) + (a_n) \cdot (c_n)
\end{align*}
which means that $R[x]$ is distributive. Since it fulfills all the axioms, $(R[x], +, \cdot)$ is a commutative ring with $1$.
\end{proof}

\begin{**}
What are the zero-divisors in $R[x]$?
\end{**}

Let $(a_n) (b_n) \in R[x]$ such that $(a_n) \cdot (b_n) = 0$ and $(a_n), (b_n) \neq (0_n)$. Then we can say that the first and last nonzero terms in $(a_n)$ and $(b_n)$ are zero divisors in $R$. This occurs because these terms will multiply and have no other terms of that degree in $(a_n) \cdot (b_n)$. That is, the highest and lowest nonzero index of $(a_n) \cdot (b_n)$ will be the product of zero divisors.

\begin{lemma}
In a commutative ring with $1$, for all $a$ we have $(-1) \cdot a = -a$.
\end{lemma}
\begin{proof}
Note that
\[
0 = a \cdot 0 = a \cdot (1 + (-1)) = a \cdot 1 + a \cdot (-1) = a + a \cdot (-1)
\]
and adding $-a$ to both sides results in $-a = a \cdot (-1)$.
\end{proof}

\begin{**}
Let $R$ be an ordered commutative ring with $1$. Show that $R$ is an integral domain.
\end{**}
\begin{proof}
Let $a,b,c \in R$ such that $a \neq 0$ and $ab = ac$. Then adding $-(ac)$ to both sides we have $ab + -(ac) = 0$. Using associativity, distributivity and Lemma 1 we have $a \cdot (b + (-c)) = 0$. Note also that from Lemma 1 we know that $-(b + (-c)) = ((-b) + c)$. Assuming that this quantity is not $0$, there are four cases which follow from the ordering of $R$.\newline

\textit{Case 1}: Let $a > 0$ and $(b + (-c)) > 0$. Then $a \cdot (b + (-c)) > 0$, which is not true.\newline

\textit{Case 2}: Let $a < 0$ and $(b + (-c)) > 0$. Then from ** Problem 6 part 1) we know $-a > 0$ and so $-a \cdot (b + (-c)) > 0$. From Lemma 1 and ** Problem 6 part 1) it follows that $a \cdot (b + (-c)) < 0$ which is not true.\newline

\textit{Case 3}: Let $a > 0$ and $(b + (-c)) < 0$. This case is similar to Case 2.\newline

\textit{Case 4}: Let $a < 0$ and $(b + (-c)) < 0$. It follows from ** Problem 6 part 4) that $a \cdot (b + (-c)) > 0$ which is not true.\newline

Since all four of the possible cases are not possible, it must be the case that $b + (-c) = 0$. Then adding $c$ to both sides results in $b = c$. Hence, $R$ is an integral domain.
\end{proof}

\begin{**}
Let $R$ be an ordered commutative ring with $1$ with $a,b,c \in R$. Show the following:\\
1) $a < 0$ if and only if $-a > 0$.\\
2) $a > 0$ if and only if $-a < 0$.\\
3) If $a < b$ and $c < 0$ then $a \cdot c > b \cdot c$.\\
4) If $a < 0$ and $b < 0$ then $a \cdot b > 0$.\\
5) If $a \neq 0$, then $a^2 > 0$.\\
6) $0<1$.
\end{**}
\begin{proof}
1) Let $a < 0$. Then add $(-a)$ to both sides. We have $0 = (-a) + a < 0 + (-a) = -a$. Similarly, assume $-a > 0$ and add $a$ to both sides. Then $0 = a + (-a) > a + 0 = a$.\\

2) Assume $a > 0$. Then add $(-a)$ to both sides. We have $0 = (-a) + a > (-a) + 0 = -a$. Similarly, assume $-a < 0$ and add $a$ to both sides. Then $0 = a + (-a) < a + 0 = a$.\newline

3) Let $a < b$ and $c < 0$. Then $(-c) > 0$. Thus $a \cdot (-c) < b \cdot (-c)$. Add $-(a \cdot (-c))$ to both sides so we have $0 < b \cdot (-c) + (-(a \cdot (-c))$. Using associativity, commutativity, distributivity and Lemma 1 we have $0 < -((b \cdot c) + (-(a \cdot c)))$. Then $0 > (b \cdot c) + (-(a \cdot c))$ and adding $a \cdot c$ to both sides we have $a \cdot c > b \cdot c$.\newline

4) Let $a < 0$ and $b < 0$. Then $-a > 0$ so $-(a \cdot b) = (-a) \cdot b < (-a) \cdot 0 = 0$ and $a \cdot b > 0$.\newline

5) Let $a \neq 0$. Then either $a > 0$ or $a < 0$. Assume first that $a > 0$. Then
\[
a^2 = a \cdot a > a \cdot 0 = 0.
\]
If $a < 0$ then $a \cdot a > 0$ by 4).\newline

6) We know $1$ is the multiplicative identity, so $1 \cdot 1 = 1$. But then $1 = 1^2 > 0$ by 5).
\end{proof}

\begin{problem}
For an ordered integral domain $(R, +, \cdot)$ let $S$ be an inductive subset of $R$ if $1 \in S$ and for all $x \in S$, $x + 1 \in S$. Then let $N$ be the intersection of all inductive subsets of $R$. Show the following:
1) Suppose that $S$ is a non-empty subset of $N$ such that $1 \in S$ and if $x \in S$ then $x + 1 \in S$. Show that $S = N$.\\
2) Show that $N$ is closed under addition.\\
3) Show that $N$ is closed under multiplication.\\
4) Show that the well ordering principle holds in $N$.\\
5) Show that $Z = N \cup \{0\} \cup -N$ is closed under addition.\\
6) Show that $Z$ is closed under multiplication.\\
7) Show that $Z$ and $\mathbb{Z}$ are order isomorphic.\newline
\end{problem}
\begin{proof}
1) By definition $S \subset N$. Also note that $1 \in S$ and $1 \in N$. Suppose that for some $n \in N$, $n \in S$. Then note that $n+1$ is in both $N$ and $S$ so by induction, $N = S$.\newline

2) Let $n \in N$. Let $S = \{m \in N \mid m + n \in N\}$. Note that $1 \in S$. Suppose $m \in S$. Then $m+n \in N$ and $m+n+1 \in S$. By induction, $N$ is closed under addition.\newline

3) Let $n \in N$ and let $S = \{m \in N \mid mn \in N\}$. Then $1 \in S$. Suppose that $m \in S$, then $n(m+1) = mn+m$ and $mn \in N$ and $N$ is closed under addition so $mn + m \in N$. Thus $m+1 \in S$ so $S = N$. Thus $N$ is closed under multiplication.\newline

4) Clearly a subset of $N$ with $1$ element is well ordered. Assume all subsets $S \subseteq N$ with $n$ elements are well ordered. Consider a subset $S' \subseteq N$ with $n+1$ elements. Let $x \in S'$ and consider $S' \backslash \{x\}$. This set is well ordered so it has a least element, $y$. There are then two cases, $x < y$ in which case $x$ is the least element of $S'$ or $x > y$ in which case $y$ is the least element of $S'$. We see then that $S'$ is well ordered. By induction, well ordering holds in $N$.\newline

5) We already know that $N$ is closed under addition and thus $-N$ is closed under addition. Addition $\{0\}$ won't change anything since it's the additive identity. Thus, the only thing we need to check is whether for $n \in N$ and $m \in -N$ we have $n + m \in Z$. Fix $n \in N$ and let $S$ be the set of $m \in N$ such that $-m + n \in Z$. We see that $n + -1 \in Z$ so $1 \in S$. Let $m \in S$. Then using Lemma 1, associativity and distributivity
\[
n + -(m+1) = n + (-m + -1) = (n + -m) + -1
\]
and $(n + -m) + -1 \in Z$. Thus the statement must hold true for all $m$.\newline

6) We know that $N$ is closed under multiplication and using ** Problem 6 we know that for $n,m \in -N$, $mn \in N$. Also, $0 \cdot n = 0$ for all $n$ so again we must consider the product of $m$ and $n$ where $n \in N$ and $m \in -N$. Let $n,m \in N$ and consider $n(-m)$. Using Lemma 1 and associativity this is just $-(nm)$ which is in $-N \subseteq Z$. Thus $Z$ is closed under multiplication.\newline

7) Note that for all $n \in N$, we have $n \in \mathbb{Z}$. To show this, note that $1 \in \mathbb{Z}$. Then for all $n \in N$ such that $n \in \mathbb{Z}$, we have $n + 1 \in \mathbb{Z}$. Since for all $n \in \mathbb{Z}$, $-n \in \mathbb{Z}$ as well, we have $-N \subseteq \mathbb{Z}$. Then let $f : Z \rightarrow \mathbb{Z}$ be the identity function such that
\[
f(n) =
\begin{cases}
n & \text{if $n \in N$}\\
0 & \text{if $n = 0$}\\
n & \text{if $n \in -N$}.
\end{cases}
\]
Then for $n,m \in Z$ we have $f(n+m) = n + m = f(n) + f(m)$ and $f(nm) = nm = f(n)f(m)$. Finally, if $n < m$ then $f(n) = n < m = f(m)$.
\end{proof}

\begin{**}
Show that addition and multiplication on $\mathbb{N}$ satisfy associativity, commutativity and distributivity.\newline
\end{**}

Associative Law of Addition
\begin{proof}
Fix $a$ and $b$ and let $S$ be the set of natural numbers for which the associative law holds. Then
\[
(a+b)+1 = (a+b)' = a + b' = a + (b+1)
\]
so $1 \in S$. Suppose that $c \in S$. Then $(a + b) + c = a + (b + c)$, and
\[
(a + b) + c' = ((a + b) + c)' = (a + (b + c))' = a + (b + c)' = a + (b + c')
\]
so $c' \in S$. Thus the law holds for all natural numbers.
\end{proof}

Commutative Law of Addition
\begin{proof}
Fix $b$ and let $S$ be the set of all $a \in \mathbb{N}$ for which the law holds. We have
\[
b + 1 = 1 + b = b'
\]
so that $1 \in S$. Let $a \in S$. Then $a + b = b + a$. Thus
\[
(a + b)' = (b + a)' = b + a'.
\]
But also, $a' + b = (a + b)'$ by the definition of addition. Thus $a' \in S$ and the law holds for all $a$.
\end{proof}

Commutative Law of Multiplication
\begin{proof}
Fix $b$ and let $S$ be the set of all $a$ for which the law holds. We have $b \cdot 1 = b$ and $1 \cdot b = b$. Thus $1 \in S$. Let $a \in S$. Then $ab = ba$. Note that
\[
ab + b = ba + b = ba'
\]
and by the definition of multiplication we have $a'b = ab + b$ so that $a'b = ba'$ and $a' \in S$. Thus the law holds for all $a$.
\end{proof}

Distributive Law
\begin{proof}
Fix $a$ and $b$ and let $S$ be the set of all $c$ for which the law holds. We have
\[
a(b+1) = ab' = ab + a = ab + a \cdot 1
\]
so $1 \in S$. Let $c \in S$. Then $a(b+c) = ab + ac$. Thus
\[
a(b+c') = a(b+c)' = a(b+c) + a = (ab + ac) + a = ab + (ac + a) = ab + ac'
\]
so that $c \in S$. Thus the law holds for all $c$.
\end{proof}

Associative Law of Multiplication
\begin{proof}
Fix $a$ and $b$ and let $S$ be the set of all $c$ such that the law holds. Note that
\[(xy) \cdot 1 = xy = x(y \cdot 1)
\]
so that $1 \in S$. Let $c \in S$. Then $(ab)c = a(bc)$. Thus
\[
(ab)c' = (ab)c + ab = a(bc) + ab = a(bc + b) = a(bc')
\]
and $c' \in S$. Thus the law holds for all $c$.
\end{proof}

\begin{lemma}
For $a,b \in \mathbb{N}$ we have $a \neq a + b$.
\end{lemma}
\begin{proof}
Fix $a$ and let $S$ be the set of all $b$ such that statement is true. We know $1 \neq a' = a+1$ so $1 \in S$. Let $y \in S$ so that $a \neq a + b$. Then $b' \neq (a+b)' = a + b'$. Thus $b' \in S$ and the statement is true for all $b$.
\end{proof}

\begin{**}
For $a,b,c \in \mathbb{N}$ show the following:\\
1) Exactly one of $a=b$, there exists $u$ such that $a=b+u$, there exists $v$ such that $b=a+v$ is true.\\
2) If $a<b$ and $b<c$ then $a<c$.\\
3) If $a<b$ then $a+c<b+c$.\\
\end{**}
\begin{proof}
1) By Lemma 2, the first and second and first and third conditions cannot both be true. Similarly the second and third conditions cannot both be true since
\[
a = b+u = (a+v) + u = a + (v + u).
\]
So at most one of the conditions is true for all $a,b \in \mathbb{N}$. Now fix $a$ and let $S$ be the set of all $b$ such that at least one of the conditions holds. For $b = 1$ we have either $a = 1 = b$ or $a = u' = u + 1 = b + u$ for some $u$. Thus $1 \in S$. Let $b \in S$. Then either $a = b$, so that
\[
b' = b + 1 = a + 1
\]
and $b'$ satisfies the third condition, or $a = b + u$ so that if $u = 1$ then $a = b + 1 = b'$ and $b'$ satisfies the first condition, else if $u \neq 1$ then for some $w$, $u = w' = 1 + w$ and
\[
a = b + u = b + w' = b + (w + 1) = b + (1 + w) = (b+1) + w = b' + w
\]
and $b'$ satisfies the second condition, or finally $b = a + v$ so that
\[
b' = (a + v)' = a + v'
\]
and $b'$ satisfies the third condition. In all cases, $b' \in S$ and so the statement holds for all $b$.\newline

2) Let $a < b$ and $b < c$. Then there exists $v,w \in \mathbb{N}$ such that $b = a + v$ and $c = b + w$. Thus
\[
c = (a + v) + w = a + (v + w)
\]
and so $a < c$.\newline

3) If $a < b$ then $a + u = b$ for some $u$. Then
\[
b + c = (a + u) + c = (u + a) + c = u + (a + c) = (a + c) + u
\]
and so $b + c > a + c$.
\end{proof}

\begin{**}
Let $\sim$ be an equivalence relation on $\mathbb{N} \times \mathbb{N}$ such that $(a,b) \sim (c,d)$ if and only if $a+d = b+c$. Show that the set of equivalence classes of this relation is the set of integers.
\end{**}
\textbf{** Definition 9.1}
\textit{Let $\mathbb{Z}$ be the set of equivalence classes of $\sim$. Let $X,Y \in \mathbb{Z}$ such that $(a_1, b_1) \in X$ and $(a_2, b_2) \in Y$. Define
\[
X + Y = \overline{(a_1 + a_2, b_1 + b_2)}
\]
\[
X \cdot Y = XY = \overline{(a_1a_2 + b_1b_2, a_1b_2 + a_2b_1)}
\]}

\textbf{** Problem 9.2}
\textit{The operations $+$ and $\cdot$ are well defined. That is, if $(a_1, b_1) \sim (c_1, d_1)$ and $(a_2, b_2) \sim (c_2, d_2)$ then
\[
(a_1 + a_2, b_1 + b_2) \sim (c_1 + c_2, d_1 + d_2)
\]
and
\[
(a_1a_2 + b_1b_2, a_1b_2 + a_2b_1) \sim (c_1c_2 + d_1d_2, c_1d_2 + c_2d_1).
\]}
\begin{proof}
Let $(a_1, b_1) \sim (c_1, d_1)$ and $(a_2, b_2) \sim (c_2, d_2)$. Then $a_1 + d_1 = b_1 + c_1$ and $a_2 + d_2 = b_2 + c_2$. Adding these equations gives us
\[
(a_1 + a_2) + (d_1 + d_2) = (b_1 + b_2) + (c_1 + c_2)
\]
which implies
\[
(a_1 + a_2, b_1 + b_2) \sim (c_1 + c_2, d_1 + d_2).
\]
A longer calculation can be done to show that
\[
a_1a_2 + b_1b_2 + c_1d_2 + c_2d_1 = a_1b_2 + a_2b_1 + c_1c_2 + d_1d_2
\]
which implies
\[
(a_1a_2 + b_1b_2, a_1b_2 + a_2b_1) \sim (c_1c_2 + d_1d_2, c_1d_2 + c_2d_1).
\]
\end{proof}

\textbf{** Problem 9.3 (Associativity of Addition)}
\textit{For all $a,b,c \in \mathbb{Z}$ we have $(a+b) + c = a + (b + c)$.}
\begin{proof}
Let $(a_1,a_2) \in a$, $(b_1,b_2) \in b$ and $(c_1,c_2) \in c$. Then we have
\begin{align*}
(a + b) + c
&= \left ( \overline{(a_1, a_2)} + \overline{(b_1,b_2)} \right ) + \overline{(c_1, c_2)}\\
&= \overline{(a_1 + b_1, a_2 + b_2)} + \overline{(c_1, c_2)}\\
&= \overline{((a_1 + b_1) + c_1, (a_1 + b_1) + c_2)}\\
&= \overline{(a_1 + (b_1 + c_1), a_2 + (b_2 + c_2))}\\
&= \overline{(a_1, a_2)} + \overline{(b_1 + c_1, b_2 + c_2)}\\
&= \overline{(a_1, a_2)} + \left ( \overline{(b_1, b_2)} + \overline{(c_1, c_2)} \right )\\
&= a + (b + c)
\end{align*}
\end{proof}

\textbf{** Problem 9.4 (Commutativity of Addition)}
\textit{For all $a,b \in \mathbb{Z}$ we have $a + b = b + a$.}
\begin{proof}
Let $(a_1, a_2) \in a$ and $(b_1, b_2) \in b$. Then
\[
a + b = \overline{(a_1, a_2)} + \overline{(b_1, b_2)} = \overline{(a_1 + b_1, a_2 + b_2)} = \overline{(b_1 + a_1, b_2 + a_2)} = \overline{(b_1, b_2)} + \overline{(a_1, a_2)} = b + a.
\]
\end{proof}

\textbf{** Problem 9.5 (Additive Identity)}
\textit{There exists $n \in \mathbb{Z}$ such that for all $a \in \mathbb{Z}$ we have $n + a = a$. From here forward we will call this $n$, $0$.}
\begin{proof}
Let $n = \overline{(1, 1)}$. Let $a \in \mathbb{Z}$ such that $(a_1, a_2) \in a$. Then
\[
n + a = \overline{(1, 1)} + \overline{(a_1, a_2)} = \overline{(1 + a_1, 1 + a_2)}.
\]
Note that $\overline{(1 + a_1, 1 + a_2)} = \overline{(a_1, a_2)}$ because
\[
1 + a_1 + a_2 = 1 + a_2 + a_1.
\]
\end{proof}

\textbf{** Problem 9.5 (Additive Inverse)}
\textit{For all $a \in \mathbb{Z}$ there exists $b \in \mathbb{Z}$ such that $b + a = 0$. From here forward we will call this $b$, $-a$.}
\begin{proof}
Let $a \in \mathbb{Z}$ such that $(a_1, a_2) \in a$ and consider $b = \overline{(a_2, a_1)}$. Then
\[
b + a = \overline{(a_2, a_1)} + \overline{(a_1, a_2)} = \overline{(a_2 + a_1, a_1 + a_2)} = \overline{(1, 1)}.
\]
\end{proof}

\textbf{** Problem 9.6 (Associativity of Multiplication)}
\textit{For all $a,b,c \in \mathbb{Z}$ we have $(ab)c = a(bc)$.}
\begin{proof}
Let $(a_1,a_2) \in a$, $(b_1,b_2) \in b$ and $(c_1,c_2) \in c$. Then we have
\begin{align*}
(ab)c
&= \left ( \overline{(a_1, a_2)} \cdot \overline{(b_1, b_2)} \right ) \cdot \overline{(c_1, c_2)}\\
&= \overline{(a_1b_1 + a_2b_2, a_1b_2 + a_2b_1)} \cdot \overline{(c_1, c_2)}\\
&= \overline{((a_1b_1 + a_2b_2)c_1 + (a_1b_2 + a_2b_1)c_2, (a_1b_1 + a_2b_2)c_2 + (a_1b_2 + a_2b_1)c_1)}\\
&= \overline{(a_1b_1c_1 + a_1b_2c_2 + a_2b_2c_1 + a_2b_1c_2, a_2b_2c_2 + a_2b_1c_1 + a_1b_1c_2 + a_1b_2c_1)}\\
&= \overline{(a_1(b_1c_1 + b_2c_2) + a_2(b_1c_2 + b_2c_1), a_2(b_1c_1 + b_2c_2) + a_1(b_1c_2 + b_2c_1))}\\
&= \overline{(a_1, a_2)} \cdot \overline{(b_1c_1 + b_2c_2, b_1c_2 + b_2c_1)}\\
&= \overline{(a_1, a_2)} \cdot \left ( \overline{(b_1, b_2)} \cdot \overline{(c_1, c_2)} \right )\\
&= a(bc)
\end{align*}
\end{proof}

\textbf{** Problem 9.7 (Commutativity of Multiplication)}
\textit{For all $a, b \in \mathbb{Z}$ we have $ab = ba$.}
\begin{proof}
Let $(a_1, a_2) \in a$ and $(b_1, b_2) \in b$. Then
\[
ab = \overline{(a_1, a_2}) \cdot \overline{(b_1, b_2)} = \overline{(a_1b_1 + a_2b_2, a_1b_2 + a_2b_1)} = \overline{(b_1a_1 + b_2a_2, b_1a_2 + b_2a_1)} = \overline{(b_1, b_2)} \cdot \overline{(a_1, a_2)} = ba.
\]
\end{proof}

\textbf{** Problem 9.8 (Multiplicative Identity)}
\textit{There exists $e \in \mathbb{Z}$ such that for all $a \in \mathbb{Z}$ we have $ea = a$. From here forward we will call this $e$, $1$.}
\begin{proof}
Let $e = \overline{(1+1, 1)}$ and let $a \in \mathbb{Z}$ such that $(a_1, a_2) \in a$. Then
\begin{align*}
ea
&= \overline{(1+1, 1)} \cdot \overline{(a_1, a_2)}\\
&= \overline{((1+1)a_1 + 1 \cdot a_2, (1+1)a_2 + 1 \cdot a_1)}\\
&= \overline{(a_1 + (a_1 + a_2), a_2 + (a_1 + a_2))}\\
&= \overline{(a_1, a_2)}\\
&= a.
\end{align*}
\end{proof}

\textbf{** Problem 9.9 (Distributivity)}
\textit{For all $a,b,c \in \mathbb{Z}$ we have $a(b+c) = ab + ac$.}
\begin{proof}
Let $(a_1,a_2) \in a$, $(b_1,b_2) \in b$ and $(c_1,c_2) \in c$. Then we have
\begin{align*}
a(b+c)
&= \overline{(a_1, a_2)} \cdot \left ( \overline{(b_1, b_2)} + \overline{(c_1, c_2)} \right )\\
&= \overline{(a_1, a_2)} \cdot \overline{(b_1 + c_1, b_2 + c_2)}\\
&= \overline{(a_1(b_1 + c_1) + a_2(b_2 + c_2), a_1(b_2 + c_2) + a_2(b_1 + c_1))}\\
&= \overline{(a_1b_1 + a_1c_1 + a_2b_2 + a_2c_2, a_1b_2 + a_1c_2 + a_2b_1 + a_2c_1)}\\
&= \overline{((a_1b_1 + a_2b_2) + (a_1c_1 + a_2c_2), (a_1b_2 + a_2b_1) + (a_1c_2 + a_2c_1))}\\
&= \overline{(a_1b_1 + a_2b_2, a_1b_2 + a_2b_1)} + \overline{(a_1c_1 + a_2c_2, a_1c_2 + a_2c_1)}\\
&= \overline{(a_1, a_2)} \cdot \overline{(b_1, b_2)} + \overline{(a_1, a_2)} \cdot \overline{(c_1, c_2)}\\
&= ab + ac.
\end{align*}
\end{proof}

\textbf{** Definition 9.10 (Embedding of $\mathbb{N}$)}
\textit{Let $f : \mathbb{N} \rightarrow \mathbb{Z}$ be a function defined by
\[
f(n) = \overline{(n+1,1)}.
\]}

\textbf{** Problem 9.11}
\textit{The function $f$ is injective.}
\begin{proof}
Let $a,b \in \mathbb{N}$ such that $f(a) = f(b)$. Then we have $\overline{(a+1, 1)} = \overline{(b+1,1)}$ and so $(a+1) + 1 = 1 + (b+1)$ which means that $a = b$. Thus $f$ is injective.
\end{proof}

\textbf{** Problem 9.12}
\textit{For all $a,b \in \mathbb{N}$ we have
\[
f(a+b) = f(a) + f(b)
\]
and
\[
f(ab) = f(a)f(b).
\]}
\begin{proof}
Let $a,b \in \mathbb{N}$, then $f(a) = \overline{(a+1, 1)}$ and $f(b) = \overline{(b+1, 1)}$. Then
\[
f(a+b) = \overline{(a+b+1, 1)} = \overline{((a+b+1) + 1, 1 + 1)} = \overline{((a+1) + (b+1), 1 + 1)} = \overline{(a+1, 1)} + \overline{(b+1, 1)} = f(a) + f(b).
\]
Similarly,
\begin{align*}
f(ab)
&= \overline{(ab+1, 1)}\\
&= \overline{(ab + 1 + a + b + 1, a + b + 1 + 1)}\\
&= \overline{((a+1)(b+1) + 1, (a+1) + (b+1))}\\
&= \overline{(a+1, 1)} \cdot \overline{(b+1, 1)}\\
&= f(a)f(b).
\end{align*}
\end{proof}

\textbf{** Definition 9.13}
\textit{Let $a,b \in \mathbb{Z}$ such that $(a_1, a_2) \in a$ and $(b_1, b_2) \in b$. Then
\[
a < b \text{ if } a_1 + b_2 < a_2 + b_1.
\]}

\textbf{** Problem 9.14}
\textit{The relation $<$ is well-defined.}
\begin{proof}
Let $\overline{(a_1, a_2)}, \overline{(b_1,b_2)}, \overline{(c_1,c_2)}, \overline{(d_1,d_2)} \in \mathbb{Z}$ such that $\overline{(a_1, a_2)} < \overline{(b_1, b_2)}$, $\overline{(a_1, a_2)} \sim \overline{(c_1, c_2)}$ and $\overline{(b_1, b_2)} \sim \overline{(d_1, d_2)}$. Then we know that
\[
a_1 + b_2 < a_2 + b_1,
\]
\[
a_1 + c_2 = a_2 + c_1
\]
and
\[
b_1 + d_2 = b_2 + d_1.
\]
Adding the desired quantities to the inequality results in
\[
a_1 + a_2 + b_1 + b_2 + c_1 + d_2 < a_1 + a_2 + b_1 + b_2 + c_2 + d_1
\]
which gives us the result
\[
\overline{(c_1, c_2)} < \overline{(d_1, d_2)}.
\]
\end{proof}

\textbf{** Problem 9.15}
\textit{The relation $<$ is an ordering on $\mathbb{Z}$.}
\begin{proof}
Let $(a_1,a_2) \in a$, $(b_1,b_2) \in b$ and $(c_1,c_2) \in c$. Then it's clear that if $a < b$ then
\[
a_1 + b_2 < a_2 + b_1
\]
and so $a \neq b$ and $a$ is not greater than $b$. The same argument holds for $a > b$. Note that $a$ must be at least greater than, less than or equal to $b$ however, because of the ordering of $\mathbb{N}$.\newline

Suppose that $a < b$ and $b < c$. Then we have
\[
a_1 + b_2 < a_2 + b_1
\]
and
\[
b_1 + c_2 < b_2 + c_1.
\]
Adding these gives the desired result that
\[
a_1 + c_2 < a_2 + c_1
\]
so $a < c$.\newline

Suppose that $a < b$. Then $a + c = \overline{(a_1 + c_1, a_2 + c_2)}$ and $b + c = \overline{(b_1 + c_1, b_2 + c_2)}$. Since
\[
a_1 + b_2 < a_2 + b_1
\]
it's clear that
\[
a_1 + b_2 + c_1 + c_2 < a_2 + b_1 + c_1 + c_2
\]
which shows that $a + c < b + c$.\newline

Finally, suppose that $a < b$ and $0 < c$. Then $a_1 + b_2 < a_2 + b_1$ and $c_2 < c_1$. Combining these inequalities gives us the desired result of
\[
(a_1c_1 + a_2c_2) + (b_1c_2 + b_2c_1) < (a_1c_2 + a_2c_1) + (b_1c_1 + b_2c_2)
\]
which implies that $ac < bc$.
\end{proof}

\textbf{** Problem 9.16}
\textit{For all $n \in \mathbb{N}$, we have $f(n) > 0$. Additionally, if $a \in \mathbb{Z}$ such that $a > 0$, then $a = f(n)$ for some $n \in \mathbb{N}$.}
\begin{proof}
Let $n \in \mathbb{N}$. Then $f(n) = \overline{(n+1, 1)}$ and $n + 2 > 2$. Thus $f(n) > 0$.\newline

Let $a \in \mathbb{Z}$ such that $(a_1, a_2) \in a$ and $a > 0$. Then $a_1 > a_2$ so there exists some $b$ such that $\overline{(a_1, a_2)} = \overline{(a_1 + b, 1)}$ so that $a = f(n)$ for some $n \in \mathbb{N}$.
\end{proof}

Thus there is a bijection between $\mathbb{N}$ and the positive elements of $\mathbb{Z}$. Hence, $\mathbb{Z}$ is a ordered integral domain where the positive elements are well ordered.

\end{flushleft}
\end{document}