\documentclass{article}
\usepackage{amsmath,amssymb,amsfonts,amsthm,fullpage}

\newtheorem{problem}{Problem}
\newtheorem{lemma}{Lemma}
\newtheorem{**}{** Problem}

\begin{document}
\begin{flushright}
Kris Harper\\

MATH 20700\\

November 24, 2008
\end{flushright}

\begin{center}
Homework 8
\end{center}

\begin{flushleft}

\begin{**}
The space $\ell_n^p (\mathbb{R})$ is complete for $1 \leq p \leq \infty$.
\end{**}
\begin{proof}
Let $(\mathbf{a}_j)$ be a Cauchy sequence on $\ell_n^p (\mathbb{R})$ and let $\varepsilon' > 0$. Then there exists $N$ such that for all $i,j > N$ we have
\[
||\mathbf{a}_i - \mathbf{a}_j||_p = \left (\sum_{k=1}^{n} |a_{i,k}-a_{j,k}|^p \right )^{\frac{1}{p}} < \varepsilon'
\]
so $|a_{i,k}-a_{j,k}|^p \leq \sum_{k=1}^{n} |a_{i,k}-a_{j,k}|^p < \varepsilon'^p$ and $|a_{i,k}-a_{j,k}| < \varepsilon'$. Thus the $k$th coordinate of the terms of $(\mathbf{a}_j)$ forms a Cauchy sequence which converges to some $b_k$. Then let $\mathbf{b}=(b_1,b_2, \dots , b_n)$, let $\varepsilon > 0$ and consider $\varepsilon/n^{1/p}$. For all $k \leq n$ there exists some $N_k$ such that for $j > N_k$ we have $|a_{j,k}-b_{k}| < \varepsilon/n^{1/p}$. Let $N$ be the largest of all such $N_k$ so that for all $j>N$ we have $|a_{j,k}-b_{k}| < \varepsilon/n^{1/p}$. Then $|a_{j,k}-b_{k}|^p < \varepsilon^p/n$ and $\sum_{k=1}^{n} |a_{i,k}-b_{k}|^p < \varepsilon^p$. Then $||\mathbf{a}_i - \mathbf{b}||_p < \varepsilon$ for all $n > N$. Thus $\lim_{n \rightarrow \infty} \mathbf{a}_n = \mathbf{b}$ which means $\ell_n^p$ is complete.
\end{proof}

\begin{**}
The space $\ell_n^p (\mathbb{C})$ is complete for $1 \leq p \leq \infty$.
\end{**}
\begin{proof}
This follows from ** Problem 1 by changing $\mathbb{R}$ to $\mathbb{C}$.
\end{proof}

\begin{**}
$\mathbb{Q}$ with the $p$-adic metric is not complete.
\end{**}
\begin{proof}
The recursive sequence $a_n = 1/2(a_{n-1} + 2/a_{n-1})$ will converge to $\sqrt{2}$ in both the usual and $p$-adic metrics.
\end{proof}

\begin{problem}
1) Every bounded sequence in $\mathbb{C}^n$ with the usual metric has a convergent subsequence.\\
2) Prove 1) for $\mathbb{R}^n$ by proceeding coordinate by coordinate.
\end{problem}
\begin{proof}
1) Let $(a_m)_{m \in \mathbb{N}})$ be a bounded sequence in $\mathbb{C}^n$. Denote $a_m = (a_{m,1}, a_{m,2}, \dots , a_{m,n})$. Use induction on $n$. For $n=1$ we simply have $\mathbb{C}$ where we know the statement is true. Assume that the statement is true for $n-1$. Let $a_m'$ be the $n-1$-tuple $(a_{m,1}, a_{m,2}, \dots , a_{m,n-1})$. Then $(a_m')$ is a bounded sequence in $\mathbb{C}^{n-1}$ and so $(a_m')$ has a convergent subsequence in $\mathbb{C}^{n-1}$. Call this subsequence $(a'_{m_j})$. Now the sequence $(a_{m_j,n})$ is a bounded sequence in $\mathbb{C}$ and has a convergent subsequence. Taking then the corresponding subsequence of $(a_{m_j})$, we get a convergent subsequence of $(a_m)$ in $\mathbb{C}^n$.\newline

2) Let $(a_m)$ be a bounded sequence in $\mathbb{R}^n$. Denote $a_m = (a_{m,1}, a_{m,2}, \dots , a_{m,n})$. Note that since $(a_m)$ is bounded in $\mathbb{R}^n$, $(a_{m,1})$ forms a bounded sequence in $\mathbb{R}$. Then this has a convergent subsequence $(a_{m_{j1},1})$. The same can be said for $(a_{m,2}$ with a convergent subsequence $(a_{m_{j2},2})$. Taking the common terms in these sequences from $(a_m)$ we have a convergent subsequence in $\mathbb{R}^2$, $(a_{m_{j12},1,2})$. Continuing in this process, we eventually arrive at a convergent subsequence of $(a_m)$, $(a_{m_{j123\dots n}1,2,3, \dots , n})$.
\end{proof}

\begin{problem}
The spaces $\mathcal{B} (X, \mathbb{R})$ and $\mathcal{B} (X, \mathbb{C})$ are complete.
\end{problem}
\begin{proof}
Let $F = \mathbb{R}$ or $F = \mathbb{C}$. Let $(f_n)_{n \in \mathbb{N}}$ be a Cauchy sequence in $\mathcal{B} (X, F)$ and let $f$ be the pointwise limit of $(f_n)_{n \in \mathbb{N}}$. Let $\varepsilon > 0$ and choose $N \in \mathbb{N}$ such that for all $n,m > N$ we have
\[
\sup_{x \in X} |f_n(x) - f_n(x)| < \varepsilon/2.
\]
For $y \in X$ choose $N' \geq N$ such that
\[
|f_{N'}(y) - f(y)| < \varepsilon/2.
\]
Then
\[
|f_n(y) - f(y)| \leq |f_n(y) - f_{N'}(y)| + |f_{N'}(y) - f(y)| < \varepsilon
\]
if $n \geq N$. But this implies that $|f(x)| < |f_n(x)| + \varepsilon$ for all $x \in X$ where $f_n$ is a bounded function. Thus $f$ is bounded and in $\mathcal{B} (X, F)$.
\end{proof}

\begin{problem}
Define $f : \mathbb{R} \rightarrow \mathbb{R}$ as
\[
f(x) =
\begin{cases}
\frac{1}{q} & \text{if $x = \frac{p}{q}$ (reduced to lowest terms, $x \neq 0$)}\\
0 & \text{if $x = 0$ or $x \notin \mathbb{Q}$}.
\end{cases}
\]
Show that $f$ is continuous at $0$ and every irrational point. Show that $f$ is not continuous at any nonzero rational point.
\end{problem}
\begin{proof}
Note that $0 \leq f(x) \leq |x|$ for all $x \in \mathbb{R}$. Consider $a \leq 0$ and for all $\varepsilon > 0$ let $\delta = \varepsilon$. Then if $0 < |a-x| < \delta$ we have $x \in (a - \delta ; a + \delta)$ and so $f(x) \in (a - \delta ; a + \delta) = (a - \varepsilon ; a + \varepsilon)$. Thus $|a-f(x)| < \varepsilon$, but $a \leq 0$ and so $|-(-a+f(x))| < \varepsilon$ which means $0 \leq |f(x)| \leq |-a+f(x)| < \varepsilon$. Now consider $a > 0$ and let $\delta = \varepsilon + a$. Then if $0 < |a-x| < \delta$ we have $f(x) \in (a - \delta ; a + \delta) = (- \varepsilon ; \varepsilon)$ and so $|f(x)| < \varepsilon$. Thus for all $\varepsilon > 0$ there exists a $\delta> 0$ such that for all $x \in \mathbb{R}$ when $0 < |a-x| < \delta$ we have $|f(x)| < \varepsilon$ and so $\lim_{x \rightarrow a} f(x) = 0$ for all $x \in \mathbb{R}$. But we know that for nonzero rationals, $f(x) \neq 0$ because of how $f$ is defined and since a function is only continuous at $a$ if $\lim_{x \rightarrow a} f(x) =f(a)$ we have $f$ is discontinuous at all nonzero rationals.
\end{proof}

\begin{problem}
1) Let $X$ and $X'$ be metric spaces and suppose that $X$ has the discrete metric. Show that any function $f : X \rightarrow X'$ is continuous.\\
2) Let $X = \mathbb{R}$ with the usual metric and let $f : X \rightarrow X$ be a polynomial function. Show that $f$ is continuous.\\
3) Let $X = \mathbb{R}$ with the usual metric and $X' = \mathbb{R}$ with the discrete metric. Describe all continuous functions from $X$ to $X'$.
\end{problem}
\begin{proof}
1) Note that for $f$ to be continuous, for every open set $A \subseteq X'$, it must be the case that $f^{-1}(A)$ is open in $X$. But since $X$ has the discrete metric, every set is open. Thus $f$ must be continuous.\newline

2) Suppose that $f$ is a polynomial function such that $f = \sum_{i=0}^{n} a_i x^i$. Let $A \subseteq X$ be open and let $x \in f^{-1} (A)$. Then $f(x) \in A$ and since $A$ is open there exists $\varepsilon > 0$ such that $B_{\varepsilon}(f(x)) \subseteq A$. Note that $x \in f^{-1}(B_{\varepsilon}(f(x)))$. Choose $\delta < \varepsilon^{1/n}$. Then $B_{\delta}(x) \subseteq f^{-1}(B_{\varepsilon}(f(x))) \subseteq f^{-1}(A)$. Thus $f^{-1}(A)$ is open and so $f$ is continuous.\newline

3) The only continuous functions from $X$ to $X'$ are constant functions. To see this, assume $f$ is continuous, let $\varepsilon = 1/2$ and let $x \in X$. Then there exists $\delta > 0$ such that $d(x,y) < \delta$ implies that $d'(f(x), f(y)) < 1/2$. But this will only happen if $f(x) = f(y)$. Thus $f(x) = f(y)$ for all $x,y \in X$ which means $f$ is constant.
\end{proof}

\begin{problem}
Suppose that $(X,d)$ and $(X',d')$ are metric spaces and that $f : X \rightarrow X'$ is continuous. Prove or disprove the following:\\
1) If $A$ is an open subset of $X$, then $f(A)$ is an open subset of $X'$.\\
2) If $B$ is a closed subset of $X'$, then $f^{-1}(B)$ is a closed subset subset of $X$.\\
3) If $A$ is a closed subset of $X$, then $f(A)$ is a closed subset $X'$.\\
4) If $A$ is a bounded subset of $X$, then $f(A)$ is a bounded subset of $X'$.\\
5) If $B$ is a bounded subset of $X'$, then $f^{-1}(B)$ is a bounded subset of $X$.\\
6) If $A \subseteq X$ and $x_0$ is an isolated point of $A$, then $f(x_0)$ is an isolated point of $f(A)$.\\
7) If $A \subseteq X$, $x_0 \in A$ and $f(x_0)$ is an isolated point of $f(A)$, then $x_0$ is an accumulation point of $f(A)$.\\
8) If $A \subseteq X$ and $x_0$ is an accumulation point of $A$, then $f(x_0)$ is an accumulation point of $f(A)$.\\
9) If $A \subseteq X$, $x_0 \in X$ and $f(x_0)$ is an accumulation point of $f(A)$, then $x_0$ is an accumulation point of $A$.\\
10) Do any of the above statements change if $X$ and $X'$ are complete?
\end{problem}
\begin{proof}
1) False. Let $X = X' = \mathbb{R}$ with the usual metric. Let $f(x) = x^2$ and let $A = (-1,1)$. Then $A$ is an open subset of $X$, but $f(A) = [0,1)$ is not open in $X'$.\newline

2) False. Let $X = \mathbb{R}^+$ and $X' = \mathbb{R}$ with the usual metric. Let $f(x) = \sin x$ and let $B = [-1,1]$. Then $B$ is a closed subset of $X'$, but $f^{-1}(B) = (0, \infty)$ is not closed in $X$.\newline

3) False. Let $X = X' = \mathbb{R}$ with the usual metric. Let $f(x) = \arctan(x)$ and let $A = \mathbb{R}$. Then $A$ is a closed subset of $X$, but $f(A) = (-\pi/2, \pi/2)$ is not closed in $X'$.\newline

4) False. Let $X = X' = \mathbb{R}$ such that $X$ has the discrete metric and $X'$ has the usual metric. Let $f(x) = x$ and let $A = \mathbb{R}$. Then $A$ is a bounded subset of $X$ because $\mathbb{R} \subseteq B_1(0)$, but $f(A) = \mathbb{R}$ is not bounded in $X'$.\newline

5) False. Let $X = X' = \mathbb{R}$ with the usual metric. Let $f(x) = \sin x$ and let $B = [-1,1]$. Then $B$ is a bounded subset of $X'$, but $f^{-1}(B) = \mathbb{R}$ is not bounded in $X$.\newline

6) False. Let $X = X' = \mathbb{R}$ such that $X$ has the discrete metric and $X'$ has the usual metric. Let $f(x) = x$ and let $A = \mathbb{R}$. Then every point in $A$ is an isolated point, but none of the points in $f(A) = \mathbb{R}$ are isolated.\newline

7) True. Let $A \subseteq X$ and let $x_0$ in $A$ then $f(x_0) \in f(A)$. The statement can be restated as, if $x_0$ is not an isolated point of $A$, then $f(x_0)$ is not an isolated point of $f(A)$. But since $x_0 \in A$, if $x_0$ is not an isolated point of $A$ we know that $x_0$ must be an accumulation point of $A$. The same can be said about $f(x_0)$ and $f(A)$. Then the proof is the same as 8).\newline

8) True. Let $x_0$ be an accumulation point of $A$. Then there exists a sequence $(x_n)$ in $A$ which converges to $x_0$. But since $f$ is continuous we have $\lim_{n \rightarrow \infty} f(x_n) = f(x_0)$. Since $f(x_n) \in f(A)$ for all $n$, it must be the case that $f(x_0)$ is an accumulation point of $f(A)$.\newline

9) False. Let $X = X' = \mathbb{R}$ such that $X$ has the discrete metric and $X'$ has the usual metric. Let $f(x) = x$ and let $A = [-1,1]$. Then $f(1) = 1$ is an accumulation point of $f(A) = [-1,1]$, but since there are no accumulation points in $X$ and so $1$ is not an accumulation point of $A$.\newline

10) No, all of the counterexamples use complete metric spaces and the statements which are true for arbitrary metric spaces will be true for compete metric spaces as well.
\end{proof}

\begin{problem}
Show that $\ell_n^p (\mathbb{C})$ and $\ell_n^q (\mathbb{C})$ are homeomorphic for $1 \leq p < q \leq \infty$.
\end{problem}
\begin{proof}
Consider the identity map $I(x) = x$ from $\ell_n^p (\mathbb{C})$ to $\ell_n^q (\mathbb{C})$. We already know that for $p < q$ the unit ball in $\ell_n^p (\mathbb{C})$ is contained in the unit ball in $\ell_n^q (\mathbb{C})$. Consider $(z_1, z_2, \dots , z_n) \in \mathbb{C}^n$ such that $\max_{1 \leq i \leq n} (|x_i|) < 1/n$. Then $\sum_{i=1}^{n} |x_i| < 1$. Thus the $1/n$ ball in the $\ell_n^{\infty}$ metric is contained in the unit ball in the $\ell_n^1$ metric. Therefore, if we take the unit ball in $\ell_n^q (\mathbb{C})$ and multiply each coordinate by a factor of $1/n$, then this set of points is in the unit ball in $\ell_n^p (\mathbb{C})$. This shows that $I$ and $I^{-1}$ are both continuous and since $I$ is a bijection, we see that it is a homeomorphism.
\end{proof}

\begin{problem}
Define an isometry as a bijection $f : X \rightarrow X'$ such that $d'(f(x_1), f(x_2)) = d(x_1, x_2)$ for all $x_1, x_2 \in X$. Show that this implies $f$ is a homeomorphism.
\end{problem}
\begin{proof}
Let $\varepsilon > 0$ such that $\delta = \varepsilon$ and let $x_1 \in X$. Then for all $x_2 \in X$ such that $d(x_1, x_2) < \delta$ we have $d(x_1, x_2) = d'(f(x_1), f(x_2)) < \varepsilon$. Thus $f$ is continuous. Also, since $f$ is a bijection, we have $d(f^{-1}(x_1), f^{-1}(x_2)) = d(x_1, x_2)$ and using a similar proof as above we see that $f^{-1}$ is continuous. Since $f$ is a bijection and $f$ and $f^{-1}$ are continuous, $f$ is a homeomorphism.
\end{proof}

\begin{problem}
Let $X = \mathbb{R}$ with the discrete metric and let $X' = \mathbb{R}$ with the usual metric. Define $f : X \rightarrow X'$ by $f(x) = x$. Show that $f$ is a continuous bijection that is not a homeomorphism.
\end{problem}
\begin{proof}
Problem 4 Part 1) shows that $f$ is continuous and $f$ is clearly a bijection since $f^{-1}(x) = x$ and $f$ is injective and surjective. Problem 4 Part 3) shows that the only continuous functions from $X'$ to $X$ are constant functions. Thus $f^{-1}$ is not continuous and $f$ is not a homeomorphism.
\end{proof}

\begin{problem}
Let $(X, d)$ be a metric space. Let $G$ be the collection of all homeomorphisms from $X$ to $X$. Prove that, under compositions of functions, $G$ is a group and the collection of all isometries of $X$ is a subgroup of $G$.
\end{problem}
\begin{proof}
Let $f,g \in G$ and consider $f \circ g$. This function is a bijection since injective and surjective properties hold under composition. Because it's a homeomorphism, $g$ is an open map which means $f \circ g$ is continuous because the preimage of an open set under $f \circ g$ is open. Similarly, $(f \circ g)^{-1}$ is continuous. Thus $G$ is closed under function composition. The identity function is $I(x) = x$. Note that by definition of function composition, $I \circ f = f$. Also, the associative law holds as usual for function composition. Finally for $f,g \in G$ consider the function $g \circ f^{-1}$. Then $(g \circ f^{-1}) \circ f = g$ and so we have solvability. Thus $G$ is a group. Let $G'$ be the set of isometries of $X$ and now let $f,g \in G'$. Then consider $(f \circ g)(x_1)$ and $(f \circ g)(x_2)$. Then $d(x_1, x_2) = d(g(x_1), g(x_2))$ and since $g(x_1), g(x_2) \in X$, we have $d(g(x_1), g(x_2)) = d(f(g(x_1)), f(g(x_2)))$. Thus $f \circ g$ is an isometry as well. Therefore $G'$ is a subgroup of $G$.
\end{proof}

\begin{**}
Show that $\mathcal{B} (X, F)$ is an algebra.
\end{**}
\begin{proof}
We see that $\mathcal{B} (X, F)$ satisfies commutativity and associativity of addition and taking the $0$ function we have an additive identity. Since for $f \in \mathcal{B} (X, F)$, $-f \in \mathcal{B}(X, F)$ we see that every element has an additive inverse. Thus $\mathcal{B} (X, F)$ is an abelian group under addition. The following statements holds for $x \in X$. Note that for $a \in F$ and $f \in \mathcal{B} (X, F)$ we have $(af)(x) = af(x)$. Then for $f, g$ we have
\[
a (f + g)(x) = a(f(x) + g(x)) = af(x) + ag(x) = (af)(x) + (ag)(x).
\]
Also for $b \in F$ we have
\[
((a+b)f)(x) = (a+b)f(x) = af(x) + bf(x) = (af)(x) + (bf)(x)
\]
and similarly
\[
(abf(x)) = (ab)f(x) = a(bf)(x).
\]
Finally, note that for $1 \in F$ we have $1 \cdot f(x) = f(x)$. These axioms show that $\mathcal{B} (X, F)$ is a vector space over $F$. Now define $(fg)(x) = f(x)g(x)$. Then for $f,g,h \in \mathcal{B} (X, F)$ we have
\[
(f(gh))(x) = f(x)(gh(x)) = f(x)g(x)h(x) = (fg)(x)h(x) = ((fg)h(x))
\]
so associativity holds. Also left and right distributivity hold since they do in $F$. Finally, for $a \in F$ we have
\[
((af)(x))g(x) = (af(x))(g(x) = af(x)g(x) = a(fg)(x) = (afg)(x) = (fag)(x) = f(x)(ag)(x).
\]
Together these axioms show that $\mathcal{B} (X, F)$ is an algebra.
\end{proof}

\begin{**}
Prove that $\mathcal{B} (X, F)$ is complete.
\end{**}
\begin{proof}
This is Problem 2.
\end{proof}

\begin{**}
Let $r \geq 1$ with $r \in \mathbb{R}$. Define
\[
f_r (x)= 
\begin{cases}
\frac{1}{q^r} & x = \frac{p}{q}, x \neq 0\\
0 & x \notin \mathbb{Q}, x = 0.
\end{cases}
\]
Show that $f_r$ is continuous for $x \notin \mathbb{Q}$ and $x = 0$. Show that it's discontinuous for $x \in \mathbb{Q} \backslash \{0\}$.
\end{**}
\begin{proof}
The same proof holds as in Problem 3.
\end{proof}

\begin{**}
Let $(X, d)$ and $(X', d')$ be metric spaces. A function $f : X \rightarrow X'$ is continuous if and only if for each open set $A \subseteq X'$, $f^{-1}(A)$ is open in $X$.
\end{**}
\begin{proof}
Suppose that $f$ is continuous and let $A \subseteq X'$ be open. Let $y \in f^{-1}(A)$. Let $\varepsilon > 0$ such that $B_{\varepsilon}(f(y)) \subseteq A$. Then there exists $\delta > 0$ such that $f(B_{\delta}(y)) \subseteq B_{\varepsilon}(f(y))$. This implies that $B_{\delta}(y) \subseteq f^{-1}(B_{\varepsilon}(f(y))) \subseteq A$. Thus $f^{-1}(A)$ is open.\newline

Conversely, assume that for every open set $A \subseteq X'$, we have $f^{-1}(A)$ is open in $X$. Then let $x \in X$ so that $f(x) \in X'$. Let $\varepsilon > 0$ so that $B_{\varepsilon}(f(x))$ is open in $X'$. Then $f^{-1}(B_{\varepsilon}(f(x)))$ is open in $X$. Note that $x \in f^{-1}(B_{\varepsilon}(f(x)))$ and so there exists $\delta > 0$ such that $B_{\delta}(x) \subseteq f^{-1}(B_{\varepsilon}(f(x)))$. Let $y \in X$ with $d(x,y) < \delta$. Then $y \in B_{\delta}(x)$ which means $y \in f^{-1}(B_{\varepsilon}(f(x)))$. But then $d'(f(x), f(y)) < \varepsilon$. Therefore $f$ is continuous for all $x \in X$.
\end{proof}

\begin{**}
Show $\mathcal{BC} (X, F)$ is closed as a subset of $\mathcal{B} (X, F)$.
\end{**}
\begin{proof}
Let $f$ be an accumulation point of $\mathcal{BC} (X, F)$. Then there exists a sequence of functions $(f_n)$ in $\mathcal{BC} (X, F)$ which converges to $f$. Let $\varepsilon > 0$. Then there exists $N$ such that for all $n > N$ we have $\sup_{x \in X} |f_n(x) - f(x)| < \varepsilon/3$. Let $x \in X$. Then for all $y \in X$ and $n > N$ we have
\[
|f(x) - f(y)| \leq |f(x) - f_n(x)| + |f_n(x) - f_n(y)| + |f(y) - f_n(y)| < \varepsilon/3 + |f_n(x) - f_n(y)| + \varepsilon/3.
\]
But since $f_n$ is continuous there exists a $\delta > 0$ such that $d(x,y) < \delta$ implies that $|f_n(x) - f_n(y)| < \varepsilon/3$. Thus $|f(x) - f(y)| < \varepsilon$ whenever $d(x,y) < \delta$. Therefore $f$ is continuous and so $f \in \mathcal{BC} (X, F)$ which means that $\mathcal{BC} (X, F)$ is closed.
\end{proof}

\begin{**}
Show that a compact subset of a metric space $(X, d)$ is closed.
\end{**}
\begin{proof}
Suppose that $C \subseteq X$ is compact and $C$ is not closed. If $C = \emptyset$ then the problem is trivial so let $C \neq \emptyset$. Let $p \notin C$ be an accumulation point of $C$. Let $\mathcal{A} = \{X \backslash \overline{B_r(p)} \mid r \in \mathbb{R}\}$. Since $p \notin C$, $\mathcal{A}$ covers $C$. Let $\mathcal{B}$ be a finite subset of $\mathcal{A}$ which covers $C$. If $\mathcal{B} = \emptyset$, $\mathcal{B}$ does not cover $C$. Then $\mathcal{B} = \{X \backslash \overline{B_{r_1}(p)}, X \backslash \overline{B_{r_2}(p)}, \dots , X \backslash \overline{B_{r_n}(p)}\}$. Take the smallest $r_i$ such that $X \backslash \overline{B_{r_i}(p)} \in \mathcal{B}$ and consider $B_{r_i/2}(p)$. This ball contains $p$, which is an accumulation point of $C$, and since balls are open, $B_{r_i/2}(p) \cap C \neq \emptyset$. But $B_{r_i/2}(p)$ is defined such that $B_{r_i/2}(p) \nsubseteq \bigcup_{B \in \mathcal{B}} B$ and so $C \nsubseteq \bigcup_{B \in \mathcal{B}} B$. But then $\mathcal{B}$ doesn't cover $C$ which is a contradiction. Therefore compact sets are closed.
\end{proof}

\begin{problem}
Define a sequence of functions $f_n : (0,1) \rightarrow \mathbb{R}$ by
\[
f_n (x)=
\begin{cases}
\frac{1}{q^n} & \text{if $x = \frac{p}{q}$}\\
0 & \text{if $x \notin \mathbb{Q}$}
\end{cases}
\]
for $n \in \mathbb{N}$. Find the pointwise limit, $f$, of the sequence $(f_n)_{n \in \mathbb{N}}$ and show that $(f_n)_{n \in \mathbb{N}}$ uniformly converges to $f$.
\end{problem}
\begin{proof}
Let $f(x) = 0$. For $x \in (0,1)$ with $x \notin \mathbb{Q}$ we have $f_n(x) = 0$ for all $n$ and so $\lim_{n \rightarrow \infty} f_n(x) = f(x) = 0$. For $x \in (0,1)$ with $x \in \mathbb{Q}$, let $x = p/q$ when reduced to lowest terms. Then $f_n(x) = 1/q^n$ which we know converges to $0$. Thus $f$ is the pointwise limit of $(f_n)$. Now let $\varepsilon > 0$. Choose $N$ such that $1/2^N < \varepsilon$. Then note that for all $n > N$ and all $x \in (0,1)$ we have $|f(x) - f_n(x)| = |f_n(x)| < 1/2^n < \varepsilon$. Thus, $(f_n)$ uniformly converges to $f$ on $(0,1)$.
\end{proof}

\begin{problem}
1) A polynomial function $p(x)$ on $\mathbb{R}$ is uniformly continuous if and only if $\deg(p(x)) < 2$.\\
2) The function $f(x) = \sin (x)$ is uniformly continuous on $\mathbb{R}$.
\end{problem}
\begin{proof}
1) Suppose $\deg(p(x)) < 2$. Then $p(x) = a_1x + a_0$ where $a_1$ may be $0$. Choose $\delta = \varepsilon/|a_1|$. Then for all $x, y \in \mathbb{R}$ with $|x-y| < \delta$ we have
\[
|f(x) - f(x)| = |a_1x + a_0 - a_1y - a_0| = |a_1(x - y)| = |a_1||x-y| < |a_1|\delta = \varepsilon.
\]
Thus $p(x)$ is uniformly continuous. Now suppose that $\deg(p(x)) \geq 2$. Then $p(x) = \sum_{i=0}^{n} a_ix^i$ where at least one of $a_2, a_3, \dots , a_n$ is not $0$. Let $\varepsilon > 0$ and for all $x,y \in \mathbb{R}$ consider some $\delta > 0$ such that $|x-y| < \delta$ implies that $|f(x) - f(y)| < \varepsilon$. Note that
\[
|f(x) - f(y)| = |\sum_{i=1}^{n} a_i(x^i-y^i)|
\]
and that this cannot be reduced to a constant multiplied by $|x-y|$. Thus $\delta$ must depend on the value of $x$ and so $p(x)$ is not uniformly continuous.\newline

2) Note that since $f'(x) = \cos x$ and $-1 \leq \cos x \leq 1$ for all $x \in \mathbb{R}$, for $x,y \in \mathbb{R}$ we have $|\sin x - \sin y|/|x-y| \leq 1$ by the mean value theorem. Then for all $\varepsilon > 0$ let $\delta = \varepsilon$ so that $|x-y| < \delta$ implies $|f(x) - f(y)| < \varepsilon$.
\end{proof}

\begin{problem}
Determine whether the following functions are uniformly continuous on $(0, \infty)$:\\
1) $f(x) = 1/x$\\
2) $f(x) = \sqrt{x}$\\
3) $f(x) = \ln (x)$\\
4) $f(x) = x \ln (x)$
\end{problem}
\begin{proof}
1) No. Let $\varepsilon = 1$. Assume there exists a $\delta > 0$ and let $\delta < 1$. Then let $x, y \in (0,1)$ such that $x = y + \delta/2$ and $y = \delta/2$. Then $|x-y| < \delta$ but
\[
|f(x) - f(y)| = |1/x - 1/y| = |1/\delta| > 1 = \varepsilon.
\]
Thus no delta can exist for $\varepsilon = 1$.\newline

2) Yes. Let $\delta = \varepsilon^2$. Then let $x,y \in \mathbb{R}$ such that $|x-y| < \delta$. Then
\[
|\sqrt{x} - \sqrt{y}| \leq < |\sqrt{x-y}| < \sqrt{\delta} = \varepsilon.
\]\newline

3) No. Let $\varepsilon = 1$ Assume there exists a $\delta > 0$ and let $\delta < 1/e$ then let $x,y \in (0,1)$ such that $x = y + \delta/2$ and $y = 1/2$. Then $|x-y| < \delta$ but
\[
|f(x) - f(y)| = |\ln (x) - \ln (y)| = |\ln (x/y)| = |\ln (\delta)| > 1 = \varepsilon.
\]\newline

4) No. The same proof as in Part 3) holds, but we have $|\ln (x^x/y^y)| > \varepsilon$.
\end{proof}

\begin{problem}
Prove the Heine-Borel theorem holds in $\mathbb{R}^n$ and $\mathbb{C}^n$ with the usual metrics.
\end{problem}
\begin{proof}
Let $S$ be a set in $\mathbb{R}^n$ such that $S$ is closed and bounded. Since $S$ is bounded it is a subset of
\[
A_1 = [a_1, b_1] \times [a_2, b_2] \times \dots \times [a_n, b_n]
\]
where $a_i, b_i \in \mathbb{R}$ and $a_i < b_i$ for $1 \leq i \leq n$. Take the bisection of each $[a_i, b_i]$ to form $2n$ intervals which make $2^n$ subboxes. Assume that $A_1$ is not compact. Then for some open cover $\mathcal{C}$ there is no finite subcover. This means that at least one of the $2^n$ subboxes contains an infinite number of open sets from $\mathcal{C}$. Let this be $A_2$. Perform the same bisection process on $A_2$ so that we have $2^n$ subboxes of $A_2$, one of which has an infinite number of sets from $\mathbb{C}$ needed to cover it. This is $A_3$. Continuing in this process we have a set of nested boxes $A_1 \supseteq A_2 \supseteq A_3 \supseteq \dots$ which have side length that tends to $0$. Since each $A_i$ is bounded and closed we have
\[
\bigcup_{i=1}^{\infty} A_i \neq \emptyset
\]
and so this intersection contains some $x \in A_1$. Since $\mathcal{C}$ covers $A_1$, there exists an open set $U \in \mathcal{C}$ such that $x \in U$. Since $U$ is open there exists $\varepsilon > 0$ such that $B_{\varepsilon}(x) \subseteq U$. Then for large enough $n$ we have $A_n \subseteq B_{\varepsilon}(x) \subseteq U$. But we've made the assumption that each $A_i$ requires an infinite number of sets from $\mathcal{C}$ to cover it and now $U$ covers $A_n$. This is a contradiction and so $S$ must be compact. If $S \subseteq \mathbb{C}^n$ such that $S$ is closed and bounded, a similar proof holds where $A_1$ is a cross product of squares. That is,
\[
A_1 = [a_1, b_1] \times [c_1, d_1] \times [a_2, b_2] \times [c_2, d_2] \times \dots \times [a_n, b_n] \times [c_n, d_n].
\]
\end{proof}

\begin{problem}
1) Let $f : X \rightarrow X'$ be a continuous map of metric spaces. Show that if $A \subseteq X$ is compact then $f(A) \subseteq X'$ is compact.\\
2) Suppose that $X$ is a compact metric space. Show that a continuous function $f : X \rightarrow \mathbb{R}$ is bounded.\\
3) Suppose that $X $ is a compact metric space. Show that a continuous function $f : X \rightarrow \mathbb{R}$ attains a maximum and minimum value on $X$.
\end{problem}
\begin{proof}
1) Let $\mathcal{A}$ be an open cover of $f(A)$. For all $x \in A$ we have $f(x) \in f(A)$ and so for all $x \in A$ there exist an open set $B \in \mathcal{A}$ such that $f(x) \in B$. But then for all $x \in A$, $x \in f^{-1}(B)$ for some $B \in \mathcal{A}$. So we have $A \subseteq \bigcup_{B \in \mathcal{A}} f^{-1}(B)$ and since $f$ is continuous $\{f^{-1}(B) \mid B \in \mathcal{A}\}$ is an open cover for $A$. But $A$ is compact so there exists a finite subcover, $\{f^{-1}(B_1), f^{-1}(B_2), \dots , f^{-1}(B_n)\}$ which covers $A$. So for all $x \in A$ there exists some $B_i \in \mathcal{A}$ such that $x \in f^{-1}(B_i)$. But then $f(x) \in B_i$ and since $f(A) = \{ y \in X' \mid x \in A, y=f(x) \}$, we have for all $y \in f(A)$, $y \in B_i$ for some $i$. Since every $B_i \in \mathcal{A}$ we have found a finite subcover of $\mathcal{A}$ which covers $f(A)$. Thus $f(A)$ is compact.\newline

2) From Part 1) we know that if $f$ is continuous and $X$ is compact, then $f(X)$ is also compact. But compact sets are bounded.\newline

3) Let $C$ be a nonempty compact set in $\mathbb{R}$ then $\sup C \in C$ because it is an accumulation point of $C$. Part 2) tells us that $f(X)$ is compact and assuming $X \neq \emptyset$ we see that $\sup f(X)$ exists and $\sup f(X) \in f(X)$. Let $f(c) = \sup f(X)$. Then there exists $d \in X$ such that $f(d) = f(c) = \sup f(X)$. But this value is greater than or equal to every value which $f$ takes on $X$. A similar proof holds for the minimum value.
\end{proof}


\begin{problem}
Suppose $X$ and $X'$ are metric spaces with $X$ compact. Show the following:\\
1) If $f : X \rightarrow X'$ is continuous on $X$, then $f$ is uniformly continuous on $X$.\\
2) If $f : X \rightarrow X'$ is a continuous bijection, then $f$ is a homeomorphism.
\end{problem}
\begin{proof}
1) Let $\varepsilon > 0$ and consider $\varepsilon/2 > 0$. We have $f$ is continuous so for all $x \in X$ there exists $\delta (x) > 0$ such that for all $y \in X$ with $d(x,y) < \delta (x)$ we have $d'(f(x),f(y)) < \varepsilon/2$. Consider the set of balls $\mathcal{A} = \{B_{\delta(x)}(x) \mid x\in X\}$ and let $\mathcal{A}' = \{B_{\delta(x)/2}(x) \mid B_{\delta(x)}(x) \in \mathcal{A}\}$. $\mathcal{A}'$ is an open cover for $X$ and since $X$ is compact there exists a finite subcover, $\mathcal{B} \subseteq \mathcal{A}'$. Let $\delta = \min \{\delta(x)/2 \mid B_{\delta(x)/2}(x) \in \mathcal{B} \}$. Then consider two points $x,y \in X$ such that $d(x,y) < \delta$. $\mathcal{B}$ is an open cover for $X$ so there exists some ball $B_{\delta(z)/2}(z) \in \mathcal{B}$ such that $x \in B_{\delta(x)/2}(z)$. Then $d(x,z) < \delta(z)/2 < \delta(z)$ and $d(x,y) < \delta \leq \delta(z)/2$ so $d(z,y) \leq d(z,x) + d(x,y) < \delta(z)$. But then $d'(f(z),f(x)) < \varepsilon/2$ and $d'(f(z),f(y)) < \varepsilon/2$ so $d'(f(x),f(y)) \leq d'(f(x),f(z)) + d'(f(z),f(y)) < \varepsilon$. Therefore for every $\varepsilon > 0$ there exists a $\delta > 0$ such that for all $x,y \in X$ with $d(x,y) < \delta$ we have $d'(f(x),f(y)) < \varepsilon$.\newline

2) From Part 1) we know that $f$ is uniformly continuous which directly implies that $f^{-1}$ is continuous. This $f$ is a homeomorphism.
\end{proof}

\end{flushleft}
\end{document}