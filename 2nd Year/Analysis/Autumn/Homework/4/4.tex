\documentclass{article}
\usepackage{amsmath,amssymb,amsfonts,amsthm,fullpage}

\newtheorem{problem}{Problem}
\newtheorem{lemma}{Lemma}
\newtheorem{**}{** Problem}

\newcommand{\aut}[1]{\textup{Aut}(#1)}

\begin{document}
\begin{flushright}
Kris Harper\\

MATH 20700\\

October 27, 2008
\end{flushright}

\begin{center}
Homework 4
\end{center}

\begin{flushleft}

\begin{**}
Let $(a_n)$ be a sequence which converges to both $a$ and $b$. Then $a = b$.
\end{**}
\begin{proof}
Let
\[
\lim_{n \rightarrow \infty} a_n = a = b
\]
and suppose $a \neq b$. Without loss of generality let $a<b$. Consider $0 < (b-a)/2$. Then there exist $N_1, N_2 \in \mathbb{N}$ such that for all $n>N_1$ we have $|a-a_n| < (b-a)/2$ and for all $n>N_2$ we have $|b-a_n| < (b-a)/2$. Let $N = \max{N_1,N_2}$ so that for all $n>N$ we have $a_n \in (a-(b-a)/2 , a + (b-a)/2)$ and $a_n \in (b-(b-a)/2 , b + (b-a)/2)$. But these sets are disjoint so this is a contradiction and $a = b$.
\end{proof}

\begin{**}
Let $S \subseteq \mathbb{R}$ be a set. Then $S$ is closed if and only if it contains all its accumulation points.
\end{**}
\begin{proof}
Let $S$ be closed. Then $\mathbb{R} \backslash S$ is open. Let $p$ be an accumulation point of $S$ such that $p \notin S$. Then $p \in \mathbb{R} \backslash S$ and so there exists $\varepsilon > 0$ such that $(p - \varepsilon, p + \varepsilon) \subseteq \mathbb{R} \backslash S$ since $\mathbb{R} \backslash S$ is open. But then
\[
((p - \varepsilon, p + \varepsilon) \backslash \{p\} ) \cap S = \emptyset
\]
which is a contradiction since $p$ is an accumulation point of $S$.\newline

Now assume that $S$ contains all its accumulation points. Let $x \in \mathbb{R} \backslash S$. Then $x$ is not an accumulation point of $S$ and so there exists some $\varepsilon > 0$ such that
\[
(x - \varepsilon, x + \varepsilon) \cap S = \emptyset
\]
since $x \notin S$. But then $(x - \varepsilon, x + \varepsilon) \subseteq \mathbb{R} \backslash S$ which means $\mathbb{R} \backslash S$ is open. Thus $S$ is closed.
\end{proof}

\begin{**}
Show that $\mathbb{Q}(\sqrt{2}) = \{a + b \sqrt{2} \mid a,b \in \mathbb{Q}\}$ is a field.
\end{**}
\begin{proof}
Let $x,y,z \in \mathbb{Q}(\sqrt{2})$ such that $x = a + b \sqrt{2}$, $y = c + d \sqrt{2}$ and $z = e + f \sqrt{2}$. Then
\[
(x + y) + z = ((a + b \sqrt{2}) + (c + d \sqrt{2})) + (e + f \sqrt{2}) = (a + b \sqrt{2}) + ((c + d \sqrt{2}) + (e + f \sqrt{2})) = x + (y + z),
\]
\[
x + y = (a + b \sqrt{2}) + (c + d \sqrt{2}) = (c + d \sqrt{2}) + (a + b \sqrt{2}),
\]
\[
(xy)z = ((a + b \sqrt{2}) (c + d \sqrt{2}))(e + f \sqrt{2}) = (a + b \sqrt{2}) ((c + d \sqrt{2})(e + f \sqrt{2}))
\]
and
\[
xy = (a + b \sqrt{2}) (c + d \sqrt{2}) = (c + d \sqrt{2}) (a + b \sqrt{2}) = yx
\]
which means associativity and commutativity of addition and multiplication are true as they are in the reals. Let $0 = 0 + 0 \sqrt{2}$. Then
\[
0 + x = (0 + 0 \cdot \sqrt{2}) + (a + b \sqrt{2}) = (0 + a) + ((0 + b)\sqrt{2}) = a + b \sqrt{2} = x.
\]
Let $-x = -(a + b \sqrt{2}) = -a -b \sqrt{2}$. Then
\[
-x + x = (-a - b \sqrt{2}) + (a + b \sqrt{2}) = (-a + a) + ((-b + b)\sqrt{2}) = 0 + 0 \cdot \sqrt{2} = 0.
\]
Let $1 = 1 + 0 \cdot \sqrt{2}$. Then
\[
1 \cdot x = (1 + 0 \cdot \sqrt{2}) (a + b \sqrt{2}) = (1 \cdot a + 2 \cdot 0 \cdot b) + (1 \cdot b + 0 \cdot a) \sqrt{2} = a + b \sqrt{2} = x.
\]
Let
\[
x^{-1} = \frac{a}{a^2 - 2b^2} - \frac{b}{a^2 - 2b^2} \sqrt{2}.
\]
Then
\[
x^{-1}x = \left ( \frac{a}{a^2 - 2b^2} - \frac{b}{a^2 - 2b^2} \sqrt{2} \right ) (a + b \sqrt{2}) = \left ( \frac{a - b\sqrt{2}}{a^2 - 2b^2} \right ) (a + b \sqrt{2}) = \frac{a^2 - 2b^2}{a^2 - 2b^2} = 1.
\]
Finally,
\[
x(y + z) = (a + b \sqrt{2})((c + d \sqrt{2}) + (e + f \sqrt{2})) = (a + b \sqrt{2}) (c + d \sqrt{2}) + (a + b \sqrt{2}) (e + f \sqrt{2}) = xy + xz
\]
from addition and multiplication in the reals. Since all the axioms are satisfied, $\mathbb{Q}(\sqrt{2})$ is a field.
\end{proof}

\begin{**}
Find $\aut{\mathbb{Q}(\sqrt{2})}$.
\end{**}
\begin{proof}
Clearly the identity is an element of $\aut{\mathbb{Q}(\sqrt{2})}$. Additionally, if we consider the elements $a + b\sqrt{2}$ in $\mathbb{Q}(\sqrt{2})$ with $b = 0$, then we have $\mathbb{Q}$, for which the only automorphism is the identity. Thus, any element of $\aut{\mathbb{Q}(\sqrt{2})}$ must keep the rational term the same. To find any other automorphims, we consider the product
\[
(a + b \sqrt{2})(c + d \sqrt{2}) = (ac + 2bd) + (ad + bc)\sqrt{2}.
\]
The term $2bd$ implies that the only other possible factorization we can have while keeping the rational terms the same is
\[
(ac + 2bd) - (ad + bc)\sqrt{2} = (a - b \sqrt{2})(c - d \sqrt{2}).
\]
Thus, there exists an automorphism $f : \mathbb{Q}(\sqrt{2}) \rightarrow \mathbb{Q}(\sqrt{2})$ such that $f(a + b \sqrt{2}) = a - b\sqrt{2}$. To show this is true consider
\[
f((a + b \sqrt{2}) + (c + d \sqrt{2})) = f((a + c) + (b + d) \sqrt{2}) = (a + c) - (b + d) \sqrt{2} = (a - b \sqrt{2}) + (c - d \sqrt{2}) = f(a + b \sqrt{2}) + f(c + d \sqrt{2})
\]
and
\[
f((a + b \sqrt{2}) (c + d \sqrt{2})) = f((ac + 2bd) + (ad + bc)\sqrt{2}) = (ac + 2bd) - (ad + bc)\sqrt{2} = (a - b \sqrt{2}) (c - d \sqrt{2}) = f(a + b \sqrt{2}) f(c + d \sqrt{2}).
\]
Thus $\aut{\mathbb{Q}(\sqrt{2})} = \{I, f\}$.
\end{proof}

\begin{**}
Find $\aut{F}$ when $F$ is a finite field.
\end{**}
\begin{proof}
There exists a unique field $\mathbb{F}_{p^n}$ with $p^n$ elements for each prime $p$ and each natural number $n$, up to isomorphism. Letting $q = p^n$ consider the function $f : \mathbb{F}_q \rightarrow \mathbb{F}_q$ such that $f(x) = x^p$. Then we see right away that
\[
f(xy) = (xy)^p = x^py^p = f(x)f(y).
\]
Additionally, using the binomial theorem we have
\[
f(x+y) = (x+y)^p = \sum_{k=0}^{p} \binom{p}{k} x^p y^{p-k} = \sum_{k=0}^{p} \frac{p!}{k!(p-k)!} x^p y^{p-k}.
\]
Since $p$ is prime, it will divide $p!$, but not $j!$ for any $1 \leq j \leq p-1$. Hence, the terms for all but $k=0$ and $k=p$ will vanish from the sum and we are left with
\[
f(x+y) = x^p + y^p = f(x) + f(y).
\]
Hence, $f$ is an automorphism for $F_{q}$. But then if we compose this function with itself, it will still be an automorphism, as long as $n \neq 1$. That is, we can compose it with itself $n$ times since the order of $F_{p^n}$ is $p^n$ and $p$ is a prime. Thus $\aut{F_{p^n}}$ is the cyclic group of order $n$ with a generating element $f$.
\end{proof}

\begin{problem}
Show the following for $z = a + bi$ and $w = c + di$:\\
1) We have $|z| \geq 0$ and $|z| = 0$ if and only if $z = 0$.\\
2) We have $|zw| = |z||w|$.\\
3) We have $|z+w| \leq |z| + |w|$.
\end{problem}
\begin{proof}
1) We have $|z| = \sqrt{a^2 + b^2} \geq 0$ since $a^2 \geq 0$ and $b^2 \geq 0$. Let $|z| = 0$. Then
\[
0 = \sqrt{z \overline{z}} = \sqrt{a^2 + b^2}
\]
so $a^2 + b^2 = 0$ and since $a^2$ and $b^2$ are both greater than or equal to $0$, they must both be $0$. Then $a = b = 0$ so $z = 0$. Now suppose $z = 0$. Then
\[
|z| = \sqrt{z \overline{z}} = \sqrt{a^2 + b^2} = \sqrt{0} = 0.
\]\newline

2) We have
\begin{align*}
|zw| &= |(ac-bd) + (ad+bc)i| \\
	&= \sqrt{(ac-bd)^2 + (ad+bc)^2} \\
	&= \sqrt{a^2c^2 -2abcd + b^2d^2 + a^2d^2 + 2abcd + b^2c^2} \\
	&= \sqrt{a^2c^2 + b^2d^2 + a^2d^2 + b^2c^2} \\
	&= \sqrt{(a^2 + b^2)(c^2 + d^2)} \\
	&= \sqrt{a^2 + b^2} \sqrt{c^2 + d^2} \\
	&= |z||w|.
\end{align*}\newline

3) We have
\[
b^2c^2 + a^2d^2 -2abcd = (ad-bc)^2 \geq 0
\]
so
\[
b^2c^2 + a^2d^2 \geq 2abcd
\]
and
\[
(a^2 + b^2)(c^2 + d^2) = a^2c^2 +b^2c^2 + a^2d^2 + b^2d^2 \geq a^2c^2 + 2abcd + b^2d^2 = (ac+bd)^2.
\]
Then we have
\[
2 \sqrt{(a^2 + b^2)(c^2 + d^2)} \geq 2(ac+bd)
\]
so
\begin{align*}
(|z| + |w|)^2 &= (\sqrt{a^2 + b^2} + \sqrt{c^2 + d^2})^2 \\
		   &= a^2 + b^2 + 2 \sqrt{(a^2 + b^2)(c^2 + d^2)} + c^2 + d^2 \\
		   & \geq a^2 + b^2 + 2(ac+bd) + c^2 + d^2 \\
		   &= (a+c)^2 + (b+d)^2 \\
		   &= |z+w|^2.
\end{align*}
Thus $|z| + |w| \geq |z+w|$.
\end{proof}

\begin{problem}
Show that $\mathbb{C}$ is not isomorphic to $\mathbb{R}$.
\end{problem}
\begin{proof}
Using the same proof which shows that $\aut{\mathbb{R}}$ contains only the identity, we see that any homomorphism from $\mathbb{R}$ to $\mathbb{C}$ must map every real number to every real number. But then the map is not surjective.
\end{proof}

\begin{problem}
Let $S = \{B_r(z) \mid r, \textup{Re}(z), \textup{Im}(z) \in \mathbb{Q}\}$ be the set of rational balls. Then any open set, $A \subseteq \mathbb{C}$, can be written as a countable union of sets in $S$.
\end{problem}
\begin{proof}
Note that if we consider the points at which the elements in $S$ are centered, we see that $S$ is simply a collection of elements of $\mathbb{Q} \times \mathbb{Q}$. Thus $S$ is countable since $\mathbb{Q}$ is countable.\newline

Let $A \subseteq \mathbb{C}$ be open such that $z \in A$ and $z = a + bi$. There exists a ball $B_r(z) \subseteq A$ where $r$ may be rational or not. If $r \notin \mathbb{Q}$ then consider some $r' \in \mathbb{Q}$ such that $0 < r' < r$ and then $B_{r'}(z) \subseteq B_r(z) \subseteq A$. We have $B_{r'/2}(z) \subseteq B_{r'}(z) \subseteq A$. Let $z' = a' + b'i$ where $a', b' \in \mathbb{Q}$ and
\[
0 < a' < r'/(2 \sqrt{2}) + a
\]
and
\[
0 < b' < r'/(2 \sqrt{2}) + b.
\]
Then
\[
a'-a < r'/(2 \sqrt{2})
\]
and
\[
b-b' < r'/(2 \sqrt{2})
\]
which means
\[
(a-a')^2 < r'^2/8,
\]
\[
(b-b')^2 < r'^2/8,
\]
\[
(a-a')^2 + (b-b')^2 < r'^2/4
\]
and $|z-z'| < r'/2$. Finally consider $z'' \in B_{r'/2}(z')$. Then $|z'-z''| < r'/2$. But also $|z -z'| < r'/2$ so we have $|z-z''| \leq |z-z'| + |z'-z''| < r'/2 + r'/2 = r'$. Thus $B_{r'/2}(z') \subseteq B_{r'}(z) \subseteq A$. Also $|z-z'| < r'/2 < r'$ so $z \in B_{r'/2}(z')$. Note that $r'/2, \textup{Re}(z'), \textup{Im}(z') \in \mathbb{Q}$. Thus for any point in $A$ there exists a set from $S$ which contains it and is a subset of $A$. But there are countably many elements of $S$ and so a countable union of them will be equal to $A$.
\end{proof}

\begin{lemma}
Every sequence has an increasing or decreasing subsequence.
\end{lemma}
\begin{proof}
Let $(a_n)$ be a sequence. Define $n$ to be a peak point if for all $m>n$ we have $a_m < a_n$. Suppose there are infinitely many peak points for $(a_n)$ and let $n_1$ be the least peak point. We can do this because peak points are natural numbers. Define the next largest peak point to be $n_2$ and so on. Note that $a_{n_i} > a_{n_{i+1}}$ for all $i \in \mathbb{N}$. Thus, we have created a decreasing subsequence $(a_{n_k})$.\newline

If there are no peak points then for all $n \in \mathbb{N}$, there exists $m>n$ such that $a_n \leq a_m$. Then we can make an increasing subsequence by letting $m_1 = 1$. Then there exists $m_2 > 1$ such that $a_1 \leq a_{m_2}$. Now there exists $m_3 > m_2$ such that $a_{m_2} \leq a_{m_3}$. Thus $(a_{m_k})$ is an increasing subsequence.\newline

Now suppose that there are finitely many peak points for $(a_n)$ and that there exists at least one peak point. Let $n \in \mathbb{N}$ be the largest peak point for $(a_n)$. Then for all $m>n$ we have $a_m < a_n$, but also $m$ is not a peak point and so there exists $m' > m$ with $a_m \leq a_{m'}$. Then create an increasing sequence as before by choosing an arbitrary $m_1>n$. Then there exists $m_2 > m_1$ such that $a_{m_1} \leq a_{m_2}$. Thus $(a_{m_k})$ is an increasing subsequence.
\end{proof}

\begin{problem}
Show that every bounded sequence in $\mathbb{C}$ has a convergent subsequence.
\end{problem}
\begin{proof}
By Lemma 1 there exists a monotonically increasing or decreasing subsequence. But then if this subsequence is bounded it will converge.
\end{proof}

\begin{problem}
Let $S \subseteq \mathbb{C}$ be a subset. Show that every neighborhood of an accumulation point of $S$ contains infinitely many points of $S$.
\end{problem}
\begin{proof}
Let $x$ be an accumulation point of $S$ and let $\varepsilon > 0$. Then there exists $x_1 \in B_{\varepsilon}(x) \cap S$ such that $x_1 \neq x$. Let $\varepsilon_1 = |x - x_1|/2$. Then there exists $x_2 \in B_{\varepsilon_2}(x) \cap S$ such that $x_2 \neq x$. Note that $x_2 \neq x_1$ as well. We can continue in this process so that there must be an infinite number of points in $S \cap B_{\varepsilon}(x)$.
\end{proof}

\begin{problem}
Show that any bounded infinite set in $\mathbb{C}$ has an accumulation point in $\mathbb{C}$.
\end{problem}
\begin{proof}
Let $S$ be a bounded infinite set in $\mathbb{C}$. Create an infinite sequence $(a_n)_{n=1}^{\infty}$ of distinct elements of $S$. We can do this since $S$ is infinite. Since $S$ is bounded, by Problem 4 there exists a convergent subsequence $(a_{n_k})_{k=1}^{\infty}$. Let $\lim_{k \rightarrow \infty} a_{n_k} = a$. Then let $\varepsilon > 0$. Then there exists $N$ such that for all $k > N$ we have $|a - a_k| < \varepsilon$. Thus there exists $k$ such that $a_{n_k} \in B_{\varepsilon}(a) \cap S$. Thus $a$ is an accumulation point for $S$.
\end{proof}

\end{flushleft}
\end{document}