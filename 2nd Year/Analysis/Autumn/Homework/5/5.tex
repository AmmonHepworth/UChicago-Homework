\documentclass{article}
\usepackage{amsmath,amssymb,amsfonts,amsthm,fullpage}

\newtheorem{problem}{Problem}
\newtheorem{lemma}{Lemma}
\newtheorem{**}{** Problem}

\newcommand{\aut}[1]{\textup{Aut}(#1)}

\begin{document}
\begin{flushright}
Kris Harper\\

MATH 20700\\

November 3, 2008
\end{flushright}

\begin{center}
Homework 5
\end{center}

\begin{flushleft}

\begin{**}
Find $f \in \aut{\mathbb{C}}$ such that $f$ is not the identity or the conjugate function.
\end{**}
\begin{proof}
Consider some element $a \in \mathbb{C}$ such that $a$ is the root of a polynomial in $\mathbb{Q}[x]$. Let the polynomial of least degree with this property be $p$. Let $f$ be an automorphism with domain $\mathbb{Q}$. Then there exists an isomorphism extending $f$ to $\mathbb{Q}(a)$ and sending $a$ to $b$ if and only if $b$ is the root of a polynomial obtained by applying $f$ to the coefficients of $p$. Here, $\mathbb{Q}(a)$ denotes the extension of $\mathbb{Q}$ generated by $a$ and is the intersection of all subfields of $\mathbb{C}$ which contain $\mathbb{Q}$ and $a$. It is now possible to use Zorn's Lemma to show that any isomorphism, $f$, with domain $\mathbb{Q}$ can be extended to an isomorphism of $\mathbb{A}$. Let $F$ be the set of isomorphisms extending $f$ to a subfield of $\mathbb{A}$. $F$ is nonempty since $f$ extends itself to $\mathbb{Q}$. Consider isomorphisms of subfields of $\mathbb{C}$ as sets of ordered paris. Let $C$ be a chain of sets from $F$ and let $D$ be the union of all the isomorphisms in $C$. Let $(a,b), (c,d) \in D$. Then $(a,b) \in f_1$ and $(c,d) \in f_2$ for isomorphisms in $F$. Since $C$ is a chain it follows that $(a,b)$ and $(c,d)$ are in the same isomorphism since $f_1 \subseteq f_2$ or $f_2 \subseteq f_1$. From here it follows that $D$ is an isomorphism in $F$. Use Zorn's Lemma to choose $g$ as a maximal element of $F$. Suppose that the domain of $g$ is not $\mathbb{A}$. Then there exists $a \in \mathbb{A}$ not in the domain. But we've already shown that we can extend $g$ to include this element. This isomorphism will be in $F$ as well and $g$ is not the maximal element. This is a contradiction and so the domain of $g$ is $\mathbb{A}$. A similar proof using Zorn's Lemma shows that for any isomorphism on a finite extension of $\mathbb{Q}$ we can create an automorphism of $\mathbb{C}$.
\end{proof}

\begin{**}
Show that $\mathbb{A}$ and $\mathbb{A}_{\mathbb{R}}$ are fields.
\end{**}
\begin{proof}
We have that $\mathbb{A}$ is the set of numbers which are roots of elements in $\mathbb{Z}[x]$. Note that we can equivalently replace $\mathbb{Z}[x]$ with $\mathbb{Q}[x]$ by taking any element of $\mathbb{Q}[x]$ and multiplying by the least common denominator of each of the coeffiecients. We first need to show that $\mathbb{A}$ is closed under addition and multiplication. Let $x, y \in \mathbb{A}$ such that $u(x) = \sum_{i=0}^{n} a_ix^i = 0$ and $v(y) = \sum_{i=0}^{m} b_i y^i = 0$. Suppose that $u$ and $v$ are the polynomials of least degree with coefficients in $\mathbb{Q}$ and $x$ and $y$ as roots. Then we can say that the sets
\[
A = \{1, x, x^2, \dots , x^{n-1}\}
\]
and
\[
B = \{1, y, y^2, \dots , y^{m-1}\}
\]
are linearly independent. Note that we can create $x^n$ from a linear combination of elements from $A$. To see this note that
\[
x^n = -\frac{1}{a_n} \sum_{i=0}^{n-1} a_i x^i.
\]
Additionally, if we multiply both sides of this equation by $x$ we have
\[
x^{n+1} = -\frac{1}{a_n} \sum_{i=0}^{n-1} a_i x^{i+1}
\]
where the sum is a linear combination of elements of $A$. Inductively this shows that $x^k \in \langle A \rangle$ where $k \in \mathbb{N} \cup \{0\}$ and $\langle A \rangle$ denotes the span of $A$ over $\mathbb{Q}$. Similarly, $y^l \in \langle B \rangle$ where $l \in \mathbb{N} \cup \{0\}$. Also note that the set
\[
C = \{1, x, x^2, \dots , x^{n-1}, xy, xy^2, \dots , x^{n-1}y^{m-2}, x^{n-1}y^{m-1}\}
\]
is finite and that $x^ky^l \in \langle C \rangle$ for all $k, l \in \mathbb{N} \cup \{0\}$ Thus there exists some finite basis $C'$ for the space spanned by $C$. Now suppose that there exists no polynomial with coefficients in $\mathbb{Q}$ where
\[
w(x+y) = \sum_{i=0}^{p} c_i (x+y)^i = \sum_{i=0}^{p} c_i \sum_{j=0}^{i} \binom{i}{j} x^i y^{i-j} = 0
\]
where we've used the binomial theorem to expand each term. But if this is the case for all $p \in \mathbb{N}$, we will never have a set of products of powers of $x$ and $y$ which is linearly independent. Thus, a basis for all products of powers of $x$ and $y$ must be infinite. But $C'$ is a finite basis. This is a contradiction and so there must exist a polynomial in $\mathbb{Q}[x]$ such that $w(x+y) = 0$. The same proof holds for a polynomial $w'(xy) = 0$. Thus, $\mathbb{A}$ and therefore $\mathbb{A}_{\mathbb{R}}$ are closed under addition and multiplication.\newline

At this point, the axioms for commutativity and associativity of multiplication and distributivity follow from the fact that $\mathbb{C}$ and $\mathbb{R}$ are fields. Note that $0$ and $1$ are algebraic numbers from the polynomials $x = 0$ and $x - 1 = 0$. Thus the additive and multiplicative identities for $\mathbb{C}$ and $\mathbb{R}$ are in $\mathbb{A}$ and $\mathbb{A}_{\mathbb{R}}$. Note also that $r(x) = x + 1 = 0$ shows that $-1 \in \mathbb{A}$ and since $\mathbb{A}$ is closed under multiplication, if $x \in \mathbb{A}$ then $-1 \cdot x = -x$ so $-x \in \mathbb{A}$. The same is true for $\mathbb{A}_{\mathbb{R}}$. Thus, additive inverses are in $\mathbb{A}$ and $\mathbb{A}_{\mathbb{R}}$. We are left with multiplicative inverses. Let $x \in \mathbb{A}_{\mathbb{R}}$ such that
\[
p(x) = \sum_{i=0}^{n} a_i x^i = 0.
\]
Then multiply both sides by $1/x^n$ so that we have
\[
0= \sum_{i=0}^{n} a_i x^{i-n} = \sum_{i=0}^{n} a_i \left ( \frac{1}{x} \right )^{n-i}.
\]
Thus, there exists a polynomial with $1/x$ as a root and so $1/x \in \mathbb{A}_{\mathbb{R}}$. Finally, let $z \in \mathbb{A}$ such that $z = a + bi$. Then there exists a polynomial in $\mathbb{Q}[x]$ such that
\[
p(z) = \sum_{i=0}^{n} a_i z^i = 0.
\]
Then
\[
\sum_{i=0}^{n} a_i \overline{z}^i = \sum_{i=0}^{n} a_i \overline{z^i} = \sum_{i=0}^{n} \overline{a_i z^i} = \overline{\sum_{i=0}^{n} a_i z^i} = \overline{0} = 0
\]
and so $\overline{z}$ is a root of $p$ as well. Note also that $z + \overline{z} \in \mathbb{A}$ and $z + \overline{z} = 2a$ and $z - \overline{z} \in \mathbb{A}$ and $z - \overline{z} = bi$. Since $i \in \mathbb{A}$ from the equation $x^2 + 1 = 0$, we see that $a, b \in \mathbb{A}$ if $a + bi \in \mathbb{A}$. Thus $|z| \in \mathbb{A}$ and so $1/|z| \in \mathbb{A}$ and finally $\overline{z}/|z| \in \mathbb{A}$ from closure under multiplication. Thus the multiplicative inverse for $z$ is algebraic. Since all the axioms are met for both $\mathbb{A}$ and $\mathbb{A}_{\mathbb{R}}$, they are both fields.
\end{proof}

\begin{**}
Find $\aut{\mathbb{A}_{\mathbb{R}}}$ and $\aut{\mathbb{A}}$.
\end{**}
\begin{proof}
Let $a \in \mathbb{A}_{\mathbb{R}}$ such that $a > 0$. Then $a = b^2$ for some $b \in \mathbb{A}_{\mathbb{R}}$. To see this note that $a$ is a root of some polynomial $p(x) = \sum_{i=0}^{n} a_n x^n$ such that $p(a) = p(b^2) = 0$. Then consider $p'(x) = \sum_{i=0}^{n} a_n (x^2)^n$ and $p'(b) = 0$. Thus $b \in \mathbb{A}_{\mathbb{R}}$. Now let $f$ be an automorphism of $\mathbb{A}_{\mathbb{R}}$. Then
\[
f(a) = f(b^2) = f(b \cdot b) = f(b) \cdot f(b) = (f(b))^2 > 0
\]
since $\mathbb{A}_{\mathbb{R}}$ is a field. Thus if $a > 0$ then $f(a) > 0$ and so automorphisms of $\mathbb{A}_{\mathbb{R}}$ preserve order. Note that $\mathbb{Q} \subseteq \mathbb{A}_{\mathbb{R}}$ because for any rational $q \in \mathbb{Q}$, $q$ is a root of $x - q$. Then the rationals are fixed under $f$. Suppose that $a < f(a)$. Then there exists $r \in \mathbb{Q}$ such that $a < r < f(a)$. But then $f(a) < f(r) = r$. Hence $a \geq f(a)$. A similar proof shows that $a \leq f(a)$ and thus $a = f(a)$. Thus $\aut{\mathbb{A}_{\mathbb{R}}} = \{I\}$ where $I$ is the identity function.\newline

Certainly the identity and complex conjugation are in $\aut{\mathbb{A}}$. From ** Problem 1 we see that there are other elements which arise from finite extensions of $\mathbb{Q}$ which are generated by algebraic numbers.
\end{proof}

\begin{**}
Complete Project 10.2 for Chapter 3.
\end{**}

\textbf{** Problem 4.1}
\textit{Determine which of the following converge:\\
1) $a_n = 1$ for all $n$.\\
2) $a_n = 1/n$.\\
3) $a_n 1/2^n$.\\
4) $a_n = (-1)^{n+1}$.\\
5) $a_n = (-i)^{n+1}/(n^2+1)$.\\
6) $a_n = e^{in\theta}/n$ for a fixed $0 \leq \theta \leq 2 \pi$.\\
7) $a_n = \sin (n \pi)/n^2$.}
\begin{proof}
1) We see that $\sum_{n=1}^{\infty} a_n$ diverges. To show this, suppose it converges to $L$. Let $\varepsilon = 1/2$. Then for all $N \in \mathbb{N}$, use the Archimedean Property choose $n$ such that $n > |L + 1|$. Then $|S_n - L| \geq 1/2 = \varepsilon$.\newline

2) Group the terms of $(a_n)$ to make a new sequence $(b_k)$ such that
\[
b_k = \sum_{i = n_{k-1} + 1}^{n_k} \frac{1}{n}
\]
where $n_k = 2^{k-1}$ for $k \in \mathbb{N}$ and $n_0 = 0$. Note that for $k \geq 2$, $b_k$ has $2^{k-1} - 2^{k-2} = 2^{k-2}$ terms, the smallest of which is $1/2^{k-1}$. Thus, for all $k \geq 2$, $b_k \geq 2^{k-2}/2^{k-1} = 1/2$. Also $b_1 = \sum_{n=1}^{1} 1/n = 1$. So for all $k \in \mathbb{N}$ we have $b_k \geq 1/2$. But then there are no terms of $(b_k)$ in $(-1/2 ; 1/2)$ so $\lim_{k \rightarrow \infty} b_k \neq 0$. Thus, $\sum_{k=1}^{\infty} b_k$ is not convergent and therefore $\sum_{n=1}^{\infty} 1/n$ is not convergent.\newline

3) Let $\varepsilon > 0$ and choose $N$ such that $1/N < \varepsilon$. Then for $n > N$ we have $1/2^n < 1/N$ since $2^n > N$. Thus for all $n > N$ we have $|1 - S_N| = |1 - 1 + 1/2^n| = 1/2^n < 1/N < \varepsilon$. Thus $\sum_{n=1}^{\infty} a_n = 1$.\newline

4) This series diverges since the partial sums are either $1$ or $0$. Thus, for any value $a \in \mathbb{C}$, there exists some ball $B_r(a)$ such that $|a| < r$ and there are infinitely many terms of $(S_N)$ which are not in $B_r(a)$.\newline

5) This sequence can be broken up into a real sequence
\[
a_n' = \frac{(-i)^{2n}}{n^2+1}
\]
and an imaginary sequence
\[
a_n'' = \frac{(-i)^{2n-1}}{n^2+1}.
\]
Each of these series converges using the comparison test and the fact that $\sum_{n=1}^{\infty} 1/n^2$ converges. Thus, the original sequence must also converge.\newline

6) This series will converge for particular values of $\theta$. For example, if $\theta = \pi$ then $e^{in\pi}/n = (\cos(n \pi) + i \sin(n \pi))/n = (-1)^n/n$. Then we have $\sum_{n=1}^{\infty} (-1)^n/n$ converges by the alternating series test.\newline

7) Note that $\sin(n \pi) = 0$ for all $n \in \mathbb{N}$ so that we have $a_n = 0$ for all $n$. Then $\sum_{n=1}^{\infty} a_n = 0$.
\end{proof}

\textbf{** Problem 4.2}
\textit{Suppose that a series $\sum_{n=1}^{\infty} a_n$ converges. Show that $\lim_{n \rightarrow \infty} a_n = 0$.}
\begin{proof}
Let $\sum_{n=1}^{\infty} a_n = S$. Then the sequence of partial sums $(S_N)$ converges to $S$ and $(S_N)$ is a Cauchy sequence. Thus for all $\varepsilon > 0$ there exists $N' \in \mathbb{N}$ such that for all $n,m > N'$ we have $|S_n - S_m| < \varepsilon$. But note that $S_{n+1} - S_n= a_n$ so for $n > N' + 1$ we have $|a_n| < \varepsilon$ which means $\lim_{n \rightarrow \infty} a_n = 0$.
\end{proof}

\textbf{** Problem 4.3}
\textit{1) If $N \in \mathbb{N}$ and $z \neq 1$ show that $S_N = \sum_{n=0}^{N} z^n = \frac{1 - z^{N+1}}{1-z}$\\
2) If $|z| < 1$, show that $\lim_{n \rightarrow \infty} z^n = 0$.\\
3) If $|z| > 1$, show that $\lim_{n \rightarrow \infty} z^n$ does not exist.}
\begin{proof}
1) Note that
\[
\sum_{n=0}^{N} z^n = 1 + z + z^2 + \dots + z^N.
\]
Multiply both sides of this equality by $1-z$. Then we have
\[
(1-z)\sum_{n=0}^{N} z^n = (1-z) (1 + z + z^2 + \dots + z^N) = 1 - z^{N+1}
\]
and since $z \neq 1$ we have $1 - z \neq 0$ so we can multiply by $1/(1-z)$ to obtain
\[
\sum_{n=0}^{N} z^n = \frac{1 - z^{N+1}}{1-z}.
\]\newline

2) Let $\varepsilon > 0$ and let $N \in \mathbb{N}$ such that $N \geq 2$ and $1/N < \varepsilon$. Then for all $n > N$ we have $|z^n| < 1/N$ since $|z| < 1$. Thus, $\lim_{n \rightarrow \infty} z^n = 0$.\newline

3) Note that since $|z| > 1$, it follows that $z^n$ is unbounded. Then for any complex number $w$ there exists a ball $B_r(w)$ with infinitely many points of $z^n$ outside of it. Thus, $z^n$ cannot converge to $w$.
\end{proof}

\textbf{** Problem 4.4}
\textit{What can you say if $|z| = 1$?}
\begin{proof}
If $z \in \mathbb{R}$ then $\lim_{n \rightarrow \infty} z^n = 1$. If $z$ is purely imaginary, then $z^n$ will not converge as $i^n$ will cycle through four different values.
\end{proof}

\textbf{** Problem 4.5}
\textit{Show that by removing an infinite number of terms from the series $\sum_{n=1}^{\infty} 1/n$, the remaining subseries can be made to converge to any real number.}
\begin{proof}
Let $c \in \mathbb{R}$ and suppose that it is not possible to remove infinitely many terms of $a_n = 1/n$ so that the subseries, $\sum_{n=1}^{\infty} b_n$, converges to $c$. Consider the partial sums $S_N = \sum_{n=1}^{N} b_n$. Then there exists some $\varepsilon > 0$ such that for all $N$ there exists an $n > N$ such that $|S_n - c| \geq \varepsilon$. But then we can add or remove terms of $(a_n)$ until the inequality is satisfied.
\end{proof}

\textbf{** Problem 4.6}
\textit{If $p \in \mathbb{R}$ show that $\sum_{n=1}^{\infty} 1/n^p$ diverges for $p < 1$ and converges for $p > 1$.}
\begin{proof}
1) Let $S_n$ be the $n$th partial sum. Then
\[
S_{2n} = 1 + \frac{1}{2^p} + \frac{1}{3^p} + \dots + \frac{1}{(2n)^p} = 1 + \left ( \frac{1}{2^p} + \frac{1}{4^p} + \dots + \frac{1}{(2n)^p} \right ) + \left ( \frac{1}{3^p} + \frac{1}{5^p} + \dots + \frac{1}{(2n-1)^p} \right ).
\]
If $p > 1$ then we have
\[
S_{2n} > 1 + \frac{1}{2^p} S_n + \left ( \frac{1}{4^p} + \frac{1}{6^p} + \dots + \frac{1}{(2n)^p} \right )
\]
which means
\[
S_{2n} > 1 + \frac{1}{2^p} S_n - \frac{1}{2^p} + \frac{1}{2^p} S_n = 1 - \frac{1}{2^p} + \frac{2}{2^p} S_n
\]
and
\[
S_{2n} < 1 + \frac{2}{2^p}S_n.
\]
Thus
\[
\frac{2^p-1}{2^p} + \frac{2}{2^p}S_n < S_{2n} < 1 + \frac{2}{2^p}S_n.
\]
A similar proof shows that for $p < 1$ we have
\[
1 + \frac{2}{2^p}S_n < S_{2n} < \frac{2^p-1}{2^p} + \frac{2}{2^p} S_n.
\]
For $p < 0$ we see that $1/n^p > 1$ for large enough values of $n$ and so the series will eventually diverge and for $p = 0$ we have the constant sequence $1$ which will diverge. Assume that $\lim_{n \rightarrow \infty} S_n = S$. Let $0 < p \leq 1$. Then from the second inequality we have
\[
1 < S - \frac{2}{2^p}S < 1 - \frac{1}{2^p}
\]
which is a contradiction. Thus, $\sum_{n=1}^{\infty} 1/n^p$ diverges for $p \leq 1$. Now consider $p > 1$. Then from the first inequality we have
\[
S - \frac{2}{2^p}S = \frac{2^p-2}{2^p} S < 1
\]
which means
\[
S < \frac{2^p}{2^p-2}.
\]
So $S_n$ is a bounded and increasing sequence. Thus it must converge.
\end{proof}

\textbf{** Problem 4.7}
\textit{1) Suppose $a_n > 0$ for $n \in \mathbb{N}$ and $\sum_{n=1}^{\infty} a_n$ converges. If $b_n \in \mathbb{C}$ satisfies $|b_n| \leq a_n$ for all $n$, then the series $\sum_{n=1}^{\infty} b_n$ converges absolutely and thus converges.\\
2) If the series $\sum_{n=1}^{\infty} a_n$ converges to $s$ and $c$ is any constant show that the series $\sum_{n=1}^{\infty} ca_n$ converges to $cs$.\\
3) Suppose that $\sum_{n=1}^{\infty} a_n$ and $\sum_{n=1}^{\infty} b_n$ are infinite series. Suppose that $a_n > 0$ and $b_n > 0$ for $n \in \mathbb{N}$ and $\lim_{n \rightarrow \infty} a_n/b_n = c > 0$. Show that $\sum_{n=1}^{\infty} a_n$ converges if and only if $\sum_{n=1}^{\infty} b_n$ converges.}
\begin{proof}
1) Note that $a_n > 0$ for all $n$ and so $\sum_{n=1}^{\infty} a_n = \sum_{n=1}^{\infty} |a_n|$ is an absolutely convergent sequence. Then the sequence of partial sums, $(S_n)$ is convergent and therefore bounded. Thus there exists $C \in \mathbb{R}$ such that $S_n \leq C$ for all $n \in \mathbb{N}$. But then since $|b_n| \leq a_n = |a_n|$ for all $n$ we have
\[
\sum_{n=1}^{N} |b_n| \leq \sum_{n=1}^{N} a_n \leq C.
\]
Thus, the sequence of partial sums, $(T_n)$, for $\sum_{n=1}^{\infty} |b_n|$ is bounded. But also $|b_n| \geq 0$ and so
\[
T_{N+1} = \sum_{n=1}^{N} |b_n| + |b_{N+1}| = T_n + |b_{N+1}| \geq T_n.
\]
Thus $(T_n)$ is a bounded increasing sequence and therefore it is convergent. Thus $\sum_{n=1}^{\infty} b_n$ is absolutely convergent.\newline

2) Suppose that $\sum_{n=1}^{\infty} a_n = s$. Then note that
\[
cS_N = c \sum_{n=1}^{N} a_n = \sum_{n=1}^{N} ca_n.
\]
We know that $\lim_{n \rightarrow \infty} S_n = s$ so let $\varepsilon > 0$ and consider $\varepsilon/|c|$. There exists $N$ such that for all $n > N$ we have $|S_n - s| < \varepsilon/|c|$. Then $|c||S_n - s| = |cS_n - cs| < \varepsilon$. Thus, $\lim_{n \rightarrow \infty} cS_n = cs$ and so $\sum_{n=1}^{\infty} ca_n = cs$.\newline

3) Assume that $\sum_{n=1}^{\infty} a_n = s$. Then note that
\[
cs = c \sum_{n=1}^{\infty} a_n = \sum_{n=1}^{\infty} ca_n = \lim_{n \rightarrow \infty} \frac{a_n}{b_n} s_n.
\]
From we see that $\sum_{n=1}^{\infty} b_n$ must converge by the Comparison Test. A similar proof holds for the converse with the fact that $c > 0$.
\end{proof}

\textbf{** Problem 4.8}
\textit{Let $\sum_{n=1}^{\infty} a_n$ be a series of nonzero numbers. Give examples to show that if $\lim_{n \rightarrow \infty} |a_{n+1}/a_n| = r = 1$, the series may converge or diverge.}
\begin{proof}
Consider $a_n = 1/n$ then $\lim_{n \rightarrow \infty} |a_{n+1}/a_n| = \lim_{n \rightarrow \infty} (n+1)/n = 1$, but $\sum_{n=1}^{\infty} a_n$ diverges. Similarly, if $b_n = 1/n^2$ then $\lim_{n \rightarrow \infty} |b_{n+1}/b_n| = \lim_{n \rightarrow \infty} (n+1)^2/n^2 = 1$, and $\sum_{n=1}^{\infty} b_n$ converges.
\end{proof}

\textbf{** Problem 4.9}
\textit{Let $(x_n)_{n \in \mathbb{N}}$ be a bounded sequence of non-negative real numbers and let $x_0 = \lim \sup_{n \rightarrow \infty} x_n$. For any $\varepsilon > 0$, show that there are only finitely many terms of the sequence greater than $x_0 + \varepsilon$, whereas there are infinitely many terms less than $x_0 + \varepsilon$.}
\begin{proof}
We know that $x_0$ is the limit of a sequence $(y_n)$ where
\[
y_n = \sup \{a_k \mid k \geq n\}.
\]
Thus $(y_n)$ is a decreasing sequence. Let $\varepsilon > 0$ and choose $n$ such that $|x_0 - y_n| < \varepsilon$. Since $(y_n)$ is decreasing we have $x_0 < y_n < \varepsilon$. By definition, $y_n$ is greater than or equal to every term of $(x_n)$ except for those with indices less than $n$. The fact that $y_n < \varepsilon$ gives us the strict inequality for finitely many terms greater than $x_0 + \varepsilon$ and infinitely many less than $x_0 + \varepsilon$.
\end{proof}

\textbf{** Problem 4.10}
\textit{Let $\sum_{n=1}^{\infty} a_n$ be a series. Give examples to show that if $\lim \sup_{n \rightarrow \infty} |a_n|^{1/n} = r = 1$ then the series may converge or diverge.}
\begin{proof}
Consider $a_n = 1^n$ then $\limsup_{n \rightarrow \infty} |a_n|^{1/n} = \limsup_{n \rightarrow \infty} (1^n)^{1/n} = 1$, but $\sum_{n=1}^{\infty} a_n$ diverges. Similarly, if $b_n = 1/n^2$ then $\lim_{n \rightarrow \infty} |b_n|^{1/n} = \lim_{n \rightarrow \infty} (1/n^2)^{1/n} = 1$, and $\sum_{n=1}^{\infty} b_n$ converges.
\end{proof}

\textbf{** Problem 4.11}
\textit{Let $\sum_{n=1}^{\infty} a_n$ is a series such that $r=\lim_{n \rightarrow \infty} |a_{n+1}|/|a_n|$ exists. Show that $\lim \sup_{n \rightarrow \infty} |a_n|^{1/n} = r$ as well.}
\begin{proof}
This follows from the fact that $|a_n|^{1/n} \leq |a_{n+1}|/|a_n|$ for large enough values of $n$. Then $||a_n|^{1/n} - r| < ||a_{n+1}|/|a_n| - r| < \varepsilon$ if given $\varepsilon > 0$.
\end{proof}

\textbf{** Problem 4.12}
\textit{Show that if a complex power series around $z_0$ converges absolutely for a complex number $z$ then it also converges for any complex number $w$ such that $|w - z_0| \leq |z - z_0|$, that is, the series converges on the disk $\{w \in \mathbb{C} \mid |w - z_0| \leq |z - z_0|\}$.}
\begin{proof}
Let $\sum_{n=1}^{\infty} a_n (z - z_0)^n$ be absolutely convergent. Then note that
\[
|w-z_0|^n \leq |z-z_0|^n
\]
if $|w-z_0| \leq |z-z_0|$. But then
\[
|a_n (w-z_0)^n| = |a_n||(w-z_0)^n| = |a_n||w-z_0|^n \leq |a_n||z-z_0|^n = |a_n||(z-z_0)^n| = |a_n (z-z_0)^n|.
\]
Then by the comparison test, the power series will converge on the disk $\{w \in \mathbb{C} \mid |w - z_0| \leq |z - z_0|\}$.
\end{proof}

\textbf{** Problem 4.13}
\textit{Determine the radius of convergence for the following power series:\\
1)
\[
\sum_{n=0}^{\infty} \frac{z^n}{n!}.
\]
2)
\[
\sum_{n=2}^{\infty} \frac{z^n}{\ln (n)}.
\]
3)
\[
\sum_{n=1}^{\infty} \frac{n^n}{n!}z^n.
\]}
\begin{proof}
1) The sequence $a_n = 1/n!$ satisfies the ratio test so that $\lim_{n \rightarrow \infty} |a_{n+1}|/|a_n| = \lim_{n \rightarrow \infty} 1/(n+1) = 0$. The result of the root test must be the same and so the radius of convergence is infinity.\newline

2) The sequence $a_n = 1/\ln (n)$ satisfies the ratio test so that $\lim_{n \rightarrow \infty} |a_{n+1}|/|a_n| = \lim_{n \rightarrow \infty} \ln (n+1)/\ln (n) = 1$. The result of the root test must be the same and so the radius of convergence is $1$.\newline

3) The sequence $a_n = n^n/n!$ satisfies the root test so that $\limsup_{n \rightarrow \infty} |a_n|^{1/n} = \lim_{n \rightarrow \infty} n/(n!)^{1/n}$ diverges. The radius of convergence must then be $0$.
\end{proof}

\begin{**}
For $x, y \in \mathbb{R}^n$ Let
\[
||x||_p = \left ( \sum_{i=1}^{n} |x_i|^p \right )^{\frac{1}{p}}
\]
and $d_p (x,y) = ||x-y||_p$. Show that $d_p$ is a metric.
\end{**}
\begin{proof}
Let $x, y \in \mathbb{R}^n$. We have $|x_i-y_i| \geq 0$ and thus $|x_i-y_i|^p \geq 0$ for each $1 \leq i \leq n$. Then $\sum_{i=1}^n |x_i - y_i|^p \geq 0$ and raising this to $1/p$ we have
\[
d_p(x,y) = ||x-y||_p = \left ( \sum_{i=1}^{n} |x_i-y_i|^p \right )^{\frac{1}{p}} \geq 0.
\]
Now suppose that $x=y$. Then $x_i = y_i$ for all $1 \leq i \leq n$ and so $|x_i-y_i| = 0$ for all $1 \leq i \leq n$. It follows that $d(x,y) = 0$. Conversely, suppose that $d(x,y) = 0$. Then
\[
\left ( \sum_{i=1}^{n} |x_i-y_i|^p \right )^{\frac{1}{p}} = 0
\]
and raising both sides to the $p$th power we have $\sum_{i=1}^{n} |x_i-y_i|^p = 0$. But since $p > 1$ we know that $|x_i-y_i|^p \geq 0$ for all $1 \leq i \leq n$. Thus $|x_i-y_i| = 0$ and so $x_i = y_i$ for all $1 \leq i \leq n$. Therefore $x=y$.\newline

Note that since $|a-b| = |-1||a-b| = |-(a-b)| = |b-a|$ for all $a,b \in \mathbb{R}$ we have
\[
d_p(x,y) = ||x-y||_p = \left ( \sum_{i=1}^{n} |x_i-y_i|^p \right )^{\frac{1}{p}} = \left ( \sum_{i=1}^{n} |y_i-x_i|^p \right )^{\frac{1}{p}} = ||y-x||_p = d(y,x).
\]\newline

Now let $z \in \mathbb{R}^n$ as well. Note that
\[
||x-z||_p^p = \sum_{i=1}^{n} |x_i - z_i|^p \leq \sum_{i=1}^{n} |x_i - z_i|^{p-1}|x_i| + \sum_{i=1}^{n} |x_i - z_i|^{p-1}|z_i|.
\]
If we now assume that $q = p/(p-1)$, then we can apply H\"{o}lder's Inequality to both terms on the right so we have
\[
||x-z||_p^p \leq \left ( \sum_{i=1}^{n} |x_i|^p \right )^{\frac{1}{p}} \left ( \sum_{i=1}^{n} |x_i - z_i|^{(p-1)q} \right )^{\frac{1}{q}} + \left ( \sum_{i=1}^{n} |z_i|^p \right )^{\frac{1}{p}} \left ( \sum_{i=1}^{n} |x_i - z_i|^{(p-1)q} \right )^{\frac{1}{q}}.
\]
Now multiply both sides by
\[
\left ( \sum_{i=1}^{n} |x_i - z_i|^{(p-1)q} \right )^{-\frac{1}{q}}
\]
and note that $1 - 1/q = 1/p$ so that we have
\[
||x-z||_p^p = \left ( \sum_{i=1}^{n} |x_i - z_i|^p \right )^{\frac{1}{p}} \leq \left ( \sum_{i=1}^{n} |x_i|^p \right )^{\frac{1}{p}} + \left ( \sum_{i=1}^{n} |z_i|^p \right )^{\frac{1}{p}} \leq \left ( \sum_{i=1}^{n} |x_i-y_i|^p \right )^{\frac{1}{p}} + \left ( \sum_{i=1}^{n} |y_i-z_i|^p \right )^{\frac{1}{p}}
\]
Thus $||x-z||_p \leq ||x-y||_p + ||y-z||_p$.
\end{proof}

\begin{**}
Define $l_n^p(\mathbb{C})$.
\end{**}
\begin{proof}
The norm $l_n^p(\mathbb{C}$ is defined as
\[
||z||_p = \left ( \sum_{j=1}^{n} |z_i|^p \right )^{\frac{1}{p}}.
\]
The proof that this is a metric is the same as the proof that $l_n^p(\mathbb{R})$ is a metric because the properties of absolute value apply in the same way.
\end{proof}

\begin{**}
Show the following for $r, s \in \mathbb{Q}$ such that $r = a/b = p^k(a'/b')$ and $s = c/d = p^l(c'/d')$:\\
1) $|r|_p \geq 0$ and $|r|_p = 0$ if and only if $r = 0$.\\
2) $|rs|_p = |r|_p |s|_p$.\\
3) $|r+s|_p \leq \max (|r|_p, |s|_p)$ and $|r+s|_p = \max(|r|_p, |s|_p)$ if and only if $|r|_p \neq |s|_p$.
\end{**}
\begin{proof}
1) Note that $|r|_p = p^{-k} \geq 0$. Also, by definition $|0|_p = 0$.\newline

2) Note that
\[
rs = \left ( p^k \frac{a'}{b'} \right ) \left ( p^l \frac{c'}{d'} \right ) = p^{k+l} \frac{a'}{b'} \frac{c'}{d'}
\]
and so $|rs|_p = p^{-(k+l)} = p^{-k}p^{-l} = |r|_p |s|_p$.\newline

3) Note that
\[
r+s = p^k \frac{a'}{b'} + p^l \frac{c'}{d'} = \frac{p^ka'd' + p^lb'c'}{b'd'} = p^m \frac{p^{k-m}a'd' + p^{l-m}b'c'}{b'd'}
\]
where $m = \min (k, l)$. Then $|r+s|_p = p^{-m} \leq \max (p^{-k}, p^{-l}) = \max (|r|_p, |s|_p)$. Note that if $|r|_p \neq |s|_p$ then $p^{-k} \neq p^{-l}$ and so $m$ must be the $\min (k, l)$ which makes $|r+s|_p = \max (|r|_p, |s|_p)$.
\end{proof}

\begin{**}
Let $d_p(r, s) = |r-s|_p$ for $r,s \in \mathbb{Q}$. Show $d_p$ is a metric on $\mathbb{Q}$.
\end{**}
\begin{proof}
Let $r,s,t \in \mathbb{Q}$ such that $r = a/b = p^k(a'/b')$ and $s = c/d = p^l(c'/d')$. ** Problem 7 Part 1) shows that $d_p(x,y) \geq 0$ and that $d(x,y) = 0$ if and only if $x-y = 0$ which means $x = y$. Now consider
\[
r-s = p^k \frac{a'}{b'} - p^l \frac{c'}{d'} = \frac{p^ka'd' - p^lb'c'}{b'd'} = p^m \frac{p^{k-m}a'd' - p^{l-m}b'c'}{b'd'}
\]
and
\[
s-r = p^l \frac{c'}{d'} - p^k \frac{a'}{b'} = \frac{p^lb'c' - p^ka'd'}{b'd'} = p^m \frac{p^{l-m}b'c' - p^{k-m}a'd'}{b'd'}
\]
where $m = \min (k, l)$. Then $|r-s|_p = p^{-m} = |s-r|_p$. Finally, note that
\[
|r-t| \leq \max (|r|_p, |t|_p) \leq \max (|r|_p, |s|_p) + \max (|s|_p, |t|_p) = |r-s|_p + |s-t|_p
\]
since $|x|_p \geq 0$ for all $x \in \mathbb{Q}$.
\end{proof}

\end{flushleft}
\end{document}